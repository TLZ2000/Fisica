\section{Elettromagnetismo}

    \subsection{Introduzione all'elettromagnetismo}
        In natura, esistono 2 tipi di materiali (elettricamente parlando):
        \begin{itemize}
            \item \textbf{conduttori}: materiali abbastanza noiosi (PER IL PROGRAMMA CHE ABBIAMO FATTO NOI QUEST'ANNO) perché \textbf{non si magnetizzano};
            \item \textbf{isolanti}: tali materiali \textbf{possono essere "elettrizzati"} (ovvero possono assumere delle cariche). Di questi possiamo applicare un'altra classificazione:
            \begin{itemize}
                \item \textbf{vetri e affini};
                \item \textbf{plastiche}.
            \end{itemize}
        \end{itemize}
    
    \subsection{Forza di Culomb}
        Vediamo ora perché questa classificazione è interessante:
        \framedImg{8}{L21A-img001}
        Ovvero, 2 cariche dello stesso materiale (vetri/vetri o plastiche/plastiche) poste vicine \textbf{si respingono tra loro}. Al contrario, 2 cariche di materiali diversi (vetri/plastiche) \textbf{si attraggono}! Per esplicitare questo concetto di attrazione/repulsione associamo alle cariche dei vetri \textbf{il segno +}, mentre alle cariche dei materiali plastici associamo \textbf{il segno -}. Altro elemento importante è che tale forza si manifesta sull'\textbf{asse congiungente le 2 cariche}. Ora, come quantifichiamo questa forza? Possiamo usare la \textbf{\textit{forza di Culomb}}:
        \begin{align*}
            \textcolor{Red}{F_{el}=\textcolor{OliveGreen}{K}*\frac{q_1*q_2}{r^2_{1,2}}*\hat{r_{1,2}}=\textcolor{OliveGreen}{\frac{1}{4\pi\varepsilon_0}}*\frac{q_1*q_2}{r^2_{1,2}}*\hat{r_{1,2}}}
        \end{align*}
        Dunque, la \textbf{forza elettrica} ($F_{el}$) è una forza appunto, quindi unità di misura $Newton$, per quanto riguarda le 2 cariche $q_1$ e $q_2$ hanno come unità di misura il $Culomb$. Per quanto riguarda $K$ invece è una costante che permette di ottenere una forza appunto, a sua volta può \textbf{essere scomposta} come $\frac{1}{4\pi\varepsilon_0}$ (utile quando usiamo il teorema di Gauss) dove compare la \textbf{\textit{costante dielettrica del vuoto}} $\textcolor{Red}{\varepsilon_0 = 8,85418*10^{-12}\frac{C^2}{N*m^2}}$. Facciamo un disegnino per visualizzare meglio la situazione:
        \framedImg{6}{L21A-img002}
        Nota che noi supponiamo sempre che le 2 cariche siano ferme.

        \subsubsection{Digressione sugli atomi}
            La materia è composta da atomi, a loro volta gli atomi sono composti da un \textbf{nucleo} di \textbf{protoni} (carica positiva), \textbf{neutroni} (carica neutra, sono la "colla" che tiene insieme i protoni [altrimenti le cariche positive si respingerebbero tra di loro]) ed \textbf{elettroni} (carica negativa). Di solito, l'atomo ha una carica elettrica complessiva nulla (\textbf{i protoni si bilanciano con gli elettroni}), però è possibile "strappare" o aggiungere degli elettroni, in modo che l'atomo abbia una carica elettrica. Ora, è possibile osservare che \textbf{non esistono cariche libere più piccole di una certa soglia} (che corrisponde alla carica dell'elettrone/protone), ovvero $\textcolor{Red}{e=1,602*10^{-19}C}$. Possiamo fare una tabellina veloce:
            \begin{center}
                \begin{tabular}{ c c c }
                    & \textbf{Carica} & \textbf{Massa} \\ 
                    \textbf{Elettrone} & $-e$ & $m_{el}=0,91*10^{-30}Kg$ \\  
                    \textbf{Protone} & $+e$ & $m_{pr}=1,67*10^{-27}Kg$\\    
                    \textbf{Neutrone} & $0$ & $m_{ne}\approx m_{pr}$
                \end{tabular}
            \end{center}
            Ora, perché mettiamo anche la massa? Possiamo notare che la \textit{forza di gravitazione universale} e la \textit{forza di Culomb} hanno una forma \textbf{estremamente simile}! Risulta quindi naturale provare a confrontare le 2 forze, possiamo immaginare di usare un altomo di idrogeno (1 protone e 1 elettrone). Abbiamo quindi:
            \begin{align*}
                &<r> = 5,3*10^{-11}m\\
                &F_{el} = -K*\frac{e^2}{<r>^2}\approx-8,2*10^{-8}N && Forza\ elettrica\\
                &F_{gr} = -G*\frac{m_e*m_P}{<r>^2}\approx-3,6*10^{-47}N && Forza\ gravitazionale
            \end{align*}
            Si nota che \textbf{la forza graviazione è diversi ordini di grandezza inferiore a quella elettrica}! Risulta quindi sensato \textbf{ignorare la forza gravitazionale all'interno dei legami atomici}!


    \subsection{Principio di sovrapposizione}
        Supponiamo di essere in questa situazione: abbiamo \textbf{2 cariche} "importanti" statiche ($Q_1$ e $Q_2$) ed una \textbf{carica di prova} ($q_0$). Ora, la carica di $q_0$ è \textbf{molto minore rispetto alle altre 2} ($|q_0|<<Q_1$ e $|q_0|<<Q_2$): questo è molto importante perché, come già detto prima, \textbf{esiste una carica "elettromagnetica" [ricontrollla questo ""] minima} ($1,6*10^{-9}C$) \textbf{sotto la quale non si può andare}! Quindi anche $q_0$ ha una carica elettrica che va ad interagire con le altre 2 cariche e, quindi, a perturbare il nostro sistema! Questa supposizione serve a perturbare il sistema il meno possibile. Tornando al discorso iniziale, supponiamo di avere queste cariche posizionate in su un piano e che interagiscano tra loro in questo modo:
        \framedImg{8}{L21-img001}
        Secondo la \textbf{legge di Culomb}, $Q_1$ eserciterà una forza $F_{1->0}$ sulla carica $q_0$ (lo stesso vale per $Q_2$ con la forza $F_{2->0}$), possiamo immaginare che le 2 forze siano quelle mostrate in figura (sono uguali solo per puro caso). Ora, per capire quale forza agisce complessivamente sulla carica $q_0$ possiamo \textbf{sommare i 2 vettori delle forze secondo la regola del parallelogramma}: per fare questo ci avvaliamo del \textbf{principio di sovrapposizione}, che ci garantisce che \textbf{l'interazione tra $Q_1$ e $q_0$ non viene perturbata in alcun modo dall'interazione tra $Q_2$ e $q_0$ e viceversa}, permettendoci quindi di \underline{\textbf{"sovrapporre" le 2 forze}!} (Sostanzialmente, il principio di sovrapposizione dice solo la parte sulla sovrapposizione, la prima parte delle interazioni è una conseguenza di questa seconda parte :P) \bigskip\\
        In questo caso specifico abbiamo solo 2 cariche puntiformi, però nulla ci vieta di usare N cariche disposte in varie forme, ad esempio un cavo:
        \framedImg{4}{L21-img002}
        In questo caso però si deve introdurre una \textbf{densità lineare di carica "$\mathbf{\lambda}$"} che, sostanzialmente, corrisponde alla \textbf{quantità di carica ($dq$) disposta presente in una certa lunghezza di filo ($dl$)}. Come per l'esempio delle cariche puntiformi, la forza totale che agisce su $q_0$ corrisponde alla \textbf{sovrapposizione di tutte le forze (infinitesime) delle varie cariche (infinitesime, infatti $dq = \lambda dl$)}. Allo stesso modo, il discorso si può estendere a 2 e 3 dimensioni, l'unica cosa che cambia è che al posto di $\lambda$ (1-D) avremmo $\sigma$ (2-D, \textit{densità superficiale di carica}) e $\rho$ (3-D, \textit{densità volumica di carica}):
        \framedImg{4}{L21-img003}

    \subsection{Campo elettrico}
        Diventa interessante analizzare l'effetto che le cariche hanno sulla carica di prova \textbf{senza considerare le cariche stesse} (senza dover quindi effettuare un integrale considerando tutte le cariche)! Per questo motivo viene introdotto il concetto di \textbf{campo elettrico} (o meglio, \textit{elettrostatico} dato che le nostre cariche sono ferme). Formalmente, definiamo il campo elettrico ($\vec{E}$) come il rapporto tra la forza provata dalla carica di prova ($\vec{F}$) e la carica di prova stessa ($q_0$), quindi:
        \begin{align*}
            \textcolor{Red}{\vec{E} \triangleq \frac{\vec{F}}{q_0}}
        \end{align*}
        Pensa a $q_0$ come ad una sonda che \textbf{va a \underline{misurare} la forza in un certo punto dello spazio}, misurando questa forza in, idealmente, tutte le possibili posizioni e mettendo tutto insieme otteniamo un \textbf{campo vettoriale} dove ad ogni posizione è assegnato un vettore!

        \subsubsection{Campo elettrico per campo puntiforme}
            Cominciamo considerando una \textbf{carica puntiforme}: se inseriamo la nostra sonda e facciamo un sistema del genere:
            \framedImg{7}{L21-img005}
            Possiamo dire che, per definizione, il campo elettrico della carica $Q$ corrisponde a:
            \begin{align*}
                &\vec{F}=K*\frac{q_0*Q}{r^2}\hat{r}&&=>\vec{E}=K*\frac{\bcancel{q_0}*Q}{r^2*\bcancel{q_0}}\hat{r}=\textcolor{Red}{K*\frac{Q}{r^2}\hat{r}}
            \end{align*}
            Si nota facilmente che, tralasciando la costante elettrica $K$, il campo elettrico è \textbf{direttamente proporzionale alla carica elettrica $Q$ ed è diretto in senso raiale alla carica}, ottenendo qualcosa del genere:
            \framedImg{7}{L21-img004}
            A livello di convenzione, si assume sempre che la nostra \textbf{carica di prova $q_0$ sia positiva}, quindi:
            \begin{itemize}
                \item se $Q>0$ allora \textbf{il campo sarà uscente};
                \item se $Q<0$ allora \textbf{il campo sarà entrante};
            \end{itemize}

        \subsubsection{Campo di dipolo}
            Se ora prendiamo 2 sorgenti puntiformi statiche di segno opposto e le mettiamo vicine. Per ogni carica puntiforme ci saranno delle \textbf{linee di campo} (uscenti per $+Q$ ed entranti per $-Q$) e, se le congiungiamo, otteniamo un campo elettrico del genere:
            \framedImg{4}{L21-img006}
            In ogni posizione associamo un vettore che indica il campo elettrico in quel punto specifico: in questo caso il campo elettrico sarà dato proprio dalla \textbf{sovrapposizione delle forze delle 2 cariche Q}! Se ci viene fornita la funzione che descrive questo \textbf{campo elettrico in funzione della posizione}, ci \textbf{risparmiamo tutto il lavoro sul calcolo degli effetti sulla carica di prova dovuti all'interazione con tutte le sorgenti del nostro sistema}! In un certo senso, la nozione di campo elettrico ci permette di descrivere la "realtà fisica" del nostro sistema \textbf{senza dover considerare in modo "geometrico, perfetto" le sorgenti} del campo (le sorgenti potrebbero essere estremamente tante, irregolari, difficili da analizzare o, adirittura, impossibili da analizzare perché magari non le conosciamo, quindi con il campo elettrico andiamo a considerare solo l'effetto che queste cariche hanno sulla nostra carica di prova).\bigskip\\
            Nota però che con questa rappresentazione grafica perdiamo la quantificazione del modulo del vettore della forza nella posizione specifica, ma questo non è un problema.

        \subsubsection{Esempio di esercizio con il campo elettrico}
            La formula fondamentale per lavorare negli esercizi con i campi elettrici è questa:
            \begin{align*}
                \textcolor{Red}{\vec{F}(\vec{r})\triangleq q* \vec{E}(\vec{r})}
            \end{align*}
            Vediamo il problema:
            \framedImg{8}{L21-img007}
            Dunque, abbiamo una \textbf{carica di prova} $p$ di \textbf{massa} $m$ che \textbf{si muove ad una velocità} $v_0$. Ad un certo punto \textcolor{Blue}{\textbf{entra nel campo elettrico}} (nota che nella realtà uno "spezzamento" così netto tra "campo elettrico" e "non campo elettrico" non è realizzabile, in questo caso facciamo delle approssimazioni e non ci interessa :P). Ora, entrando nel campo elettrico, la nostra carica di prova "devia" dal suo percorso: quando uscirà dal campo elettrico, lo farà con un \textbf{certo angolo $\mathbf{\theta}$}. Conoscendo il \textbf{campo elettrico} $\vec{E}$, quanto vale l'\textbf{l'angolo $\mathbf{\theta}$}?\bigskip\\
            Allora, cominciamo con la seconda legge della dinamica:
            \begin{align*}
                \vec{F} = m*\vec{a}=m*\frac{d\vec{v}}{dt}
            \end{align*}
            Però a cosa corrisponde quella forza specifica? Proprio alla forza dovuta all campo elettrico:
            \begin{align*}
                \vec{F}=q*\vec{E}=q*E*\hat{y}
            \end{align*}
            Ora, quest'ultimo pezzo è importante: la forza legata al \textbf{campo elettrico agisce solo sull'asse y}! La velocità della carica sull'asse x quindi resterà costante:
            \begin{align*}
                &\begin{cases}
                    x)\frac{dv_x}{dt}=0 & Perche\ v_x\ è\ costante\\
                    y)
                    \begin{cases}
                        \vec{F} = m*a_y =m*\frac{dv_y}{dt}\\
                        \vec{F} = q*\vec{E} =q*E
                    \end{cases}&=>\textcolor{Red}{\frac{dv_y}{dt}=\frac{q*E}{m}}
                \end{cases}
            \end{align*}
            Facciamo qualche magheggio da fisico su \textcolor{Red}{questo pezzo}, ottenendo:
            \begin{align*}
                & \frac{dv_y}{dt}=\frac{q*E}{m} && => dv_y=\frac{q*E}{m} * dt\\
                & && => \int dv_y = \int \frac{q*E}{m} * dt\\
                & v_y\ dipende\ da\ t\ perche\ tutto\ il\ resto\ e'\ costante && => v_y(t) = \frac{q*E}{m}*t
            \end{align*}
            Bene, abbiamo ottenuto $v_y$ \textbf{in funzione del tempo $t$}: quanto vale $t$? Possiamo scoprilo sfruttando la x:
            \begin{align*}
                &\frac{dv_x}{dt}=0 && => v_x = const = v_0\\
                & && => \frac{dx}{dt}= v_0\\
                & && => dx = v_0 dt\\
                & && => \int dx = \int v_0 dt\\
                & && => \int dx = v_0 \int dt\\
                & && => x(t) = v_0*t\\
                & && => \textcolor{Red}{\frac{x(t)}{v_0} = t}
            \end{align*}
            Ora che abbiamo il \textcolor{Red}{tempo} per percorrere lo spazio L (fai riferimento al disegno) possiamo \textbf{sostituirla nell'equazione della velocità}!
            \begin{align*}
                &\begin{cases}
                    t = \frac{x(t)}{v_0}\\
                    v_y(t) = \frac{q*E}{m}*t
                \end{cases}
                && => v_y(t) = \frac{q*E}{m}*\frac{x(t)}{v_0}
            \end{align*}
            Abbiamo la \textcolor{Orange}{velocità su y}, come la riconduciamo all'\textbf{angolo $\mathbf{theta}$}? Ricorda che:
            \begin{align*}
                &\vec{v} =
                \begin{cases}
                    v_y = sin(\theta)*|\vec{v}|\\
                    v_x = cos(\theta)*|\vec{v}|\\
                \end{cases} && => \mathbf{\textcolor{Red}{\frac{v_y}{v_x} = \frac{sin(\theta)*\bcancel{|\vec{v}|}}{cos(\theta)*\bcancel{|\vec{v}|}} = tan(\theta)}}
            \end{align*}
            \framedImg{2}{L21-img008}
            Ora ci basta mettere tutto a sistema:
            \begin{align*}
                \frac{v_y}{v_x} = tan(\theta) = \frac{q*E}{m}*\frac{x(t)}{v_0} *\frac{1}{v_0} = \frac{q*E}{m} * \frac{L}{v_0^2}
            \end{align*}
            Ricorda che $x(t) = L$. Abbiamo finito, questo è il nostro risultato finale:
            \begin{align*}
                \mathbf{tan(\theta) = \frac{q*E}{m} * \frac{L}{v_0^2}} = \frac{q*E*L}{2*(\frac{1}{2}*m*v_0^2)} = \textcolor{Orange}{\frac{q*E*L}{2*E_{K, ini}}}
            \end{align*}
            Volendo è possibile fare qualche trasformazione algebrica per ricondurci alla \textcolor{Orange}{all'energia cinetica della particella di prova} (utile per fare delle considerazioni teoriche, esperimento di Rutherford).

    \subsection{Circuitazione e potenziale elettrico}
        Dato che parliamo di "\textit{forza elettrica}" (quella di \textit{Culomb}) ha senso cercare di \textbf{calcolare il lavoro compiuto da tale forza}! Noi consideriamo solo il \textbf{caso elettrostatico} (dato che abbiamo fatto solo quello). Dunque, la definizione generale di lavoro ci dice che, il lavoro ampiuto da una forza tra il punto A ed il punto B corrisponde a:
        \begin{align*}
            W_{A\rightarrow B}=\int^A_B\vec{F}d\vec{s}
        \end{align*}
        A questa definizione possiamo applicare quella di \textbf{campo elettrico} (per la precisione, \textit{elettrostatico}), ottenendo:
        \begin{align*}
            W_{A\rightarrow B}=\int^A_B\vec{F}ds=\int^A_Bq*\vec{E}d\vec{s}
        \end{align*}
        E togliamo la "$q$" dall'integrale in quanto costante:
        \begin{align*}
            W_{A\rightarrow B}=\int^A_B\vec{F}ds=\int^A_Bq*\vec{E}d\vec{s}=q*\int^A_B\vec{E}d\vec{s}
        \end{align*}
        Ora \textcolor{Red}{questo pezzo} viene chiamato \textbf{circuitazione}, nel caso delle cariche statiche vale:
        \begin{align*}
            \oint\vec{E}\ d\vec{s} = 0
        \end{align*}
        \subsubsection{Energia potenziale elettrica e potenziale (sono 2 cose relativamente diverse :P) nel caso di cariche puntiformi singole}
            Allora, nel caso delle cariche singole possiamo dimostrare che la circuitazione vale 0 in modo molto semplice. Per farlo, "calcoliamo" quanto vale il lavoro per gli spostamenti "elementrari", in particolare quelli "\textcolor{Red}{radiali alla carica}" (paralleli al versore con la carica) e quelli "\textcolor{OliveGreen}{ortogonali alla carica}" (ortogonali al versore con la carica):
            \framedImg{8}{L22-img001}
            Il tutto si basa sui \textbf{versori}: nel caso rosso abbiamo $\hat{r}*\hat{r} = 1$, mentre nel caso verde abbiamo $\hat{r}*\hat{\theta} = 0$ (infatti $\hat{\theta}$ è ortogonale a $\hat{r}$, quindi il prodotto è uguale a 0), ciò porta tutto a 0!
            Ora, vediamo un esempio, supponiamo di avere questo spostamento da A a B:
            \framedImg{4}{L22-img002}
            L'idea è quella di \textbf{scomporre lo spostamento in tratti \textcolor{OliveGreen}{ortogonali} e tratti \textcolor{Red}{radiali}}. Alla fine possiamo calcolare il lavoro:
            \begin{align*}
                W_{A\rightarrow B} = \int_B^AdW=\textcolor{OliveGreen}{\int_B^AdW_{ort}} + \textcolor{Red}{\int_B^AdW_{rad}}
            \end{align*}
            Però, sappiamo dalla dimostrazione data prima che il lavoro ortogonale è \textbf{uguale a 0}! Quindi:
            \begin{align*}
                W_{A\rightarrow B} = \int_B^AdW=\bcancel{\textcolor{OliveGreen}{\int_B^AdW_{ort}} }+ \textcolor{Red}{\int_B^AdW_{rad}}
            \end{align*}
            Allora sviluppiamo l'itegrale del lavoro radiale:
            \begin{align*}
                W_{A\rightarrow B} = \int_B^AdW_{rad} =q*\int^A_B\frac{KQ}{r^2}dr
            \end{align*}
            Ora, supponiamo di avere "segmenti infinitamente piccoli", quindi avremmo \textcolor{Blue}{qualcosa del tipo}:
            \begin{align*}
                W_{A\rightarrow B} = \int_B^AdW_{rad} =q*\int^A_B\frac{KQ}{r^2}dr = -qQK[\textcolor{Blue}{(\frac{1}{r_1}-\frac{1}{r_A})+(\frac{1}{r_2}-\frac{1}{r_1})+...+(\frac{1}{r_{N-1}}-\frac{1}{r_{N-2}})+(\frac{1}{r_{B}}-\frac{1}{r_{N-1}})}]
            \end{align*}
            Di questa serie infinita di addizioni e sottrazioni ci sono \textbf{un sacco di elementi che si annullano tra di loro}! In particolare lasceranno SOLO i punti agli estremi:
            \begin{align*}
                W_{A\rightarrow B} = \int_B^AdW_{rad} =q*\int^A_B\frac{KQ}{r^2}dr = -qQK[(\frac{1}{r_1}-\frac{1}{r_A})+(\frac{1}{r_2}-\frac{1}{r_1})+...+(\frac{1}{r_{N-1}}-\frac{1}{r_{N-2}})+(\frac{1}{r_{B}}-\frac{1}{r_{N-1}})]=\\
                \textcolor{Red}{\frac{-qQK}{r_B}-\frac{-qQK}{r_A}}
            \end{align*}
            La cosa importante da ricordare qui è che \textcolor{Red}{il lavoro dipende solo dagli estremi $A$ e $B$}! Non da altri punti!\\
            Facciamo un po' di considerazioni:
            \begin{itemize}
                \item \textbf{energia potenziale elettrostatica}: dato che la forza elettrica è conservativa (ovvero il lavoro su un percorso chiuso è 0), possiamo dire che la \textbf{variazione di energia potenziale  elettrica $\Delta U$ corrisponde al lavoro * -1}:
                \begin{align*}
                    W_{AB} = -\Delta U_{AB}\\
                    ---dove---\\
                    U_X=\frac{KQq}{r_X}+const
                \end{align*}
                Nota che la $const$ è una costante arbitraria (che nella pratica non è importante dato che quando facciamo la differenza si semplifica);
                \item \textbf{La forza potenziale elettrica è conservativa}, quindi abbiamo che:
                \begin{align*}
                    \oint\vec{E}\ dr = 0
                \end{align*}
                Infatti sappiamo che "$W_{A\rightarrow A} = 0$", possiamo scomporre tale lavoro come "$W_{A\rightarrow A} = q\oint \vec{E}dr = 0$". Dato che "$q=const$", allora deve essere che "$\oint \vec{E}dr = 0$"!
                \item\textbf{Differenza di potenziale $\Delta V$}: come abbiamo fatto per il campo elettrico, viene naturale pensare di fare lo stesso con l'energia potenziale elettrica:
                \begin{align*}
                    \Delta V = \frac{\Delta U}{q}
                \end{align*}
                Quindi possiamo dire che il \textbf{potenziale nella posizione A} corrisponde a:
                \begin{align*}
                    V_A=\frac{KQ}{r_A}+const'
                \end{align*}
                Ora, anche qui la costante $const'$ è arbitraria e relativamente poco importante (a noi interessa la \textbf{differenza} di potenziale, quindi se facciamo la differenza questa costante si semplifica).
            \end{itemize}

        \subsubsection{Energia potenziale elettrica e potenziale nel caso di cariche puntiformi multiple}
            Abbiamo visto il caso delle cariche puntiformi singole, ma nel caso ne avessimo di più? Sostanzialmente la stessa cosa di prima, però usiamo \textbf{il principio di sovreapposizione}! In pratica consiste nel calcolare il \textbf{lavoro/energia potenziale/potenziale di tutte le cariche e li sommiamo}!
            \begin{align*}
                &W_{TOT, A}=\sum_i W_{i, A}\\
                &\Delta U_{TOT, A} = \sum_i \Delta U_{i, A}\\
                &\Delta V_{TOT, A}=\sum_i V_{i, A}
            \end{align*}

    \subsection{Teorema di Gauss}
        Prima di introdurre il teorema vero e proprio, dobbiamo introdurre diversi concetti importanti:
        \subsubsection{Concetto di "linee di forza"}
            Possiamo rappresentare un campo vettoriale (un campo elettrico nel nostro caso) tramite delle \textbf{linee di forza}, ad esempio:
            \framedImg{4}{L21-img006}
            Da queste linee di forza possiamo "estrarre" le informazioni relative al vettore ($\vec{v}$) di una certa posizione:
            \begin{itemize}
                \item \textbf{direzione di $\vec{v_A}$}: corrisponde alla tangente della linea di forza nella posizione A;
                \item \textbf{verso di $\vec{v_A}$}: corrisponde al verso di percorrenza della linea di forza nella posizione A;
                \item \textbf{modulo di $\vec{v_A}$}: corrisponde alla densità delle linee di forza nel punto A, non offre spunti di calcolo particolari ma è comunque utile per capire dove il campo è più o meno forte;
            \end{itemize}

        \subsubsection{Concetto di "flusso di $\vec{v}$"}
            Supponiamo di avere una \textbf{superficie infinitesima $d\Sigma$ rappresentata da un versore $\hat{n}$} (normale alla superficie). Ora, il flusso infinitesimo viene identificato come:
            \begin{align*}
                d\Phi = \vec{v}*\hat{n}*d\Sigma
            \end{align*}
            Possiamo vederlo come la "\textit{quantità infinitesima di forza che passa attraverso la superficie infinitesima}". Se immaginiamo di estendere tale superficie in modo che non sia più infinitesima, possiamo calcolare il flusso semplicemente applicando un integrale (inteso come la somma di tutte le aree infinitesime che rappresentano la nostra superficie non infinitesima):
            \begin{align*}
                \Phi = \int_\Sigma\vec{v}*\hat{n}*d\Sigma
            \end{align*}
            Vediamo un esempio veloce:
            \framedImg{4}{L22-img003.jpg}
            Supponiamo di avere un fiume dove l'acqua scorre con una velocità di $10m/s$ lungo il versore $\hat{x}$ e di avere una sezione del fiume con un'area di $20m^2$, identificata da un versore normale $\hat{x}$ che ha la stessa direzione di $\hat{x}$. Ora, se volessimo calcolare il flusso di acqua che passa attraverso la sezione del fiume dovremmo semplicemente applicare la formula vista prima! Espandiamo un po':
            \begin{align*}
                \Phi = \int_\Sigma d\Phi = \int_\Sigma \vec{v}*\hat{n}*d\Sigma = \textcolor{Red}{(\vec{v}*\hat{n})}*\textcolor{OliveGreen}{\int_\Sigma d\Sigma}
            \end{align*}
            Dunque, \textcolor{Red}{questo pezzo} lo portiamo fuori dato che non dipende da $\Sigma$, mentre \textcolor{OliveGreen}{quest'altro} corrisponde, idealmente, alla somma di tutte le aree infinitesime che compongono la nostra area non infinitesima (quindi corrisponde prorpio alla nostra superficie "completa" $\Sigma$):
            \begin{align*}
                \Phi = \textcolor{Red}{(\vec{v}*\hat{n})}*\textcolor{OliveGreen}{\int_\Sigma d\Sigma} = 10m/s*\textcolor{Orange}{(\hat{x}*\hat{n})}*\Sigma = 10m/s*\textcolor{Orange}{(1)}*20m^2 = 200m^3/s
            \end{align*}
            Molto semplice!
        
        \subsubsection{Concetto di "angolo solido"}
            Anche qui il concetto è molto semplice, partiamo da quello di "\textit{angolo piano}":
            \framedImg{3}{L22-img004.jpg}
            L'angolo $\alpha$ corrisponde alla lunghezza della sezione circolare $a$ fratto il raggio $R$. Ora, possiamo pensare di prendere questo concetto ed allargarlo aggiungendo una dimensione in più, ottenendo appunto l'\textit{\textbf{angolo solido $\omega$}}!
            \framedImg{3}{L22-img005.jpg}
            Cosa succede però se abbiamo una calotta sferica (la $S$) \textbf{non ortogonale} al raggio? Ricorda che possiamo identificare una superficie con il versore ortogonale associato, in altre parole cosa facciamo quanto il versore di S non è parallelo al raggio? Basta semplicemente fare la proiezione della calotta non ortogonale in modo che diventi ortogonale al raggio! In pratica basta fare così:
            \framedImg{3}{L22-img006.jpg}
            Ora un po' di considerazioni che ci serviranno in seguito, pensa all'angolo piano: quanto vale l'angolo piano totale? Il massimo che possiamo avere come arco è l'intero cerchio, quindi:
            \framedImg{7}{L22-img007.jpg}
            Se ci pensi è proprio \textbf{l'angolo giro espresso in radianti}! Ora, facciamo la stessa cosa con gli angoli solidi, la massima calotta sferica che possiamo avere è l'intera sfera, quindi:
            \begin{align*}
                \omega_{max}=\frac{<area\ sfera>}{r^2}=\frac{4\pi r^2}{r^2}=4\pi
            \end{align*}
            Tieni a mente questa roba, torna dopo nella dimostrazione del teorema di gauss!
        
        \subsubsection{Il teorema vero e proprio}
            Ora che abbiamo tutti i concetti fondamentali, possiamo introdurre il \textbf{teorema di Gauss} vero e proprio! La domanda a cui vogliamo rispondere è, quanto vale:
            \begin{align*}
                \textcolor{Red}{\oint \vec{E}*\hat{n}*d\Sigma = ?}
            \end{align*}
            Ovvero, \textbf{quanto vale il flusso che passa attraverso una superficie chiusa \textnormal{(il "$\oint$" indica una superficie chiusa)}?}
            Vediamo un po' di casistiche:
            \paragraph{Cariche puntiformi}
                A sua volta, possiamo considerare anche altre casistiche:
                \subparagraph{Cariche puntiformi singole interne a superfici chiuse}
                    Dunque, ci troviamo in una situazione di questo tipo:
                    Abbiamo una \textbf{carica $q$} che si trova all'interno di una certa superficie chiusa. Ora, scegliamo una generica \textcolor{Red}{$d\Sigma$} che avrà un suo versore ortogonale \textcolor{Red}{$\hat{n}$}. Quindi facciamo delle considerazione sull'integrando:
                    \begin{align*}
                        \vec{E}*\hat{n}*d\Sigma
                    \end{align*}
                    Come prima cosa espandiamo la definizione di campo:
                    \begin{align*}
                        \vec{E}*\hat{n}*d\Sigma=\frac{q}{4\pi*\varepsilon_0*r^2}\textcolor{Blue}{\hat{r}*\hat{n}d\Sigma}
                    \end{align*}
                    Bene, adesso \textcolor{Blue}{questo pezzo è la definizione vista nell'angolo solido con la calotta non ortogonale},quindi la applichiamo ed otteniamo la \textbf{superficie ortogonale}:
                    \begin{align*}
                        \vec{E}*\hat{n}*d\Sigma=\frac{q}{4\pi*\varepsilon_0*r^2}\textcolor{Blue}{\hat{r}*\hat{n}d\Sigma} = \frac{q}{4\pi*\varepsilon_0*r^2}*\textcolor{Blue}{d\Sigma_\perp}
                    \end{align*}
                    Però sappiamo per la definizione di angolo solido, sappiamo che la superficie ortogonale corrisponde proprio all'angolo solido moltiplicato per il raggio$^2$ (dobbiamo fare una trasformazione veloce rispetto alla formula di prima, ma nienete di speciale):
                    \begin{align*}
                        \vec{E}*\hat{n}*d\Sigma=\frac{q}{4\pi*\varepsilon_0*r^2}\hat{r}*\hat{n}d\Sigma = \frac{q}{4\pi*\varepsilon_0*r^2}*\textcolor{Purple}{d\Sigma_\perp} = \frac{q}{4\pi*\varepsilon_0*\bcancel{r^2}}*\textcolor{Purple}{\bcancel{r^2}*d\omega} = \underline{\textcolor{Red}{\frac{q}{4\pi*\varepsilon_0}d\omega}}
                    \end{align*}
                    Si può notare che \textbf{è possibile semplificare un $r^2$}! Proprietà estremamente importante del teorema di gauss perchè ci dice che il flusso \textbf{NON dipende dall'orientamento della superficie}, infatti in qualsiasi modo sia orietata noi andremo solo a considerare l'angolo solido da cui "si vede" quella superficie. Ottimo! Possiamo reintrodurre l'integrale:
                    \begin{align*}
                        \oint\vec{E}*\hat{n}*d\Sigma= \oint\textcolor{Red}{\frac{q}{4\pi*\varepsilon_0}d\omega}
                    \end{align*}
                    Portiamo fuori tutto quello che non dipende da $\omega$ (quindi tutto :P):
                    \begin{align*}
                        \oint\vec{E}*\hat{n}*d\Sigma = \oint\frac{q}{4\pi*\varepsilon_0}d\omega = \frac{q}{4\pi*\varepsilon_0}\oint d\omega
                    \end{align*}
                    "Risolviamo" l'integrale e quello ci darà \textbf{l'angolo solido totale} che, abbiamo visto prima, \textbf{vale proprio $4\pi$}, quindi possiamo fare un'altra semplificazione:
                    \begin{align*}
                        \oint\vec{E}*\hat{n}*d\Sigma = \oint\frac{q}{4\pi*\varepsilon_0}d\omega = \frac{q}{4\pi*\varepsilon_0}\oint d\omega = \frac{q}{4\pi*\varepsilon_0}\omega_{TOT}=\frac{q}{\bcancel{4\pi}*\varepsilon_0}\bcancel{4\pi}=\underline{\textcolor{Red}{\frac{q}{\varepsilon_0}}}
                    \end{align*}
                    \textcolor{Red}{QUESTO} è il risultato finale! Ci dice che il flusso passante attraverso una superficie chiusa dipende solo dalla carica!

                \subparagraph{Cariche puntiformi singole esterne a superfici chiuse}
                    Risposta veloce: \textcolor{Red}{il flusso in questo caso vale 0}! Ma perché?
                    \framedImg{7}{L22-img008.jpg}
                    Dall'immagine possiamo notare che il flusso attraversa la superficie \textbf{2 volte}: la prima \textbf{entra} (quindi il flusso sarà negativo) e la seconda volta \textbf{esce} (quindi il flusso sarà positivo). Facciamo un po' di passaggi:
                    \begin{align*}
                        \textcolor{Red}{\vec{E}_1*\hat{n}_1*dS_1} = ... = \textcolor{Red}{-\frac{q}{4\pi \varepsilon_0}d\omega_1}\\
                        \textcolor{OliveGreen}{\vec{E}_2*\hat{n}_2*dS_2} = ... = \textcolor{OliveGreen}{\frac{q}{4\pi \varepsilon_0}d\omega_2}
                    \end{align*}
                    Ora, la "fregatura" sta nel fatto che, in entrambe le equazioni, \textbf{l'angolo solido è lo stesso}! Nel caso di $dS_1$ la calotta sferica è più piccola, però è anche più vicina alla carica! Nel caso di $dS_2$ invece è più grande, ma è anche più lontana dalla carica! Tutto si bilancia in modo che l'angolo solido è lo stesso! Per calcolare il flusso totale infinitesimo possiamo quindi sommare tutto, ottenendo:
                    \begin{align*}
                        \vec{E}_{TOT}=\textcolor{OliveGreen}{\frac{q}{4\pi \varepsilon_0}d\omega_2}\textcolor{Red}{-\frac{q}{4\pi \varepsilon_0}d\omega_1}=\textcolor{OliveGreen}{\frac{q}{4\pi \varepsilon_0}d\omega}\textcolor{Red}{-\frac{q}{4\pi \varepsilon_0}d\omega} = 0
                    \end{align*}
                    Ma nel caso di forme strambe?
                    \framedImg{7}{L22-img009.jpg}
                    Nulla cambia, è un semplice concetto di topologia: se entro una volta, per forza devo anche uscire! Quindi ogni entrata sarà bilanciata da un uscita.

                \subparagraph{Cariche puntiformi multiple}
                    Molto semplice! Facciamo le stesse viste prima (sia nel caso delle cariche interne che esterne) ma usando il \textbf{principio di sovrapposizione}! In particolare avremmo che:
                    \begin{align*}
                        \vec{E}_{TOT} = \vec{E}_{1}+\vec{E}_{2}+...+\vec{E}_{n-1}+\vec{E}_{n}
                    \end{align*}
                    Di conseguenza:
                    \begin{align*}
                        \vec{\Phi}_{TOT} = \oint \vec{E}_{TOT}*\hat{n}*d\Sigma= \oint(\vec{E}_{1}+...+\vec{E}_{n})*\hat{n}*d\Sigma
                    \end{align*}
                    Dato che abbiamo una somma, possiamo spezzare l'integrale:
                    \begin{align*}
                        \vec{\Phi}_{TOT}=\oint(\vec{E}_{1}+...+\vec{E}_{n})*\hat{n}*d\Sigma=\textcolor{Red}{\oint \vec{E}_{1}*\hat{n}*d\Sigma+...+\oint \vec{E}_{n}*\hat{n}*d\Sigma}
                    \end{align*}
                    Ma \textcolor{Red}{questi} sono proprio i flussi delle varie cariche, che per definizione (dimostrato prima) corrispondono prorpio alla cariche fratto la costante dielettrica del vuoto:
                    \begin{align*}
                        \vec{\Phi}_{TOT}=\oint(\vec{E}_{1}+...+\vec{E}_{n})*\hat{n}*d\Sigma=\textcolor{Red}{\oint \vec{E}_{1}*\hat{n}*d\Sigma+...+\oint \vec{E}_{n}*\hat{n}*d\Sigma}= \Phi_1+...+\Phi_n = \frac{q_1}{\varepsilon_0}+...+\frac{q_n}{\varepsilon_0}
                    \end{align*}
                    Quindi possiamo riassumere il tutto come:
                    \begin{align*}
                        \textcolor{Red}{\vec{\Phi}_{TOT}=\sum_i\frac{q_i}{\varepsilon_0}}
                    \end{align*}

            \subsubsection{Applicazioni del teorema di Gauss}
                Calcolare il campo elettrico delle varie distribuzioni di carica può essere motlo complesso, pensa alle distribuzioni lineari e piane (le puntiformi sono abbastanza semplici). Però possiamo usare il \textbf{teorema di Gauss per calcolarlo}! Vediamo i vari casi:

                \paragraph{Cariche puntiformi}
                    Siamo già riusciti a calcolarlo usando la forza di Culomb, però possiamo dimostrarlo facilmente anche con Gauss. Allora, tutto parte da questa uguaglianza:
                    \begin{align*}
                        \oint\vec{E}*\hat{n}*d\Sigma = \sum_i\frac{q_i}{\varepsilon_0}
                    \end{align*}
                    Supponiamo inotre che la nostra \textbf{superficie chiusa sia una sfera} (anche se fosse una forma più complessa non cambierebbe molto dato che poi andiamo solo a considerare gli angoli solidi). 
                    \framedImg{5}{L22-img010}
                    A sinistra dell'uguale abbiamo \textbf{la parte "matematica"}, mentre a destra \textbf{quella fisica} potremmo dire. Ora espandiamo le 2 parti e vediamo quanto vale il campo elettrico! Come prima cosa sappiamo che nel caso di cariche puntiformi la forza elettrica si \textbf{diffonde in modo radiale}, quindi possiamo dire che:
                    \begin{align*}
                        \vec{E}=E(r)*\hat{r}
                    \end{align*}
                    Ora espandiamo la parte a sx dell'uguale:
                    \begin{align*}
                        d\Phi=\vec{E}*\hat{n}*d\Sigma = E(r)*\textcolor{Red}{(\hat{r}*\hat{n})}*\textcolor{OliveGreen}{d\Sigma} = \underline{E(r)*\textcolor{Red}{(1)}*\textcolor{OliveGreen}{r^2*d\omega}}
                    \end{align*}
                    Ottimo, abbiamo questo risultato, ora per calcolare il flusso ci basta applicare un'integrale sulla superficie chiusa:
                    \begin{align*}
                        \Phi = \oint E(r)*\textcolor{Red}{(1)}*\textcolor{OliveGreen}{r^2*d\omega} = E(r)*r^2*\textcolor{Purple}{\oint d\omega} = E(r)*r^2*\textcolor{Purple}{4\pi}
                    \end{align*}
                    Perfetto, ora consideriamo la parte a dx:
                    \begin{align*}
                        \Phi = \sum_i\frac{q_i}{\varepsilon_0}
                    \end{align*}
                    Qui non c'è niente di particolare, sappiamo che c'è una carica sola di valore q, quindi:
                    \begin{align*}
                        \Phi = \frac{q}{\varepsilon_0}
                    \end{align*}
                    Adesso semplicemente eguagliamo il tutto ed otteniamo il risultato finale:
                    \begin{align*}
                        E(r)*r^2*\textcolor{Purple}{4\pi} = \frac{q}{\varepsilon_0}
                    \end{align*}
                    Con un po' di trasformazioni algebriche molto semplici otteniamo il valore del campo elettrico:
                    \begin{align*}
                        E(r)= \frac{q}{4\pi*\varepsilon_0*r^2}
                    \end{align*}
                    Questo è lo stesso risutato che avevamo ottenuto con Culomb! Figo!

                \paragraph{Cariche lineari}
                    Il procedimento è simile a quello per le cariche puntiformi, però in questo caso abbiamo un cavo di densità lineare $\lambda$ e di lunghezza infinita (in realtà non è infinita, ma è molto più grande di tutte le altre lunghezze in gioco). A differenza di prima, in questo caso dobbiamo usare le simmetrie in modo intelligente.
                    \framedImg{8}{L22-img011}
                    Come si vede in figura possiamo immaginare che il campo elettrico si diffonda come un cilindro, infatti i contributi "laterali" si compensano tra di loro (data la lunghezza infinita del cavo). Anche in questo caso quindi, possiamo pensare che il campo elettrico si diffonda in direzione radiale rispetto al cavo:
                    \begin{align*}
                        \vec{E}=E(r)*\hat{r}
                    \end{align*}
                    Ora, proviamo a calcolare il flusso delle superfici circolari laterali:
                    \begin{align*}
                        d\Phi_{B1}= E(r)*(\hat{r}*\hat{n})*d\Sigma_{B1} = E(r)*(0)*d\Sigma_{B1} = 0
                    \end{align*}
                    Ci rendiamo conto però che \textbf{i 2 versori sono ortogonali tra loro}, quindi il prodotto scalare si \textbf{annulla}! Lo stesso vale anche per l'altra superficie circolare:
                    \begin{align*}
                        d\Phi_{B2} = 0
                    \end{align*}
                    Consideriamo ora la superficie "radiale" del cilindro:
                    \begin{align*}
                        d\Phi_{L} = E(r)*\textcolor{Red}{(\hat{r}*\hat{n_l})}*d\Sigma_L = E(r)*\textcolor{Red}{(1)}*d\Sigma_L
                    \end{align*}
                    Bene, ora il "problema" è calcolare la $d\Sigma_L$. Allora, per definizione di angolo piano, sappiamo che l'arco della nostra area corrisponde a $a=\alpha*r$, quindi abbiamo che:
                    \begin{align*}
                        d\Sigma_L=d\alpha*r*dl
                    \end{align*}
                    Possiamo sostituirlo nell'equazione ottenuto finora:
                    \begin{align*}
                        d\Phi_{L} = E(r)*(\hat{r}*\hat{n_l})*d\Sigma_L = E(r)*(1)*d\Sigma_L = E(r)*(1)*d\alpha*r*dl
                    \end{align*}
                    Ora come prima calcoliamo l'integrale chiuso:
                    \begin{align*}
                        \Phi_{L} = \oint E(r)*d\alpha*r*dl = E(r)*r*\textcolor{Red}{\oint d\alpha}\textcolor{OliveGreen}{\oint dl} = E(r)*r*\textcolor{Red}{2\pi}*\textcolor{OliveGreen}{D}
                    \end{align*}
                    Completata la parte "matematica" possiamo passare a quella fisica:
                    \begin{align*}
                        \Phi_{L} = \sum_i\frac{q_i}{\varepsilon_0}
                    \end{align*}
                    Ma quanto vale la carica interna totale? Per calcolarla, semplicemente \textbf{moltiplichiamo la distribuzione di carica per la lunghezza del filo contenuta nel cilindro}:
                    \begin{align*}
                        \Phi_{L} = \sum_i\frac{q_i}{\varepsilon_0}=\frac{\lambda*D}{\varepsilon_0}
                    \end{align*}
                    Uguaglimo il tutto:
                    \begin{align*}
                        E(r)*r*2\pi*\bcancel{D} = \frac{\lambda*\bcancel{D}}{\varepsilon_0}
                    \end{align*}
                    Ancora un po' di passaggi algebrici:
                    \begin{align*}
                       \underline{\textcolor{Red}{E(r) = \frac{\lambda}{2\pi*r*\varepsilon_0}}}
                    \end{align*}