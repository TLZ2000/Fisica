\section{Elettromagnetismo}

    \subsection{Introduzione all'elettromagnetismo}
        In natura, esistono 2 tipi di materiali (elettricamente parlando):
        \begin{itemize}
            \item \textbf{conduttori}: materiali abbastanza noiosi (PER IL PROGRAMMA CHE ABBIAMO FATTO NOI QUEST'ANNO) perché \textbf{non si magnetizzano};
            \item \textbf{isolanti}: tali materiali \textbf{possono essere "elettrizzati"} (ovvero possono assumere delle cariche). Di questi possiamo applicare un'altra classificazione:
            \begin{itemize}
                \item \textbf{vetri e affini};
                \item \textbf{plastiche}.
            \end{itemize}
        \end{itemize}
    
    \subsection{Forza di Culomb}
        Vediamo ora perché questa classificazione è interessante:
        \framedImg{8}{L21A-img001}
        Ovvero, 2 cariche dello stesso materiale (vetri/vetri o plastiche/plastiche) poste vicine \textbf{si respingono tra loro}. Al contrario, 2 cariche di materiali diversi (vetri/plastiche) \textbf{si attraggono}! Per esplicitare questo concetto di attrazione/repulsione associamo alle cariche dei vetri \textbf{il segno +}, mentre alle cariche dei materiali plastici associamo \textbf{il segno -}. Altro elemento importante è che tale forza si manifesta sull'\textbf{asse congiungente le 2 cariche}. Ora, come quantifichiamo questa forza? Possiamo usare la \textbf{\textit{forza di Culomb}}:
        \begin{align*}
            \textcolor{Red}{F_{el}=\textcolor{OliveGreen}{K}*\frac{q_1*q_2}{r^2_{1,2}}*\hat{r_{1,2}}=\textcolor{OliveGreen}{\frac{1}{4\pi\varepsilon_0}}*\frac{q_1*q_2}{r^2_{1,2}}*\hat{r_{1,2}}}
        \end{align*}
        Dunque, la \textbf{forza elettrica} ($F_{el}$) è una forza appunto, quindi unità di misura $Newton$, per quanto riguarda le 2 cariche $q_1$ e $q_2$ hanno come unità di misura il $Culomb$. Per quanto riguarda $K$ invece è una costante che permette di ottenere una forza appunto, a sua volta può \textbf{essere scomposta} come $\frac{1}{4\pi\varepsilon_0}$ (utile quando usiamo il teorema di Gauss) dove compare la \textbf{\textit{costante dielettrica del vuoto}} $\textcolor{Red}{\varepsilon_0 = 8,85418*10^{-12}\frac{C^2}{N*m^2}}$. Facciamo un disegnino per visualizzare meglio la situazione:
        \framedImg{6}{L21A-img002}
        Nota che noi supponiamo sempre che le 2 cariche siano ferme.

        \subsubsection{Digressione sugli atomi}
            La materia è composta da atomi, a loro volta gli atomi sono composti da un \textbf{nucleo} di \textbf{protoni} (carica positiva), \textbf{neutroni} (carica neutra, sono la "colla" che tiene insieme i protoni [altrimenti le cariche positive si respingerebbero tra di loro]) ed \textbf{elettroni} (carica negativa). Di solito, l'atomo ha una carica elettrica complessiva nulla (\textbf{i protoni si bilanciano con gli elettroni}), però è possibile "strappare" o aggiungere degli elettroni, in modo che l'atomo abbia una carica elettrica. Ora, è possibile osservare che \textbf{non esistono cariche libere più piccole di una certa soglia} (che corrisponde alla carica dell'elettrone/protone), ovvero $\textcolor{Red}{e=1,602*10^{-19}C}$. Possiamo fare una tabellina veloce:
            \begin{center}
                \begin{tabular}{ c c c }
                    & \textbf{Carica} & \textbf{Massa} \\ 
                    \textbf{Elettrone} & $-e$ & $m_{el}=0,91*10^{-30}Kg$ \\  
                    \textbf{Protone} & $+e$ & $m_{pr}=1,67*10^{-27}Kg$\\    
                    \textbf{Neutrone} & $0$ & $m_{ne}\approx m_{pr}$
                \end{tabular}
            \end{center}
            Ora, perché mettiamo anche la massa? Possiamo notare che la \textit{forza di gravitazione universale} e la \textit{forza di Culomb} hanno una forma \textbf{estremamente simile}! Risulta quindi naturale provare a confrontare le 2 forze, possiamo immaginare di usare un altomo di idrogeno (1 protone e 1 elettrone). Abbiamo quindi:
            \begin{align*}
                &<r> = 5,3*10^{-11}m\\
                &F_{el} = -K*\frac{e^2}{<r>^2}\approx-8,2*10^{-8}N && Forza\ elettrica\\
                &F_{gr} = -G*\frac{m_e*m_P}{<r>^2}\approx-3,6*10^{-47}N && Forza\ gravitazionale
            \end{align*}
            Si nota che \textbf{la forza graviazione è diversi ordini di grandezza inferiore a quella elettrica}! Risulta quindi sensato \textbf{ignorare la forza gravitazionale all'interno dei legami atomici}!


    \subsection{Principio di sovrapposizione}
        Supponiamo di essere in questa situazione: abbiamo \textbf{2 cariche} "importanti" statiche ($Q_1$ e $Q_2$) ed una \textbf{carica di prova} ($q_0$). Ora, la carica di $q_0$ è \textbf{molto minore rispetto alle altre 2} ($|q_0|<<Q_1$ e $|q_0|<<Q_2$): questo è molto importante perché, come già detto prima, \textbf{esiste una carica "elettromagnetica" [ricontrollla questo ""] minima} ($1,6*10^{-9}C$) \textbf{sotto la quale non si può andare}! Quindi anche $q_0$ ha una carica elettrica che va ad interagire con le altre 2 cariche e, quindi, a perturbare il nostro sistema! Questa supposizione serve a perturbare il sistema il meno possibile. Tornando al discorso iniziale, supponiamo di avere queste cariche posizionate in su un piano e che interagiscano tra loro in questo modo:
        \framedImg{8}{L21-img001}
        Secondo la \textbf{legge di Culomb}, $Q_1$ eserciterà una forza $F_{1->0}$ sulla carica $q_0$ (lo stesso vale per $Q_2$ con la forza $F_{2->0}$), possiamo immaginare che le 2 forze siano quelle mostrate in figura (sono uguali solo per puro caso). Ora, per capire quale forza agisce complessivamente sulla carica $q_0$ possiamo \textbf{sommare i 2 vettori delle forze secondo la regola del parallelogramma}: per fare questo ci avvaliamo del \textbf{principio di sovrapposizione}, che ci garantisce che \textbf{l'interazione tra $Q_1$ e $q_0$ non viene perturbata in alcun modo dall'interazione tra $Q_2$ e $q_0$ e viceversa}, permettendoci quindi di \underline{\textbf{"sovrapporre" le 2 forze}!} (Sostanzialmente, il principio di sovrapposizione dice solo la parte sulla sovrapposizione, la prima parte delle interazioni è una conseguenza di questa seconda parte :P) \bigskip\\
        In questo caso specifico abbiamo solo 2 cariche puntiformi, però nulla ci vieta di usare N cariche disposte in varie forme, ad esempio un cavo:
        \framedImg{4}{L21-img002}
        In questo caso però si deve introdurre una \textbf{densità lineare di carica "$\mathbf{\lambda}$"} che, sostanzialmente, corrisponde alla \textbf{quantità di carica ($dq$) disposta presente in una certa lunghezza di filo ($dl$)}. Come per l'esempio delle cariche puntiformi, la forza totale che agisce su $q_0$ corrisponde alla \textbf{sovrapposizione di tutte le forze (infinitesime) delle varie cariche (infinitesime, infatti $dq = \lambda dl$)}. Allo stesso modo, il discorso si può estendere a 2 e 3 dimensioni, l'unica cosa che cambia è che al posto di $\lambda$ (1-D) avremmo $\sigma$ (2-D, \textit{densità superficiale di carica}) e $\rho$ (3-D, \textit{densità volumica di carica}):
        \framedImg{4}{L21-img003}

    \subsection{Campo elettrico}
        Diventa interessante analizzare l'effetto che le cariche hanno sulla carica di prova \textbf{senza considerare le cariche stesse} (senza dover quindi effettuare un integrale considerando tutte le cariche)! Per questo motivo viene introdotto il concetto di \textbf{campo elettrico} (o meglio, \textit{elettrostatico} dato che le nostre cariche sono ferme). Formalmente, definiamo il campo elettrico ($\vec{E}$) come il rapporto tra la forza provata dalla carica di prova ($\vec{F}$) e la carica di prova stessa ($q_0$), quindi:
        \begin{align*}
            \textcolor{Red}{\vec{E} \triangleq \frac{\vec{F}}{q_0}}
        \end{align*}
        Pensa a $q_0$ come ad una sonda che \textbf{va a \underline{misurare} la forza in un certo punto dello spazio}, misurando questa forza in, idealmente, tutte le possibili posizioni e mettendo tutto insieme otteniamo un \textbf{campo vettoriale} dove ad ogni posizione è assegnato un vettore!

        \subsubsection{Campo elettrico per campo puntiforme}
            Cominciamo considerando una \textbf{carica puntiforme}: se inseriamo la nostra sonda e facciamo un sistema del genere:
            \framedImg{7}{L21-img005}
            Possiamo dire che, per definizione, il campo elettrico della carica $Q$ corrisponde a:
            \begin{align*}
                &\vec{F}=K*\frac{q_0*Q}{r^2}\hat{r}&&=>\vec{E}=K*\frac{\bcancel{q_0}*Q}{r^2*\bcancel{q_0}}\hat{r}=\textcolor{Red}{K*\frac{Q}{r^2}\hat{r}}
            \end{align*}
            Si nota facilmente che, tralasciando la costante elettrica $K$, il campo elettrico è \textbf{direttamente proporzionale alla carica elettrica $Q$ ed è diretto in senso raiale alla carica}, ottenendo qualcosa del genere:
            \framedImg{7}{L21-img004}
            A livello di convenzione, si assume sempre che la nostra \textbf{carica di prova $q_0$ sia positiva}, quindi:
            \begin{itemize}
                \item se $Q>0$ allora \textbf{il campo sarà uscente};
                \item se $Q<0$ allora \textbf{il campo sarà entrante};
            \end{itemize}

        \subsubsection{Campo di dipolo}
            Se ora prendiamo 2 sorgenti puntiformi statiche di segno opposto e le mettiamo vicine. Per ogni carica puntiforme ci saranno delle \textbf{linee di campo} (uscenti per $+Q$ ed entranti per $-Q$) e, se le congiungiamo, otteniamo un campo elettrico del genere:
            \framedImg{4}{L21-img006}
            In ogni posizione associamo un vettore che indica il campo elettrico in quel punto specifico: in questo caso il campo elettrico sarà dato proprio dalla \textbf{sovrapposizione delle forze delle 2 cariche Q}! Se ci viene fornita la funzione che descrive questo \textbf{campo elettrico in funzione della posizione}, ci \textbf{risparmiamo tutto il lavoro sul calcolo degli effetti sulla carica di prova dovuti all'interazione con tutte le sorgenti del nostro sistema}! In un certo senso, la nozione di campo elettrico ci permette di descrivere la "realtà fisica" del nostro sistema \textbf{senza dover considerare in modo "geometrico, perfetto" le sorgenti} del campo (le sorgenti potrebbero essere estremamente tante, irregolari, difficili da analizzare o, adirittura, impossibili da analizzare perché magari non le conosciamo, quindi con il campo elettrico andiamo a considerare solo l'effetto che queste cariche hanno sulla nostra carica di prova).\bigskip\\
            Nota però che con questa rappresentazione grafica perdiamo la quantificazione del modulo del vettore della forza nella posizione specifica, ma questo non è un problema.

        \subsubsection{Esempio di esercizio con il campo elettrico}
            La formula fondamentale per lavorare negli esercizi con i campi elettrici è questa:
            \begin{align*}
                \textcolor{Red}{\vec{F}(\vec{r})\triangleq q* \vec{E}(\vec{r})}
            \end{align*}
            Vediamo il problema:
            \framedImg{8}{L21-img007}
            Dunque, abbiamo una \textbf{carica di prova} $p$ di \textbf{massa} $m$ che \textbf{si muove ad una velocità} $v_0$. Ad un certo punto \textcolor{Blue}{\textbf{entra nel campo elettrico}} (nota che nella realtà uno "spezzamento" così netto tra "campo elettrico" e "non campo elettrico" non è realizzabile, in questo caso facciamo delle approssimazioni e non ci interessa :P). Ora, entrando nel campo elettrico, la nostra carica di prova "devia" dal suo percorso: quando uscirà dal campo elettrico, lo farà con un \textbf{certo angolo $\mathbf{\theta}$}. Conoscendo il \textbf{campo elettrico} $\vec{E}$, quanto vale l'\textbf{l'angolo $\mathbf{\theta}$}?\bigskip\\
            Allora, cominciamo con la seconda legge della dinamica:
            \begin{align*}
                \vec{F} = m*\vec{a}=m*\frac{d\vec{v}}{dt}
            \end{align*}
            Però a cosa corrisponde quella forza specifica? Proprio alla forza dovuta all campo elettrico:
            \begin{align*}
                \vec{F}=q*\vec{E}=q*E*\hat{y}
            \end{align*}
            Ora, quest'ultimo pezzo è importante: la forza legata al \textbf{campo elettrico agisce solo sull'asse y}! La velocità della carica sull'asse x quindi resterà costante:
            \begin{align*}
                &\begin{cases}
                    x)\frac{dv_x}{dt}=0 & Perche\ v_x\ è\ costante\\
                    y)
                    \begin{cases}
                        \vec{F} = m*a_y =m*\frac{dv_y}{dt}\\
                        \vec{F} = q*\vec{E} =q*E
                    \end{cases}&=>\textcolor{Red}{\frac{dv_y}{dt}=\frac{q*E}{m}}
                \end{cases}
            \end{align*}
            Facciamo qualche magheggio da fisico su \textcolor{Red}{questo pezzo}, ottenendo:
            \begin{align*}
                & \frac{dv_y}{dt}=\frac{q*E}{m} && => dv_y=\frac{q*E}{m} * dt\\
                & && => \int dv_y = \int \frac{q*E}{m} * dt\\
                & v_y\ dipende\ da\ t\ perche\ tutto\ il\ resto\ e'\ costante && => v_y(t) = \frac{q*E}{m}*t
            \end{align*}
            Bene, abbiamo ottenuto $v_y$ \textbf{in funzione del tempo $t$}: quanto vale $t$? Possiamo scoprilo sfruttando la x:
            \begin{align*}
                &\frac{dv_x}{dt}=0 && => v_x = const = v_0\\
                & && => \frac{dx}{dt}= v_0\\
                & && => dx = v_0 dt\\
                & && => \int dx = \int v_0 dt\\
                & && => \int dx = v_0 \int dt\\
                & && => x(t) = v_0*t\\
                & && => \textcolor{Red}{\frac{x(t)}{v_0} = t}
            \end{align*}
            Ora che abbiamo il \textcolor{Red}{tempo} per percorrere lo spazio L (fai riferimento al disegno) possiamo \textbf{sostituirla nell'equazione della velocità}!
            \begin{align*}
                &\begin{cases}
                    t = \frac{x(t)}{v_0}\\
                    v_y(t) = \frac{q*E}{m}*t
                \end{cases}
                && => v_y(t) = \frac{q*E}{m}*\frac{x(t)}{v_0}
            \end{align*}
            Abbiamo la \textcolor{Orange}{velocità su y}, come la riconduciamo all'\textbf{angolo $\mathbf{theta}$}? Ricorda che:
            \begin{align*}
                &\vec{v} =
                \begin{cases}
                    v_y = sin(\theta)*|\vec{v}|\\
                    v_x = cos(\theta)*|\vec{v}|\\
                \end{cases} && => \mathbf{\textcolor{Red}{\frac{v_y}{v_x} = \frac{sin(\theta)*\bcancel{|\vec{v}|}}{cos(\theta)*\bcancel{|\vec{v}|}} = tan(\theta)}}
            \end{align*}
            \framedImg{2}{L21-img008}
            Ora ci basta mettere tutto a sistema:
            \begin{align*}
                \frac{v_y}{v_x} = tan(\theta) = \frac{q*E}{m}*\frac{x(t)}{v_0} *\frac{1}{v_0} = \frac{q*E}{m} * \frac{L}{v_0^2}
            \end{align*}
            Ricorda che $x(t) = L$. Abbiamo finito, questo è il nostro risultato finale:
            \begin{align*}
                \mathbf{tan(\theta) = \frac{q*E}{m} * \frac{L}{v_0^2}} = \frac{q*E*L}{2*(\frac{1}{2}*m*v_0^2)} = \textcolor{Orange}{\frac{q*E*L}{2*E_{K, ini}}}
            \end{align*}
            Volendo è possibile fare qualche trasformazione algebrica per ricondurci alla \textcolor{Orange}{all'energia cinetica della particella di prova} (utile per fare delle considerazioni teoriche, esperimento di Rutherford).
