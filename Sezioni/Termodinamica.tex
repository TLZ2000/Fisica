\section{Termodinamica}
    La \textbf{termodinamica} è il ramo della fisica classica e della chimica che studia e descrive le trasformazioni termodinamiche indotte da calore a lavoro e viceversa in un sistema termodinamico, in seguito a processi che coinvolgono cambiamenti delle variabili di stato temperatura ed energia.\\
    Ci ricordiamo ora che il lavoro delle forze non conservative è la variazione di energia meccanica, ovvero come segue,
    \begin{align*}
        W_{N.C.} = E_f - E_i
    \end{align*}

    \subsection{Numero di Avogadro}
        Consideriamo ora un numero molto alto di di costituenti del sistema. Le relazioni che ci sono tra di essi sono espresse in un'unità di misura che è il \textbf{numero di Avogadro},
        \begin{align*}
            N_A = 6,022 * 10^{23} = 1mol
        \end{align*}
        esso indica quanti costituenti fondamentali sono contenuti in una certa quantità di sostanza detta \textbf{mole}.

    \subsection{Sistema termodinamico}
        Nella termodinamica vediamo l'universo composto da due parti, il \textbf{sistema termodinamico} e l'\textbf{ambiente}. Tra questi due c'è un'iterazione/scambio continuo ed in base a questo scambio definiamo i diversi tipi di sistemi termodinamici:
        \begin{center}
            \begin{tabular} { |c|c|c|c| }
                \hline
                & ENERGIA & MATERIA & ESEMPIO\\
                \hline
                APERTO & si & si & pentola d'acqua che bolle senza coperchio\\
                \hline
                CHIUSO & si & no & pentola d'acqua che bolle con coperchio\\
                \hline
                ISOLATO & no & no & contenitore isolato che contiene acqua\\
                \hline
                IMPOSSIBILE & no & si & impossibile, scambio materia => scambio energia\\
                \hline
            \end{tabular}
        \end{center}

    \subsection{Variabili termodinamiche}
        Le \textbf{variabili termodinamiche}, anche dette \textbf{coordinate termodinamiche}, si distinguono in \textbf{grandezze estensive}, ovvero quelle che dipendono dall'estensione del sistema termodinamico (eg. volume, massa), e \textbf{grandezze intensive}, ovvero quelle che non dipendono dall'estensione del sistema termodinamico (eg. pressione, temperatura, densità).\\

        Queste variabili sono anche distinte in \textbf{variabili globali}, ovvero quelle che sono riferibili a tutto il sistema (come la temperatura, nel caso in cui si lasci abbastanza tempo ad un corpo di scaldarsi/raffreddarsi), e \textbf{variabili locali}, ovvero quelle che sono riferibili ad una parte del sistema (come la densità, se si pensa ad una torta con crema e frutta, queste due avranno densità diversa).\\

        Il loro numero dipende dal sistema che sto considerando e le più famose sono \textbf{pressione}, \textbf{temperatura}, \textbf{volume}, \textbf{densità}, \textbf{massa},...\\

        Osserviamo che per i sistemi di gas useremo pressione, volume e temperatura ed inoltre il numero di moli e di componenti.

    \subsection{Equilibrio termodinamico}
        L'\textbf{equilibrio termodinamico} di un sistema termodinamico si ha quando tutte le suoi corpi/componenti sono in:
        \begin{itemize}
            \item \textbf{equilibrio meccanico}: ovvero quando non ci sono momenti delle forze che agiscono tra i suoi corpi/componenti, qualsiasi questi siano.\\
            \item \textbf{equilibrio termico}: ovvero quando non c'è nessuna differenza di temperatura tra i suoi corpi/componenti, qualsiasi questi siano.\\
            \item \textbf{equilibrio chimico}: ovvero quando non c'è nessuna reazione chimica tra i suoi corpi/componenti, qualsiasi questi siano.
        \end{itemize}

    \subsection{Trasformazione termodinamica}
        Una \textbf{trasformazione termodinamica} è il passaggio del sistema termodinamico da uno stato termodinamico ad un altro.\\
        Un esempio è il passaggio da caldo a freddo, in esso ci sono tanti passaggi/variazioni:
        \begin{itemize}
            \item \textbf{espansione/contrazione} di liquidi e gas\\
            \item variazione resistenza elettrica\\
            \item variazione differenza di potenziale\\
            \item variazione di riflettività/trasmittanza
        \end{itemize}
        È utile rappresentare le trasformazioni come spostamenti in un piano di coordinate. Per i gas di solito si usa il piano $V, p$.

    \subsection{Temperatura}
        La \textbf{temperatura} si misura su una scala e corrisponde alla media dell'energia cinetica dei componenti del sistema, ovvero quanto i componenti "sono agitati".

        \subsubsection{Unità di misura temperatura}
            La temperatura si misura con diverse unità di misura, \textbf{Celsius} (\degree{}C), \textbf{Fahrenheit} (\degree{}F) e \textbf{Kelvin} (K).\\
            Osserviamo che il simbolo della temperatura, per il Kelvin è \textbf{T}, mentre per le altre unità di misura è \textbf{t}, quindi la maiuscola è solo per il Kelvin.\\
            Per passare da una u.d.m. all'altra le formule sono le seguenti,
            \begin{align*}
                &t(\degree{}C)=T(K) - 273.16\\
                &t(\degree{}F)=\frac{9}{5}t(\degree{}C) + 32
            \end{align*}
            e le restanti si possono ricavare da queste.

        \subsubsection{Punto triplo dell'$H_2O$}
            Il \textbf{punto triplo dell'$H_2O$} è il punto in cui coesistono i 3 stati della materia, che nel caso dell'$H_2O$ sono ghiaccio, acqua e vapore acqueo, e corrisponde a
            \begin{align*}
                T_0=273.16K=0\degree{}C
            \end{align*}

        \subsubsection{Termometro}
            Un \textbf{termometro} è un sistema fisico, usato come strumento, che ci permette di esprimere le variazioni di temperatura in funzione di una grandezza fisica,
            \begin{align*}
                &\Delta T\rightarrow\Delta X\\
                &T=T(x)
            \end{align*}
            Nel caso di termometro a mercurio, la grandezza fisica è la lunghezza della colonnina di mercurio che leggiamo.\\
            Si cerca di far si che questa lunghezza sia lineare, ovvero che cresca linearmente con la temperatura, in modo che,
            \begin{align*}
                T=T(X)=\beta X
            \end{align*}
            Quando si costruisce un termometro si deve calibrarlo, per farlo si prende una temperatura di riferimento ($T_0$) e si misura la lunghezza ($X_0$), poi con questi due dati si calcola,
            \begin{align*}
                \beta=\frac{T_0}{X_0}
            \end{align*}

        \subsubsection{Contatto termico}
            \framedImg{25}{L13-img002}
            Supponiamo di avere tre corpi, A, B e C a contatto tra loro come in figura, con temperature iniziali $T_{A,i}$, $T_{B,i}$ e $T_{C,i}$ corrispondentemente, diverse tra di loro.\\
            Dopo un periodo di tempo abbastanza lungo, le tre temperature, $T_{A,f}$, $T_{B,f}$ e $T_{C,f}$, saranno uguali. Se invece aspettiamo poco tempo, ci si può trovare in diverse situazioni a seconda del tipo di contatto tra i corpi.\\
            Le pareti si dividono quindi in:
            \begin{itemize}
                \item \textbf{pareti diatermiche}: permettono lo scambio di calore grazie all'ottimo contatto\\
                \item \textbf{pareti adiabatiche}: non permettono scambio di calore, sono quindi "isolanti"
            \end{itemize}
            Il \textbf{contenitore adiabatico} è un contenitore isolante termicamente.\\
            Osserviamo che in pratica, se si aspetta un tempo sufficientemente lungo, niente è perfettamente isolante.

        \subsubsection{Principio zero della termodinamica}
            Il principio zero della termodinamica dice che se due corpi sono entrambi in equilibrio termico con un terzo corpo, allora lo sono anche fra loro, ovvero se A e B sono in equilibrio termico e B e C sono in equilibrio termico, allora anche A e C sono in equilibrio termico.\\
            Questo si traduce nel fatto che se,
            \begin{align*}
                T_A=T_B\quad AND\quad T_B=T_C\quad =>\quad T_A=T_C
            \end{align*}

    \subsection{Esperimento di Joule}
        \framedImg{40}{L14-img001}
        Supponiamo di avere un contenitore con pareti adiabatiche, riempito di un fluido al cui interno è inserito un \textbf{mulinello} agganciato ad una puleggia a cui è appesa una massa $m$. La massa andando su e giù fa girare il mulinello, che a sua volta fa scaldare l'acqua.\\
        Ora si può legare $\Delta h$ a $\Delta T$. Il lavoro che compio è il seguente,
        \begin{align*}
            W=mg\Delta h
        \end{align*}
        Analogamente al mulinello posso immergere una \textbf{resistenza} nell'acqua e in questo caso ottengo lavoro,
        \begin{align*}
            W=\frac{\Delta V^2}{R}\Delta t
        \end{align*}
        Stessa cosa se immergo in acqua un'\textbf{elica} e la faccio muovere su e giù, o due \textbf{piastre} che faccio sfregare tra loro producendo attrito.\\
        Posso anche immergere un \textbf{pallone} nell'acqua e far variare il suo volume.\\
        In conclusione quindi se faccio lavoro (o casino) dentro ad un sistema termodinamico, la temperatura (o agitazione) aumenta, infatti il lavoro è legato alla variazione di temperatura, a lavori uguali corrispondono variazioni di temperatura uguali,
        \begin{align*}
            W=-\Delta U_{int}
        \end{align*}
        osserviamo che il segno "-" è una convenzione adottata in termodinamica, dato che quello che interessa della macchina termica è il lavoro che "l'acqua" può compiere. Se il sistema da energia, il lavoro sarà $>0$, se invece il sistema riceve energia, il lavoro sarà $<0$.\\
        Con questo esperimento abbiamo appreso che l'energia interna dipende solo dalla temperatura ovvero,
        \begin{align*}
            U = U(T)
        \end{align*}

    \subsection{Calore}
        Il \textbf{calore} è quindi ciò che determina la variazione di temperatura,
        \begin{align*}
            \Delta T = \gamma Q => Q = \frac{1}{\gamma}\Delta T = \frac{1}{\gamma}(T_f - T_i) = \frac{1}{\gamma}(U_f - U_i) = \frac{1}{\gamma}\Delta U = \eta \Delta U
        \end{align*}
        con $\eta = \frac{1}{\gamma}$.
        Ora quindi se confrontiamo due differenze di temperature,
        \begin{align*}
            &\Delta T_1=\gamma Q_1\\
            &\Delta T_2=\gamma Q_2
        \end{align*}
        otteniamo,
        \begin{align*}
            \frac{\Delta T_2}{\Delta T_1}=\frac{\bcancel{\gamma} Q_2}{\bcancel{\gamma} Q_1}
        \end{align*}
        Ora sperimentalmente si prova che $\eta = 1$ e quindi ottengo che,
        \begin{align*}
            Q = \Delta U_{int}
        \end{align*}
        ovvero il calore è tutto energia interna.\\
        Osserviamo che introducendo il calore abbiamo generalizzato il principio di conservazione dell'energia, infatti il calore è un modo per scambiare l'energia.

        \subsubsection{Esempio di calore}
            La massa (e.g. carbone) che brucio per scaldare una pentola d'acqua darà un certo calore che, a parità di massa, dipende da tante cose come l'umidità, la qualità del carbone, il tipo di pentola che uso,... Tutte queste sono delle costanti quindi ho,
            \begin{align*}
                Q = c_p c_h c_q ... m
            \end{align*}
            Facendo un po' di esperimenti in condizioni diverse si potranno osservare delle variazioni di temperatura diverse, quello che però osservo sempre è che se raddoppio la massa che brucio, raddoppia anche la variazione di temperatura, la costante $\gamma$ tiene conto di queste costanti.

    \subsection{Principio di equivalenza calore-lavoro}
        Dato ora quindi che il calore è $Q = \Delta U_{int}$ e ricordandoci da prima che $W=-\Delta U_{int}$ possiamo ricavare che,
        \begin{align*}
            Q = -W <=> W = -Q
        \end{align*}
        Il calore e il lavoro quindi sono energia in trasferimento/movimento, identificano una differenza tra lo stato iniziale e lo stato finale.

    \subsection{I principio della termodinamica}
        Tornando ora a parlare della variazione d'energia interna, essa è causata in parte dal lavoro che il sistema compie/subisce e in parte dal calore che il sistema assorbe/cede, ovvero
        \begin{align*}
            &\Delta U_{int} = \Delta U_{int}^{(W)} + \Delta U_{int}^{(Q)}\\
            &\Delta U_{int}^{(W)}=-W\\
            &\Delta U_{int}^{(Q)}=Q
        \end{align*}
        La variazione di energia interna dovuta al lavoro è $-W$ dato che se il sistema compie lavoro, questo viene a "danno" della sua energia interna. Per la variazione di energia dovuta al calore invece è $Q$ dato che se ricevo calore questo si "trasforma" in energia interna.\\
        In conclusione quindi ho che la variazione di energia interna è la differenza tra il calore e il lavoro,
        \begin{align*}
            \Delta U{int} = Q - W
        \end{align*}
        Osserviamo ora che per passare dallo stato iniziale a quello finale ci sono tanti modi/traiettorie, ognuna delle quali ha il suo $Q_i$ e il suo $W_i$, la $\Delta U$ però è sempre uguale, indipendentemente dalla traiettoria.\\\\
        Ora posso concentrarmi sulle quantità infinitesimali, quindi mi verrebbe da scrivere,
        \begin{align*}
            dU = dQ - dW
        \end{align*}
        La parte sinistra all'"=" è corretta, infatti la variazione di energia dipende solo dallo stato iniziale e da quello finale, è differenziabile. La parte destra invece, quindi la variazione di calore e di lavoro dipende anche dal percorso e non è un differenziale esatto, quindi si usa $\delta$ come segue,
        \begin{align*}
            dU = \delta Q - \delta W
        \end{align*}
        La variazione infinitesima di calore e di lavoro dipende quindi dalla trasformazione $\gamma$ che effettuo sul piano,
        \begin{align*}
            &\delta Q = \delta Q_{\gamma}(p, V, T)\\
            &\delta W = \delta W_{\gamma}(p, V, T)
        \end{align*}

    \subsection{Tipi di trasformazioni termodinamiche}

        \subsubsection{Trasformazione ciclica}
            Una \textbf{trasformazione ciclica} è una trasformazione per cui lo stato iniziale e quello finale coincidono. Dato quindi che $i\equiv f$ ho che,
            \begin{align*}
                &T_i = T_f\\
                &U_i = U_f
            \end{align*}
            e quindi,
            \begin{align*}
                \Delta U = 0 => Q = W
            \end{align*}
            Infatti se ho incamerato energia, devo in qualche modo restituirla all'ambiente esterno, per far si stato iniziale e finale coincidano.\\
            I motori per esempio fanno trasformazioni cicliche.

        \subsubsection{Trasformazione quasi-statica}
            Una \textbf{trasformazione quasi-statica} è una trasformazione in cui ogni stato intermedio è uno stato di equilibrio, dove le "cose" sono fatte con "dolcezza" tale da non perturbare ne il sistema ne l'ambiente esterno.

        \subsubsection{Trasformazione reversibile}
            Una \textbf{trasformazione reversibile} è una trasformazione in cui si "può tornare indietro". Per essere reversibile, una trasformazione deve essere quasi statica e non ci devono essere dissipazioni.

            \paragraph{Esempio trasformazione reversibile}
                Supponiamo di avere un contenitore, con del gas all'interno, aperto sopra e nell'apertura è inserito un pistone, supponiamo anche che tra il pistone e le pareti non ci siano perdite. Se ora metto dei pesetti sopra il pistone, riesco a farlo abbassare. Posso assumere che gli infiniti stati intermedi infinitesimi sono in equilibrio, quindi la trasformazione è quasi-statica. Inoltre è una trasformazione reversibile, infatti togliendo i pesetti che ho messo sopra il pistone, il sistema ritorna allo stato in cui era prima di metterci i pesetti.
                \framedImg{4}{L14-img002}

            \paragraph{Esempio trasformazione non reversibile}
                Questo esempio riguarda l'espansione libera dei gas, infatti ho un contenitore adiabatico separato all'interno in due camere con un rubinetto che le collega. All'inizio il rubinetto è chiuso e nella camera destra c'è gas, mentre in quella sinistra il vuoto. Se ora apro il rubinetto ed aspetto un certo tempo, finisco nella situazione in cui il gas si è distribuito nelle due camere. Ora per far si che il gas torni tutto nella camera destra (situazione iniziale) devo aspettare tempo infinito.
                \framedImg{8}{L14-img003}

        \subsubsection{Lentezza}
            Ci riferiamo alla \textbf{lentezza} come la caratteristica/velocità di un processo (trasformazione termodinamica) che ci consente di trattare tutti gli stati intermedi come stati di equilibrio e che ci consente di raffigurare la trasformazione come passaggi su infiniti stati intermedi tutti di equilibrio.

    \subsection{Calorimetria}
        \framedImg{40}{L14-img004}
        Consideriamo un sistema termodinamico composto da un contenitore adiabatico con all'interno $H_2O$ a $T_{H_2O,i}$ ed un corpo $c$ immerso nell'acqua a $T_{c,i}$ (analogamente se considero due masse solide a contatto tra loro in un contenitore adiabatico ottengo gli stessi risultati). Aspettando un istante di tempo abbastanza lungo passo dal sistema descritto sopra, ad un sistema in cui l'$H_2O$ e $c$ hanno la stessa temperatura $T_*$, quindi ipotizzando che l'$H_2O$ abbia temperatura inferiore a $c$ abbiamo che $T_{H_2O,i}<T_*<T_{c,i}$.\\
        Ora essendo il sistema isolato abbiamo che,
        \begin{align*}
            Q = 0 \quad AND\quad W = 0 => \Delta U = 0
        \end{align*}
        e quindi,
        \begin{align*}
            0=\Delta U=\Delta U_{H_2O} + \Delta U_c => \Delta U_{H_2O} = - \Delta U_c
        \end{align*}
        ovvero l'energia interna che l'$H_2O$ "guadagna" la "toglie" dal corpo $c$. Osserviamo ora che $\Delta V = 0$, ovvero il volume non cambia, e quindi $\Delta W = 0$ sia per l'$H_2O$ che per $c$ e quindi ho che,
        \begin{align*}
            &\Delta U_{H_2O}=Q_{H_2O} - \bcancel{W_{H_2O}}\\
            &\Delta U_c=Q_c - \bcancel{W_c}
        \end{align*}
        La variazione di energia interna avviene tutta per cambio di calore, e quindi ora ottengo,
        \begin{align*}
            \Delta U_{H_2O} = - \Delta U_c => Q_{H_2O} = -Q_c
        \end{align*}
        Ora devo capire a che energia corrisponde la variazione di temperatura, questa mi consente di calcolare il lavoro che mi serve per fare aumentare la temperatura dell'${H_2O}$. Questo sarà il calore $Q_{H_2O}$, che sarà $-Q_c$.\\
        Ora ottengo che
        \begin{align*}
            Q = c_{\gamma}m\Delta T
        \end{align*}
        dove $c_{\gamma}$ è il calore specifico.\\
        Osserviamo che questa è una trasformazione non reversibile, infatti devo aspettare infinito tempo per far si che le due temperature tornino ai valori iniziali.

        \subsubsection{Calore specifico}
            Il \textbf{calore specifico} di un corpo è una costante definita come segue,
            \begin{align*}
                c = \frac{1}{m}\left[\frac{dQ}{dT}\right]_{\gamma}
            \end{align*}
            La sua unità di misura è quindi $\frac{J}{kgK}$.\\
            Osserviamo ora che il calore specifico si cita per una particolare trasformazione, infatti scrivere,
            \begin{align*}
                c = \frac{1}{m}\frac{dQ}{dT}
            \end{align*}
            oppure,
            \begin{align*}
                c = \frac{1}{m}\frac{\delta Q}{dT}
            \end{align*}
            sarebbe sbagliato.\\
            Osserviamo ora che il fatto di scambiare calore con l'esterno non implica che ci sia una variazione di temperatura, infatti quest'ultima si ha solo quando varia l'energia interna.\\
            Il calore specifico è utile solo nel caso in cui c'è variazione di temperatura. Per esempio quando l'acqua nel freezer diventa ghiaccio (passaggio di fase), non c'è variazione di temperatura anche se il frigo ha spento energia per ghiacciare l'acqua, quindi c'è scambio di calore ma non ha senso parlare di calore specifico.\\

            Per una qualsiasi trasformazione quasi-statica, da $P_i$ a $P_f$, il calore è definito come segue,
            \begin{align*}
                Q = \int_{P_i}^{P_f} \delta Q = \int_{P_i}^{P_f} mc_{\gamma}dT
            \end{align*}
            ed in alternativa,
            \begin{align*}
                Q = C(T_f - T_i)
            \end{align*}
            dove $C$ è la capacità termica.

        \subsubsection{Calorimetro}
            Il \textbf{calorimetro} è un sistema controllato per il quale so connettere la variazione di temperatura al calore ricevuto. Quello dell'esempio precedente (contenitore adiabatico contenente acqua ed un oggetto immerso), è un calorimetro a bagno.

        \subsubsection{Capacità termica}
            La \textbf{capacità termica} è il prodotto tra massa e calore specifico della trasformazione che stiamo considerando, ovvero come segue,
            \begin{align*}
                C=mc_{\gamma}
            \end{align*}



    \subsection{Cambi di fase}
        La materia può trovarsi in \textbf{3 diversi stati}:
        \begin{itemize}
            \item solido: il corpo ha forma e volume propri;
            \item liquido: il corpo ha un volume proprio ma assume la forma del recipiente che lo contiene;
            \item gassoso: il corpo, molto spesso aeroforme, assume forma e volume del recipiente che lo contiene;
        \end{itemize}
        Sotto certe condizioni, un corpo può passare da uno stato all'altro, questo passaggio, in base agli stati iniziale e finale, ha doversi nomi:
        \framedImg{5}{L15-img001}
        Quando si parla di termodinamica, le "scale di misura" importanti sono volume, \textbf{pressione e temperatura}, se prendiamo le ultime 2 e le rappresentiamo su un piano cartesiano otteniamo quello che viene detto \textbf{diagramma di fase}:
        \framedImg{5}{L15-img002}
        Ovviamente varia da materiale a materiale, ma sostanzialmente la forma è quella. Prendendo ad esempio l'acqua, in corrispondenza di $T = 273,16 K$ e $P = 611 Pa$ abbiamo quello che si definisce \textbf{punto triplo}, ovvero quel punto in cui le \textbf{3 linee di demarcazione si intersecano}. In questo punto, i 3 stati possono coesistere contemporaneamente (in rosso nell'immagine)! È particolarmente importante perché \textbf{definisce la scala di temperatura}. Un altro punto particolarmente importante è quello che viene definito \textbf{punto critico}, ovvero il punto in cui \textbf{finisce la linea di demarcazione tra liquido e gas}. In particolare, oltre quel punto, gli stati gassoso e liquido non sono più distinguibili (fluido supercritico).

        \subsubsection{Calori latenti}
            Immaginiamo di voler far passare dell'acqua dallo stato liquido a quello gassoso: osservando il diagramma di fase, possiamo notare che per farlo potremmo far variare la pressione, mantenendo la temperatura stabile, oppure modificare la temperatura mantenendo la pressione stabile (oppure anche modificare entrambe ovviamente). Se decidessimo di mantenere la temperatura stabile, e quindi spostarci solo sull'asse P, potremmo notare che per mantenere la temperatura effettivamente stabile \textbf{dobbiamo fornire calore al sistema!} Il passaggio da liquido a gassoso, in genere, \textbf{ha bisogno di assorbire calore dall'ambiente circostante per effettuare il cambio di fase}. Allo stesso modo, se passassimo da gassoso a liquido dovremmo \textbf{assorbire noi calore} per poter mantenere la temperatura stabile.\\
            In particolare, questo "calore che non fa variare la temperatura", definito \textbf{calore latente di fusione/evaporazione/...}, è proporzionale alla massa dell'oggetto sottoposto al cambio di fase, in particolare la formula è:
            \begin{align*}
                \textcolor{Red}{Q_\lambda=\lambda*m}
            \end{align*}
            Il paramentro importante di questa formula è il $\textcolor{Red}{\lambda}$, ovvero la \textbf{costante di calore latente} e varia da materiale a materiale e, soprattutto, in base alla temperatura alla quale si trova il corpo sottoposto al cambio di fase. Facciamo alcuni esempi con l'acqua:
            \begin{align*}
                & \lambda_{fusione} = 3,3 *10^5\frac{J}{Kg} <==> 273,16 K && \lambda_{evaporazione} = 22,6 *10^5\frac{J}{Kg} <==> 373,16 K
            \end{align*}
            Idealmente, se andiamo verso lo stato solido (nel diagramma di fase saliamo verso l'alto sull'asse della pressione) è \textbf{il cambio di fase a darci calore} (siamo noi che dobbiamo assorbirlo), quindi abbiamo:
            \begin{align*}
                Q_\lambda=\textcolor{Red}{-}\lambda*m
            \end{align*}
            Se invece andiamo verso lo stato gassoso (nel diagramma di fase sceniamo verso il basso sull'asse della pressione) siamo \textbf{noi che dobbiamo dare calore al cambio di fase} (è il cambio di fase che lo assorbe), quindi abbiamo:
            \begin{align*}
                Q_\lambda=\textcolor{Red}{+}\lambda*m
            \end{align*}
            Nota che facciamo questo discorso (noi che forniamo calore, noi che assorbiamo calore). I \textbf{calori latenti} sono molto interessanti in quanto sono uno dei pochi casi in cui \textbf{abbiamo uno scambio di calore senza un'effettiva variazione di temperatura}.

            \paragraph{Esempio}
                Facciamo un veloce esercizio: \textit{abbiamo \textbf{1 litro} di acqua \textbf{liquida} ad una temperatura di \textbf{273,16 K}, \textbf{quanto calore dobbiamo fornire/sottrarre dal sistema per solidificare l'acqua mantenendo la temperatura costante?}} \\
                Allora, la prima cosa da notare è che \textbf{da liquido vogliamo passare a solido}, quindi la costante che ci interessa sarà \textbf{quella della fusione "al contrario"}. Avremmo allora:
                \begin{align*}
                    Q_\lambda = \lambda_{solidificazione}*m=\textcolor{Red}{-\lambda_{fusione}}*m
                \end{align*}
                Ora, un altro problema è che noi abbiamo \textbf{litri} di acqua, non chili. Ricorda che i \textbf{i litri sono un'unità di misura del volume}, quindi ci basta semplicemente \textbf{moltiplicare il volume per la densità dell'acqua} (in particolare $1\frac{Kg}{dm^3}$, aka. $1\frac{Kg}{l}$, infatti $1 litro = 1 dm^3$). Tornando a noi, avremmo:
                \begin{align*}
                    &Q_\lambda &&= -\lambda_{fusione}*m\\
                    & &&= -\lambda_{fusione}*\textcolor{Red}{\delta_{H2O}*V_{H2O}}\\
                    & &&= -3,3 *10^5\frac{J}{\bcancel{Kg}}*1\frac{\bcancel{Kg}}{\bcancel{l}}*1\bcancel{l}\\
                    & &&= -3,3 *10^5J = -330 KJ
                \end{align*}

            \paragraph{Esempio temperatura di equilibrio}

    \subsection{Trasferimento di calore}
        Iniziamo col ricordare che il calore non è altro che "\textit{energia in movimento}": fisicamente parlando, un corpa \textbf{non ha calore}, ma \textbf{ha energia} (ricordalo per l'orale :P). Detto ciò, introduciamo i 3 principali metodi per il trasferimento del calore:
        \begin{itemize}
            \item conduzione;
            \item convezione;
            \item irraggiamento;
        \end{itemize}

        \subsubsection{Conduzione}
            Ovvero il \textbf{trasferimento di energia tramite contatto}, immaginamo di avere una barra di "\textit{qualcosa}" collegata ad una \textbf{sorgente di calore ideale} (definiamo sorgente di calore ideale un \textit{corpo ideale capace di fornire/assorbire calore all'infinito}):
            \framedImg{7}{L15-img003}
            La barra inizialmente si trova ad una sua \textbf{temperatura iniziale $T_i$}, mentre la sorgente si trova costantemente ad una \textbf{temperatura $T_s$}. Supponendo che le dimensioni $L_x$ ed $L_y$ della barra (la sua sezione trasversale) siano \textbf{trascurabili}, possiamo immaginarla di \textbf{dividerla in tratti infinitesimali} per lunghezza. Se lasciamo scorrere il tempo, i \textbf{tratti vicini alla sorgente cambieranno la loro temperatura}, avvicinandosi (fino a raggiungere) $T_s$. Dopo un certo tempo, avremmo che \textbf{tutta la barra raggiungerà la temperatura} $T_2$.\\

            Come facciamo a quantificare il tutto? Come detto prima, immaginiamo di dividere la barra in molte sezioni trasversali:
            \framedImg{7}{L15-img004}
            Ora, possiamo notare diverse "proporzionalità":
            \begin{itemize}
                \item Q $\propto \Delta$ T: il calore assorbito dalla barra è \textbf{proporzionale alla differenza di temperatura tra le sue sezioni trasversali};
                \item Q $\propto \Delta$ t: il calore assorbito dalla barra è \textbf{proporzionale al tempo per cui resta a contatto con la sorgente ideale} (più tempo resto a contatto e più calore assorbo);
                \item Q $\propto$ S: il calore assorbito dalla barra è \textbf{proporzionale alla superficie di contatto tra la barra e la sorgete} (più superficie ho a contatto con la sorgente e più calore assorbo);
            \end{itemize}
            Se consideriamo \textbf{tratti infinitesimi} possiamo ricavare la \textbf{legge di Fourier}, ovvero:
            \begin{align*}
                \textcolor{Red}{dQ=-K*\frac{dT}{dz}*dS*dt}
            \end{align*}
            Analizziamola un po': sostanzialmente ci sta dicendo che la \textit{quantità infinitesima di calore \textbf{assorbito dalla barra}} (e quindi sottratto alla sorgente, per quello il segno -) \textit{è proporzionale alla differenza di temperatura per unità di lunghezza($\frac{\Delta T}{dz}$), alla superficie trasversale della barra ($dS$) e al tempo di contatto ($dt$). Tutto ciò è regolato da una costante di proporzionalità ($K$, nota che è DIVERSA dal Kelvin) che rappresenta la \textbf{conducibilità del materiale}}.

            \paragraph{La costante di conducibilità}
                Se ci rigiriamo un po' la formula di Fourier, possiamo facilmente notare che l'unità di misura di questa costante corrisponde a $\frac{[energ]}{[lung]*[temp]*[temperat]}$, in particolare avremmo $\frac{J}{m*s*K}$. Più è grande questa costante e più facilmente il notro materiale condurrà calore, vediamo qualche esempio:
                \begin{itemize}
                    \item alluminio: $200\frac{J}{msK}$, buon conduttore di calore;
                    \item sughero: $0,04\frac{J}{msK}$, cattivo conduttore di calore
                \end{itemize}
                Nota che, la maggior parte delle volte, i buoni conduttori di calore sono anche buni conduttori di energia elettrica (quindi i metalli).


        \subsubsection{Convezione}
            Il metodo di trasferimento del calore principale quando si parla di liquidi e di atmosfera. Non lo trattiamo approfonditamente (dato che i meccanismi algebrici che lo riguardano sono molto complessi), ma lo vediamo velocemente comunque. Supponiamo di essere in questa situazione:
            \framedImg{3}{L15-img005}
            Abbiamo una sorgente a contatto con un liquido più freddo. Ora, vediamo cosa succede se alziamo la temperatura della sorgente e aspettiamo un certo tempo (non specificato, un tempo generico):
            \framedImg{7}{L15-img006}
            Consideriamo diversi momenti:
            \begin{enumerate}
                \item partiremo in un momento in cui tutto il liquido (sia a contatto con la sorgente che non) è alla \textbf{stessa pressione $P_0$ e temperatura $T_0$};
                \item col passare del tempo, il liquido a contatto con la sorgente \textbf{inizia a scaldarsi} e ciò comporta un \textbf{aumento della sua pressione} (al momento prendi per buono);
                \item dato che la pressione $P_1 > P_0$, allora il liquido caldo \textbf{inizia ad espandersi in maniera irregolare};
                \item dato che l'espansione è irregolare, può succedere che \textbf{bolle calde si "staccano" dalla sorgente} ed iniziano a "fluttuare" nel liquido freddo, \textbf{permettendo ad altro linquido freddo di scaldarsi con la sorgente}. Un'altro punto importante è che \textbf{anche la bolla calda scalderà il liquido freddo!}
            \end{enumerate}
            Concludendo, l'idea è che, con il passare del tempo, il numero di bolle calde che navigano nel fluido continua ad aumentare finché \textbf{tutto il fluido non arriverà alla temperatura della sorgente}.

        \subsubsection{Irraggiamento}
            Ogni corpo emette delle \textbf{onde elettromagnetiche}, queste non sono altro che \textbf{fasci di energia pura}: trasferendo energia possiamo trasferire calore! Ora, la formula che regola questo fenomeno è la \textbf{legge di Stefan-Boltzmann}, ovvero:
            \begin{align*}
                \textcolor{Red}{\epsilon=\sigma*e*T^4}
            \end{align*}
            In particolare abbiamo che:
            \begin{itemize}
                \item $\varepsilon$: l'energia emessa da un corpo;
                \item $\sigma$: la \textbf{costante di Stefan-Boltzmann}, vale \textcolor{Red}{$\sigma=5,67*10^-8\frac{J}{m^2*s*K^4}$};
                \item $e$: emissività del corpo, possiamo vederla anche come la capacità del corpo di riflettere le onde elettromagnetiche. Vale un valore $\in [0, 1]$, dove 1 identifica la "massima rifelttività" del corpo, mentre 0 il "massimo assorbimento" (un corpo totalmente nero) NON SICURO!!! RICONTROLLARE;
                \item $T^4$: la temperatura $^4$ del corpo che consideriamo.
            \end{itemize}
            Se analizziamo l'unità di misura di questa quantità $\varepsilon$ vediamo che corrisponde a $\textcolor{Red}{\frac{J}{m^2*K}}$, in altre parole $[\varepsilon]=[\frac{E}{L^2*T}]$, ovvero Energia su Lunghezza$^2$*Temperatura.
            Nota che tutti i corpi emettono queste onde elettromagnetiche, un esempio interessante (e soprattutto che potrebbe tornare all'esame) è quello della \textbf{costante solare}, ovvero la quantità di energia "irragiata" dal sole. Identifichiamo qusta quantità (così come un sacco di altre cose che non centrano niente, quindi ATTENZIANE) con la lettera $c$ e la chiamiamo \textbf{costante solare}. In particolare, vale (mediamente):
            \begin{align*}
                c=1,36\frac{J}{m^2*s}
            \end{align*}
            Una piccola nota conclusiva: è abbastanza improprio considerare l'irraggiamento come forma di trasmissione del calore "normale", infatti \textbf{non c'è un vero e proprio trasferimento meccanico del calore}

            \paragraph{Esercizio con costante solare}

        \subsubsection{Contenitore adiabatico}
            Ora che conosciamo i vari metodi per il trasferimento di calore, possiamo porci il quesito \textbf{"come costruire un contenitore veramente adiabatico"} (isolante)? Uno dei metodi possibili è quello del \textbf{\textit{vaso Dewar}}, che possiamo schematizzare in questo modo:
            \framedImg{4}{L15-img007}
            Per evitare tutti i tipi di trasferimento di calore, il contenitore deve avere diverse caratteristiche, in particolare:
            \begin{enumerate}
                \item essere a tenuta stagna, in modo da \textbf{evitare la convezione};
                \item avere uno "strato di vuoto" il più vicino al liquido da mantenere a temperatura, in modo da \textbf{evitare la conduzione};
                \item avere delle pareti \textbf{riflettenti} (Dewar usava delle pareti argentate) in modo da \textbf{evitare l'irraggiamento}. Nota che le pareti devono essere riflettenti \textbf{sia verso l'esterno} (per riflettere le onde dell'ambiente esterno) \textbf{sia verso l'interno} (in modo da impedire al liquido di disperdere il suo calore).
            \end{enumerate}

    \subsection{Gas ideali}
        I gas sono un sistema termodinamico composto da un \textbf{gran numero di molecole libere di muoversi nello spazio}. Noi consideriamo i cosidetti \textbf{"gas ideali"}, ovvero quei gas per cui valgono le seguenti condizioni:
        \begin{itemize}
          \item gli elementi che compongono il gas \textbf{non interagiscono tra di loro in alcun modo};
          \item gli elementi che compongono il gas \textbf{hanno volume nullo} (li approssimiamo come dei punti materiali, infatti, rispetto a loro, il contenitore che le contiene è infinitamente grande [di solito]);
          \item \textbf{non ci sono reazioni chimiche} tra le componenti del gas o col contenitore.
        \end{itemize}
        Come si intuisce dal nome, i \textbf{gas ideali} (in quanto ideali) \textbf{sono impossibili}, però sono una \textbf{buona approssimazione per quei gas a bassa pressione} (in genere $< 100 atm$).

        \subsubsection{Le variabili principali}
          Per descrivere i gas, usiamo principalmente \textbf{3 variabili}:
          \begin{itemize}
              \item volume ($V$);
              \item pressione ($P$);
              \item temperatura ($T$);
          \end{itemize}
          Analizziamole velocemente:
            \paragraph{Volume}
                La più semplice delle 3: quando parliamo di gas supponiamo che \textbf{questo sia contenuo in un qualche contenitore}, questo contenitore \textbf{"contiene" una porzione di spazio} che rappresenta appunto il volume. Nota che, se non diversamente specificato, supponiamo che il gas \textbf{occupi tutto il volume a sua disposizione}. Il volume si misura come una \textbf{lunghezza al qubo} ($L^3$), in particolare il \textcolor{Red}{\textbf{metro$^3$}} oppure il \textbf{litro} (che corrisponde a un $dm^3 = 10^{-3}m^3$).
            \paragraph{Pressione}
                Il fatto che il gas occupi tutto lo spazio a sua disposizione implica che \textbf{sarà a contatto col contenitore che lo contiene}. Inoltre, le molecole del gas sono in costante movimento quindi ci saranno \textbf{degli urti tra il contenitore e le molecole del gas}: ciò darà origine ad una forza che premerà sul contenitore.
                \framedImg{4}{L15-img008}
                Formalizzando il tutto, ogni molecola del gas si muove liberamente nel contenitore con una \textbf{certa velocità} \textcolor{Red}{$v$} ed ha anche una \textbf{certa massa} \textcolor{Red}{$m$}. Possiamo quindi dire che \textbf{ogni molecola ha una certa quantità di moto} \textcolor{Red}{$m*v = p$}. Ora, quando le molecole urtano il contenitore avviene \textbf{un urto perfettamente elastico} tra la molecola e la parete del contenitore. C'è da notare che la massa della molecola ($m$) è \textbf{infinitesimale} rispetto alla massa della parete del contenitore ($m_{parete}$), quindi potremmo dire che c'è una specie di "\textbf{inversione del moto}". Questa variazione della quantità di moto $\Delta p$ avviene in un certo tempo $\Delta t$ e sappiamo che non è nulla, quindi abbiamo \textbf{una forza impulsiva!}
                \begin{align*}
                    \frac{m*\vec{v}_i - m*\vec{v}_f}{\Delta t} = \frac{\Delta p}{\Delta t} => \vec{F}\neq \vec{0}
                \end{align*}
                Questo per ogni singola molecola, se però sommiamo le forze impulsive di tutte le molecole del gas otteniamo una forza (ortogonale alla parete del contenitore) non trascurabile. Se poi andiamo a considerare la \textbf{forza ortogonale per unità di superficie otteniamo proprio la pressione!} In particolare:
                \begin{align*}
                    p = \frac{F_\bot}{S}
                \end{align*}
                Dimensionalmente, ci troviamo con una Forza fratto una Superficie, avremmo quindi:
                \begin{align*}
                    [p]=\frac{[F]}{[L^2]}=\frac{N}{m^2}=\textcolor{Red}{Pascal=Pa}
                \end{align*}
                L'unità di misura fondamentale della pressione è il Pascal, esistono poi delle "derivate", in particolare citiamo:
                \begin{align*}
                    &\textcolor{Red}{bar = 10^5Pa}&&\textcolor{Red}{atmosfera=atm=1,013bar}
                \end{align*}
                Terminiamo la pressione con una precisazione per quanto riguarda i \textbf{gas in equilibrio}. Immaginiamo di avere una situazione di questo tipo:
                \framedImg{5}{L15-img009}
                Una supposizione importante per capire questo "esempio" è che il contenitore (in verde) possa, volendo, espandersi all'infinito. Ora, abbiamo detto che le molecole del gas all'interno del contenitore, scontrandosi con le sue pareti, creano una forza (che noi chiamiamo \textbf{pressione}) che \textbf{spinge verso l'esterno}. Se prendiamo ad esempio un palloncino e lo gonfiamo, notiamo che inizia ad \textbf{aumentare di volume}. Ad un certo punto però \textbf{si ferma}. Ciò avviene perché si è raggiunto uno \textbf{stato di equilibrio}, ovvero abbiamo la forza della pressione che spinge verso l'esterno ma c'è anche una forza uguale ma opposta che bilancia la pressione. Nel caso del palloncino, questa è la \textbf{risultante della somma tra forza elastica del pallone} ($\vec{F}_{gas->S_1}$) \textbf{e pressione dell'ambiente esterno} che preme sul pallove dall'esterno verso l'interno ($\vec{F}_{amb->S_1}$). Se noi "estremizziamo" questo esempio e togliamo la forza elastica del pallone, abbiamo che il pallone smette di crescere solo quando \textbf{la forza del gas che preme sulla superficie è uguale alla forza dell'ambiente che preme sulla superficie}, ovvero:
                \begin{align*}
                    &\vec{F}_{amb->S_1} = -\vec{F}_{gas->S_1}&&=>\frac{F_{\bot,a}}{S_1}=\frac{F_{\bot,g}}{S_1}\\
                    & && =>\textcolor{Red}{P_a=P_g}
                \end{align*}
                Questa equazione in \textcolor{Red}{rosso} è molto importante per i \textbf{sistemi in equilibrio}. Di fatto, la parete del contenitore trasmette meccanicamente le forze dall'interno verso l'esterno e vice versa. Possiamo quindi pensare di "togliere virtualmente" la parete e fare dei confronti tra i 2 gas. In molti casi, infatti, non possiamo direttamente misurare la pressione dei gas che ci interessano, però, se questi si trovano in equilibrio con l'ambiente esterno, possiamo calcolarla considerando la pressione dell'ambiente esterno! Ad esempio:
                \framedImg{5}{L15-img010}
                Abbiamo un pistone che preme del gas contenuto in un contenitore a tenuta stagna. Di questo pistone conosciamo la \textbf{massa} ($m$) e la \textbf{superficie} (S), come facciamo a calcolare la pressione del gas in rosso? Se il sistema si trova in uno stato di equilibrio, ovvero il pistone è fermo e non scende (comprimento ulteriormente il gas), possiamo dire che la pressione del gas rosso è \textbf{uguale alla pressione dell'ambiente esterno}, ovvero:
                \begin{align*}
                    &P_{ambiente}&&=P_{atmosferica}+P_{pistone}\\
                    & &&=P_{atmosferica}+\frac{F_{pesoPistone}}{S}\\
                    & &&=P_{atmosferica}+\frac{m*g}{S}
                \end{align*}
                Concludendo, dato che la pressione del gas rosso è uguale alla pressione dell'ambiente esterno (perchè il sistema è in equilibrio), abbiamo che $P_{gasRosso}=P_{atmosferica}+\frac{m*g}{S}$!

                \paragraph{Temperatura}
                    La temperatura è \textbf{fortemente legata all'energia interna del gas}. Possiamo vedere l'energia interna del gas come la \textbf{somma del'energia cinetica di ogni molecola del gas} ($\frac{1}{2}mv^2$). Per quanto riguarda le unità di misura, abbiamo già detto prima che utilizziamo il \textbf{Kelvin} ($K$) e, in alcuni casi, i gradi \textit{Celsius} ($^\circ C$) o \textit{Fahrenheit} ($^\circ F$)

            \subsubsection{Equazione di stato dei gas ideali (o perfetti)}
                Introduciamo ora l'\textbf{equazione di stato di questi gas ideali}, prima però una veloce definizione (tratta da Wikipedia) di equazione di stato:
                \begin{center}
                    \textit{In termodinamica e chimica fisica, una equazione di stato è una legge costitutiva che descrive lo stato della materia sotto un dato insieme di condizioni fisiche. Fornisce una relazione matematica tra due o più variabili di stato associate alla materia, come temperatura, pressione, volume o energia interna.}
                \end{center}
                In pratica, è una funzione (= 0) che ci descrive il comportamento del nostro gas in base alla variazione delle variabili termodinamiche che consideriamo. Come detto prima, per i gas usiamo come variabili \textbf{pressione, volume e temperatura}, dato che sono 3 rappresentiamo il tutto in un \textbf{diagramma bidimensionale} ($3-1$, infatti se avessimo N variabili dovremmo usare una dimensione $N-1$, questo perché, di solito, siamo interessati a scoprire il valore di una particolare variabile date le altre). In particolare, per i gas l'equazione di stato è:
                \begin{align*}
                    \textcolor{Red}{p*V - n R T = 0}
                \end{align*}

                Questa equazione la si ottiene basandosi sui risultati di 3 (o meglio 4) leggi, vediamole.
                \paragraph{Legge isocora di Gay-Lussac}
                    Se noi teniamo il \textbf{volume costante} ($V=const$, isocora), abbiamo che pressione la corrisponde a:
                    \begin{align*}
                        p=p_0(1+\beta t)
                    \end{align*}
                    Dove $p_0$ corrsiponde alla pressione ad una temperatura di \textbf{0 gradi Celsious}. Se noi rappresentassimo questa equazione sul piano cartesiano in funzione della temperatura, otterremo qualcosa del genere:
                    \framedImg{5}{L16-img001}
                    Nota come la retta è crescente, questo implica che, per qualche valore di temperatura ($t^*$), avremmo \textbf{pressione = 0}, e ciò è particolarmente interessante! Idealmente, più è bassa la temperatura e, a parità di volume, la pressione si abbassa: però la pressione \textbf{non può essere negativa}! Su questo punto ci torneremo in seguito;
                \paragraph{Legge isobara di Gay-Lussac}
                    Se noi teniamo la \textbf{pressione costante} ($p=const$, isobara), abbiamo che il volume in funzione della temperatura corrisponde a:
                    \begin{align*}
                        V=V_0(1+\alpha t)
                    \end{align*}
                    Dove $V_0$ corrsiponde al volume ad una temperatura di \textbf{0 gradi Celsious}. Se noi rappresentassimo questa equazione sul piano cartesiano in funzione della temperatura, otterremo qualcosa del genere:
                    \framedImg{5}{L16-img002}
                    Nota come la retta è crescente, questo implica che, per qualche valore di temperatura ($t^{**}$), avremmo \textbf{volume = 0}, e anche questo è particolarmente interessante! Idealmente, più è bassa la temperatura e, a parità di pressione, il volume si abbassa: però il volume \textbf{non può essere negativo}! Cosa ancora più interessante, $t^* = t^{**}$ (i 2 punti sono uguali)! Dopo ci torneremo più in dettaglio;
                \paragraph{Legge isoterma di Boyle}
                    A parità di temperatura, il \textbf{prodotto $p*V$ resta costante}! Ovvero:
                    \begin{align*}
                        p_i*V_i = p_f*V_f = const
                    \end{align*}
                    In soldoni, se tengo la temperatura costante e abbasso la pressione, il volume \textbf{aumenta per compensare} e \textbf{viceversa}. Se prendiamo questa equazione e la rappresentiamo su un grafico bidimensionale, otteniamo un'iperbole equilatera:
                    \framedImg{5}{L16-img003}

                \paragraph{Legge di Avogadro}
                    Gas diversi nelle stesse condizioni di temperatura, pressione e volume \textbf{contengono lo stesso numero di oggetti} ($N$). In particolare, abbiamo che:
                    \begin{align*}
                        N=\frac{1}{K_B}\frac{pV}{T}
                    \end{align*}
                    Dove $K_B$ rappresenta la \textbf{costante di Boltzmann} (\textcolor{Red}{$K_B = 1,38*10^{-23}\frac{J}{K}$}). Da dove deriva la sua unità di misura? Se controlliamo l'equazione precedente, abbiamo che $N$, ovvero il numero di "molecole" del gas (adimensionale) è (tralasciando $K_B$) uguale al $[\frac{pV}{T}] =[\frac{\frac{F}{L^2}*L^3}{T}]=[\frac{F*L}{T}]=[\frac{E}{T}]=\frac{J}{K}$ ricorda che, per definizione, \textbf{una forza per uno spostamente è un'energia} (che ha unità di misura Joule). Dato che dobbiamo "semplificare" queste unità di misura per rendere il tutto adimensionale, la costante di Boltzmann avrà come unità di misura $\frac{J}{K}$!
                    Molto spesso però, ci troveremo a lavorare con il \textbf{numero di moli} $n$ invece che con il numero esatto di elementi $N$. Possiamo quindi modificare la formula precedente in questo modo:
                    \begin{align*}
                        n=\frac{1}{R}\frac{pV}{T}
                    \end{align*}
                    Sappiamo infatti che le moli $n=\frac{N}{N_A}$ (ricorda che il \textit{numero di Avogadro} $N_A=6,022*10^{23}$), quindi ci basta \textbf{divedere da entrambe le parti per $\mathbf{N_A}$}. Infatti $R$ corrisponde proprio a $R=\frac{1}{K_B*N_A}\approx8,314\frac{J}{K*mol}$. Concludiamo con un po' di numeri, supponiamo di avere un gas con queste caratteristiche:
                    \begin{align*}
                        &n = 1mol&&p=1atm&&t=0^\circ C =>273,16 K
                    \end{align*}
                    Allora il suo volume è fissato, e vale:
                    \begin{align*}
                        &V = 22,414 dm^3 = 22,414 l
                    \end{align*}
                    Questo viene definito \textbf{volume molare}.

                \paragraph{Mettiamo tutte le leggi insieme}
                    L'equazione di stato dei gas si ottine \textbf{mettendo insieme le 4 leggi viste prima}, cominciamo da quelle di \textit{Gay-Lussac}:
                    \doubleFramedImg{4}{L16-img001}{L16-img002}
                    Avevamo questi 2 grafici, ovviamente la retta dipende dal gas: per gas diversi otterremo rette diverse. La cosa interessante però è che i punto $t^*$ e $t^{**}$ \textbf{puntano allo stesso valore di temperatura}! Questo succede indipendentemente dal valore  di $\alpha$ e $\beta$ (che dipendono dal tipo di gas, dal contenitore, ...). Prendendo 2 misurazioni (2 punti sul grafico) siamo capaci di tracciare la retta associata e stimare questa temperatura. Col tempo, siamo arrivati ad un'ottima approssimazione, ovvero:
                    \begin{align*}
                        t^*=t^{**}=t_0=-273,15^\circ C = 0 K
                    \end{align*}
                    L'idea ora è quella di fare una cosa del genere:
                    \doubleFramedImg{4}{L16-img004}{L16-img005}
                    Ovvero \textbf{spostiamo il sistema di riferimento}, in modo da usare il \textbf{sistema di riferimento assoluto per la temperatura} (i gradi \textit{Kelvin} appunto). Se facciamo cos', le equazioni delle leggi fondamentali cambiano leggermente, in particolare:
                    \begin{align*}
                        &\begin{cases}
                            gay-lussac:\ p=p_0(1+\beta t)\\
                            gay-lussac:\ p=V_0(1+\alpha t)\\
                            boyle:\ p_i*V_i = p_f*V_f = const\\
                            avogadro:\ n=\frac{1}{R}\frac{pV}{T}\\
                        \end{cases}&&=>
                        &&\begin{cases}
                            \textcolor{Red}{gay-lussac:\ p=p_0\beta T}\\
                            \textcolor{Red}{gay-lussac:\ p=V_0\alpha T}\\
                            boyle:\ p*V=const\\
                            avogadro:\ \frac{pV}{T}=nR\\
                        \end{cases}
                    \end{align*}
                    Il significato resta sempre lo stesso, le stiamo solo considerando con la scala della temperatura assoluta. Questa formulazione delle \textbf{leggi sperimentali} ci permette di capire che questi sono solo dei "casi particolari" di una formulazione più generica, che è proprio la nostra \textbf{equazione di stato dei gas ideali}:
                    \begin{align*}
                        pV=nRT
                    \end{align*}
                    Ricorda che questa equazione vale \textbf{solo per gli stati di equilibrio del nostro sistema termodinamico}! Un'altra cosa interessante da notare è che l'equazione di stato è uguale all'equazione di Avogadro: in realtà la formulazione della legge di Avogadro non è esattamente quella data. Nell'originale ci sono delle costanti moltiplicative che la rendono effettivamente diversa dall'equazione di stato. Al momento però ce la facciamo andare bene così.

                \paragraph{Esempio}

            \subsubsection{Lavoro dei gas}
                In questa parte è importante \textbf{ricordare la prima legge della termodinamica}:
                \begin{align*}
                    \Delta U = Q-W
                \end{align*}
                Ora, immaginiamo di essere in una situazione in cui c'è un gas che si espande:
                \framedImg{8}{L16-img006}
                Partiamo con un gas a volume $V_0$ e a pressione $p_0$, ora supponiamo che si facciam un \textbf{incremento infinitesimale del volume}, passando al volume $V_1=V_0+dV$, mantenendo una pressione $p_0$ costante. Come calcoliamo il lavoro infinitesimale ($dW$) compiuto con questo incremento di volume? Sappiamo che \textbf{il lavoro corrisponde a Forza * Spostamento} ($dW = \vec{F}*d\vec{s}$, volendo possiamo togliere anche il simbolo del vettore, dato che tutto avvien sullo stesso asse). Possiamo fare un po' di passaggi algebrici:
                \begin{align*}
                    &dW= &&= \vec{F}*d\vec{s}\\
                    &\textrm{\textit{Sostituiamo i parametri}}&& =\textcolor{Purple}{F}*\textcolor{Orange}{dh}\\
                    &\textrm{\textit{Moltiplichiamo e dividiamo per la Superficie}}&& = \frac{F}{\textcolor{OrangeRed}{S}}*\textcolor{OrangeRed}{S}dh\\
                    &\textrm{\textit{Abbiamo che $\frac{F}{S}=pressione$ e $S*dh=L^2*L=volume$}} &&=p*dV
                \end{align*}
                Questo ragionamento vale solo per \textbf{lavori infinitesimi} e, quindi, per \textbf{spostamenti infinitesimi}: cosa succede quando abbiamo spostamenti più grandi? "Semplicemente" facciamo l'integrale di tutti i lavori infinitesimi! In particolare:
                \begin{align*}
                    \int_i^f dW = \int_i^f p(v)*dV
                \end{align*}
                Nota come \textbf{la pressione dipende dal volume}: nell'esempio prima potevamo mettere che la pressione restasse costante dato che la variazione di volume è infinitesima, ma nel caso in cui abbiamo una variazione maggire dobbiamo tenerne conto! Il messaggio da portare a casa è che \textbf{il gas compie del lavoro quando cambia di volume}!

            \subsubsection{Energia interna dei gas}
                Diciamolo subito: \textbf{l'energia interna di un gas dipende SOLO dalla sua temperatura}. Possiamo arrivare a questa conclusione con diversi metodi, vediamoli.
                \paragraph{Esperimento dell'espasione libera del gas}
                    A questo punto è molto importante citare l'\textbf{esperimento dell'espansione libera di Joule}:
                    \framedImg{5}{L16-img007}
                    In questo esperimento abbiamo un vaso Dewar che contiene un liquido, in questo liquido sono immerse 2 camere: una contiene del gas, l'altra il vuoto. Nota che il liquido è in equilibrio termico col gas. Queste camere sono collegate da un piccolo passaggio, inizialmente chiuso, e sono fatte di un materiale che \textbf{conduce calore}. Ad un certo punto \textbf{apriamo il passaggio tra le 2 camere}, permettendo al gas di passare nella camera vuota. Dopo questa \textbf{espansione libera del gas} "succedono delle cose", in particolare:
                    \begin{align*}
                        & V_i < V_f && \textrm{Il volume occupata dal gas aumenta (occupa entrambe le camere)}\\
                        & p_i > p_f && \textrm{La pressione del gas diminuisce (stessa quantità in volume maggiore)}\\
                        T_i \approx T_f
                    \end{align*}
                    L'ultima riga è la più interessante: nonostante il volume e la pressione siano cambiati, la temperatura resta uguale (in realtà cambia, ma in modo quasi impercettibile). In generale, l'energia interna del sistema dipende da pressione, volume e temperatura ($U(p, V, T)$). Inoltre, per il primo principio della termodinamica, abbiamo che $\Delta U = Q - W$. Ora, il \textbf{contenitore è rigido}  quindi il \textbf{volume resta effettivamente costante}, di conseguenza \textbf{NON C'E' LAVORO} ($W=0$, è vero che il gas aumenta di volume però lo fa liberamente senza opporsi a nessuno [es. un pistone], quindi NON c'è lavoro). Per quanto riguarda il calore, tutto il nostro sistema si trova in un \textbf{contenitore adiabatico}, allora \textbf{non abbiamo scambio di calore con l'esterno} ($Q=0$, il calore del sistema è già all'interno del sistema, non si può prendere calore da nessuna parte). Dunque, per la prima legge della dinamica abbiamo che:
                    \begin{align*}
                        \Delta U = Q-W = 0-0 = 0 => \mathbf{\Delta U = 0}
                    \end{align*}
                    Quindi, se $\Delta U == 0$, allora che $\Delta T == 0$!
                    Abbiamo detto prima che, \textbf{in genere}, l'energia interna di un sistema dipende da pressione, volume e temperatura, \textbf{PERO'} abbiamo visto sperimentalmente che, pur cambiando la pressione ed il volume, \textbf{l'energia interna del sistema resta ugugale}: possiamo quindi dire che \textbf{l'energia interna del gas dipende solo dalla Temperatura}!


                \paragraph{Teoria cinetica dei gas}
                    Prima abbiamo visto un \textbf{metodo sperimentale}, ora passiamo a vedere il "metodo" matematico. Per fare ciò dobbiamo introdurre quella che viene definita \textbf{teoria cinentica dei gas}, ovvero "\textit{come si muovono le molecole dei gas?}" Immaginiamo di avere un contenitore tridimensionale che contiene del gas:
                    \framedImg{3}{L16-img008}
                    Le molecole del gas si \textbf{muovono in modo caotico}, \textbf{\underline{non hanno una direzione preferita}} (molto importante). Supponiamo che il contenitore sia un qubo di \textbf{lato $\mathbf{L}$}. Iniziamo considerando una parete singola:
                    \framedImg{3}{L16-img009}
                    La nostra molecola entrerà in collisione con la "parete $L_y/L_z$" con \textbf{un angolo $\mathbf{\theta}$, una certa velocità $\mathbf{v_x}$ ed una massa $m$} (la stessa per tutte le molecole): in questo caso si verificherà un \textbf{urto perfettamente elastico} dove la \textbf{parete del contenitore resterà perfettamente immobile} (dato che la sua massa è praticamente infinita rispetto a quella della molecola). Ora, l'angolo di uscita della molecola sarà uguale a quello di entrata e, più importante, la \textbf{velocità sull'asse x} (che nel caso della parete ortogonale all'asse x è l'unica che ci interessa) sarà \textbf{uguale e contraria alla velocità di entrata}. Avremmo quindi un $\Delta p$ (quantità di moto) molto specifico:
                    \begin{align*}
                        \textcolor{Red}{\Delta p} = p_f-p_i = m*v_{x, f}-m*v_{x, i}=m*(v_{x, f}-v_{x, i}) =2m*v_{x, i}= \textcolor{Red}{2m*v_{x}}
                    \end{align*}
                    Ora possiamo chiederci "\textit{ogni quanto, in termini di tempo, questa molecola urta QUESTA parete}"? Dato che in ogni urto l'angolo d'uscita è lo stesso di quello in entrata, possiamo immaginare che le molecole del gas si muovano in questo modo "speculare":
                    \framedImg{5}{L16-img010}
                    Dunque, per percorrere nella sua interezza la lunghezza $L_x$ (che si trova sull'asse x) alla velocità $v_x$ impiegheremo esattamente $\frac{L_x}{v_x}$, poi moltiplichiamo il tutto per 2 dato che vogliamo gli urti sulla stessa parete (praticamente si deve percorrere $2L_x$). Otteniamo quindi:
                    \begin{align*}
                        \textcolor{Red}{\Delta t = \frac{2L_x}{v_x}}
                    \end{align*}
                    Abbiamo un $\Delta p$ e abbiamo un $\Delta t$, quindi \textbf{abbiamo una forza impulsiva}! In particolare:
                    \begin{align*}
                        F_x=\frac{\Delta p}{\Delta t}= \frac{2m*v_{x}}{\frac{2L_x}{v_x}} = \frac{2m*v_{x}*v_{x}}{2L_x}= \frac{m*v_{x}^2}{L_x}
                    \end{align*}
                    Questa è la forza esercitata da una paticella sola, se vogliamo calcolare la \textbf{forza totale esercitata sulla parete} dobbiamo sommare la forza esercitata da tutte le particelle, quindi:
                    \begin{align*}
                        F_x^{tot}=\sum_{i=1}^N  \frac{m*v_{x}^2}{L_x}=  \frac{m}{L_x}\sum_{i=1}^Nv_{x}^2
                    \end{align*}
                    Nota come $m$ ed $L_x$ siano 2 valore costanti per tutte le particelle, quindi li portiamo fuori dalla sommatoria. A questo punto, le particelle sono davvero tante e trovare la velocità di tutte potrebbe essere un procedimento un po' lungo, quindi \textbf{iniziamo a considerare la velocità media delle particelle} ($<v_x>$), quindi possiamo scrivere così:
                    \begin{align*}
                        &F_x^{tot}=\frac{m}{L_x}\sum_{i=1}^Nv_{x}^2=\frac{m}{L_x}N*<v_{x}^2>&&=>&&\textcolor{Red}{F_x^{tot}=\frac{m}{L_x}N*<v_{x}^2>}
                    \end{align*}
                    Bene, a questo punto decidiamo di \textbf{dividere da entrambe le parti per} $\mathbf{L_y}$ \textbf{e} $\mathbf{L_z}$, ottenendo:
                    \begin{align*}
                        &F_x^{tot}=\frac{m}{L_x}N*<v_{x}^2> &&=>\frac{F_x^{tot}}{L_y*L_z}=\frac{m}{L_x*L_y*L_z}N*<v_{x}^2>\\
                        & Superficie\ e\ Volume&&=>\frac{F_x^{tot}}{S}=\frac{m}{V}N*<v_{x}^2>\\
                        &Pressione\ su\ X&&=>P_x=\frac{m}{V}N*<v_{x}^2>
                    \end{align*}
                    Questo è un risultato abbastanza importante: abbiamo ottenuto la \textbf{pressione misurata sull'asse x}. Ora facciamo un passaggio abbastanza "scivoloso": \textbf{la pressione che misuriamo sull'asse x deve essere uguale a quella che misuriamo sull'asse y e sull'asse z} (la pressione del gas deve essere uguale in tutto il gas)!
                    \begin{align*}
                        P_x=P_y=P_z=P
                    \end{align*}
                    Da questo punto, considerando la formula della $P-x$ che abbiamo trovato prima, possiamo dedurre che:
                    \begin{align*}
                        <v_x^2>=<v_y^2>=<v_z^2>
                    \end{align*}
                    Questo è vero perché il gas \textbf{non può avere una direzione preferita \underline{in media}}! Tutte le particelle si muovono in moto assolutamente caotico. Ora, considerando la velocità al quadrato media del gas:
                    \begin{align*}
                        <v^2>=<v_x^2+v_y^2+v_z^2>=<v_x^2>+<v_y^2>+<v_z^2>
                    \end{align*}
                    PERO', dato che $<v_x^2>=<v_y^2>=<v_z^2>$, abbiamo:
                    \begin{align*}
                        <v^2>=<v_x^2>+<v_y^2>+<v_z^2>=3*<v_x^2>
                    \end{align*}
                    Allora $<v_x^2>$ sarà ugugale a:
                    \begin{align*}
                        <v^2>=3*<v_x^2>=>\textcolor{Red}{<v_x^2>=\frac{<v^2>}{3}}
                    \end{align*}
                    Possiamo usare questa nuova informazione nella formula precedente:
                    \begin{align*}
                        P=\frac{m}{V}N*<v_{x}^2>=\frac{m*N*<v^2>}{3V}=\frac{N*<m*v^2>}{3V}
                    \end{align*}
                    Questo è un risultato molto interessante, infatti possiamo \textbf{moltiplicare e dividere per 2}, ottenendo \textbf{l'energia cinetica media}:
                    \begin{align*}
                        &P=\frac{N*<m*v^2>}{3V}&&=>P=\frac{N}{3V}*<m*v^2>\\
                        & &&=>P=\frac{2N}{3V}*<\frac{1}{2}*m*v^2>\\
                        & &&=>\textcolor{Red}{P=\frac{2N}{3V}*<E_K>}
                    \end{align*}
                    Ma non ci fermiamo qui, infatti possiamo "spostare" il volume così:
                    \begin{align*}
                        &P=\frac{2N}{3V}*<E_K>&&=>PV=\frac{2N}{3}*<E_K>
                    \end{align*}
                    Ora, per l'equazione di stato dei gas ideali, possiamo \textbf{sostituire quel $\mathbf{PV}$}:
                    \begin{align*}
                        & PV=\frac{2N}{3}*<E_K> && => \bcancel{N}K_BT=\frac{2\bcancel{N}}{3}*<E_K>
                    \end{align*}
                    Alla fine di tutto questo procedimento molto verboso, possiamo finalmente concludere con:
                    \begin{align*}
                        \textcolor{Red}{<E_K>=\frac{3}{2}*K_BT}
                    \end{align*}
                    \textbf{L'energia cinetica media delle particelle del gas dipende solo dalla sua Temperatura} (moltiplicata per una costante). Questo è un risultato molto interessante. Avevamo detto in precedenza che, nel nostro modello molto easy dei gas ideali, l'\textbf{energia interna del sistema corrisponde alla dell'energia cinetica delle particelle che compongono in nostro gas} ($N<E_K>=U$), possiamo  quindi dire che:
                    \begin{align*}
                        U=\textcolor{Red}{\frac{3}{2}*NK_B}T
                    \end{align*}
                    Questa formula è fantastica: abbiamo stabilito per \textbf{deduzione dal modello} che \textbf{l'energia cinetica} delle particelle del gas e quindi l'energia interna del gas \textbf{dipende solo la temperatura}! Infatti tutta la \textcolor{Red}{parte in rosso} è \textbf{costante}, l'unica roba che può cambiare e far variare l'energia interna è la Temperatura. Rigirando la formula, possiamo dire che la temperatura del gas dipende dall'energia cinetica delle sue particelle! Da questo possiamo dire che, per raggiungere lo \textit{zero assoluto} (0 K o -273,16 $^\circ$ C) solo quando tutte le particelle sono completamente ferme.

                    \subparagraph{Gradi di libertà}
                        Per quanto riguarda questo metodo "matematico" c'è un discorso sui \textit{gradi di libertà}. Partiamo da questa formul:
                        \begin{align*}
                            U=\frac{\textcolor{Orange}{3}}{2}*NK_BT
                        \end{align*}
                        Ora, \textbf{da dove deriva \textcolor
                        {Orange}{questo 3}}? Se ricordi, quel 3 deriva da come abbiamo "semplificato" la velocità della particella del gas:
                        \begin{align*}
                            <v^2>=<v_x^2>+<v_y^2>+<v_z^2>=3<v_x^2>
                        \end{align*}
                        Questo ragionamento va bene per \textbf{i gas mono-atomici} (A), ovvero quelli composti da "particelle uniche". Se però abbiamo \textbf{gas bi-atomici} (B) o \textbf{poli-atomici} (C)?
                        \framedImg{5}{L16-img011}
                        Non si deve cambiare molto il ragionamento, si deve solo tenere conto del fatto che le particelle bi-atomiche e poli-atomiche \textbf{possono ruotare}. Non ci fermiamo troppo, basti sapere che la \textbf{formula generale per l'energia interna è}:
                        \begin{align*}
                            \textcolor{Purple}{U=\frac{l}{2}*NK_BT}
                        \end{align*}
                        Dove $l$ rappresenta i \textbf{gradi di libertà} e vale:
                        \begin{itemize}
                            \item \textbf{gas mono-atomici}, la particella può muoversi su 3 assi: \textcolor{Red}{$l=3$};
                            \item \textbf{gas bi-atomici}, la particella è composta da 2 atomi, può muoversi su 3 assi e ruotare in modo "significativo" su 2 assi (una rotazione sull'asse che connette i 2 atomi non comporta nessun cambiamento significativo della particella): \textcolor{Red}{$l=3 + 2 = 5$};
                            \item \textbf{gas poli-atomici}, la particella è composta da più di 2 atomi, può muoversi su 3 assi e ruotare in modo "significativo" su 3 assi: \textcolor{Red}{$l=3 + 3 = 6$};
                        \end{itemize}
                        Negli esercizi verrà detto il tipo di gas: basta \textbf{sostituire la \textit{l}}!

        \subsection{Trasformazioni particolari dei gas ideali}
            Vediamo alcune trasformazioni dei gas ideali, questo vengono definite "particolari" perché \textbf{mantengono una delle variabili termodinamiche costanti}. Per rappresentare queste trasformazioni usiamo il \textbf{piano PV}:
            \framedImg{5}{L16-img012}
            Una cosa importante in queste trasformazioni è la \textbf{rappresentazione del lavoro}, abbiamo detto che:
            \begin{align*}
                W_{I\rightarrow F}= \int_I^Fp(V)dV
            \end{align*}
            Ovvero, il lavoro compiuto dal gas corrisponde all'integrale della pressione (che dipende dal Volume) rispetto al Volume dal punto iniziale $I$ al punto finale $F$. Graficamente sul piano corrisponde all'area di piano sotto la curva della nostra trasformazione. Non dimenticare come funzionano le integrali però:
            \doubleFramedImg{4}{L16-img013}{L16-img014}
            Se andiamo da $A$ verso $B$ (ovvero il Volume aumenta) allora \textbf{il segno dell'integrale sarà \textcolor{OliveGreen}{POSITIVO}}, altrimenti se andiamo da $B$ verso $A$ (ovvero il Volume diminuisce) allora \textbf{il segno dell'integrale sarà \textcolor{Red}{NEGATIVO}}.
            Finiamo questa introduzione ricordando \textbf{la prima legge della termodinamica}:
            \begin{align*}
                \Delta U=Q-W
            \end{align*}
            Ci tornerà molto utile nell'analisi delle trasformazioni.

            \subsubsection{Trasformazione isocora}
                \framedImg{40}{L17-img001}
                La \textbf{trasformazione isocora} è una trasformazione per cui il volume rimane costante. La variazione di energia interna è quindi legata solamente al calore che il sistema scambia con l'esterno.
                \begin{align*}
                    V = cost => dV = 0
                \end{align*}
                Osserviamo ora che anche la variazione di lavoro infinitesimale e la variazione di calore sono,
                \begin{align*}
                    &\delta W = p*\bcancel{dV} = 0\\
                    &\Delta W = 0\\
                    &\delta Q = nc_VdT\\
                    &\Delta Q = nc_V\Delta T
                \end{align*}
                Osserviamo che in questo caso (quando si parla di gas) non si una il calore specifico inteso come capacità termica ma introduciamo il \textbf{calore specifico molare} e quindi usiamo il numero di moli $n$ e non la massa $m$.\\
                Ora calcoliamo la variazione di energia interna,
                \begin{align*}
                    &dU=\delta Q - \bcancel{\delta W} = nc_VdT\\
                    &\Delta U = nc_V\Delta T
                \end{align*}
                \textcolor{Red}{NOTA IMPORTANTE!!!!!!!!!!!!!!!!!!!!!!!!!!!!!}\\
                Questa formulazione dell'energia interna vale anche per le altre trasformazioni, non solo per l'isocora.

            \subsubsection{Trasformazione isobara}
                \framedImg{40}{L17-img002}
                La \textbf{trasformazione isobara} è una trasformazione per cui la pressione rimane costante.
                \begin{align*}
                    p = cost => dp = 0
                \end{align*}
                Ora dall'\textbf{equazione di stato dei gas ideali} possiamo ricavarci la pressione,
                \begin{align*}
                    pV=nRT => p = \frac{nRT}{V} = cost
                \end{align*}
                ed otteniamo variazione di lavoro e variazione di calore,
                \begin{align*}
                    &\delta W=pdV = \frac{nRT}{V}dV\\
                    &\delta Q = nc_PdT
                \end{align*}
                e quindi variazione di energia interna,
                \begin{align*}
                    dU=\delta Q - \delta W=nc_PdT-pdV
                \end{align*}
                In conclusione abbiamo quindi che,
                \begin{align*}
                    &W=p\Delta V\\
                    &Q = nc_P\Delta T\\
                    &\Delta U=Q - W=nc_P\Delta T-p\Delta V=nc_V\Delta T
                \end{align*}

            \subsubsection{Relazione di Mayer}
                La \textbf{relazione di Mayer} è una relazione tra il calore specifico a pressione costante ed il calore specifico a volume costante per i gas ideali. Per calcolarla facciamo come segue.\\
                Per la trasformazione isocora avevamo ottenuto,
                \begin{align*}
                    dU=nc_VdT
                \end{align*}
                che valeva anche per le altre trasformazioni. Ora osserviamo che la variazione di energia interna può essere definita anche come,
                \begin{align*}
                    dU=\delta Q - \delta W
                \end{align*}
                e quindi da questa possiamo ricavare la variazione di calore,
                \begin{align*}
                    \delta Q =dU+\delta W
                \end{align*}
                e sostituendo all'energia interna il valore ottenuto per l'isocora e al lavoro la sua definizione, ottengo,
                \begin{align*}
                    \delta Q=nc_VdT + pdV
                \end{align*}
                Mettendo insieme quindi quest'ultima formula e il valore di $\delta Q$ ottenuto per l'isobara, ottengo che,
                \begin{align*}
                    nc_PdT=nc_VdT + pdV
                \end{align*}
                Ora posso osservare che derivando l'equazione di stato dei gas ideali, dato che siamo nell'isobara ho pressione costante, ed inoltre $n$ ed $R$ sono costanti e quindi ottengo,
                \begin{align*}
                    d[pV] = d[nRT] => pdV = nRdT
                \end{align*}
                Sostituendo questa alla formula precedente ottengo quindi,
                \begin{align*}
                    &nc_PdT=nc_VdT + nRdT && =>  \bcancel{n}c_P \bcancel{dT} =\bcancel{n}c_V \bcancel{dT} + \bcancel{n}R \bcancel{dT}\\
                    & && => \textcolor{Red}{c_P - c_V = R}
                \end{align*}

            \subsubsection{Calore specifico a volume costante}
                Dalla teoria cinetica dei gas abbiamo che l'energia interna è,
                \begin{align*}
                    U = N\frac{l}{2}k_BT
                \end{align*}
                e se considero il numero di moli invece che il numero di componenti diventa,
                \begin{align*}
                    U = n\frac{l}{2}RT
                \end{align*}
                questo perchè se moltiplico e divido per $N_A$ a destra, ottengo,
                \begin{align*}
                    U = \frac{N}{N_A}\frac{l}{2}k_BN_AT
                \end{align*}
                e $\frac{N}{N_A}=n$ e $k_B N_A=R$.\\
                Ora abbiamo anche dalla isocora che,
                \begin{align*}
                    U = nc_VT
                \end{align*}
                e quindi mettendole insieme ottengo,
                \begin{align*}
                    n\frac{l}{2}RT=nc_VT => c_V=\frac{l}{2}R
                \end{align*}
                Per i gas mono-atomici ho quindi che $c_V=\frac{3}{2}R$ e per i gas bi-atomici ho che $c_V=\frac{5}{2}R$.

            \subsubsection{Calore specifico a pressione costante}
                Per calcolare il calore specifico a pressione costante utilizzo la relazione di Mayer ed il calore specifico a volume costante appena calcolato,
                \begin{align*}
                    c_P = c_V + R = \frac{l}{2}R + R = \frac{l + 2}{2}R
                \end{align*}
                Per i gas mono-atomici ho quindi che $c_P=\frac{5}{2}R$ e per i gas bi-atomici ho che $c_P=\frac{7}{2}R$.

            \subsubsection{Trasformazione isoterma}
                \framedImg{40}{L17-img003}
                La \textbf{trasformazione isoterma} è una trasformazione per cui la temperatura rimane costante. Di conseguenza anche l'energia rimarrà costante.
                \begin{align*}
                    &T = cost => \Delta T = 0\\
                    &U = cost => dU = 0
                \end{align*}
                e quindi l'equazione di stato dei gas sarà,
                \begin{align*}
                    pV = nRT = cost
                \end{align*}
                Ora la variazione di energia è,
                \begin{align*}
                    dU = \delta Q - \delta W => \delta Q = \delta W
                \end{align*}
                Lo scambio di calore quindi coincide con lo scambio di lavoro.\\
                In questo caso non vale,
                \begin{align*}
                    \delta Q = nc_TdT
                \end{align*}
                ma il calore si calcola tramite il lavoro,
                \begin{align*}
                    \delta W = pdV = nRT\frac{dV}{V}
                \end{align*}
                Il lavoro ora sarà quindi,
                \begin{align*}
                    W=\int_i^f \delta W = \int_i^f nRT\frac{dV}{V} = nRT \int_i^f \frac{dV}{V} = nRTln\bigg(\frac{V_f}{V_i}\bigg)
                \end{align*}
                quindi il lavoro compiuto durante un'espansione/compressione isoterma è proporzionale ad $nRT$ e quindi alla temperatura in cui avviene la trasformazione.\\
                In conclusione ho quindi che,
                \begin{align*}
                    Q = W = nRTln\bigg(\frac{V_f}{V_i}\bigg)
                \end{align*}
                Se ho un'espansione allora calore e lavoro saranno positivi, se ho una compressione invece saranno negativi.

                \paragraph{Esempio espansione isoterma}
                \quad\\
                \nFramedImg{80}{L17-imgEsIsoterma}

            \subsubsection{Trasformazione adiabatica}
                \framedImg{40}{L17-img004}
                \nFramedImg{80}{L17-imgTrasAdiabatica1}
                \nFramedImg{80}{L17-imgTrasAdiabatica2}

        \subsection{Trasformazioni cicliche}
            %\framedImg{40}{L17-img005}
            Le \textbf{trasformazioni cicliche} sono trasformazioni dove il punto iniziale coincide con quello finale. La variazione di energia interna è 0.
            \begin{align*}
                \Delta U = 0 => Q = W \bcancel{=>}\delta Q = \delta W
            \end{align*}
            Osserviamo che però le variazioni di calore e lavoro infinitesimali non sono uguali.\\
            Il lavoro totale è quindi l'area del ciclo nel piano p,V, il lavoro ha segno che dipende dal senso di percorrenza.

            \subsubsection{Macchine termiche}
                Le \textbf{macchine termiche} sono macchine per cui il lavoro e quindi di conseguenza il calore è positivo. Quindi prende calore dall'esterno e lo trasforma in lavoro.
                \begin{align*}
                    W > 0 => Q > 0
                \end{align*}
                Il rendimento delle macchine termiche è il lavoro che la macchina compie diviso il calore che assorbe,
                \begin{align*}
                    \eta = \frac{W}{Q_A} = \frac{Q_A + Q_C}{Q_A} = 1 + \frac{Q_C}{Q_A} = 1 - \frac{|Q_C|}{Q_A}
                \end{align*}
                dato che il calore ceduto $Q_C < 0$.\\
                In pratica è impossibile che il rendimento sia $\eta=1$, perchè c'è sempre una piccola quantità di calore che cedo all'ambiente esterno.\\
                Per calcolare il rendimento devo mettere a numeratore ciò di cui voglio misurare l'efficenza e a denominatore ciò che devo pagare.

            \subsubsection{Macchine frigorifere}
                Le \textbf{macchine frigorifere} sono macchine per cui il lavoro e quindi di conseguenza il calore è negativo. Quindi viene portato via calore.
                \begin{align*}
                    W < 0 => Q < 0
                \end{align*}
                Il coefficiente di prestazione delle macchine frigorifere è il calore che assorbo diviso il lavoro che subisco.
                \begin{align*}
                    \eta = \frac{Q_A}{|W|}
                \end{align*}
                Per calcolare il coefficiente di prestazione devo mettere a numeratore ciò di cui voglio misurare l'efficenza e a denominatore ciò che devo pagare.

        \subsection{Ciclo di Carnot}
            \framedImg{40}{L17-img006}
            Il \textbf{ciclo di Carnot} è un ciclo reversibile, quindi composto da trasformazioni reversibili. È composto infatti da una espansione isoterma, da A a B, e da una adiabatica, da B a C, e da una compressione isoterma, da C a D, e da una adiabatica, da D ad A, tutte reversibili. Per l'espansione isoterma ho $T_2$ e per la compressione isoterma ho $T_1$.\\
            Posso rappresentare il ciclo di Carnot come una macchina come segue,
            %\framedImg{60}{L17-img007}
            Per i vari tratti (trasformazioni) ho quindi,
            \begin{center}
                \begin{tabular}{|cr|cr|cr|cr|}
                    \hline
                    & \quad & $\Delta U$ & $Q$ & $W$\tabularnewline
                    \hline
                    & AB & $0$ & $-nRT2ln\bigg(\frac{V_B}{V_A}\bigg)$ & $nRT2ln\bigg(\frac{V_B}{V_A}\bigg)$\tabularnewline
                    \hline
                    & BC & $mc_V(T_1 - T_2)$ & $0$ & $-mc_V(T_1 - T_2)$\tabularnewline
                    \hline
                    & CD & $0$ & $-nRT1ln\bigg(\frac{V_D}{V_C}\bigg)$ & $nRT1ln\bigg(\frac{V_D}{V_C}\bigg)$\tabularnewline
                    \hline
                    & DA & $mc_V(T_2 - T_1)$ & $0$ & $-mc_V(T_2 - T_1)$\tabularnewline
                    \hline
                \end{tabular}
            \end{center}
            Ora calcolo il rendimento,
            \begin{align*}
                &W = nRT2ln\bigg(\frac{V_B}{V_A}\bigg) + nRT1ln\bigg(\frac{V_D}{V_C}\bigg) \bcancel{- mc_V(T_1 - T_2)} \bcancel{- mc_V(T_2 - T_1)}\\
                &Q_A = nRT2ln\bigg(\frac{V_B}{V_A}\bigg)\\
                &\eta = \frac{nRT2ln\bigg(\frac{V_B}{V_A}\bigg) + nRT1ln\bigg(\frac{V_D}{V_C}\bigg)}{nRT2ln\bigg(\frac{V_B}{V_A}\bigg)}=1 + \frac{ T1ln\bigg(\frac{V_D}{V_C}\bigg)}{T2ln\bigg(\frac{V_B}{V_A}\bigg)} = 1 + \frac{T1}{T2}\frac{ln\bigg(\frac{V_D}{V_C}\bigg)}{ln\bigg(\frac{V_B}{V_A}\bigg)} = 1 + \frac{T1}{T2}
            \end{align*}
            Se ho quindi $T_1$ molto bassa e $T_2$ molto alta, il rendimento è $\eta\simeq 1$, quindi ho rendimento massimo.

        \subsection{Secondo principio della termodinamica}
            Possiamo dividere il secondo principio della termodinamica in 2 parti (+ 1):
            \begin{enumerate}
                \item \textbf{principio di Kelvin}: \textit{è impossibile realizzare una macchina termica ciclica il cui \textbf{unico} risultato sia la conversione in lavoro di tutto il calore assorbito da una sorgente omogenea};
                \item \textbf{principio di Clausius}: \textit{è impossibile realizzare una trasformazione il cui \textbf{unico} risultato sia quello di trasferire calore da un corpo più freddo a uno più caldo senza l'apporto di lavoro esterno};
                \item \textit{è impossibile realizzare una macchina termica il cui rendimento sia pari al 100\%} (questa parte non viene citata dal professore, la aggiungo io per completezza [trovata su Wikipedia], da questo punto in poi lo ignorerò).
            \end{enumerate}
            Questi, in quanto principi, \textbf{non vanno dimostrati}! Però, è possibile (con una dimostrazione per assurdo) dimostrare che entrambi i principi \textbf{sono equivalenti}.

            \subsubsection{Veloce recap sulle macchine}
                Prima di passare alla dimostrazione per assurdo precedentemente citata, facciamo un veloce ripasso per quanto riguarda le macchine termiche e le macchine frigorifere. In entrambi i casi, nota che la temperatura $T_2$ è sempre \textbf{maggiore} della temperatura $T_1$
                \paragraph{Macchine termiche}
                    Definiamo una \textbf{macchina termica} una macchina che ha come scopo \textbf{produrre lavoro} ($W$) tramite \textbf{l'assorbimento di calore} ($Q_A$). Come visto in precedenza, questo NON è un procedimento efficiente al 100\% (tutto il \textit{calore assorbito} ($Q_A$) non verrà mai convertito 1:1 in lavoro), ma ci sarà anche del \textit{calore ceduto} ($Q_C$). In particolare, il rendimento della nostra macchina corrisponderà a:
                    \begin{align*}
                        \eta = \frac{W}{Q_A} = \frac{Q_A-|Q_C|}{Q_A} =  \frac{Q_A}{Q_A}- \frac{Q_C|}{Q_A} = 1- \frac{|Q_C|}{Q_A}
                    \end{align*}
                    Vediamo un rapido esempio:
                    \framedImg{7}{L18-img001}
                    La macchina ($M$) assorbe calore ($Q_A$, segno positivo dato che lo assorbe) e produce un certo lavoro in uscita ($W$, segno positivo perché lo produce). In contemporanea però, cede del calore ($Q_C$, segno negativo perché lo perde). Ora, seguendo la formuletta vista prima, possiamo dire che $M$ \textbf{lavora con un rendimento del 30\%}, però possiamo anche andare oltre! Supponendo che questa sia una \textbf{macchina di Carnot}, possiamo anche dire il \textbf{rapporto tra le temperature in gioco}! In particolare, avevamo detto che il rendimento di una macchina di Carnot corrisponde a:
                    \begin{align*}
                        \eta=1-\frac{T_1}{T_2}
                    \end{align*}
                    Quindi possiamo dire che:
                    \begin{align*}
                        \eta=1-\frac{T_1}{T_2}=30\%=\frac{3}{10} => \frac{T_1}{T_2}= \frac{7}{10}
                    \end{align*}
                    Data una delle 2 temperature possiamo dire quale deve essere l'altra temperatura per lavorare con un rendimento del 30\%!

                \paragraph{Macchine frigorifere}
                    Definiamo una \textbf{macchina frigorifera} una macchina che ha come scopo \textbf{assorbire calore} ($Q_A$) tramite \textbf{l'assorbimento di lavoro} ($W$). In questo caso, il rendimento di questa macchina dipende dal rappotro tra calore assorbito e quantità di lavoro impiegata per assorbire tale calore. La formula sarà quindi:
                    \begin{align*}
                        \eta = \frac{Q_A}{|W|}
                    \end{align*}
                    Ricorda che nelle macchine frigorifere, il lavoro viene "rubato" a noi, quindi (nel nostro sistema di riferimento) avrà segno negativo  (da qui il modulo). Ricorda inoltre che $W = Q_A-|Q_C|$ (ancora, nel nostro sistema di riferimento il calore ceduto è negativo, quindi mettiamo il modulo). Vediamo ora un rapido esempio:
                    \framedImg{7}{L18-img002}
                    La macchina ($M$) assorbe calore ($Q_A$, segno positivo dato che lo assorbe)  da una zona di \textbf{bassa temperatura} ($T_1$) utilizzando una certa quantità di lavoro in entrata ($W$, segno negativo perché lo "ruba" a noi). In contemporanea cede del calore ($Q_C$, segno negativo perché lo perde) ad una zona ad \textbf{alta temperatura} ($T_2$).

            \subsubsection{Dimostrazione dell'uguaglianza delle 2 principi}
                I 2 princip (di Kelvin e Clausius) sono equivalenti, ciò è facilmente dimostrabile con una dimostrazione per assurdo, vediamo entrambi i casi.
                \paragraph{Kelvin => Clausius}
                    Immaginiamo di avere una macchina "$\bcancel{K}$" capace di violare il principio di Kelvin: se tale macchina esistesse, allora sarebbe anche possibile violare il principio di Clausius! Quindi, immaginiamo di avere questa macchina "$\bcancel{K}$", capace di \textbf{assorbire una quantità $Q_2$} di calore (da una sorgente a temperatura $T_2$) e convertire tutto questo calore un un \textbf{lavoro $W$} \textbf{senza dispersione}:
                    \framedImg{2}{L18-img003}
                    Ora, possiamo affiancarci una normalissima macchina che compie una \textbf{trasformazione isoterma}. Questa macchina \textbf{assorbirà tutto il lavoro} prodotto da $\bcancel{K}$ e \textbf{restituirà una quantità $Q_2$} di calore ad una sorgente a temperatura $T_1$. Visto che possiamo, supponiamo anche che $\mathbf{T_1>T_2}$ (molto importante):
                    \framedImg{4}{L18-img004}
                    A questo punto possiamo "\textbf{compattare}" \textcolor{Red}{queste macchine} in una \textbf{macchina unica} che andremmo a chiamare $\bcancel{C}$:
                    \framedImg{6}{L18-img005}
                    Se andiamo ad analizzare la macchina $\bcancel{C}$ notiamo che \textbf{viola il principio di Clausius}! Infatti è capace di \textbf{trasferire calore dalla sorgente a temperatura $T_2$ a quella a temperatura $T_1$} e , come abbiamo detto prima, $\mathbf{T_1>T_2}$: in questa macchina il calore passa da una zona fredda ad una zona più calda "naturalmente", senza l'assorbimento di lavoro esterno (il lavoro viene generato tutto internamente alla macchina $\bcancel{C}$, quindi quello possiamo ignorarlo)!

                \paragraph{Clausius => Kelvin}
                    In un modo simile, possiamo dimostrare che l'esistenza di una macchina capace di violare il principio di Clausius implicherebbe l'esistenza di una macchina capace di violare il principio di Kelvin. Come prima, immaginiamo di avere questa macchina "$\bcancel{C}$", capace di \textbf{assorbire una quantità $Q$} di calore da una sorgente a temperatura $T_1$ e \textbf{cedere la stessa quantità di calore} (quindi $-Q$) ad una sorgente a temperatura $T_2$ (vale sempre $\mathbf{T_1>T_2}$):
                    \framedImg{2}{L18-img006}
                    Ora, a questa macchina impossibile ne affianchiamo una normalissima che assorbe un calore $Q^*$ dalla sorgente $T_2$ e produce un certo lavoro $W$. La parte importante qui è il \textbf{calore ceduto}: supponimao che \textbf{corrisponda esattamente a $-Q$} ($Q^*$ può essere arbitrariamente grande, quindi gestiamo il rendimento della macchina e l'assorbimento di calore in modo da cederne quanto vogliamo noi):
                    \framedImg{4}{L18-img007}
                    E adesso comprimiamo \textcolor{Red}{queste macchine}: dato che \textbf{assorbiamo e cediamo una quantità di calore Q dalla sorgente $T_1$} è come se \textbf{non assorbissimo effettivamente calore}! Quindi otteniamo una \textbf{macchina} (che possiamo chiamare $\bcancel{K}$) \textbf{capace di violare il principio di Kelvin}!
                    \framedImg{6}{L18-img008}

        \subsection{Teorema di Carnot}
            Questo teorema nasce da una domanda molto semplice: "\textit{\textbf{a parità di temperature, e più efficiente una macchina termica reversibile o una irreversibile?}}". Nota MOLTO bene quel "\textbf{a parità di temperature}": durante l'orale è importante citarlo! Per la dimostrazione ci troviamo in questa situazione:
            \framedImg{6}{L18-img009}
            \begin{enumerate}
                \item abbiamo 2 macchine termiche:
                \begin{itemize}
                    \item una \textbf{macchina generica} $X$ di cui non conosciamo le specifiche, in particolare \textbf{non sappiamo se è reversibile} (può diventare una macchina frigorifera) \textbf{o irreversibile};
                    \item una \textbf{macchina NON generica} $R$ che \textbf{sappiamo essere reversibile}
                \end{itemize}
                \item abbiamo 2 sorgenti di temperatura:
                \begin{itemize}
                    \item una alla temperatura $T_2$;
                    \item una alla temperatura $T_1$ che sappiamo essere \textbf{più fredda} dell'altra ($T_1<T_2$);
                \end{itemize}
                \item entrambe le macchine \textbf{assorbono calore da $\mathbf{T_2}$} e \textbf{cedono calore a $\mathbf{T_1}$}, nota che \textbf{non è detto che le 2 macchine assorbano/cedano le stesse quantità di calore};
                \item entrambe le macchine \textbf{producono la stessa quantità di lavoro $\mathbf{W}$}.
            \end{enumerate}

            Ora, noi ci chiediamo:
            \begin{align*}
                \eta_{X, (T_1, T_2)} >=< \eta_{R, (T_1, T_2)}
            \end{align*}
            ovvero, \textbf{come si relazionano i rendimenti delle 2 macchine tra di loro}? Il teorema di Carnot risponde proprio a questo quesito: \textcolor{Red}{\textbf{il rendimento di una macchina termica generica X è SEMPRE minore o uguale al rendimento di una macchina termica reversibile}} (l'= vale quando entrambe le macchine sono reversibili). Come arriviamo a questa conclusione? Ci arriviamo tramite delle assunzioni, in particolare assumiamo (per assurdo) che :
            \begin{align*}
                \eta_{X, (T_1, T_2)} > \eta_{R, (T_1, T_2)}
            \end{align*}
            Ok, dato che la macchina R è reversibile, possiamo farne la \textbf{versione "negata" $\overline{R}$}:
            \framedImg{3}{L18-img010}
            e dato che assorbe proprio una \textbf{quantità $\mathbf{-W}$ di lavoro} la possiamo \textbf{connettere in serie alla macchina generica X}:
            \framedImg{4}{L18-img011}
            Possiamo quindi immaginare di \textbf{comprimere le 2 macchine} (annullando di fatto l'apporto del lavoro, è tutta roba interna alla macchine quindi noi non perdiamo niente), ottenendo la macchina $\mathbf{M_1}$:
            \framedImg{4}{L18-img012}
            Ora, ci sono dei trasferimenti di energia tra $T_1<->M_1$ e $T_2<->M_1$, però in quale verso? Prima abbiamo supposto "$\eta_{X, (T_1, T_2)} > \eta_{R, (T_1, T_2)}$", quindi possiamo ragionarci sopra:
            \begin{align*}
                &\eta_{X, (T_1, T_2)} > \eta_{R, (T_1, T_2)} &&=> \frac{\bcancel{W}}{Q_2} > \frac{\bcancel{W}}{Q_2'}\\
                & && =>\frac{1}{Q_2} > \frac{1}{Q_2'}\\
                & && =>Q_2' > Q_2\\
            \end{align*}
            Dato il lavoro prodotto dalle macchine e i calori che assorbono possiamo facilmente calcolare il loro rendimento, dato che sappiamo (o meglio, abbiamo supposto) che il rendimento di $X$ è maggiore di quello di $R$ possiamo applicare dei semplici passaggi algebrici e scoprire che \textbf{$\mathbf{Q_2'}$ deve essere maggiore di $\mathbf{Q_2}$}. Fai attenzione qui: se il calore è $> 0$ allora \textbf{la macchina lo sta assorbendo}, se è $< 0$ allora \textbf{lo sta cedendo}. Dato che non stiamo producendo lavoro, \textbf{i 2 calori devono essere uguali e contrari}, cioè \textbf{stiamo trasferendo calore da una sorgente ad un'altra}. Facciamo un disegno:
            \framedImg{4}{L18-img013}
            Qui ci accorgiamo però che \textbf{c'è un problema}: abbiamo detto prima che $T_1<T_2$, quindi \textbf{$\mathbf{T_1}$ è più fredda di $\mathbf{T_2}$}! Stiamo \textbf{trasferendo calore da una fonte fredda verso una fonte calda}, ma questo è \textbf{impossibile per il principio di Clausius}! Non possiamo trasferire calore da una fonte fredda verso una fonte calda senza l'apporto di lavoro esterno, possiamo quindi dire che:
            \begin{align*}
                \textcolor{Red}{\bcancel{\eta_{X, (T_1, T_2)} > \eta_{R, (T_1, T_2)}}}
            \end{align*}
            Non si può fare, è impossibile. \textbf{Quindi deve essere}:
            \begin{align*}
                \mathbf{\eta_{X, (T_1, T_2)} <= \eta_{R, (T_1, T_2)}}
            \end{align*}
            Ora, possiamo considerare un altro caso: e se \textbf{anche la macchina $\mathbf{X}$ fosse reversibile}? Ricorda che è una macchina generica, potrebbe essere reversibile come no: nel caso fosse reversibile potremmo invertire $X$ e fare \textbf{gli stessi identici passaggi fatti prima}. Da questi otteremo che:
            \begin{align*}
                \mathbf{\eta_{X, (T_1, T_2)} => \eta_{R, (T_1, T_2)}}
            \end{align*}
            Se ora mettiamo insieme quanto abbiamo scoperto, otteniamo che:
            \begin{align*}
                \begin{rcases*}
                    \eta_{X, (T_1, T_2)} <= \eta_{R, (T_1, T_2)} \\
                    \eta_{X, (T_1, T_2)} => \eta_{R, (T_1, T_2)}
                \end{rcases*}
                \textcolor{Red}{\mathbf{\eta_{X, (T_1, T_2)} = \eta_{R, (T_1, T_2)}}}
            \end{align*}
            \textbf{Se le 2 macchine sono entrambe reversibili} (e lavorano nello stesso range di temperatura) \textbf{devono avere lo stesso rendimento}! Sostanziamlmente, questa è la legge di Carnot:
            \begin{align}
                \textcolor{Red}{\eta_{X, (T_1, T_2)} <= \eta_{R, (T_1, T_2)}}
            \end{align}
            \textcolor{Red}{\textbf{Il rendimento di una macchina termica qualsiasi è, sempre minore o ugugale del rendimento di una qualsiasi macchina termica reversibile che lavora nello stesso range di temperatura}} (se anche la prima macchina è reversibile, le 2 macchine DEVONO avere lo stesso rendimento).

            \subsubsection{Corollario e osservazioni}
                A questo punto possiamo mettere insieme un corollario:
                \begin{enumerate}
                    \item \textbf{corollario}: come già detto in precendenza, \textbf{tutte le macchine reversibili che lavorano alle stesse temperature hanno lo stesso rendimento};
                    \item \textbf{osservazione 1}: abbiamo detto che \textbf{tutte le macchine reversibili DEVONO avere la stessa efficienza} (a parità di temperatura), dato che la \textbf{macchina di carnot è reversibile} ed ha un rendimento che corrisponde a "$1-\frac{T_1}{T_2}$" (ricorda che $T_1<T_2$) possiamo dire che \textbf{tutte le macchine termiche reversibili hanno rendimento \textcolor{Red}{$\-\frac{T_1}{T_2}$}};
                    \item \textbf{osservazione 2}: supponendo che la macchina $X$ \textbf{ceda il calore $Q_1$ a $T_1$ ed assorba $Q_2$ da $T_2$}, possiamo dire che il suo rendimento corrisponde a $1-\frac{|Q_1|}{Q_2}$ (ricorda che $Q_1$ è negativo perché è il calore ceduto dalla macchina). Ora, per il teorema di Carnot, possiamo dire che:
                    \begin{align*}
                        &\eta_{X, (T_1, T_2)} <= \eta_{R, (T_1, T_2)}&&===>1-\frac{|Q_{1, X}|}{Q_{2, X}} <= 1-\frac{|Q_{1, R}|}{Q_{2, R}}\\
                        & Per\ osservazione\ precedente\ (R\ reversibile) &&===>1-\frac{|Q_{1, X}|}{Q_{2, X}} <= 1-\frac{T_1}{T_2}\\
                        & Tolgo\ _X\ per\ semplicita && ===>\bcancel{1}-\frac{|Q_1|}{T_1} <= \bcancel{1}-\frac{Q_2}{T_2}\\
                        & Q_1\ ==\ -|Q_1|&& ===>+\frac{Q_1}{T_1} <= -\frac{Q_2}{T_2}\\
                        & && ===>\textcolor{Red}{\frac{Q_1}{T_1}+ \frac{Q_2}{T_2} <= 0}\\
                    \end{align*}
                    \textcolor{Red}{Questa} è una formulazione alternativa del teorema di Carnot che ci tornerà molto utile in futuro;
                    \item \textbf{osservazione 3}: dato che, all'\textit{osservazione 2}, abbiamo stabilito che "$\frac{|Q_{1, X}|}{Q_{2, X}} <= 1-\frac{T_1}{T_2}$". Bene, possiamo quindi dire che il \textbf{massimo rendimento possibile per qualsiasi macchina} che lavora nel range di temperatura \textbf{[$\mathbf{T_1}$, $\mathbf{T_2}$]} corrispone esattamente a:
                    \begin{align*}
                        \eta_{MAX, (T_1, T_2)} = 1-\frac{T_1}{T_2}
                    \end{align*}
                    Questa è una cosa abbastanza sorprendente: a livello teorico siamo riusciti a trovare un limite effettivamente impossibile da superare nella realtà! Supponendo di avere $T_1 = 283K$ e $T_2 = 300K$, possiamo subito dire che qualsiasi macchina che lavora in questo range di temperature avrà di sicuro un rendimento che non supera $1-\frac{283}{300} = \frac{17}{300}$ (piuttosto basso come rendimento): questo è il trucco! Più è piccolo è il divario delle temperature e minore sarà il rendimento! Questa è un limite teorico, dimostrabile con i teoremi!
                    \item \textbf{osservazione 4}: facendo riferimento all'\textit{osservazione 2}, se abbiamo 2 macchine reversibili possiamo dire che: $\frac{Q_1}{T_1}+ \frac{Q_2}{T_2} \textcolor{Red}{=} 0$. Se esplicitiamo i segni dei 2 calori (usando i moduli), otteniamo: $\frac{-|Q_1|}{T_1}+ \frac{|Q_2|}{T_2} = 0$. Questo "cambio di segni" ci permette di ottenere questa equazione:
                    \begin{align*}
                        \frac{|Q_1|}{T_1} = \frac{|Q_2|}{T_2}
                    \end{align*}
                    Questa equazione ci permette di stabilire una \textbf{relazione tra calore trasferito $\mathbf{Q}$ e la temperatura $\mathbf{T}$}: questo rapporto ci permette di creare degli strumenti di misurazione della temperatura assoluta. Misurando lo scambio del calore, possiamo misurare la temperatura!
                \end{enumerate}
        \subsection{Teorema di Clausius}
            Il teorema di Clausius è un'estensione del teorema di Carnot su \textbf{cicli generici}. Allora, partiamo dalla formulazione di Carnot vista nel corollario:
            \begin{align*}
                \frac{Q_1}{T_1}+ \frac{Q_2}{T_2} <= 0
            \end{align*}
            In questa formula notiamo facilmente che il ciclo in questione \textbf{lavora tra 2 temperature fisse} (ha 2 scambi di calore):
            \framedImg{4}{L19-img001}
            Ma se avessimo un "ciclo termodinamico generico" che lavora \textbf{a contatto con più temperature diverse}? Ad esempio:
            \framedImg{4}{L19-img002}
            Possiamo immaginare di prendere questo ciclo e \textbf{dividerlo in piccoli cicli di carnot}, ottenendo qualcosa del genere:
            \framedImg{4}{L19-img003}
            Questa è un'apporssiamzione molto grossolana: devi immaginarla con tratti infinitesimi, prossimi al ciclo generico vero e proprio (se abbiamo N tratti, allora N dovrebbe tendere ad $\infty$).  Immaginiamo di ingrandire un tratto di questa approssimazione e disegnami i cicli di carnot:
            \framedImg{5}{L19-img004}
            Che info possiamo estrarre da ogni ciclo di carnot? Per il ciclo i-esimo possiamo dire che:
            \begin{align*}
                \frac{Q_1}{T_1}+ \frac{Q_2}{T_2} <= 0
            \end{align*}
            (L'= vale solo se il ciclo è reversibile). Ora, se immaginiamo che i cicli interni siano tutti reversibili, abbiamo che i loro \textbf{tratti adiabatici non scambiano calore} mentre \textbf{quelli isotermi si!} In particolare, in un tratto \textbf{assorbono calore} (da dx a sx), mentre in un tratto \textbf{cedono calore} (da sx a dx). Ora un passaggio importante: dal disegno precedente si può notare che ogni tratto interno al ciclo (quelli tratteggiati) viene condiviso tra 2 diversi cicli. Uno \textbf{cederà calore}, mentre l'altro \textbf{lo assorbità}: avremmo la stessa quantità $\frac{Q}{T}$ \textbf{2 volte ma di segno opposto}! Se sommiamo tutte queste quantità, \textbf{quelle interne si bilanciano tra di loro} e \textbf{resteremmo solo con i contributi dei tratti esterni}! Quindi, se facciamo:
            \begin{align*}
                \sum_{MicroCicli}(\frac{Q_1}{T_1}+\frac{Q_2}{T_2}) \leq 0
            \end{align*}
            Se poi espandessimo questa sommatoria, potremmo notare che i termini interni si semplificherebbero tra di loro, \textbf{lasciando solo le quantità relative ai tratti esterni}. Quindi, possiamo sommare i contributi di tutti i tratti esterni (supponiamo N) ottenendo:
            \begin{align*}
                \sum_{j=1}^N\frac{Q_{j}}{T_{j}} \leq 0
            \end{align*}
            Ovvero \textit{\textcolor{Red}{la somma dei rapporti tra i calori scambiati e le rispettive temperature delle sorgenti è negativa (macchine irriversibili) o uguale a 0 (macchine reversibili)}}. Sostanzialmente questo è il teorema di Clausius. Esiste anche la sua versione "continua", ovvero l'integrale:
            \begin{align*}
                \oint\frac{dQ}{T} \leq 0
            \end{align*}
            Anche in questo caso, il "<" vale per le macchine (i cicli) irreversibili, mentre l'"=" per quelli reversibili. Questa poi è la base per l'\textbf{\textit{integrale di Clausius}}:
            \begin{align*}
                \oint\frac{dQ}{T}
            \end{align*}
            Fondamentale per andare poi a definire quella che è l'\textbf{entropia}.

        \subsection{Entropia}
            Supponiamo di cominciare con una \textbf{trasformazione ciclica "divisia" in 2 parti}, ad esempio:
            \framedImg{4}{L19-img005}
            Possiamo qundi individuare 2 tratti:
            \begin{itemize}
                \item $A\rightarrow B$: lo chiamiamo tratto $I$;
                \item $B\rightarrow A$: lo chiamiamo tratto $II$;
            \end{itemize}
            Ora possiamo dire che, per il teorema di Clausius visto prima, abbiamo:
            \begin{align*}
                \oint\frac{dQ}{T} \leq 0
            \end{align*}
            In particolare per i cicli reversibili:
            \begin{align*}
                \oint\frac{dQ}{T} = 0
            \end{align*}
            Concentriamoci su quest'ultimi per il momento. Per le proprietà degli integrali possiamo dire che:
            \begin{align*}
                &\oint\frac{dQ}{T} = 0 &&=> \oint_{A}^B(\frac{dQ}{T})_{(I)}+\oint_{B}^A(\frac{dQ}{T})_{(II)} = 0\\
                & Applico\ reversibilità\ integrali&&=> \oint_{A}^B(\frac{dQ}{T})_{(I)}\textcolor{Orange}{-\oint_{A}^B}(\frac{dQ}{T})_{(II)} = 0\\
                & &&=> \oint_{A}^B(\frac{dQ}{T})_{(I)} = \oint_{A}^B(\frac{dQ}{T})_{(II)}\\
            \end{align*}
            Tutto questo \textbf{indipendentemente dal percorso compiuto dal tratto}! Volendo avrei potuto prendere dei percorsi completamente diversi, tipo:
            \framedImg{6}{L19-img006}
            ed avrei \textbf{ottenuto la stessa identica formula}. Morale della favola: $\oint_{A}^B(\frac{dQ}{T})$ \textbf{NON dipende dal percorso che scelgo}. Dato che il percorso non ci interessa possiamo dire che il risultato di $\oint_{A}^B(\frac{dQ}{T})$ è un \textbf{integrando perfetto}, quindi lo possiamo "risolvere" in questo modo:
            \begin{align*}
                \oint_{A}^B(\frac{dQ}{T}) \triangleq S(B)-S(A)
            \end{align*}
            [Small nota di servizio: "$\triangleq$" è "uguale per definizione"] Dove $S$ corrisponde ad un nuova funzione di stato chiamata \textbf{entropia}. É anche possibile avere la "versione infinitesima" della variazione di entropia, ovvero $\frac{dQ}{T} = dS$ (praticamente abbiamo solo tolto l'integrale, molto probabilmente è solo una notazione).


            \subsubsection{Variazione di entropia $\Delta S$}
                Ora un punto molto importante, possiamo dire che:
                \begin{align*}
                    \Delta S_{A\rightarrow B}=\oint_{A}^B(\frac{dQ}{T})_{\textcolor{Red}{perRevQualsiasi}}
                \end{align*}
                Concentrati su quel "\textcolor{Red}{per una reversibile qualsiasi}" (perRevQualsiasi): \textbf{la variazione di entropia, per una trasformazione qualsiasi} (sia reversibile che non)\textbf{, dallo stato termodinamico A allo stato termodinamico B dipende soltanto dalla posizione di questi 2 punti, ed è completamente \underline{indipendendte dal percorso} che li congiunge}!\bigskip\\
                Calcolare la variazione di entropia è molto semplici: dato che non ci interessa il percorso, ci basta prendere \textbf{il punto iniziale ed il punto finale, scegliere la trasformazione REVERSIBILE più semplice tra questi 2 punti e calcolare l'integrale} $\oint_{A}^B(\frac{dQ}{T})$.

            \subsubsection{Alcune osservazioni}
                Facciamo alcune osservazioni sull'entropia:
                \begin{enumerate}
                    \item l'entropia è una \textbf{quantità additiva}, ovvero: se abbiamo 2 sistemi (con relative entropie $S_1$ ed $S_2$) e creiamo un terzo sistema dall'\textbf{unione dei 2 sistemi precedenti}, l'entropia del terzo sistema corrisponderà a $S_3 = S_1 + S_2$;
                    \item l'entropia è una \textbf{grandezza fisica estensiva}, ovvero \textbf{dipende dalle dimensioni del sistema}, in particolare dal calore scambiato col sistema esterno (a sua volta il calore è anche un grandezza fisica estensiva);
                    \item per il calcolo della $\Delta S$ \textbf{occore scegliere la trasformazione termodinamica REVERSIBILE più conveniente}, ad esempio supponiamo di avere questa trasformazione irreversibile.
                    \framedImg{4}{L19-img007}
                    Dato che la trasformazione fornitn è IRREVERSIBILE, non possiamo calcolarti $\oint_{A}^B(\frac{dQ}{T})_{IRR}$, questo perché $\oint_{A}^B(\frac{dQ}{T})_{IRR}\neq S_B-S_A$ (e anche perché calcolalo tu questo integrale :P). Supponendo che (il testo dell'esercizio ci dica che) $T_A = T_B$, allora possiamo \textbf{collegare i 2 stati termodinamici con una ISOTERMA REVERSIBILE}. Per quel caso specifico (dell'isoterma reversibile) \textbf{noi SAPPIAMO} che:
                    \begin{align*}
                        &dQ = \frac{n*R*T}{V}dV&&=>\frac{dQ}{T}=n*R*\frac{dV}{V}&&=> \int_A^B(\frac{dQ}{T}) = \int_A^Bn*R*\frac{dV}{V}
                    \end{align*}
                    A questo punto risolviamo semplicemente l'integrale, ed otteniamo:
                    \begin{align*}
                        &\int_A^B(\frac{dQ}{T}) &&= \int_A^Bn*R*\frac{dV}{V}n*R*\int_A^B\frac{dV}{V}\\
                         & && = n*R*(ln(V_f)-ln(V_i))\\
                         & && = \textcolor{Red}{n*R*(ln(\frac{V_f}{V_i}))}\\
                    \end{align*}
                    \textcolor{Red}{Questa} è la nostra \textbf{variazione di entropia}!
                \end{enumerate}

            \subsubsection{Variazione di entropia per le trasformazioni termodinamiche notevoli}
                Vediamo la formulazione della variazione di entropia per le principali trasformazioni termodinamiche:
                \paragraph{Trasformazione isoterma}
                    Come abbiamo già visto prima, abbiamo che $V_i \rightarrow V_f$ e sappiamo che $dQ = \frac{n*R*T}{V}dV$, quindi mettiamo insieme l'integrale e risolviamo:
                    \begin{align*}
                        &\int_A^B(\frac{dQ}{T}) &&= \int_A^Bn*R*\frac{dV}{V}n*R*\int_A^B\frac{dV}{V}\\
                         & && = n*R*(ln(V_f)-ln(V_i))\\
                         & && = n*R*ln(\frac{V_f}{V_i})\\
                    \end{align*}
                    Quindi possiamo dire che per le isoterme \textcolor{OliveGreen}{$\mathbf{\Delta S = n*R*(ln(\frac{V_f}{V_i}))}$}.
                \paragraph{Trasformazione isocora}
                    In questo caso abbiamo $dV = 0$, di conseguenza anche $W=0$. Per definizione, abbiamo quindi che $dQ = dU = nc_vdT$, di conseguenza la variazione infinitesima di entropia corrisponderà a $\frac{dQ}{T} = nC_v\frac{dT}{T}$. Facciamo l'integrale e, similmente a prima, finiamo con: \textcolor{OliveGreen}{$\mathbf{\Delta S = n*R*ln(\frac{T_f}{T_i})}$}.
                \paragraph{Trasformazione isobara}
                    Qui abbiamo che $p = const$, allora $dQ = nc_pdT$, quindi $\frac{dQ}{T} = nc_p\frac{dT}{T} = dS$, svolgendo l'integrale abbiamo: \textcolor{OliveGreen}{$\mathbf{\Delta S = \int_i^fdS = n*c_p*ln(\frac{T_f}{T_i})}$}.
                \paragraph{Trasformazione adiabatica}
                    Sappiamo che il $dQ = 0$ (nelle adiabatiche non c'è scambio di temperatura), di conseguenza $\int_A^B(\frac{dQ}{T})\triangleq 0$ (per definizione degli integrali);

                \paragraph{Cambi di fase (non necessariamente gas ideali)}
                    Allora, sappiamo che nei cambi di fase $dQ = \lambda*dm$ (ricorda che $\lambda$ corrisponde al calore latente, mentre $dm$ corrisponde alla massa che cambia di fase). Ora basta applicare tutti i passaggi visti prima ed otteniamo che $\frac{dQ}{T} =\frac{\lambda}{T}*dm = dS$ e fare l'integrale per ottenere: \textcolor{OliveGreen}{$\mathbf{\Delta S = \int dS =\frac{\lambda*m}{T}}$}

            \subsubsection{Diagrammi T-S}
                Tutte le trasformazion che abbiam visto prima le possiamo anche rappresentare sui \textbf{diagrammi T-S}, che mettono in rapporto la temperatura con l'entropia:
                \framedImg{8}{L19-img008}

            \subsubsection{Teorema dell'entropia (formulazione matematica del secondo principio della termodinamica)}
                Supponiamo di avere una trasformazione ciclica divisa in 2 tratti: uno IRREVERSIBILE (I) ed uno REVERSIBILE (II).
                \framedImg{4}{L19-img009}
                Ora, per il \textbf{teorema di Clausius} abbiamo che:
                \begin{align*}
                    \oint \frac{dQ}{T}<0
                \end{align*}
                Dovrebbe essere "$<=$", ma dato che la trasformazione comprende un tratto irreversibile va considerata tutta irreversibile! Possiamo quindi togliere l'"$=$" (che vale solo nel caso reversibile). Adesso facciamo un po' di passaggi algebrici:
                \begin{align*}
                    & \oint \frac{dQ}{T}<0 &&=>\int_A^B (\frac{dQ}{T})_{IRR} + \int_B^A (\frac{dQ}{T})_{REV}<0\\
                    & Invertiamo\ la\ reversibile &&=>\int_A^B (\frac{dQ}{T})_{IRR} - \int_A^B (\frac{dQ}{T})_{REV}<0\\
                    & &&=>\int_A^B (\frac{dQ}{T})_{IRR} < \int_A^B (\frac{dQ}{T})_{REV}\\
                    & Giriamo\ l'equazione\ per\ comodita &&=>\textcolor{Orange}{\int_A^B (\frac{dQ}{T})_{REV} > \int_A^B (\frac{dQ}{T})_{IRR}}\\
                \end{align*}
                Teniamo a mente \textcolor{Orange}{questo} pezzo, ci torna utile dopo. Fatto ciò, consideriamo una \textbf{sistema isolato} dove, per definizione, $\mathbf{dQ = 0}$. Non abbiamo scambio di calore con l'esterno, quindi qualsiasi integrale della forma $\int_A^B (\frac{dQ}{T}) = 0$. Ora, ricordiamo a cosa corrisponde la variazione di entropia:
                \begin{align*}
                    \Delta S_{A\rightarrow B} = S_B-S_A=\int_A^B (\frac{dQ}{T})_{REV}
                \end{align*}
                Quindi, secondo questa definizione se $dQ = 0$, e se la trasformazione che facciamo è \textbf{reversibile}, allora anche $\Delta S_{A\rightarrow B} = 0$. Fin qui niente di strano. Ora però \textbf{riprendiamo \textcolor{Orange}{quel pezzo di prima}}:
                \begin{align*}
                    \Delta S_{A\rightarrow B} = S_B-S_A=\int_A^B (\frac{dQ}{T})_{REV} \textcolor{Orange}{>\int_A^B (\frac{dQ}{T})_{IRR}}
                \end{align*}
                Questo è molto interessante: se lo scambio di calore è nullo, allora anche l'integrale di Clausius è nullo. Però, se questo è \textbf{relativo ad una trasformazione irreversibile rappresenterà} (matematicamente, per quanto abbiamo dimostrato prima) \textbf{una quantità inferiore alla variazione di entropia}! Di conseguenza, se $dQ = 0$ e quindi $\int_A^B (\frac{dQ}{T})_{IRR} = 0$ avremmo:
                \begin{align*}
                    \Delta S_{A\rightarrow B} = S_B-S_A \textcolor{Orange}{>\int_A^B (\frac{dQ}{T})_{IRR}} = 0
                \end{align*}
                Quindi avremmo che $S_B-S_A > 0$. Mettendo tutto insieme, possiamo dire che:
                \begin{align*}
                    \textcolor{Red}{S_B-S_A >= 0
                    \begin{cases}
                        S_B = S_A \ \ \ \ \ \ \ \ Se\ trasformazione\ reversibile\\
                        S_B > S_A \ \ \ \ \ \ \ \ Se\ trasformazione\ irreversibile
                    \end{cases}}
                \end{align*}
                Questo è un risultato molto importante: \textbf{\textit{nei sistemi isolati, l'entropia NON PUO' MAI CALARE!}} Ciò viene anche chiamato \textit{formulazione matematica del secondo principio della termodinamica}. Se poi andiamo a pensare che l'\textbf{l'universo può essere considerato un sistema isolato}, possiamo dire che \textbf{l'entropia dell'universo continua ad aumentare} (o al massimo resta uguale). Questo è un concetto interessante, da un senso fisico al tempo se ci pensiamo. In natura (fisicamente parlando), un evento avviene solo nel senso in cui aumenta l'entropia dell'universo!


    \subsection{Esercizi termodinamica}
        \subsubsection{Esercizio n.10, esame giugno2021b}
            \framedImg{60}{L20-img001}
            Nel testo del problema manca $T_f = 600K$ temperatura di fusione del piombo!!!!\\
            Possiamo subito notare che il calore assorbito dal cubo di piombo ($Q_A > 0$) è dovuto all'attrito con l'aria, il lavoro dell'attrito sarà quindi $W_{ATT} < 0$. Ora abbiamo variazione di energia interna,
            \begin{align*}
                \Delta U = Q - W = 0 => Q_A = -W_{ATT}
            \end{align*}
            Osserviamo ora che il calore di fusione è,
            \begin{align*}
                Q_F = m\lambda
            \end{align*}
            Ora osserviamo che l'energia potenziale iniziale è $=0$ dato che il cubo di piombo parte da fermo, la variazione di energia potenziale è quindi,
            \begin{align*}
                \Delta E_p = E_{p,f} - E_{p,i} = Q_A + Q_F \bcancel{- 0}
            \end{align*}
            da questa sostituiamo e otteniamo,
            \begin{align*}
                mgh = mc\Delta T + m\lambda => gh = c(T_f - T_0) + \lambda
            \end{align*}
            e da questa ci ricaviamo la temperatura iniziale $T_0$ come segue,
            \begin{align*}
                T_0 = T_f - \frac{gh - \lambda}{c}
            \end{align*}
            Ora convertiamo $c$ e $\lambda$,
            \begin{align*}
                &c=0.031\frac{kcal}{Kkg}= 0.031*4184\frac{J}{Kkg}= 0.129*10^3\frac{J}{Kkg}\\
                &\lambda=6\frac{kcal}{kg}=6*4184\frac{J}{kg}=25.1*10^3\frac{J}{kg}
            \end{align*}
            Ora sostituiamo i valori e calcoliamo la temperatura iniziale,
            \begin{align*}
                &T_0 = 600K - \frac{9.8\frac{m}{s^2}*6,6*10^3m - 25,2*10^3 \frac{J}{kg}}{0.126* 10^3\frac{J}{Kkg}}= 600K - \frac{6.4*10^4\frac{m^2}{s^2} - 2.52* 10^4 \frac{\frac{\bcancel{kg}m^2}{s^2}}{\bcancel{kg}}}{0.0126*10^4\frac{\frac{\bcancel{kg}m^2}{s^2}}{K\bcancel{kg}}}\\
                &= 600K - \frac{6.4*\bcancel{10^4} - 2.52* \bcancel{10^4}}{0.0126*\bcancel{10^4}\frac{1}{K}} = 600K - \frac{6.4 - 2.52}{0.0126}K = 600K - 308K = 292K
            \end{align*}
            Ora convertiamo la temperatura in $^{\circ}C$,
            \begin{align*}
                292 - 273.16 = 18.84^{\circ}C
            \end{align*}

        \subsubsection{Entropia esercizio n.10, esame giugno2021b}
            Arrivati ora alla soluzione dell'esercizio precedente proviamo a calcolare l'entropia,
            \begin{align*}
                \Delta S_{AC} = \Delta S_{AB} + \Delta S_{BC}
            \end{align*}
            dove $AB$ indica la caduta e $BC$ indica la fusione.//
            Ai dati precedenti aggiungiamo massa $m=70g$ e ora possiamo procedere con l'esercizio. Calcoliamo prima la parte di caduta che è una isobara,
            \begin{align*}
                \Delta S_{AB} = mcln\bigg(\frac{T_f}{T_0}\bigg)=70*10^{-3}\bcancel{kg}*1.26*10^2\frac{J}{K\bcancel{kg}}ln\bigg(\frac{600\bcancel{K}}{292 \bcancel{K}}\bigg)= 8.8\frac{J}{K}ln(2.05)=6.34\frac{J}{K}
            \end{align*}
            Calcoliamo ora la parte di fusione che è un cambiamento di fase,
            \begin{align*}
                \Delta S_{BC} = \frac{\lambda m}{T_F} = \frac{2.5*10^4\frac{J}{\bcancel{kg}}70*10^{-3}\bcancel{kg}}{600K} = \frac{1750J}{600K} = 2.92\frac{J}{K}
            \end{align*}
            Ora quindi ho che,
            \begin{align*}
                \Delta S = 6.34 + 2.92 = 9.26\frac{J}{K}
            \end{align*}
