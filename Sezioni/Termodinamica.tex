\section{Termodinamica}
    La \textbf{termodinamica} è il ramo della fisica classica e della chimica che studia e descrive le trasformazioni termodinamiche indotte da calore a lavoro e viceversa in un sistema termodinamico, in seguito a processi che coinvolgono cambiamenti delle variabili di stato temperatura ed energia.\\
    Ci ricordiamo ora che il lavoro delle forze non conservative è la variazione di energia meccanica, ovvero come segue,
    \begin{align*}
        W_{N.C.} = E_f - E_i
    \end{align*}

    \subsection{Numero di Avogadro}
        Consideriamo ora un numero molto alto di di costituenti del sistema. Le relazioni che ci sono tra di essi sono espresse in un'unità di misura che è il \textbf{numero di Avogadro},
        \begin{align*}
            N_A = 6,022 * 10^{23} = 1mol
        \end{align*}
        esso indica quanti costituenti fondamentali sono contenuti in una certa quantità di sostanza detta \textbf{mole}.

    \subsection{Sistema termodinamico}
        Nella termodinamica vediamo l'universo composto da due parti, il \textbf{sistema termodinamico} e l'\textbf{ambiente}. Tra questi due c'è un'iterazione/scambio continuo ed in base a questo scambio definiamo i diversi tipi di sistemi termodinamici:
        \begin{center}
            \begin{tabular} { |c|c|c|c| }
                \hline
                & ENERGIA & MATERIA & ESEMPIO\\
                \hline
                APERTO & si & si & pentola d'acqua che bolle senza coperchio\\
                \hline
                CHIUSO & si & no & pentola d'acqua che bolle con coperchio\\
                \hline
                ISOLATO & no & no & contenitore isolato che contiene acqua\\
                \hline
                IMPOSSIBILE & no & si & impossibile, scambio materia => scambio energia\\
                \hline
            \end{tabular}
        \end{center}

    \subsection{Variabili termodinamiche}
        Le \textbf{variabili termodinamiche}, anche dette \textbf{coordinate termodinamiche}, si distinguono in \textbf{grandezze estensive}, ovvero quelle che dipendono dall'estensione del sistema termodinamico (eg. volume, massa), e \textbf{grandezze intensive}, ovvero quelle che non dipendono dall'estensione del sistema termodinamico (eg. pressione, temperatura, densità).\\

        Queste variabili sono anche distinte in \textbf{variabili globali}, ovvero quelle che sono riferibili a tutto il sistema (come la temperatura, nel caso in cui si lasci abbastanza tempo ad un corpo di scaldarsi/raffreddarsi), e \textbf{variabili locali}, ovvero quelle che sono riferibili ad una parte del sistema (come la densità, se si pensa ad una torta con crema e frutta, queste due avranno densità diversa).\\

        Il loro numero dipende dal sistema che sto considerando e le più famose sono \textbf{pressione}, \textbf{temperatura}, \textbf{volume}, \textbf{densità}, \textbf{massa},...\\

        Osserviamo che per i sistemi di gas useremo pressione, volume e temperatura ed inoltre il numero di moli e di componenti.

    \subsection{Equilibrio termodinamico}
        L'\textbf{equilibrio termodinamico} di un sistema termodinamico si ha quando tutte le suoi corpi/componenti sono in:
        \begin{itemize}
            \item \textbf{equilibrio meccanico}: ovvero quando non ci sono momenti delle forze che agiscono tra i suoi corpi/componenti, qualsiasi questi siano.\\
            \item \textbf{equilibrio termico}: ovvero quando non c'è nessuna differenza di temperatura tra i suoi corpi/componenti, qualsiasi questi siano.\\
            \item \textbf{equilibrio chimico}: ovvero quando non c'è nessuna reazione chimica tra i suoi corpi/componenti, qualsiasi questi siano.
        \end{itemize}

    \subsection{Trasformazione termodinamica}
        Una \textbf{trasformazione termodinamica} è il passaggio del sistema termodinamico da uno stato termodinamico ad un altro.\\
        Un esempio è il passaggio da caldo a freddo, in esso ci sono tanti passaggi/variazioni:
        \begin{itemize}
            \item \textbf{espansione/contrazione} di liquidi e gas\\
            \item variazione resistenza elettrica\\
            \item variazione differenza di potenziale\\
            \item variazione di riflettività/trasmittanza
        \end{itemize}
        È utile rappresentare le trasformazioni come spostamenti in un piano di coordinate. Per i gas di solito si usa il piano $V, p$.

    \subsection{Temperatura}
        La \textbf{temperatura} si misura su una scala e corrisponde alla media dell'energia cinetica dei componenti del sistema, ovvero quanto i componenti "sono agitati".

        \subsubsection{Unità di misura temperatura}
            La temperatura si misura con diverse unità di misura, \textbf{Celsius} (\degree{}C), \textbf{Fahrenheit} (\degree{}F) e \textbf{Kelvin} (K).\\
            Osserviamo che il simbolo della temperatura, per il Kelvin è \textbf{T}, mentre per le altre unità di misura è \textbf{t}, quindi la maiuscola è solo per il Kelvin.\\
            Per passare da una u.d.m. all'altra le formule sono le seguenti,
            \begin{align*}
                &t(\degree{}C)=T(K) - 273.16\\
                &t(\degree{}F)=\frac{9}{5}t(\degree{}C) + 32
            \end{align*}
            e le restanti si possono ricavare da queste.

        \subsubsection{Punto triplo dell'$H_2O$}
            Il \textbf{punto triplo dell'$H_2O$} è il punto in cui coesistono i 3 stati della materia, che nel caso dell'$H_2O$ sono ghiaccio, acqua e vapore acqueo, e corrisponde a
            \begin{align*}
                T_0=273.16K=0\degree{}C
            \end{align*}

        \subsubsection{Termometro}
            Un \textbf{termometro} è un sistema fisico, usato come strumento, che ci permette di esprimere le variazioni di temperatura in funzione di una grandezza fisica,
            \begin{align*}
                &\Delta T\rightarrow\Delta X\\
                &T=T(x)
            \end{align*}
            Nel caso di termometro a mercurio, la grandezza fisica è la lunghezza della colonnina di mercurio che leggiamo.\\
            Si cerca di far si che questa lunghezza sia lineare, ovvero che cresca linearmente con la temperatura, in modo che,
            \begin{align*}
                T=T(X)=\beta X
            \end{align*}
            Quando si costruisce un termometro si deve calibrarlo, per farlo si prende una temperatura di riferimento ($T_0$) e si misura la lunghezza ($X_0$), poi con questi due dati si calcola,
            \begin{align*}
                \beta=\frac{T_0}{X_0}
            \end{align*}

        \subsubsection{Contatto termico}
            %\framedImg{25}{L13-img001}
            Supponiamo di avere tre corpi, A, B e C a contatto tra loro come in figura, con temperature iniziali $T_{A,i}$, $T_{B,i}$ e $T_{C,i}$ corrispondentemente, diverse tra di loro.\\
            Dopo un periodo di tempo abbastanza lungo, le tre temperature, $T_{A,f}$, $T_{B,f}$ e $T_{C,f}$, saranno uguali. Se invece aspettiamo poco tempo, ci si può trovare in diverse situazioni a seconda del tipo di contatto tra i corpi.\\
            Le pareti si dividono quindi in:
            \begin{itemize}
                \item \textbf{pareti diatermiche}: permettono lo scambio di calore grazie all'ottimo contatto\\
                \item \textbf{pareti adiabatiche}: non permettono scambio di calore, sono quindi "isolanti"
            \end{itemize}
            Il \textbf{contenitore adiabatico} è un contenitore isolante termicamente.\\
            Osserviamo che in pratica, se si aspetta un tempo sufficientemente lungo, niente è perfettamente isolante.

        \subsubsection{Principio zero della termodinamica}
            Il principio zero della termodinamica dice che se due corpi sono entrambi in equilibrio termico con un terzo corpo, allora lo sono anche fra loro, ovvero se A e B sono in equilibrio termico e B e C sono in equilibrio termico, allora anche A e C sono in equilibrio termico.\\
            Questo si traduce nel fatto che se,
            \begin{align*}
                T_A=T_B\quad AND\quad T_B=T_C\quad =>\quad T_A=T_C
            \end{align*}

    \subsection{Esperimento di Joule}
        %\framedImg{40}{L14-img001}
        Supponiamo di avere un contenitore con pareti adiabatiche, riempito di un fluido al cui interno è inserito un \textbf{mulinello} agganciato ad una puleggia a cui è appesa una massa $m$. La massa andando su e giù fa girare il mulinello, che a sua volta fa scaldare l'acqua.\\
        Ora si può legare $\Delta h$ a $\Delta T$. Il lavoro che compio è il seguente,
        \begin{align*}
            W=mg\Delta h
        \end{align*}
        Analogamente al mulinello posso immergere una \textbf{resistenza} nell'acqua e in questo caso ottengo lavoro,
        \begin{align*}
            W=\frac{\Delta V^2}{R}\Delta t
        \end{align*}
        Stessa cosa se immergo in acqua un'\textbf{elica} e la faccio muovere su e giù, o due \textbf{piastre} che faccio sfregare tra loro producendo attrito.\\
        Posso anche immergere un \textbf{pallone} nell'acqua e far variare il suo volume.\\
        In conclusione quindi se faccio lavoro (o casino) dentro ad un sistema termodinamico, la temperatura (o agitazione) aumenta, infatti il lavoro è legato alla variazione di temperatura, a lavori uguali corrispondono variazioni di temperatura uguali,
        \begin{align*}
            W=-\Delta U_{int}
        \end{align*}
        osserviamo che il segno "-" è una convenzione adottata in termodinamica, dato che quello che interessa della macchina termica è il lavoro che "l'acqua" può compiere. Se il sistema da energia, il lavoro sarà $>0$, se invece il sistema riceve energia, il lavoro sarà $<0$.\\
        Con questo esperimento abbiamo appreso che l'energia interna dipende solo dalla temperatura ovvero,
        \begin{align*}
            U = U(T)
        \end{align*}

    \subsection{Calore}
        Il \textbf{calore} è quindi ciò che determina la variazione di temperatura,
        \begin{align*}
            \Delta T = \gamma Q => Q = \frac{1}{\gamma}\Delta T = \frac{1}{\gamma}(T_f - T_i) = \frac{1}{\gamma}(U_f - U_i) = \frac{1}{\gamma}\Delta U = \eta \Delta U
        \end{align*}
        con $\eta = \frac{1}{\gamma}$.
        Ora quindi se confrontiamo due differenze di temperature,
        \begin{align*}
            &\Delta T_1=\gamma Q_1\\
            &\Delta T_2=\gamma Q_2
        \end{align*}
        otteniamo,
        \begin{align*}
            \frac{\Delta T_2}{\Delta T_1}=\frac{\bcancel{\gamma} Q_2}{\bcancel{\gamma} Q_1}
        \end{align*}
        Ora sperimentalmente si prova che $\eta = 1$ e quindi ottengo che,
        \begin{align*}
            Q = \Delta U_{int}
        \end{align*}
        ovvero il calore è tutto energia interna.\\
        Osserviamo che introducendo il calore abbiamo generalizzato il principio di conservazione dell'energia, infatti il calore è un modo per scambiare l'energia.

        \subsubsection{Esempio di calore}
            La massa (e.g. carbone) che brucio per scaldare una pentola d'acqua darà un certo calore che, a parità di massa, dipende da tante cose come l'umidità, la qualità del carbone, il tipo di pentola che uso,... Tutte queste sono delle costanti quindi ho,
            \begin{align*}
                Q = c_p c_h c_q ... m
            \end{align*}
            Facendo un po' di esperimenti in condizioni diverse si potranno osservare delle variazioni di temperatura diverse, quello che però osservo sempre è che se raddoppio la massa che brucio, raddoppia anche la variazione di temperatura, la costante $\gamma$ tiene conto di queste costanti.

    \subsection{Principio di equivalenza calore-lavoro}
        Dato ora quindi che il calore è $Q = \Delta U_{int}$ e ricordandoci da prima che $W=-\Delta U_{int}$ possiamo ricavare che,
        \begin{align*}
            Q = -W <=> W = -Q
        \end{align*}
        Il calore e il lavoro quindi sono energia in trasferimento/movimento, identificano una differenza tra lo stato iniziale e lo stato finale.

    \subsection{I principio della termodinamica}
        Tornando ora a parlare della variazione d'energia interna, essa è causata in parte dal lavoro che il sistema compie/subisce e in parte dal calore che il sistema assorbe/cede, ovvero
        \begin{align*}
            &\Delta U_{int} = \Delta U_{int}^{(W)} + \Delta U_{int}^{(Q)}\\
            &\Delta U_{int}^{(W)}=-W\\
            &\Delta U_{int}^{(Q)}=Q
        \end{align*}
        La variazione di energia interna dovuta al lavoro è $-W$ dato che se il sistema compie lavoro, questo viene a "danno" della sua energia interna. Per la variazione di energia dovuta al calore invece è $Q$ dato che se ricevo calore questo si "trasforma" in energia interna.\\
        In conclusione quindi ho che la variazione di energia interna è la differenza tra il calore e il lavoro,
        \begin{align*}
            \Delta U{int} = Q - W
        \end{align*}
        Osserviamo ora che per passare dallo stato iniziale a quello finale ci sono tanti modi/traiettorie, ognuna delle quali ha il suo $Q_i$ e il suo $W_i$, la $\Delta U$ però è sempre uguale, indipendentemente dalla traiettoria.\\\\
        Ora posso concentrarmi sulle quantità infinitesimali, quindi mi verrebbe da scrivere,
        \begin{align*}
            dU = dQ - dW
        \end{align*}
        La parte sinistra all'"=" è corretta, infatti la variazione di energia dipende solo dallo stato iniziale e da quello finale, è differenziabile. La parte destra invece, quindi la variazione di calore e di lavoro dipende anche dal percorso e non è un differenziale esatto, quindi si usa $\delta$ come segue,
        \begin{align*}
            dU = \delta Q - \delta W
        \end{align*}
        La variazione infinitesima di calore e di lavoro dipende quindi dalla trasformazione $\gamma$ che effettuo sul piano,
        \begin{align*}
            &\delta Q = \delta Q_{\gamma}(p, V, T)\\
            &\delta W = \delta W_{\gamma}(p, V, T)
        \end{align*}

    \subsection{Tipi di trasformazioni termodinamiche}

        \subsubsection{Trasformazione ciclica}
            Una \textbf{trasformazione ciclica} è una trasformazione per cui lo stato iniziale e quello finale coincidono. Dato quindi che $i\equiv f$ ho che,
            \begin{align*}
                &T_i = T_f\\
                &U_i = U_f
            \end{align*}
            e quindi,
            \begin{align*}
                \Delta U = 0 => Q = W
            \end{align*}
            Infatti se ho incamerato energia, devo in qualche modo restituirla all'ambiente esterno, per far si stato iniziale e finale coincidano.\\
            I motori per esempio fanno trasformazioni cicliche.

        \subsubsection{Trasformazione quasi-statica}
            Una \textbf{trasformazione quasi-statica} è una trasformazione in cui ogni stato intermedio è uno stato di equilibrio, dove le "cose" sono fatte con "dolcezza" tale da non perturbare ne il sistema ne l'ambiente esterno.

        \subsubsection{Trasformazione reversibile}
            Una \textbf{trasformazione reversibile} è una trasformazione in cui si "può tornare indietro". Per essere reversibile, una trasformazione deve essere quasi statica e non ci devono essere dissipazioni.

            \paragraph{Esempio trasformazione reversibile}
                %\framedImg{40}{L13-img002}
                Supponiamo di avere un contenitore, con del gas all'interno, aperto sopra e nell'apertura è inserito un pistone, supponiamo anche che tra il pistone e le pareti non ci siano perdite. Se ora metto dei pesetti sopra il pistone, riesco a farlo abbassare. Posso assumere che gli infiniti stati intermedi infinitesimi sono in equilibrio, quindi la trasformazione è quasi-statica. Inoltre è una trasformazione reversibile, infatti togliendo i pesetti che ho messo sopra il pistone, il sistema ritorna allo stato in cui era prima di metterci i pesetti.

            \paragraph{Esempio trasformazione non reversibile}
                %\framedImg{40}{L13-img003}
                Questo esempio riguarda l'espansione libera dei gas, infatti ho un contenitore adiabatico separato all'interno in due camere con un rubinetto che le collega. All'inizio il rubinetto è chiuso e nella camera destra c'è gas, mentre in quella sinistra il vuoto. Se ora apro il rubinetto ed aspetto un certo tempo, finisco nella situazione in cui il gas si è distribuito nelle due camere. Ora per far si che il gas torni tutto nella camera destra (situazione iniziale) devo aspettare tempo infinito.

        \subsubsection{Lentezza}
            Ci riferiamo alla \textbf{lentezza} come la caratteristica/velocità di un processo (trasformazione termodinamica) che ci consente di trattare tutti gli stati intermedi come stati di equilibrio e che ci consente di raffigurare la trasformazione come passaggi su infiniti stati intermedi tutti di equilibrio.

    \subsection{Calorimetria}
        %\framedImg{40}{L13-img004}
        Consideriamo un sistema termodinamico composto da un contenitore adiabatico con all'interno $H_2O$ a $T_{H_2O,i}$ ed un corpo $c$ immerso nell'acqua a $T_{c,i}$ (analogamente se considero due masse solide a contatto tra loro in un contenitore adiabatico ottengo gli stessi risultati). Aspettando un istante di tempo abbastanza lungo passo dal sistema descritto sopra, ad un sistema in cui l'$H_2O$ e $c$ hanno la stessa temperatura $T_*$, quindi ipotizzando che l'$H_2O$ abbia temperatura inferiore a $c$ abbiamo che $T_{H_2O,i}<T_*<T_{c,i}$.\\
        Ora essendo il sistema isolato abbiamo che,
        \begin{align*}
            Q = 0 \quad AND\quad W = 0 => \Delta U = 0
        \end{align*}
        e quindi,
        \begin{align*}
            0=\Delta U=\Delta U_{H_2O} + \Delta U_c => \Delta U_{H_2O} = - \Delta U_c
        \end{align*}
        ovvero l'energia interna che l'$H_2O$ "guadagna" la "toglie" dal corpo $c$. Osserviamo ora che $\Delta V = 0$, ovvero il volume non cambia, e quindi $\Delta W = 0$ sia per l'$H_2O$ che per $c$ e quindi ho che,
        \begin{align*}
            &\Delta U_{H_2O}=Q_{H_2O} - \bcancel{W_{H_2O}}\\
            &\Delta U_c=Q_c - \bcancel{W_c}
        \end{align*}
        La variazione di energia interna avviene tutta per cambio di calore, e quindi ora ottengo,
        \begin{align*}
            \Delta U_{H_2O} = - \Delta U_c => Q_{H_2O} = -Q_c
        \end{align*}
        Ora devo capire a che energia corrisponde la variazione di temperatura, questa mi consente di calcolare il lavoro che mi serve per fare aumentare la temperatura dell'${H_2O}$. Questo sarà il calore $Q_{H_2O}$, che sarà $-Q_c$.\\
        Ora ottengo che
        \begin{align*}
            Q = c_{\gamma}m\Delta T
        \end{align*}
        dove $c_{\gamma}$ è il calore specifico.\\
        Osserviamo che questa è una trasformazione non reversibile, infatti devo aspettare infinito tempo per far si che le due temperature tornino ai valori iniziali.

        \subsubsection{Calore specifico}
            Il \textbf{calore specifico} di un corpo è una costante definita come segue,
            \begin{align*}
                c = \frac{1}{m}\left[\frac{dQ}{dT}\right]_{\gamma}
            \end{align*}
            La sua unità di misura è quindi $\frac{J}{kgK}$.\\
            Osserviamo ora che il calore specifico si cita per una particolare trasformazione, infatti scrivere,
            \begin{align*}
                c = \frac{1}{m}\frac{dQ}{dT}
            \end{align*}
            oppure,
            \begin{align*}
                c = \frac{1}{m}\frac{\delta Q}{dT}
            \end{align*}
            sarebbe sbagliato.\\
            Osserviamo ora che il fatto di scambiare calore con l'esterno non implica che ci sia una variazione di temperatura, infatti quest'ultima si ha solo quando varia l'energia interna.\\
            Il calore specifico è utile solo nel caso in cui c'è variazione di temperatura. Per esempio quando l'acqua nel freezer diventa ghiaccio (passaggio di fase), non c'è variazione di temperatura anche se il frigo ha spento energia per ghiacciare l'acqua, quindi c'è scambio di calore ma non ha senso parlare di calore specifico.\\

            Per una qualsiasi trasformazione quasi-statica, da $P_i$ a $P_f$, il calore è definito come segue,
            \begin{align*}
                Q = \int_{P_i}^{P_f} \delta Q = \int_{P_i}^{P_f} mc_{\gamma}dT
            \end{align*}
            ed in alternativa,
            \begin{align*}
                Q = C(T_f - T_i)
            \end{align*}
            dove $C$ è la capacità termica.

        \subsubsection{Calorimetro}
            Il \textbf{calorimetro} è un sistema controllato per il quale so connettere la variazione di temperatura al calore ricevuto. Quello dell'esempio precedente (contenitore adiabatico contenente acqua ed un oggetto immerso), è un calorimetro a bagno.

        \subsubsection{Capacità termica}
            La \textbf{capacità termica} è il prodotto tra massa e calore specifico della trasformazione che stiamo considerando, ovvero come segue,
            \begin{align*}
                C=mc_{\gamma}
            \end{align*}



    \subsection{Cambi di fase}





        \subsubsection{Calore latente per il cambio di fase}
