\section{Termodinamica}
    La \textbf{termodinamica} è il ramo della fisica classica e della chimica che studia e descrive le trasformazioni termodinamiche indotte da calore a lavoro e viceversa in un sistema termodinamico, in seguito a processi che coinvolgono cambiamenti delle variabili di stato temperatura ed energia.\\
    Ci ricordiamo ora che il lavoro delle forze non conservative è la variazione di energia meccanica, ovvero come segue,
    \begin{align*}
        W_{N.C.} = E_f - E_i
    \end{align*}

    \subsection{Numero di Avogadro}
        Consideriamo ora un numero molto alto di di costituenti del sistema. Le relazioni che ci sono tra di essi sono espresse in un'unità di misura che è il \textbf{numero di Avogadro},
        \begin{align*}
            N_A = 6,022 * 10^{23} = 1mol
        \end{align*}
        esso indica quanti costituenti fondamentali sono contenuti in una certa quantità di sostanza detta \textbf{mole}.

    \subsection{Sistema termodinamico}
        Nella termodinamica vediamo l'universo composto da due parti, il \textbf{sistema termodinamico} e l'\textbf{ambiente}. Tra questi due c'è un'iterazione/scambio continuo ed in base a questo scambio definiamo i diversi tipi di sistemi termodinamici:
        \begin{center}
            \begin{tabular} { |c|c|c|c| }
                \hline
                & ENERGIA & MATERIA & ESEMPIO\\
                \hline
                APERTO & si & si & pentola d'acqua che bolle senza coperchio\\
                \hline
                CHIUSO & si & no & pentola d'acqua che bolle con coperchio\\
                \hline
                ISOLATO & no & no & contenitore isolato che contiene acqua\\
                \hline
                IMPOSSIBILE & no & si & impossibile, scambio materia => scambio energia\\
                \hline
            \end{tabular}
        \end{center}

    \subsection{Variabili termodinamiche}
        Le \textbf{variabili termodinamiche}, anche dette \textbf{coordinate termodinamiche}, si distinguono in \textbf{grandezze estensive}, ovvero quelle che dipendono dall'estensione del sistema termodinamico (eg. volume, massa), e \textbf{grandezze intensive}, ovvero quelle che non dipendono dall'estensione del sistema termodinamico (eg. pressione, temperatura, densità).\\

        Queste variabili sono anche distinte in \textbf{variabili globali}, ovvero quelle che sono riferibili a tutto il sistema (come la temperatura, nel caso in cui si lasci abbastanza tempo ad un corpo di scaldarsi/raffreddarsi), e \textbf{variabili locali}, ovvero quelle che sono riferibili ad una parte del sistema (come la densità, se si pensa ad una torta con crema e frutta, queste due avranno densità diversa).\\

        Il loro numero dipende dal sistema che sto considerando e le più famose sono \textbf{pressione}, \textbf{temperatura}, \textbf{volume}, \textbf{densità}, \textbf{massa},...\\

        Osserviamo che per i sistemi di gas useremo pressione, volume e temperatura ed inoltre il numero di moli e di componenti.

    \subsection{Equilibrio termodinamico}
        L'\textbf{equilibrio termodinamico} di un sistema termodinamico si ha quando tutte le suoi corpi/componenti sono in:
        \begin{itemize}
            \item \textbf{equilibrio meccanico}: ovvero quando non ci sono momenti delle forze che agiscono tra i suoi corpi/componenti, qualsiasi questi siano.\\
            \item \textbf{equilibrio termico}: ovvero quando non c'è nessuna differenza di temperatura tra i suoi corpi/componenti, qualsiasi questi siano.\\
            \item \textbf{equilibrio chimico}: ovvero quando non c'è nessuna reazione chimica tra i suoi corpi/componenti, qualsiasi questi siano.
        \end{itemize}

    \subsection{Trasformazione termodinamica}
        Una \textbf{trasformazione termodinamica} è il passaggio del sistema termodinamico da uno stato termodinamico ad un altro.\\
        Un esempio è il passaggio da caldo a freddo, in esso ci sono tanti passaggi/variazioni:
        \begin{itemize}
            \item \textbf{espansione/contrazione} di liquidi e gas\\
            \item variazione resistenza elettrica\\
            \item variazione differenza di potenziale\\
            \item variazione di riflettività/trasmittanza
        \end{itemize}
        È utile rappresentare le trasformazioni come spostamenti in un piano di coordinate. Per i gas di solito si usa il piano $V, p$.

    \subsection{Temperatura}
        La \textbf{temperatura} si misura su una scala e corrisponde alla media dell'energia cinetica dei componenti del sistema, ovvero quanto i componenti "sono agitati".

        \subsubsection{Unità di misura temperatura}
            La temperatura si misura con diverse unità di misura, \textbf{Celsius} (\degree{}C), \textbf{Fahrenheit} (\degree{}F) e \textbf{Kelvin} (K).\\
            Osserviamo che il simbolo della temperatura, per il Kelvin è \textbf{T}, mentre per le altre unità di misura è \textbf{t}, quindi la maiuscola è solo per il Kelvin.\\
            Per passare da una u.d.m. all'altra le formule sono le seguenti,
            \begin{align*}
                &t(\degree{}C)=T(K) - 273.16\\
                &t(\degree{}F)=\frac{9}{5}t(\degree{}C) + 32
            \end{align*}
            e le restanti si possono ricavare da queste.

        \subsubsection{Punto triplo dell'$H_2O$}
            Il \textbf{punto triplo dell'$H_2O$} è il punto in cui coesistono i 3 stati della materia, che nel caso dell'$H_2O$ sono ghiaccio, acqua e vapore acqueo, e corrisponde a
            \begin{align*}
                T_0=273.16K=0\degree{}C
            \end{align*}

        \subsubsection{Termometro}
            Un \textbf{termometro} è un sistema fisico, usato come strumento, che ci permette di esprimere le variazioni di temperatura in funzione di una grandezza fisica,
            \begin{align*}
                &\Delta T\rightarrow\Delta X\\
                &T=T(x)
            \end{align*}
            Nel caso di termometro a mercurio, la grandezza fisica è la lunghezza della colonnina di mercurio che leggiamo.\\
            Si cerca di far si che questa lunghezza sia lineare, ovvero che cresca linearmente con la temperatura, in modo che,
            \begin{align*}
                T=T(X)=\beta X
            \end{align*}
            Quando si costruisce un termometro si deve calibrarlo, per farlo si prende una temperatura di riferimento ($T_0$) e si misura la lunghezza ($X_0$), poi con questi due dati si calcola,
            \begin{align*}
                \beta=\frac{T_0}{X_0}
            \end{align*}

        \subsubsection{Contatto termico}
            %\framedImg{25}{L13-img001}
            Supponiamo di avere tre corpi, A, B e C a contatto tra loro come in figura, con temperature iniziali $T_{A,i}$, $T_{B,i}$ e $T_{C,i}$ corrispondentemente, diverse tra di loro.\\
            Dopo un periodo di tempo abbastanza lungo, le tre temperature, $T_{A,f}$, $T_{B,f}$ e $T_{C,f}$, saranno uguali. Se invece aspettiamo poco tempo, ci si può trovare in diverse situazioni a seconda del tipo di contatto tra i corpi.\\
            Le pareti si dividono quindi in:
            \begin{itemize}
                \item \textbf{pareti diatermiche}: permettono lo scambio di calore grazie all'ottimo contatto\\
                \item \textbf{pareti adiabatiche}: non permettono scambio di calore, sono quindi "isolanti"
            \end{itemize}
            Il \textbf{contenitore adiabatico} è un contenitore isolante termicamente.\\
            Osserviamo che in pratica, se si aspetta un tempo sufficientemente lungo, niente è perfettamente isolante.

        \subsubsection{Principio zero della termodinamica}
            Il principio zero della termodinamica dice che se due corpi sono entrambi in equilibrio termico con un terzo corpo, allora lo sono anche fra loro, ovvero se A e B sono in equilibrio termico e B e C sono in equilibrio termico, allora anche A e C sono in equilibrio termico.\\
            Questo si traduce nel fatto che se,
            \begin{align*}
                T_A=T_B\quad AND\quad T_B=T_C\quad =>\quad T_A=T_C
            \end{align*}

    \subsection{Esperimento di Joule}
        %\framedImg{40}{L14-img001}
        Supponiamo di avere un contenitore con pareti adiabatiche, riempito di un fluido al cui interno è inserito un \textbf{mulinello} agganciato ad una puleggia a cui è appesa una massa $m$. La massa andando su e giù fa girare il mulinello, che a sua volta fa scaldare l'acqua.\\
        Ora si può legare $\Delta h$ a $\Delta T$. Il lavoro che compio è il seguente,
        \begin{align*}
            W=mg\Delta h
        \end{align*}
        Analogamente al mulinello posso immergere una \textbf{resistenza} nell'acqua e in questo caso ottengo lavoro,
        \begin{align*}
            W=\frac{\Delta V^2}{R}\Delta t
        \end{align*}
        Stessa cosa se immergo in acqua un'\textbf{elica} e la faccio muovere su e giù, o due \textbf{piastre} che faccio sfregare tra loro producendo attrito.\\
        Posso anche immergere un \textbf{pallone} nell'acqua e far variare il suo volume.\\
        In conclusione quindi se faccio lavoro (o casino) dentro ad un sistema termodinamico, la temperatura (o agitazione) aumenta, infatti il lavoro è legato alla variazione di temperatura, a lavori uguali corrispondono variazioni di temperatura uguali,
        \begin{align*}
            W=-\Delta U_{int}
        \end{align*}
        osserviamo che il segno "-" è una convenzione adottata in termodinamica, dato che quello che interessa della macchina termica è il lavoro che "l'acqua" può compiere. Se il sistema da energia, il lavoro sarà $>0$, se invece il sistema riceve energia, il lavoro sarà $<0$.\\
        Con questo esperimento abbiamo appreso che l'energia interna dipende solo dalla temperatura ovvero,
        \begin{align*}
            U = U(T)
        \end{align*}

    \subsection{Calore}
        Il \textbf{calore} è quindi ciò che determina la variazione di temperatura,
        \begin{align*}
            \Delta T = \gamma Q => Q = \frac{1}{\gamma}\Delta T = \frac{1}{\gamma}(T_f - T_i) = \frac{1}{\gamma}(U_f - U_i) = \frac{1}{\gamma}\Delta U = \eta \Delta U
        \end{align*}
        con $\eta = \frac{1}{\gamma}$.
        Ora quindi se confrontiamo due differenze di temperature,
        \begin{align*}
            &\Delta T_1=\gamma Q_1\\
            &\Delta T_2=\gamma Q_2
        \end{align*}
        otteniamo,
        \begin{align*}
            \frac{\Delta T_2}{\Delta T_1}=\frac{\bcancel{\gamma} Q_2}{\bcancel{\gamma} Q_1}
        \end{align*}
        Ora sperimentalmente si prova che $\eta = 1$ e quindi ottengo che,
        \begin{align*}
            Q = \Delta U_{int}
        \end{align*}
        ovvero il calore è tutto energia interna.\\
        Osserviamo che introducendo il calore abbiamo generalizzato il principio di conservazione dell'energia, infatti il calore è un modo per scambiare l'energia.

        \subsubsection{Esempio di calore}
            La massa (e.g. carbone) che brucio per scaldare una pentola d'acqua darà un certo calore che, a parità di massa, dipende da tante cose come l'umidità, la qualità del carbone, il tipo di pentola che uso,... Tutte queste sono delle costanti quindi ho,
            \begin{align*}
                Q = c_p c_h c_q ... m
            \end{align*}
            Facendo un po' di esperimenti in condizioni diverse si potranno osservare delle variazioni di temperatura diverse, quello che però osservo sempre è che se raddoppio la massa che brucio, raddoppia anche la variazione di temperatura, la costante $\gamma$ tiene conto di queste costanti.

    \subsection{Principio di equivalenza calore-lavoro}
        Dato ora quindi che il calore è $Q = \Delta U_{int}$ e ricordandoci da prima che $W=-\Delta U_{int}$ possiamo ricavare che,
        \begin{align*}
            Q = -W <=> W = -Q
        \end{align*}
        Il calore e il lavoro quindi sono energia in trasferimento/movimento, identificano una differenza tra lo stato iniziale e lo stato finale.

    \subsection{I principio della termodinamica}
        Tornando ora a parlare della variazione d'energia interna, essa è causata in parte dal lavoro che il sistema compie/subisce e in parte dal calore che il sistema assorbe/cede, ovvero
        \begin{align*}
            &\Delta U_{int} = \Delta U_{int}^{(W)} + \Delta U_{int}^{(Q)}\\
            &\Delta U_{int}^{(W)}=-W\\
            &\Delta U_{int}^{(Q)}=Q
        \end{align*}
        La variazione di energia interna dovuta al lavoro è $-W$ dato che se il sistema compie lavoro, questo viene a "danno" della sua energia interna. Per la variazione di energia dovuta al calore invece è $Q$ dato che se ricevo calore questo si "trasforma" in energia interna.\\
        In conclusione quindi ho che la variazione di energia interna è la differenza tra il calore e il lavoro,
        \begin{align*}
            \Delta U{int} = Q - W
        \end{align*}
        Osserviamo ora che per passare dallo stato iniziale a quello finale ci sono tanti modi/traiettorie, ognuna delle quali ha il suo $Q_i$ e il suo $W_i$, la $\Delta U$ però è sempre uguale, indipendentemente dalla traiettoria.\\\\
        Ora posso concentrarmi sulle quantità infinitesimali, quindi mi verrebbe da scrivere,
        \begin{align*}
            dU = dQ - dW
        \end{align*}
        La parte sinistra all'"=" è corretta, infatti la variazione di energia dipende solo dallo stato iniziale e da quello finale, è differenziabile. La parte destra invece, quindi la variazione di calore e di lavoro dipende anche dal percorso e non è un differenziale esatto, quindi si usa $\delta$ come segue,
        \begin{align*}
            dU = \delta Q - \delta W
        \end{align*}
        La variazione infinitesima di calore e di lavoro dipende quindi dalla trasformazione $\gamma$ che effettuo sul piano,
        \begin{align*}
            &\delta Q = \delta Q_{\gamma}(p, V, T)\\
            &\delta W = \delta W_{\gamma}(p, V, T)
        \end{align*}

    \subsection{Tipi di trasformazioni termodinamiche}

        \subsubsection{Trasformazione ciclica}
            Una \textbf{trasformazione ciclica} è una trasformazione per cui lo stato iniziale e quello finale coincidono. Dato quindi che $i\equiv f$ ho che,
            \begin{align*}
                &T_i = T_f\\
                &U_i = U_f
            \end{align*}
            e quindi,
            \begin{align*}
                \Delta U = 0 => Q = W
            \end{align*}
            Infatti se ho incamerato energia, devo in qualche modo restituirla all'ambiente esterno, per far si stato iniziale e finale coincidano.\\
            I motori per esempio fanno trasformazioni cicliche.

        \subsubsection{Trasformazione quasi-statica}
            Una \textbf{trasformazione quasi-statica} è una trasformazione in cui ogni stato intermedio è uno stato di equilibrio, dove le "cose" sono fatte con "dolcezza" tale da non perturbare ne il sistema ne l'ambiente esterno.

        \subsubsection{Trasformazione reversibile}
            Una \textbf{trasformazione reversibile} è una trasformazione in cui si "può tornare indietro". Per essere reversibile, una trasformazione deve essere quasi statica e non ci devono essere dissipazioni.

            \paragraph{Esempio trasformazione reversibile}
                %\framedImg{40}{L13-img002}
                Supponiamo di avere un contenitore, con del gas all'interno, aperto sopra e nell'apertura è inserito un pistone, supponiamo anche che tra il pistone e le pareti non ci siano perdite. Se ora metto dei pesetti sopra il pistone, riesco a farlo abbassare. Posso assumere che gli infiniti stati intermedi infinitesimi sono in equilibrio, quindi la trasformazione è quasi-statica. Inoltre è una trasformazione reversibile, infatti togliendo i pesetti che ho messo sopra il pistone, il sistema ritorna allo stato in cui era prima di metterci i pesetti.

            \paragraph{Esempio trasformazione non reversibile}
                %\framedImg{40}{L13-img003}
                Questo esempio riguarda l'espansione libera dei gas, infatti ho un contenitore adiabatico separato all'interno in due camere con un rubinetto che le collega. All'inizio il rubinetto è chiuso e nella camera destra c'è gas, mentre in quella sinistra il vuoto. Se ora apro il rubinetto ed aspetto un certo tempo, finisco nella situazione in cui il gas si è distribuito nelle due camere. Ora per far si che il gas torni tutto nella camera destra (situazione iniziale) devo aspettare tempo infinito.

        \subsubsection{Lentezza}
            Ci riferiamo alla \textbf{lentezza} come la caratteristica/velocità di un processo (trasformazione termodinamica) che ci consente di trattare tutti gli stati intermedi come stati di equilibrio e che ci consente di raffigurare la trasformazione come passaggi su infiniti stati intermedi tutti di equilibrio.

    \subsection{Calorimetria}
        %\framedImg{40}{L13-img004}
        Consideriamo un sistema termodinamico composto da un contenitore adiabatico con all'interno $H_2O$ a $T_{H_2O,i}$ ed un corpo $c$ immerso nell'acqua a $T_{c,i}$ (analogamente se considero due masse solide a contatto tra loro in un contenitore adiabatico ottengo gli stessi risultati). Aspettando un istante di tempo abbastanza lungo passo dal sistema descritto sopra, ad un sistema in cui l'$H_2O$ e $c$ hanno la stessa temperatura $T_*$, quindi ipotizzando che l'$H_2O$ abbia temperatura inferiore a $c$ abbiamo che $T_{H_2O,i}<T_*<T_{c,i}$.\\
        Ora essendo il sistema isolato abbiamo che,
        \begin{align*}
            Q = 0 \quad AND\quad W = 0 => \Delta U = 0
        \end{align*}
        e quindi,
        \begin{align*}
            0=\Delta U=\Delta U_{H_2O} + \Delta U_c => \Delta U_{H_2O} = - \Delta U_c
        \end{align*}
        ovvero l'energia interna che l'$H_2O$ "guadagna" la "toglie" dal corpo $c$. Osserviamo ora che $\Delta V = 0$, ovvero il volume non cambia, e quindi $\Delta W = 0$ sia per l'$H_2O$ che per $c$ e quindi ho che,
        \begin{align*}
            &\Delta U_{H_2O}=Q_{H_2O} - \bcancel{W_{H_2O}}\\
            &\Delta U_c=Q_c - \bcancel{W_c}
        \end{align*}
        La variazione di energia interna avviene tutta per cambio di calore, e quindi ora ottengo,
        \begin{align*}
            \Delta U_{H_2O} = - \Delta U_c => Q_{H_2O} = -Q_c
        \end{align*}
        Ora devo capire a che energia corrisponde la variazione di temperatura, questa mi consente di calcolare il lavoro che mi serve per fare aumentare la temperatura dell'${H_2O}$. Questo sarà il calore $Q_{H_2O}$, che sarà $-Q_c$.\\
        Ora ottengo che
        \begin{align*}
            Q = c_{\gamma}m\Delta T
        \end{align*}
        dove $c_{\gamma}$ è il calore specifico.\\
        Osserviamo che questa è una trasformazione non reversibile, infatti devo aspettare infinito tempo per far si che le due temperature tornino ai valori iniziali.

        \subsubsection{Calore specifico}
            Il \textbf{calore specifico} di un corpo è una costante definita come segue,
            \begin{align*}
                c = \frac{1}{m}\left[\frac{dQ}{dT}\right]_{\gamma}
            \end{align*}
            La sua unità di misura è quindi $\frac{J}{kgK}$.\\
            Osserviamo ora che il calore specifico si cita per una particolare trasformazione, infatti scrivere,
            \begin{align*}
                c = \frac{1}{m}\frac{dQ}{dT}
            \end{align*}
            oppure,
            \begin{align*}
                c = \frac{1}{m}\frac{\delta Q}{dT}
            \end{align*}
            sarebbe sbagliato.\\
            Osserviamo ora che il fatto di scambiare calore con l'esterno non implica che ci sia una variazione di temperatura, infatti quest'ultima si ha solo quando varia l'energia interna.\\
            Il calore specifico è utile solo nel caso in cui c'è variazione di temperatura. Per esempio quando l'acqua nel freezer diventa ghiaccio (passaggio di fase), non c'è variazione di temperatura anche se il frigo ha spento energia per ghiacciare l'acqua, quindi c'è scambio di calore ma non ha senso parlare di calore specifico.\\

            Per una qualsiasi trasformazione quasi-statica, da $P_i$ a $P_f$, il calore è definito come segue,
            \begin{align*}
                Q = \int_{P_i}^{P_f} \delta Q = \int_{P_i}^{P_f} mc_{\gamma}dT
            \end{align*}
            ed in alternativa,
            \begin{align*}
                Q = C(T_f - T_i)
            \end{align*}
            dove $C$ è la capacità termica.

        \subsubsection{Calorimetro}
            Il \textbf{calorimetro} è un sistema controllato per il quale so connettere la variazione di temperatura al calore ricevuto. Quello dell'esempio precedente (contenitore adiabatico contenente acqua ed un oggetto immerso), è un calorimetro a bagno.

        \subsubsection{Capacità termica}
            La \textbf{capacità termica} è il prodotto tra massa e calore specifico della trasformazione che stiamo considerando, ovvero come segue,
            \begin{align*}
                C=mc_{\gamma}
            \end{align*}



    \subsection{Cambi di fase}
        La materia può trovarsi in \textbf{3 diversi stati}:
        \begin{itemize}
            \item solido: il corpo ha forma e volume propri;
            \item liquido: il corpo ha un volume proprio ma assume la forma del recipiente che lo contiene;
            \item gassoso: il corpo, molto spesso aeroforme, assume forma e volume del recipiente che lo contiene;
        \end{itemize}
        Sotto certe condizioni, un corpo può passare da uno stato all'altro, questo passaggio, in base agli stati iniziale e finale, ha doversi nomi:
        \framedImg{5}{L15-img001}
        Quando si parla di termodinamica, le "scale di misura" importanti sono volume, \textbf{pressione e temperatura}, se prendiamo le ultime 2 e le rappresentiamo su un piano cartesiano otteniamo quello che viene detto \textbf{diagramma di fase}:
        \framedImg{5}{L15-img002}
        Ovviamente varia da materiale a materiale, ma sostanzialmente la forma è quella. Prendendo ad esempio l'acqua, in corrispondenza di $T = 273,16 K$ e $P = 611 Pa$ abbiamo quello che si definisce \textbf{punto triplo}, ovvero quel punto in cui le \textbf{3 linee di demarcazione si intersecano}. In questo punto, i 3 stati possono coesistere contemporaneamente (in rosso nell'immagine)! È particolarmente importante perché \textbf{definisce la scala di temperatura}. Un altro punto particolarmente importante è quello che viene definito \textbf{punto critico}, ovvero il punto in cui \textbf{finisce la linea di demarcazione tra liquido e gas}. In particolare, oltre quel punto, gli stati gassoso e liquido non sono più distinguibili (fluido supercritico).

        \subsubsection{Calori latenti}
            Immaginiamo di voler far passare dell'acqua dallo stato liquido a quello gassoso: osservando il diagramma di fase, possiamo notare che per farlo potremmo far variare la pressione, mantenendo la temperatura stabile, oppure modificare la temperatura mantenendo la pressione stabile (oppure anche modificare entrambe ovviamente). Se decidessimo di mantenere la temperatura stabile, e quindi spostarci solo sull'asse P, potremmo notare che per mantenere la temperatura effettivamente stabile \textbf{dobbiamo fornire calore al sistema!} Il passaggio da liquido a gassoso, in genere, \textbf{ha bisogno di assorbire calore dall'ambiente circostante per effettuare il cambio di fase}. Allo stesso modo, se passassimo da gassoso a liquido dovremmo \textbf{assorbire noi calore} per poter mantenere la temperatura stabile.\\
            In particolare, questo "calore che non fa variare la temperatura", definito \textbf{calore latente di fusione/evaporazione/...}, è proporzionale alla massa dell'oggetto sottoposto al cambio di fase, in particolare la formula è:
            \begin{align*}
                \textcolor{Red}{Q_\lambda=\lambda*m}
            \end{align*}
            Il paramentro importante di questa formula è il $\textcolor{Red}{\lambda}$, ovvero la \textbf{costante di calore latente} e varia da materiale a materiale e, soprattutto, in base alla temperatura alla quale si trova il corpo sottoposto al cambio di fase. Facciamo alcuni esempi con l'acqua:
            \begin{align*}
                & \lambda_{fusione} = 3,3 *10^5\frac{J}{Kg} <==> 273,16 K && \lambda_{evaporazione} = 22,6 *10^5\frac{J}{Kg} <==> 373,16 K
            \end{align*}
            Idealmente, se andiamo verso lo stato solido (nel diagramma di fase saliamo verso l'alto sull'asse della pressione) è \textbf{il cambio di fase a darci calore} (siamo noi che dobbiamo assorbirlo), quindi abbiamo:
            \begin{align*}
                Q_\lambda=\textcolor{Red}{-}\lambda*m
            \end{align*}
            Se invece andiamo verso lo stato gassoso (nel diagramma di fase sceniamo verso il basso sull'asse della pressione) siamo \textbf{noi che dobbiamo dare calore al cambio di fase} (è il cambio di fase che lo assorbe), quindi abbiamo:
            \begin{align*}
                Q_\lambda=\textcolor{Red}{+}\lambda*m
            \end{align*}
            Nota che facciamo questo discorso (noi che forniamo calore, noi che assorbiamo calore). I \textbf{calori latenti} sono molto interessanti in quanto sono uno dei pochi casi in cui \textbf{abbiamo uno scambio di calore senza un'effettiva variazione di temperatura}.

            \paragraph{Esempio}
                Facciamo un veloce esercizio: \textit{abbiamo \textbf{1 litro} di acqua \textbf{liquida} ad una temperatura di \textbf{273,16 K}, \textbf{quanto calore dobbiamo fornire/sottrarre dal sistema per solidificare l'acqua mantenendo la temperatura costante?}} \\
                Allora, la prima cosa da notare è che \textbf{da liquido vogliamo passare a solido}, quindi la costante che ci interessa sarà \textbf{quella della fusione "al contrario"}. Avremmo allora:
                \begin{align*}
                    Q_\lambda = \lambda_{solidificazione}*m=\textcolor{Red}{-\lambda_{fusione}}*m
                \end{align*}
                Ora, un altro problema è che noi abbiamo \textbf{litri} di acqua, non chili. Ricorda che i \textbf{i litri sono un'unità di misura del volume}, quindi ci basta semplicemente \textbf{moltiplicare il volume per la densità dell'acqua} (in particolare $1\frac{Kg}{dm^3}$, aka. $1\frac{Kg}{l}$, infatti $1 litro = 1 dm^3$). Tornando a noi, avremmo:
                \begin{align*}
                    &Q_\lambda &&= -\lambda_{fusione}*m\\
                    & &&= -\lambda_{fusione}*\textcolor{Red}{\delta_{H2O}*V_{H2O}}\\
                    & &&= -3,3 *10^5\frac{J}{\bcancel{Kg}}*1\frac{\bcancel{Kg}}{\bcancel{l}}*1\bcancel{l}\\
                    & &&= -3,3 *10^5J = -330 KJ
                \end{align*}

            \paragraph{Esempio temperatura di equilibrio}

    \subsection{Trasferimento di calore}
        Iniziamo col ricordare che il calore non è altro che "\textit{energia in movimento}": fisicamente parlando, un corpa \textbf{non ha calore}, ma \textbf{ha energia} (ricordalo per l'orale :P). Detto ciò, introduciamo i 3 principali metodi per il trasferimento del calore:
        \begin{itemize}
            \item conduzione;
            \item convezione;
            \item irraggiamento;
        \end{itemize}

        \subsubsection{Conduzione}
            Ovvero il \textbf{trasferimento di energia tramite contatto}, immaginamo di avere una barra di "\textit{qualcosa}" collegata ad una \textbf{sorgente di calore ideale} (definiamo sorgente di calore ideale un \textit{corpo ideale capace di fornire/assorbire calore all'infinito}):
            \framedImg{7}{L15-img003}
            La barra inizialmente si trova ad una sua \textbf{temperatura iniziale $T_i$}, mentre la sorgente si trova costantemente ad una \textbf{temperatura $T_s$}. Supponendo che le dimensioni $L_x$ ed $L_y$ della barra (la sua sezione trasversale) siano \textbf{trascurabili}, possiamo immaginarla di \textbf{dividerla in tratti infinitesimali} per lunghezza. Se lasciamo scorrere il tempo, i \textbf{tratti vicini alla sorgente cambieranno la loro temperatura}, avvicinandosi (fino a raggiungere) $T_s$. Dopo un certo tempo, avremmo che \textbf{tutta la barra raggiungerà la temperatura} $T_2$.\\

            Come facciamo a quantificare il tutto? Come detto prima, immaginiamo di dividere la barra in molte sezioni trasversali:
            \framedImg{7}{L15-img004}
            Ora, possiamo notare diverse "proporzionalità":
            \begin{itemize}
                \item Q $\propto \Delta$ T: il calore assorbito dalla barra è \textbf{proporzionale alla differenza di temperatura tra le sue sezioni trasversali};
                \item Q $\propto \Delta$ t: il calore assorbito dalla barra è \textbf{proporzionale al tempo per cui resta a contatto con la sorgente ideale} (più tempo resto a contatto e più calore assorbo);
                \item Q $\propto$ S: il calore assorbito dalla barra è \textbf{proporzionale alla superficie di contatto tra la barra e la sorgete} (più superficie ho a contatto con la sorgente e più calore assorbo);
            \end{itemize}
            Se consideriamo \textbf{tratti infinitesimi} possiamo ricavare la \textbf{legge di Fourier}, ovvero:
            \begin{align*}
                \textcolor{Red}{dQ=-K*\frac{dT}{dz}*dS*dt}
            \end{align*}
            Analizziamola un po': sostanzialmente ci sta dicendo che la \textit{quantità infinitesima di calore \textbf{assorbito dalla barra}} (e quindi sottratto alla sorgente, per quello il segno -) \textit{è proporzionale alla differenza di temperatura per unità di lunghezza($\frac{\Delta T}{dz}$), alla superficie trasversale della barra ($dS$) e al tempo di contatto ($dt$). Tutto ciò è regolato da una costante di proporzionalità ($K$, nota che è DIVERSA dal Kelvin) che rappresenta la \textbf{conducibilità del materiale}}.

            \paragraph{La costante di conducibilità}
                Se ci rigiriamo un po' la formula di Fourier, possiamo facilmente notare che l'unità di misura di questa costante corrisponde a $\frac{[energ]}{[lung]*[temp]*[temperat]}$, in particolare avremmo $\frac{J}{m*s*K}$. Più è grande questa costante e più facilmente il notro materiale condurrà calore, vediamo qualche esempio:
                \begin{itemize}
                    \item alluminio: $200\frac{J}{msK}$, buon conduttore di calore;
                    \item sughero: $0,04\frac{J}{msK}$, cattivo conduttore di calore
                \end{itemize}
                Nota che, la maggior parte delle volte, i buoni conduttori di calore sono anche buni conduttori di energia elettrica (quindi i metalli).


        \subsubsection{Convezione}
            Il metodo di trasferimento del calore principale quando si parla di liquidi e di atmosfera. Non lo trattiamo approfonditamente (dato che i meccanismi algebrici che lo riguardano sono molto complessi), ma lo vediamo velocemente comunque. Supponiamo di essere in questa situazione:
            \framedImg{3}{L15-img005}
            Abbiamo una sorgente a contatto con un liquido più freddo. Ora, vediamo cosa succede se alziamo la temperatura della sorgente e aspettiamo un certo tempo (non specificato, un tempo generico):
            \framedImg{7}{L15-img006}
            Consideriamo diversi momenti:
            \begin{enumerate}
                \item partiremo in un momento in cui tutto il liquido (sia a contatto con la sorgente che non) è alla \textbf{stessa pressione $P_0$ e temperatura $T_0$};
                \item col passare del tempo, il liquido a contatto con la sorgente \textbf{inizia a scaldarsi} e ciò comporta un \textbf{aumento della sua pressione} (al momento prendi per buono);
                \item dato che la pressione $P_1 > P_0$, allora il liquido caldo \textbf{inizia ad espandersi in maniera irregolare};
                \item dato che l'espansione è irregolare, può succedere che \textbf{bolle calde si "staccano" dalla sorgente} ed iniziano a "fluttuare" nel liquido freddo, \textbf{permettendo ad altro linquido freddo di scaldarsi con la sorgente}. Un'altro punto importante è che \textbf{anche la bolla calda scalderà il liquido freddo!}
            \end{enumerate}
            Concludendo, l'idea è che, con il passare del tempo, il numero di bolle calde che navigano nel fluido continua ad aumentare finché \textbf{tutto il fluido non arriverà alla temperatura della sorgente}.

        \subsubsection{Irraggiamento}
            Ogni corpo emette delle \textbf{onde elettromagnetiche}, queste non sono altro che \textbf{fasci di energia pura}: trasferendo energia possiamo trasferire calore! Ora, la formula che regola questo fenomeno è la \textbf{legge di Stefan-Boltzmann}, ovvero:
            \begin{align*}
                \textcolor{Red}{\epsilon=\sigma*e*T^4}
            \end{align*}
            In particolare abbiamo che:
            \begin{itemize}
                \item $\varepsilon$: l'energia emessa da un corpo;
                \item $\sigma$: la \textbf{costante di Stefan-Boltzmann}, vale \textcolor{Red}{$\sigma=5,67*10^-8\frac{J}{m^2*s*K^4}$};
                \item $e$: emissività del corpo, possiamo vederla anche come la capacità del corpo di riflettere le onde elettromagnetiche. Vale un valore $\in [0, 1]$, dove 1 identifica la "massima rifelttività" del corpo, mentre 0 il "massimo assorbimento" (un corpo totalmente nero) NON SICURO!!! RICONTROLLARE;
                \item $T^4$: la temperatura $^4$ del corpo che consideriamo.
            \end{itemize}
            Se analizziamo l'unità di misura di questa quantità $\varepsilon$ vediamo che corrisponde a $\textcolor{Red}{\frac{J}{m^2*K}}$, in altre parole $[\varepsilon]=[\frac{E}{L^2*T}]$, ovvero Energia su Lunghezza$^2$*Temperatura.
            Nota che tutti i corpi emettono queste onde elettromagnetiche, un esempio interessante (e soprattutto che potrebbe tornare all'esame) è quello della \textbf{costante solare}, ovvero la quantità di energia "irragiata" dal sole. Identifichiamo qusta quantità (così come un sacco di altre cose che non centrano niente, quindi ATTENZIANE) con la lettera $c$ e la chiamiamo \textbf{costante solare}. In particolare, vale (mediamente):
            \begin{align*}
                c=1,36\frac{J}{m^2*s}
            \end{align*}
            Una piccola nota conclusiva: è abbastanza improprio considerare l'irraggiamento come forma di trasmissione del calore "normale", infatti \textbf{non c'è un vero e proprio trasferimento meccanico del calore}

            \paragraph{Esercizio con costante solare}

        \subsubsection{Contenitore adiabatico}
            Ora che conosciamo i vari metodi per il trasferimento di calore, possiamo porci il quesito \textbf{"come costruire un contenitore veramente adiabatico"} (isolante)? Uno dei metodi possibili è quello del \textbf{\textit{vaso Dewar}}, che possiamo schematizzare in questo modo:
            \framedImg{4}{L15-img007}
            Per evitare tutti i tipi di trasferimento di calore, il contenitore deve avere diverse caratteristiche, in particolare:
            \begin{enumerate}
                \item essere a tenuta stagna, in modo da \textbf{evitare la convezione};
                \item avere uno "strato di vuoto" il più vicino al liquido da mantenere a temperatura, in modo da \textbf{evitare la conduzione};
                \item avere delle pareti \textbf{riflettenti} (Dewar usava delle pareti argentate) in modo da \textbf{evitare l'irraggiamento}. Nota che le pareti devono essere riflettenti \textbf{sia verso l'esterno} (per riflettere le onde dell'ambiente esterno) \textbf{sia verso l'interno} (in modo da impedire al liquido di disperdere il suo calore).
            \end{enumerate}

    \subsection{Gas ideali}
        I gas sono un sistema termodinamico composto da un \textbf{gran numero di molecole libere di muoversi nello spazio}. Noi consideriamo i cosidetti \textbf{"gas ideali"}, ovvero quei gas per cui valgono le seguenti condizioni:
        \begin{itemize}
          \item gli elementi che compongono il gas \textbf{non interagiscono tra di loro in alcun modo};
          \item gli elementi che compongono il gas \textbf{hanno volume nullo} (li approssimiamo come dei punti materiali, infatti, rispetto a loro, il contenitore che le contiene è infinitamente grande [di solito]);
          \item \textbf{non ci sono reazioni chimiche} tra le componenti del gas o col contenitore.
        \end{itemize}
        Come si intuisce dal nome, i \textbf{gas ideali} (in quanto ideali) \textbf{sono impossibili}, però sono una \textbf{buona approssimazione per quei gas a bassa pressione} (in genere $< 100 atm$).

        \subsubsection{Le variabili principali}
          Per descrivere i gas, usiamo principalmente \textbf{3 variabili}:
          \begin{itemize}
              \item volume ($V$);
              \item pressione ($P$);
              \item temperatura ($T$);
          \end{itemize}
          Analizziamole velocemente:
            \paragraph{Volume}
                La più semplice delle 3: quando parliamo di gas supponiamo che \textbf{questo sia contenuo in un qualche contenitore}, questo contenitore \textbf{"contiene" una porzione di spazio} che rappresenta appunto il volume. Nota che, se non diversamente specificato, supponiamo che il gas \textbf{occupi tutto il volume a sua disposizione}. Il volume si misura come una \textbf{lunghezza al qubo} ($L^3$), in particolare il \textcolor{Red}{\textbf{metro$^3$}} oppure il \textbf{litro} (che corrisponde a un $dm^3 = 10^{-3}m^3$).
            \paragraph{Pressione}
                Il fatto che il gas occupi tutto lo spazio a sua disposizione implica che \textbf{sarà a contatto col contenitore che lo contiene}. Inoltre, le molecole del gas sono in costante movimento quindi ci saranno \textbf{degli urti tra il contenitore e le molecole del gas}: ciò darà origine ad una forza che premerà sul contenitore.
                \framedImg{4}{L15-img008}
                Formalizzando il tutto, ogni molecola del gas si muove liberamente nel contenitore con una \textbf{certa velocità} \textcolor{Red}{$v$} ed ha anche una \textbf{certa massa} \textcolor{Red}{$m$}. Possiamo quindi dire che \textbf{ogni molecola ha una certa quantità di moto} \textcolor{Red}{$m*v = p$}. Ora, quando le molecole urtano il contenitore avviene \textbf{un urto perfettamente elastico} tra la molecola e la parete del contenitore. C'è da notare che la massa della molecola ($m$) è \textbf{infinitesimale} rispetto alla massa della parete del contenitore ($m_{parete}$), quindi potremmo dire che c'è una specie di "\textbf{inversione del moto}". Questa variazione della quantità di moto $\Delta p$ avviene in un certo tempo $\Delta t$ e sappiamo che non è nulla, quindi abbiamo \textbf{una forza impulsiva!}
                \begin{align*}
                    \frac{m*\vec{v}_i - m*\vec{v}_f}{\Delta t} = \frac{\Delta p}{\Delta t} => \vec{F}\neq \vec{0}
                \end{align*}
                Questo per ogni singola molecola, se però sommiamo le forze impulsive di tutte le molecole del gas otteniamo una forza (ortogonale alla parete del contenitore) non trascurabile. Se poi andiamo a considerare la \textbf{forza ortogonale per unità di superficie otteniamo proprio la pressione!} In particolare:
                \begin{align*}
                    p = \frac{F_\bot}{S}
                \end{align*}
                Dimensionalmente, ci troviamo con una Forza fratto una Superficie, avremmo quindi:
                \begin{align*}
                    [p]=\frac{[F]}{[L^2]}=\frac{N}{m^2}=\textcolor{Red}{Pascal=Pa}
                \end{align*}
                L'unità di misura fondamentale della pressione è il Pascal, esistono poi delle "derivate", in particolare citiamo:
                \begin{align*}
                    &\textcolor{Red}{bar = 10^5Pa}&&\textcolor{Red}{atmosfera=atm=1,013bar}
                \end{align*}
                Terminiamo la pressione con una precisazione per quanto riguarda i \textbf{gas in equilibrio}. Immaginiamo di avere una situazione di questo tipo:
                \framedImg{5}{L15-img009}
                Una supposizione importante per capire questo "esempio" è che il contenitore (in verde) possa, volendo, espandersi all'infinito. Ora, abbiamo detto che le molecole del gas all'interno del contenitore, scontrandosi con le sue pareti, creano una forza (che noi chiamiamo \textbf{pressione}) che \textbf{spinge verso l'esterno}. Se prendiamo ad esempio un palloncino e lo gonfiamo, notiamo che inizia ad \textbf{aumentare di volume}. Ad un certo punto però \textbf{si ferma}. Ciò avviene perché si è raggiunto uno \textbf{stato di equilibrio}, ovvero abbiamo la forza della pressione che spinge verso l'esterno ma c'è anche una forza uguale ma opposta che bilancia la pressione. Nel caso del palloncino, questa è la \textbf{risultante della somma tra forza elastica del pallone} ($\vec{F}_{gas->S_1}$) \textbf{e pressione dell'ambiente esterno} che preme sul pallove dall'esterno verso l'interno ($\vec{F}_{amb->S_1}$). Se noi "estremizziamo" questo esempio e togliamo la forza elastica del pallone, abbiamo che il pallone smette di crescere solo quando \textbf{la forza del gas che preme sulla superficie è uguale alla forza dell'ambiente che preme sulla superficie}, ovvero:
                \begin{align*}
                    &\vec{F}_{amb->S_1} = -\vec{F}_{gas->S_1}&&=>\frac{F_{\bot,a}}{S_1}=\frac{F_{\bot,g}}{S_1}\\
                    & && =>\textcolor{Red}{P_a=P_g}
                \end{align*}
                Questa equazione in \textcolor{Red}{rosso} è molto importante per i \textbf{sistemi in equilibrio}. Di fatto, la parete del contenitore trasmette meccanicamente le forze dall'interno verso l'esterno e vice versa. Possiamo quindi pensare di "togliere virtualmente" la parete e fare dei confronti tra i 2 gas. In molti casi, infatti, non possiamo direttamente misurare la pressione dei gas che ci interessano, però, se questi si trovano in equilibrio con l'ambiente esterno, possiamo calcolarla considerando la pressione dell'ambiente esterno! Ad esempio:
                \framedImg{5}{L15-img010}
                Abbiamo un pistone che preme del gas contenuto in un contenitore a tenuta stagna. Di questo pistone conosciamo la \textbf{massa} ($m$) e la \textbf{superficie} (S), come facciamo a calcolare la pressione del gas in rosso? Se il sistema si trova in uno stato di equilibrio, ovvero il pistone è fermo e non scende (comprimento ulteriormente il gas), possiamo dire che la pressione del gas rosso è \textbf{uguale alla pressione dell'ambiente esterno}, ovvero:
                \begin{align*}
                    &P_{ambiente}&&=P_{atmosferica}+P_{pistone}\\
                    & &&=P_{atmosferica}+\frac{F_{pesoPistone}}{S}\\
                    & &&=P_{atmosferica}+\frac{m*g}{S}
                \end{align*}
                Concludendo, dato che la pressione del gas rosso è uguale alla pressione dell'ambiente esterno (perchè il sistema è in equilibrio), abbiamo che $P_{gasRosso}=P_{atmosferica}+\frac{m*g}{S}$!

                \paragraph{Temperatura}
                    La temperatura è \textbf{fortemente legata all'energia interna del gas}. Possiamo vedere l'energia interna del gas come la \textbf{somma del'energia cinetica di ogni molecola del gas} ($\frac{1}{2}mv^2$). Per quanto riguarda le unità di misura, abbiamo già detto prima che utilizziamo il \textbf{Kelvin} ($K$) e, in alcuni casi, i gradi \textit{Celsius} ($^\circ C$) o \textit{Fahrenheit} ($^\circ F$)

            \subsubsection{Equazione di stato dei gas ideali (o perfetti)}
                Introduciamo ora l'\textbf{equazione di stato di questi gas ideali}, prima però una veloce definizione (tratta da Wikipedia) di equazione di stato:
                \begin{center}
                    \textit{In termodinamica e chimica fisica, una equazione di stato è una legge costitutiva che descrive lo stato della materia sotto un dato insieme di condizioni fisiche. Fornisce una relazione matematica tra due o più variabili di stato associate alla materia, come temperatura, pressione, volume o energia interna.}
                \end{center}
                In pratica, è una funzione (= 0) che ci descrive il comportamento del nostro gas in base alla variazione delle variabili termodinamiche che consideriamo. Come detto prima, per i gas usiamo come variabili \textbf{pressione, volume e temperatura}, dato che sono 3 rappresentiamo il tutto in un \textbf{diagramma bidimensionale} ($3-1$, infatti se avessimo N variabili dovremmo usare una dimensione $N-1$, questo perché, di solito, siamo interessati a scoprire il valore di una particolare variabile date le altre). In particolare, per i gas l'equazione di stato è:
                \begin{align*}
                    \textcolor{Red}{p*V - n R T = 0}
                \end{align*}

                Questa equazione la si ottiene basandosi sui risultati di 3 (o meglio 4) leggi, vediamole.
                \paragraph{Legge isocora di Gay-Lussac}
                    Se noi teniamo il \textbf{volume costante} ($V=const$, isocora), abbiamo che pressione la corrisponde a:
                    \begin{align*}
                        p=p_0(1+\beta t)
                    \end{align*}
                    Se noi rappresentassimo questa equazione sul piano cartesiano in funzione della temperatura, otterremo qualcosa del genere:
                    \framedImg{5}{L16-img001}
                    Nota come la retta è crescente, questo implica che, per qualche valore di temperatura ($t^*$), avremmo \textbf{pressione = 0}, e ciò è particolarmente interessante! Idealmente, più è bassa la temperatura e, a parità di volume, la pressione si abbassa: però la pressione \textbf{non può essere negativa}! Su questo punto ci torneremo in seguito;
                \paragraph{Legge isobara di Gay-Lussac}
                    Se noi teniamo la \textbf{pressione costante} ($p=const$, isobara), abbiamo che il volume in funzione della temperatura corrisponde a:
                    \begin{align*}
                        V=V_0(1+\alpha t)
                    \end{align*}
                    Se noi rappresentassimo questa equazione sul piano cartesiano in funzione della temperatura, otterremo qualcosa del genere:
                    \framedImg{5}{L16-img002}
                    Nota come la retta è crescente, questo implica che, per qualche valore di temperatura ($t^{**}$), avremmo \textbf{volume = 0}, e anche questo è particolarmente interessante! Idealmente, più è bassa la temperatura e, a parità di pressione, il volume si abbassa: però il volume \textbf{non può essere negativo}! Cosa ancora più interessante, $t^* = t^{**}$ (i 2 punti sono uguali)! Dopo ci torneremo più in dettaglio;
                \paragraph{Legge isoterma di Boyle}
                    A parità di temperatura, il \textbf{prodotto $p*V$ resta costante}! Ovvero:
                    \begin{align*}
                        p_i*V_i = p_f*V_f = const
                    \end{align*}
                    In soldoni, se tengo la temperatura costante e abbasso la pressione, il volume \textbf{aumenta per compensare} e \textbf{viceversa}. Se prendiamo questa equazione e la rappresentiamo su un grafico bidimensionale, otteniamo un'iperbole equilatera:
                    \framedImg{5}{L16-img003}

                \paragraph{Legge di Avogadro}
                    Gas diversi nelle stesse condizioni di temperatura, pressione e volume \textbf{contengono lo stesso numero di oggetti} ($N$). In particolare, abbiamo che:
                    \begin{align*}
                        N=\frac{1}{K_B}\frac{pV}{T}
                    \end{align*}
                    Dove $K_B$ rappresenta la \textbf{costante di Boltzmann} (\textcolor{Red}{$K_B = 1,38*10^{-23}\frac{J}{K}$}). Da dove deriva la sua unità di misura? Se controlliamo l'equazione precedente, abbiamo che $N$, ovvero il numero di "molecole" del gas (adimensionale) è (tralasciando $K_B$) uguale al $[\frac{pV}{T}] =[\frac{\frac{F}{L^2}*L^3}{T}]=[\frac{F*L}{T}]=[\frac{E}{T}]=\frac{J}{K}$ ricorda che, per definizione, \textbf{una forza per uno spostamente è un'energia} (che ha unità di misura Joule). Dato che dobbiamo "semplificare" queste unità di misura per rendere il tutto adimensionale, la costante di Boltzmann avrà come unità di misura $\frac{J}{K}$!
                    Molto spesso però, ci troveremo a lavorare con il \textbf{numero di moli} $n$ invece che con il numero esatto di elementi $N$. Possiamo quindi modificare la formula precedente in questo modo:
                    \begin{align*}
                        n=\frac{1}{R}\frac{pV}{T}
                    \end{align*}
                    Sappiamo infatti che le moli $n=\frac{N}{N_A}$ (ricorda che il \textit{numero di Avogadro} $N_A=6,022*10^{23}$), quindi ci basta \textbf{divedere da entrambe le parti per $\mathbf{N_A}$}. Infatti $R$ corrisponde proprio a $R=\frac{1}{K_B*N_A}\approx8,314\frac{J}{K*mol}$. Concludiamo con un po' di numeri, supponiamo di avere un gas con queste caratteristiche:
                    \begin{align*}
                        &n = 1mol&&p=1atm&&t=0^\circ C =>273,16 K
                    \end{align*}
                    Allora il suo volume è fissato, e vale:
                    \begin{align*}
                        &V = 22,414 dm^3 = 22,414 l
                    \end{align*}
                    Questo viene definito \textbf{volume molare}.

                \paragraph{Mettiamo tutte le leggi insieme}
                    L'equazione di stato dei gas si ottine \textbf{mettendo insieme le 4 leggi viste prima}, cominciamo da quelle di \textit{Gay-Lussac}:
                    \doubleFramedImg{4}{L16-img001}{L16-img002}
                    Avevamo questi 2 grafici, ovviamente la retta dipende dal gas: per gas diversi otterremo rette diverse. La cosa interessante però è che i punto $t^*$ e $t^{**}$ \textbf{puntano allo stesso valore di temperatura}! Prendendo 2 misurazioni (2 punti sul grafico) siamo capaci di tracciare la retta associata e stimare questa temperatura. Col tempo, siamo arrivati ad un'ottima approssimazione, ovvero:
                    \begin{align*}
                        t^*=t^{**}=t_0=-273,15^\circ C = 0 K
                    \end{align*}
                    L'idea ora è quella di fare una cosa del genere:
                    \doubleFramedImg{4}{L16-img004}{L16-img005}
                    Ovvero \textbf{spostiamo il sistema di riferimento}, in modo da usare il \textbf{sistema di riferimento assoluto per la temperatura} (i gradi \textit{Kelvin} appunto). Se facciamo cos', le equazioni delle leggi fondamentali cambiano leggermente (il significato resta sempre lo stesso), permettendoci di ricavare \textbf{l'equazione di stato dei gas ideali}:
                    \begin{align*}

                    \end{align*}
