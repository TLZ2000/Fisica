\section{Introduzione}
	\subsection{Il metodo sperimentale}
		Distingue discipline sperimentali da discipline non sperimentali. Si compone di diverse fasi:
		\begin{enumerate}
			\item \textbf{formulazione ipotesi}: si fa un'\textbf{ipostesi descrittiva} (in \textbf{linguaggio matematico}) della porzione di mondo che si vuole analizzare, di conseguenza si decide di \textbf{non considerare} altre caratteristiche del mondo che non centrano con l'ipotesi che stiamo formulando;
			\item \textbf{esperimento}: si va a ricreare una situazione dove l'aspetto che vogliamo analizzare è \textbf{sicuramente presente} e \textbf{influenzato il meno possibile da fattori esterni};
			\item \textbf{esecuzione dell'esperimento}: si verifica l'ipotesi, formulata in modo matematico, confrontando i valori ottenuti con l'esperimento con quelli che si ottengono dalla nostra ipotesi.
			\medskip\\
		\end{enumerate}

		In base alla "verifica" dell'ipotesi possiamo fare una differenziazione:
		\begin{itemize}
			\item \textbf{teoria}: l'ipotesi \textbf{non è ancora verificata}, o è verificata \textbf{parzialmente};
			\item \textbf{legge fisica}: l'ipotesi \textbf{è verificata} (in un certo ambito);
		\end{itemize}

\section{Cinetica dei punti}
	Descrive il movimento dei corpi.
	\subsection{Sistema di riferimento}
		Specifichiamo un sistema di riferimento per il seguente argomento:
		\framedImg{85}{L01-img001}
		Una cosa importante da notare è che un numero singolo può rappresentare solo cose "\textbf{mono-dimensionali}" e che, soprattutto, non tutte le unità di misura possono rappresentare qualsiasi cosa (ad es.: l'età dell'universo non si può rappresentare con i metri).

	\subsection{Diagramma dello spazio}
		Rappresentiamo lo spostamento nel tempo tramite un "\textit{\textbf{diagramma dello spazio}}":
		\framedImg{40}{L01-img002}
		In particolare, in questo diagramma rappresentiamo sull'asse Y lo \textbf{spostamento} (S) (rappresentato come \textbf{valore uni-dimensionale}) e sull'asse X il \textbf{tempo} (t) (anche rappresentato come \textbf{valore uni-dimensionale}). \importante{Nota che il diagrmma NON RAPPRESENTA una posizione, ma lo spostamento in relazione al tempo}.

	\subsection{Caso semplice}
		Vediamo un semplice caso di utilizzo per capire come usare i diagrammi dello spazio:
		\framedImg{40}{L01-img003}
		Possiamo immaginare di avere un oggetto in movimento su una retta tra i punti A e B, come possiamo rappresentare questo movimento nel diagramma? Come prima cosa posizioniamo i "fenomeni" (\definizione{qualcosa che appare evidente all'osservazione}{fenomeno}), ovvero i \textbf{punti A e B}, nota che non è detto che questi punti coincidano con dei "punti particolari" (ad esempio l'origine) nel nostro diagramma. In particolare, a questi punti associamo \textbf{un valore sull'asse del tempo ($t_i$, $t_f$)} ed \textbf{un valore sull'asse dello spazio ($S_i$, $S_f$)}.
		A questo punto esistono \textbf{infiniti possibili percorsi} tra il punto A ed il punto B, ad esempio:
		\framedImg{40}{L01-img004}
		Importante notare che \importante{non tutti questi percorsi, pur avendo senso matematico, hanno senso fisico}! Ad esempio, il percorso in rosso "torna indietro nel tempo"!

	\subsection{Moto rettilineo uniforme}
		STUB\#\#\#\#\#\#\#\#\#\#\#\#\#\#\#\#\#\#\# (In teoria lo fa dopo, controllare)

	\subsection{Velocità}
		Possiamo immaginare la velocità (\textit{v}) come la "\definizione{variazione dello spazio rapportato al tempo impiegato per percorrerlo}{velocità}", in particolare la velocità è data dalla formula:
		\begin{align*}
			v=\frac{S_f - S_i}{t_f - t_i} = \frac{\Delta S}{\Delta t}
		\end{align*}
		Vediamo un semplice esempio:
		\begin{align*}
			S_i = 400m,\ S_f = 700m,\ t_i = 7:30 = 450 min,\ t_f = 7:40 = 460 min
		\end{align*}
		\begin{align*}
			v=\frac{700m - 400m}{460min - 450min}=\frac{300m}{600s}=0,5m/s
		\end{align*}
		Nota che nella seconda uguaglianza nell'esempio abbiamo \textbf{convertito i minuti in secondi}, puoi immaginare che abbiamo posto "$min = (60s)$", quindi abbiamo fatto "$10 min = 10 * (60s)= 600s$".

		\subsubsection{Velocità istantanea}
			Quella che abbiamo calcolato prima possiamo vederla come "velocità media" di tutto il percorso, la \textbf{velocità istantanea} invece possiamo vederla come la \definizione{velocità in un punto specifico del percorso}{velocità istantanea}. Immagina quindi di fare la formula:
			\begin{align*}
				v_{ist}=lim_{\Delta t\rightarrow 0} \frac{\Delta S}{\Delta t} = \frac{\delta S}{\delta t}
			\end{align*}
			Nota che quando si usa la lettera "$\delta$" stiamo ad indicare una \textbf{piccola} (infinitesima) \textbf{variazione}. Ora, se il valore di S viene espresso \textbf{in funzione di t}, quindi abbiamo $S(t)$, e la funzione "$S(t)$" è \textbf{derivabile}, allora la \textbf{velocità istantanea corrisponde alla derivata prima della funzione $S(t)$}, che a sua volta corrisponde a $\frac{dS}{dt}$.
			\framedImg{40}{L01-img005}
			Supponendo che il \textbf{moto del nostro punto} venga identificato dalla curva in verde, il rapporto tra la lunghezza dei 2 cateti $C_1C_2$ ($\Delta t$) e $C_2C_3$ ($\Delta S$) rappresenta la \textbf{tangente $\alpha$}, che in questo caso rappresenta la \textbf{velocità media}. Ora, se restringiamo l'intervallo di $t$ in modo che tenda a 0 e calcoliamo il valore della derivata in quel punto otterremo la velocità istantanea.

		\subsubsection{Accelerazione}
			Nel paragrafo precedente abbiamo visto che la \textbf{velocità in un punto corrisponde al valore della derivata prima} (della funzione che rappresenta il moto del nostro corpo) \textbf{in quel punto}, per quanto riguarda l'accelerazione abbiamo che \textbf{l'accelerazione corrisponde al rapporto tra la derivata della velocità e la derivata del tempo}, ottenendo quindi la formula $\frac{dv}{dt}$, operativamente dobbiamo fare la \textbf{derivata seconda della funzione che rappresenta il moto del nostro punto}.

		\subsubsection{Esercizi vari sui moti con formule}
			Vediamo alcuni esempi:
			\paragraph{Esempio 1}
				Supponiamo di avere un oggetto che si sposta da un punto A ($t_0, S_0$) ad un punto B ($t_1, S_1$) tramite un \textbf{moto rettilineo uniforme}, abbiamo i seguenti dati:
				\begin{align*}
					&t_0 = ?&&S_0=1,5Km&&v=36m/s\\
					&S_1 = 11,5Km&&t_1=0,3h
				\end{align*}
				L'obiettivo è trovare i dati mancanti (ovvero $t_0$). Noi sappiamo che la velocità "v" corrisponde a:
				\begin{align*}
					v=\frac{\Delta S}{\Delta t}=\frac{S_1-S_0}{t_1-t_0}=> ... => t_0 = t_1 -\frac{S_1-S_0}{v}
				\end{align*}
				Sostituendo i valori forniti, otteniamo che $t_0 \approx 802,22s$
