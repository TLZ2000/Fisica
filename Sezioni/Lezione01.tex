\section{Introduzione}
	\subsection{Il metodo sperimentale}
		Distingue discipline sperimentali da discipline non sperimentali. Si compone di diverse fasi:
		\begin{enumerate}
			\item \textbf{formulazione ipotesi}: si fa un'\textbf{ipostesi descrittiva} (in \textbf{linguaggio matematico}) della porzione di mondo che si vuole analizzare, di conseguenza si decide di \textbf{non considerare} altre caratteristiche del mondo che non centrano con l'ipotesi che stiamo formulando;
			\item \textbf{esperimento}: si va a ricreare una situazione dove l'aspetto che vogliamo analizzare è \textbf{sicuramente presente} e \textbf{influenzato il meno possibile da fattori esterni};
			\item \textbf{esecuzione dell'esperimento}: si verifica l'ipotesi, formulata in modo matematico, confrontando i valori ottenuti con l'esperimento con quelli che si ottengono dalla nostra ipotesi.
			\medskip\\
		\end{enumerate}

		In base alla "verifica" dell'ipotesi possiamo fare una differenziazione:
		\begin{itemize}
			\item \textbf{teoria}: l'ipotesi \textbf{non è ancora verificata}, o è verificata \textbf{parzialmente};
			\item \textbf{legge fisica}: l'ipotesi \textbf{è verificata} (in un certo ambito);
		\end{itemize}

\section{Cinetica dei punti}
	Descrive il movimento dei corpi.
	\subsection{Sistema di riferimento}
		Specifichiamo un sistema di riferimento per il seguente argomento:
		\framedImg{85}{L01-img001}
		Una cosa importante da notare è che un numero singolo può rappresentare solo cose "\textbf{mono-dimensionali}" e che, soprattutto, non tutte le unità di misura possono rappresentare qualsiasi cosa (ad es.: l'età dell'universo non si può rappresentare con i metri).

	\subsection{Diagramma dello spazio}
		Rappresentiamo lo spostamento nel tempo tramite un "\textit{\textbf{diagramma dello spazio}}":
		\framedImg{40}{L01-img002}
		In particolare, in questo diagramma rappresentiamo sull'asse Y lo \textbf{spostamento} (s) (rappresentato come \textbf{valore uni-dimensionale}) e sull'asse X il \textbf{tempo} (t) (anche rappresentato come \textbf{valore uni-dimensionale}). \importante{Nota che il diagrmma NON RAPPRESENTA una posizione, ma lo spostamento in relazione al tempo}.

	\subsection{Caso semplice}
		Vediamo un semplice caso di utilizzo per capire come usare i diagrammi dello spazio:
		\framedImg{40}{L01-img003}
		Possiamo immaginare di avere un oggetto in movimento su una retta tra i punti A e B, come possiamo rappresentare questo movimento nel diagramma? Come prima cosa posizioniamo i "fenomeni" (\definizione{qualcosa che appare evidente all'osservazione}{fenomeno}), ovvero i \textbf{punti A e B}, nota che non è detto che questi punti coincidano con dei "punti particolari" (ad esempio l'origine) nel nostro diagramma. In particolare, a questi punti associamo \textbf{un valore sull'asse del tempo ($t_i$, $t_f$)} ed \textbf{un valore sull'asse dello spazio ($s_i$, $s_f$)}.
		A questo punto esistono \textbf{infiniti possibili percorsi} tra il punto A ed il punto B, ad esempio:
		\framedImg{40}{L01-img004}
		Importante notare che \importante{non tutti questi percorsi, pur avendo senso matematico, hanno senso fisico}! Ad esempio, il percorso in rosso "torna indietro nel tempo"!

	\subsection{Moto rettilineo uniforme}
		STUB\#\#\#\#\#\#\#\#\#\#\#\#\#\#\#\#\#\#\# (In teoria lo fa dopo, controllare)

	\subsection{Velocità}
		Possiamo immaginare la velocità (\textit{v}) come la "\definizione{variazione dello spazio rapportato al tempo impiegato per percorrerlo}{velocità}", in particolare la velocità è data dalla formula:
		\begin{align*}
			v=\frac{s_f - s_i}{t_f - t_i} = \frac{\Delta s}{\Delta t}
		\end{align*}
		Vediamo un semplice esempio:
		\begin{align*}
			s_i = 400m,\ s_f = 700m,\ t_i = 7:30 = 450 min,\ t_f = 7:40 = 460 min
		\end{align*}
		\begin{align*}
			v=\frac{700m - 400m}{460min - 450min}=\frac{300m}{600s}=0,5m/s
		\end{align*}
		Nota che nella seconda uguaglianza nell'esempio abbiamo \textbf{convertito i minuti in secondi}, puoi immaginare che abbiamo posto "$min = (60s)$", quindi abbiamo fatto "$10 min = 10 * (60s)= 600s$".

		\subsubsection{Velocità istantanea}
			Quella che abbiamo calcolato prima possiamo vederla come "velocità media" di tutto il percorso, la \textbf{velocità istantanea} invece possiamo vederla come la \definizione{velocità in un punto specifico del percorso}{velocità istantanea}. Immagina quindi di fare la formula:
			\begin{align*}
				v_{ist}=lim_{\Delta t\rightarrow 0} \frac{\Delta s}{\Delta t} = \frac{\delta s}{\delta t}
			\end{align*}
			Nota che quando si usa la lettera "$\delta$" stiamo ad indicare una \textbf{piccola} (infinitesima) \textbf{variazione}. Ora, se il valore di $s$ viene espresso \textbf{in funzione di t}, quindi abbiamo $s(t)$, e la funzione "$s(t)$" è \textbf{derivabile}, allora la \textbf{velocità istantanea corrisponde alla derivata prima della funzione $s(t)$}, che a sua volta corrisponde a $\frac{ds}{dt}$.
			\framedImg{40}{L01-img005}
			Supponendo che il \textbf{moto del nostro punto} venga identificato dalla curva in verde, il rapporto tra la lunghezza dei 2 cateti $C_1C_2$ ($\Delta t$) e $C_2C_3$ ($\Delta s$) rappresenta la \textbf{tangente $\alpha$}, che in questo caso rappresenta la \textbf{velocità media}. Ora, se restringiamo l'intervallo di $t$ in modo che tenda a 0 e calcoliamo il valore della derivata in quel punto otterremo la velocità istantanea.

		\subsubsection{Accelerazione}
			Nel paragrafo precedente abbiamo visto che la \textbf{velocità in un punto corrisponde al valore della derivata prima} (della funzione che rappresenta il moto del nostro corpo) \textbf{in quel punto}, per quanto riguarda l'accelerazione abbiamo che \textbf{l'accelerazione corrisponde al rapporto tra la derivata della velocità e la derivata del tempo}, ottenendo quindi la formula $\frac{dv}{dt}$, operativamente dobbiamo fare la \textbf{derivata seconda della funzione che rappresenta il moto del nostro punto}.

		\subsubsection{Moto rettilineo uniformemente accellerato}
			Cominciamo col dire che:
			\begin{align*}
				a=\frac{dv}{dt}
			\end{align*}
			Ricorda che con $dv$ e $dt$ intendiamo le \textbf{derivate}. Da questa ricaviamo $dv$, ovvero:
			\begin{align*}
				dv = a*dt\ => \int_A^B dv = \int_A^B (a*dt)\ => v_B-v_A = a(t_B-t_A)
			\end{align*}
			Da questo otteniamo quindi che la velocità in funzione del tempo corrisponde a:
			\begin{align*}
				\underline{v(t)=v_0+a(t-t_0)}
			\end{align*}
			Ottenuta questa formula, possiamo passare a calcolare lo \textbf{spazio in funzione del tempo}, ovvero:
			\begin{align*}
				&v(t) = \frac{ds}{dt}\ =>\ ds = v*dt\ =>\ \int_A^B ds = \int_A^B v*dt\ =>\ s_B-s_A = \int_A^B [v_0+a(t-t_0)] dt\ =>\ \\
				&=>\ s_B-s_A = \bigg[v_0*t+a\frac{(t-t_0)^2}{2}\bigg]_A^B\ =>\ s_B-s_A = v_0*t_B+a\frac{(t_B-t_0)^2}{2} - v_0*t_A+a\frac{(t_A-t_0)^2}{2}
			\end{align*}
			Da questo otteniamo quindi che la velocità in funzione del tempo corrisponde a:
			\begin{align*}
				\underline{s(t)=s_0+v_0(t-t_0)+\frac{1}{2}a(t-t_0)^2}
			\end{align*}
			Terminiamo dicendo che in questo moto \textbf{l'accelerazione è costante}, quindi:
			\begin{align*}
				\underline{a(t)= a}
			\end{align*}

		\subsubsection{Esercizi vari sui moti con formule}
			Vediamo alcuni esempi:
			\paragraph{Esempio 1 (moto rettilineo uniforme)}
				Supponiamo di avere un oggetto che si sposta da un punto A ($t_0, s_0$) ad un punto B ($t_1, s_1$) tramite un \textbf{moto rettilineo uniforme}, abbiamo i seguenti dati:
				\begin{align*}
					&t_0 = ?&&s_0=1,5Km&&v=36m/s\\
					&s_1 = 11,5Km&&t_1=0,3h
				\end{align*}
				L'obiettivo è trovare i dati mancanti (ovvero $t_0$). Noi sappiamo che la velocità "v" corrisponde a:
				\begin{align*}
					v=\frac{\Delta s}{\Delta t}=\frac{s_1-s_0}{t_1-t_0}=> ... => t_0 = t_1 -\frac{s_1-s_0}{v}
				\end{align*}
				Sostituendo i valori forniti, otteniamo che $t_0 \approx 802,22s$

			\paragraph{Esempio 2 (moto rettilineo uniformemente accellerato)}
				Supponiamo di avere un oggetto all'altezza $h_0$ e di lanciarlo verso l'alto con una velocità $v_0$ nell'istante $t_0$ con un'accelerazione $a$. Dobbiamo trovare l'altezza ($h_c$) ed il tempo ($t_c$) di culmine e, supponendo che alla fine l'oggetto raggiunga l'altezza finale "$h_f$", trovare il tempo finale "$t_f$". Supponiamo di avere i seguenti dati:
				\begin{align*}
					&h_0 = 100m&&t_0=0s&&v_0=5m/s&&a=-9,8m/s^2\\
					&t_c = ? && h_c=? && t_f = ? && h_f = 0m
				\end{align*}
				Includiamo delle immagini complementari:
				\framedImg{20}{L02-img001}
				\framedImg{50}{L02-img002}

				Procediamo per punti:
				\begin{enumerate}
					\item Vogliamo trovare il tempo di culmine ($t_c$), quindi poniamo $v(t)=0$ e troviamo la $t$ che rende vera l'equazione:
					\begin{align*}
						v(t)=0\ =>\ v_0+a(t-0)\ =>\ t_c=-\frac{v_0}{a} = -\frac{5m/s}{-9,8m/s^2}\approx0,51s
					\end{align*}
					\item Vogliamo calcolare l'altezza di culmine ($h_c$), per farlo usiamo la formula dello spazio:
					\begin{align*}
						&h_c = s(t_c)=s_0+v_0(t_c-0)+\frac{1}{2}a(t_c-0)^2=\\
						&=100m + 5m/s * (0,51s)+1/2(-9,8m/s^2)*(0,51s)^2\approx 101,28m
					\end{align*}
					\item Vogliamo calcolare il tempo "finale" ($t_f$), per farlo usiamo sempre la formula dello spazio:
					\begin{align*}
						&s(t_f)= h_f = 0\ =>\ \\
						&=>\ s_0+v_0(t_f-0)+\frac{1}{2}a(t_f-0)^2=0
					\end{align*}
					A questo punto abbiamo una funzione di secondo grado con $x = t_f$, quindi usiamo la formula solita:
					\begin{align*}
						&t_{f\ 1/2}=-\frac{v_0}{a}\pm\sqrt{(-\frac{v_0}{a})^2-2\frac{s_0}{a}}\\
						&t_f = 0,51s + \sqrt{(0,51 s)^2-2*\frac{100m}{-9,8m/s^2}} \approx 5,06s
					\end{align*}
					Nota che possiamo subito sostituire il "$\pm$" con un "$+$" dato che la radice sarà sicuramente più grande di quel $0,51$ che la precede, quindi non avrebbe fisicamente senso fare altrimenti (tempo negativo).
				\end{enumerate}

		\subsection{Moto armonico}
			Nel moto armonico abbiamo un'\textbf{accelerazione oscillante}, nella forma \underline{$a_0*sin(t)$}. Il problema è che il sin (come tutte le funzioni matematiche) è adimensionale, quindi dobbiamo aggiungere delle componenti aggiuntive per \textbf{rendere il tempo "t" adimensionale}, in paricolare abbiamo che:
			\begin{align*}
				&a(t) = a_0 * sin(\omega t + \varphi)
			\end{align*}
			dove "$\omega$" rappresenta la \textbf{pulsazione} e "$\varphi$" la \textbf{fase}. Nota che \textbf{abbiamo già l'accelerazione}, ovvero $a_0*sin(t)$, quindi per calcolare velocità e spazio procediamo per \textbf{integrazioni successive}, con gli estremi di integrazione che corrispondono al \textbf{punto di inizio e di fine} della nostra misurazione.
			\begin{align*}
				&v(t)= v_0 +\int_{t_0}^ta(\tau) d\tau = v_0 + \frac{1}{\omega} \int_{t_0}^t \omega a_0 sin(\omega t+\varphi)d\tau=\\
				&=v_0 + \frac{1}{\omega} \bigg[-cos(\omega t+\varphi)\bigg]_{t_0}^t=\textcolor{Red}{v_0} - \frac{a_0}{\omega} cos(\omega t+\varphi) + \textcolor{Red}{\frac{a_0}{\omega} cos(\omega t_0+\varphi)}=\\
				&=\textcolor{Red}{V}-\frac{a_0}{\omega} cos(\omega t+\varphi)
			\end{align*}
			Nota che il \textcolor{Red}{testo in rosso sopra}, in quanto costante, viene raccolto in \textit{\textcolor{Red}{V}}, passiamo ora a calcolare lo spazio (che corrisponde all'integrazione della velocità):
			\begin{align*}
				&s(t)= s_0 +\int_{t_0}^tv(\tau) d\tau =\\
				&=\textcolor{Red}{s_0} + V(t-t_0) - \frac{a_0}{\omega^2} sin(\omega t+\varphi) + \textcolor{Red}{\frac{a_0}{\omega^2} sin(\omega t_0+\varphi)}=\\
				&=\textcolor{Red}{S} + V(t-t_0) -\frac{a_0}{\omega^2} sin(\omega t+\varphi)
			\end{align*}

			In definitiva, le formule che interessano a noi sono:
			\begin{align*}
				&a(t)=a_0 * sin(\omega t + \varphi)\\
				&v(t)=\textcolor{Red}{V}-\frac{a_0}{\omega} cos(\omega t+\varphi)\\
				&s(t)=\textcolor{Red}{S} + V(t-t_0) -\frac{a_0}{\omega^2} sin(\omega t+\varphi)\\
			\end{align*}
			Ricorda che \textcolor{Red}{le parti in rosso} sono costanti (di solito per noi varranno 0), mentre l'accelerazione ci è stata fornita all'intizio, quindi teniamo quella. Vediamo un "esempio":

			\subsubsection{Esempio di moto armonico}
				Ipotiziamo di avere una situazione del genere: vogliamo misuare l'andamento dell'ombra di un'altalena (che va solo avanti e indietro) sulla superficie.
				\framedImg{25}{L02-img003}
				Noi \importante{assumiamo sempre che $\varphi$ (ovvero la fase)$ = 0$} e che \importante{cominciamo da $t_0 = 0$}, quindi le nostre formule diventano:
				\begin{align*}
					&a(t)=a_0 * sin(\omega t)\\
					&v(t)=-\frac{a_0}{\omega} cos(\omega t)\\
					&s(t)=-\frac{a_0}{\omega^2} sin(\omega t)\\
				\end{align*}
				Prima di passare al grafico dobbiamo calcolare il valore della nostra variabile $t$, ora noi sappiamo che $\omega t$, dato che $\varphi = 0$, deve rappresentare una rotazione completa ($2\pi$):
				\begin{align*}
					&\omega t = 2\pi\ =>\ t = \frac{2\pi}{\omega} = T
				\end{align*}
				Nota che il nostro $T$ rappresenta il \textbf{periodo}. Con queste funzioni/variabili, possiamo passare al calcolo dei grafici temporali:
				\framedImg{45}{L02-img004}

		\subsection{I moti piani}
			Prima di partice con i moti veri e propri, introduciamo velocemente i \textbf{vettori}.

			\subsubsection{I vettori}
				Passiamo ora a considerare i \textbf{moti con 2 dimensioni}, per questo motivo dobbiamo introdurre i \textbf{vettori} composti da:
				\begin{itemize}
					\item punto di inizio;
					\item verso;
					\item modulo (la lunghezza del vettore);
					\item direzione (la retta su cui giace il vettore);
				\end{itemize}

				I vettori, si comportano in modi leggermente diversi rispetto ai numeri "normali", in particolare a noi interessa:
				\begin{itemize}
					\item somma: si fa con la \textbf{regola del parallelogramma}, ovvero:
					\framedImg{70}{L02-img005}
					\item prodotto per scalare: quando si moltiplica un vettore per uno scalare, semplicemente si va a \textbf{moltiplicare il modulo del vettore}, in particolare "$\vec{a} = b * \vec{c} => |\vec{a}|= b * |\vec{c}|$"
				\end{itemize}

			\subsubsection{Sistema di riferimento}
				Introduciamo ora il sistema di riferimento per questo moto:
				\framedImg{45}{L03-img001}
				Da questo punto in poi, rappresentiamo il moto sul \textbf{piano cartesiano}: rappresenteremo quindi il \textbf{movimento "fisico"} del moto in quanto \textbf{non più unidimensionale!} Per quanto riguarda gli assi, si usano quelli che vengono definiti \textbf{versori} che matematicamente si rappresentano come $\hat{x} = \vec{x}/|\vec{x}|$. In questo modo otteniamo qualcosa di \textbf{adimensionale} e che ha \textbf{modulo 1 per definizione}.
				\medskip\\
				Quando vogliamo rappresentare un punto, possiamo farlo \textbf{attraverso un vettore}, che a sua volta si può rappresentare come la \textbf{somma di 2 vettori "unidimensionali"} (uno per ogni asse) che a loro volta si possono rappresentare come \textbf{spostamenti sui vari assi moltiplicati per il versore associato}:
				\begin{align*}
					\vec{P} = \vec{P}_x + \vec{P}_y = S_x * \hat{x} + S_y * \hat{y}
				\end{align*}
				\framedImg{45}{L03-img002}
				Allo stesso modo possiamo rappresentare velocità ed accelerazione!



			\subsection{Moto circolare uniforme}

			%IMMAGINE (moto circolare uniforme)

				Nel moto circolare uniforme la velocità è \textbf{costante}, infatti in tutti i punti la velocità \textbf{non} cambia lunghezza, cambia solo la sua direzione. La velocità in un punto, inoltre, è \textbf{sempre tangente} alla traiettoria in quel punto.
				Si hanno le seguenti formule:
				\begin{align*}
					&\alpha(t)=\alpha_0 + \omega t
					&&r(t)=R\\
				\end{align*}
				dove $R$ è il raggio della circonferenza e $\omega$ è la velocità angolare, definita come $\omega=\frac{2\pi}{T}$ con $T$ periodo.
				La \textbf{frequenza} è definita come $f = \frac{1}{T}$ e la \textbf{velocità} come:
				\begin{align*}
					&v=\frac{2\pi R}{T}=\omega R
				\end{align*}
				Osservando la formula qui sopra, possiamo notare che a meno che il raggio $R$ non cambi, la velocità sarà costante, nel caso in cui cambia, invece, cambierà anche la velocità.
				Ora, ponendo l'origine del piano cartesiano come il centro della circonferenza, calcoliamo lo \textbf{spazio}, la \textbf{velocità} e l'\textbf{accelerazione} in funzione del tempo come segue:
				\begin{align*}
					&\vec{s}(t)=
					\begin{cases}
						x(t)=Rcos(\alpha(t))\\\\
						y(t)=Rsin(\alpha(t))\\
					\end{cases}\\\\
					&\vec{v}(t)=
					\begin{cases}
						v_x(t)=-R\frac{d\alpha}{dt}sin(\alpha(t))=-\omega Rsin(\alpha(t))\\\\
						v_y(t)=R\frac{d\alpha}{dt}cos(\alpha(t))=\omega Rcos(\alpha(t))\\
					\end{cases}\\\\
					&\vec{a}(t)=
					\begin{cases}
						a_x(t)=-\omega^2 Rcos(\alpha(t))\\\\
						a_y(t)=-\omega^2 Rsin(\alpha(t))\\
					\end{cases}\\
				\end{align*}

			%IMMAGINE (moto circolare uniforme accelerazione)

				Ora l'accelerazione si divide in due componenti, $\vec{a_t}$ (\textbf{accelerazione tangente}) e $\vec{a_n}$ (\textbf{accelerazione normale o centripeta}). La prima è parallela alla tangente nel punto e modifica il \textbf{modulo della velocità}, mentre la seconda è ortogonale alla tangente nel punto e modifica la \textbf{traiettoria(direzione)}.
				Dato un qualsiasi \textbf{moto piano}, se prendo una piccola parte di questo, allora può essere immaginato come un arco di circonferenza. Più il tratto è \textbf{grande} e più l'arco di circonferenza sembrerà \textbf{rettilineo}, più il tratto è \textbf{piccolo} e più l'arco di circonferenza sembrerà \textbf{curvo} e avrò corrispondentemente una circonferenza \textbf{grande} e una \textbf{piccola}.

				%IMMAGINE (moto piano come moto circolare uniforme)

				In questo caso avrò quindi che le due componenti dell'accelerazione varranno:
				\begin{align*}
					&|\vec{a_n}|=\frac{v^2}{R}\\\\
					&|\vec{a_t}|=\vec{a}\frac{\vec{v}}{|\vec{v}|}=\vec{a}\hat{v}
				\end{align*}
				dove $\hat{v}$ è il \textbf{versore velocità}.
