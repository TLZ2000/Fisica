\section{Meccanica}
    \subsection{Lavoro}
        \framedImg{50}{L08-img001}
        Supponiamo che ci sia un corpo che si muove lungo una traiettoria. Ora prendiamo un punto d'origine del sistema di riferimento e descriviamo il movimento del corpo con dei raggi vettore dall'origine. Ora consideriamo un punto iniziale $p_i$ ed uno finale $p_f$ molto vicini tra loro e quindi otteniamo uno spostamento infinitesimale, $d\vec{s}=\vec{r_f} - \vec{r_i}$.\\
        Sul corpo inoltre agisce una forza $\vec{F}$ che è quella che lo fa spostare. Osserviamo ora che a seconda del valore di questa forza, potrebbe variare la velocità dell'oggetto. Se $d\vec{s}\perp\vec{F}$, allora il corpo non rallenta, quindi se l'angolo è $90_{\circ}$, l'effetto sarà minimo, mentre se l'angolo è $0_{\circ}$ l'effetto sarà massimo e infine se l'angolo è $180_{\circ}$ l'effetto sarà massimo ma in senso opposto.\\
        In conclusione quindi l'effetto che la forza ha sul corpo è descritta dal coseno ed è definito come \textbf{lavoro infinitesimo}, in formula:
        \begin{align*}
            dW=|\vec{F}||d\vec{s}|cos\alpha_{F,ds}
        \end{align*}
        Se il lavoro infinitesimo è positivo, allora il corpo accellera, se è negativo invece, il corpo rallenta. L'iterazione quindi tra il corpo e l'agente esterno, ovvero quello che applica la forza, è definita come energia. In questo caso,l'agente esterno perde energia.\\
        \begin{mdframed}
            Osserviamo ora che:
            \begin{align*}
                |\vec{v}||\vec{w}|cos(\alpha_{v,w})=\vec{v}\cdot\vec{w}
            \end{align*}
            dove "$\cdot$" rappresenta il prodotto scalare tra due vettori.
            Spesso questo viene detto $v_T w$, ovvero la componente di $v$ tangente a $w$, quindi:
            \begin{align*}
                |\vec{v}|cos(\alpha_{v,w})=v_T\\
                |\vec{w}|cos(\alpha_{v,w})=w_T
            \end{align*}
        \end{mdframed}
        Ora usiamo quindi questa osservazione e otteniamo:
        \begin{align*}
            dW=\vec{F}\cdot d\vec{s}
        \end{align*}
        Osserviamo che l'unità di misura del lavoro è il Joule, $J = 1 Nm$. Una caloria sono, $1cal=4.18J$\\.
        Ora calcoliamo il \textbf{lavoro medio} come segue:
        \begin{align*}
            \overline{W}=\vec{F}\cdot\Delta\vec{s}
        \end{align*}
        Osserviamo che il lavoro è una quantità scalare, dato che il prodotto scalare prende due vettori e li trasforma in uno scalare.
        \begin{mdframed}
            Facciamo ora un'esempio molto semplice:
            \begin{align*}
                |\vec{F}| = 10N
                |\Delta\vec{s}| = 100m
            \end{align*}
            Ora in base all'angolo $\alpha_{F,ds}$ abbiamo dei diversi valori di lavoro:
            \begin{center}
                \begin{tabular}{ |c|c| }
                    \hline
                    $\alpha_{F,ds}$ & $W$ \\
                    \hline
                    $0$ & $1000J$ \\
                    \hline
                    $\pi$ & $-1000J$ \\
                    \hline
                    $\pi/2$ & $0J$ \\
                    \hline
                    $\pi/4$ & $707J$ \\
                    \hline
                \end{tabular}
            \end{center}
        \end{mdframed}
        Ora consideriamo tutti gli spostamenti infinitesimi e supponiamo che ogni spostamento infinitesimo abbia un valore di forza diversa (ovviamente la forza è esercitata sempre dallo stesso agente esterno, ma in istanti diversi). Abbiamo quindi per ogni spostamento infinitesimale un lavoro infinitesimale diverso. Rappresentiamo questa cosa come segue:
        \begin{align*}
            &d\vec{s_1},d\vec{s_2},d\vec{s_3},...,d\vec{s_n}\\
            &\vec{F_1},\vec{F_2},\vec{F_3},...,\vec{F_n}\\
            &dW_1,dW_2,dW_3,...,dW_n
        \end{align*}
        Il \textbf{lavoro} quindi sarà la somma di tutti questi lavori infinitesimi:
        \begin{align*}
            W_{TOT}=\sum_{i=1}^n dW_i
        \end{align*}
        però considerando che la n tende a infinito, $n\rightarrow\infty$, otteniamo:
        \begin{align*}
            W_{i\rightarrow f}=\int_i^n dW_i=\int_i^n \vec{F}\cdot d\vec{s}
        \end{align*}
        Osserviamo che la $\vec{F}$ si può tirare fuori dall'integrale solo se non varia per nessun spostamento infinitesimale.\\\\
        Ora consideriamo il lavoro $W>0$, ovvero per un angolo compreso tra $[0,\pi/2]$, ovvero in cui la forza sta aiutando il moto/movimento, in questo caso il lavoro si chiama \textbf{lavoro motore}.\\
        Se invece il lavoro $W<0$, ovvero per un angolo compreso tra $[\pi/2,\pi]$, ovvero in cui la forza non sta aiutando il moto/movimento, in questo caso il lavoro si chiama \textbf{lavoro resistente}.\\\\

    \subsection{Esempi di lavoro}
        \subsubsection{Lavoro forza peso}
            Data l'equazione della forza peso:
            \begin{align*}
                \vec{F_p}=-mh\hat{h}
            \end{align*}
            abbiamo lavoro infinitesimo:
            \begin{align*}
                dW=\vec{F_p}\cdot d\vec{h}=-mh\hat{h}\cdot dh\hat{h}=-mgdh
            \end{align*}
            Osserviamo che $\hat{h}\cdot\hat{h}=1$.
            Ora quindi sommando tutti questi lavori infinitesimi otteniamo il lavoro $W_p$:
            \begin{align*}
                W_p=\int_i^f dW=\int_i^f -mgdh=-mg(h_f-h_i)=-mg\Delta h
            \end{align*}
            Osserviamo ora che se $\Delta h>0$, ovvero $h_f>h_i$, avremo $W_p<0$, quindi lavoro resistente, mentre se $\Delta h<0$, ovvero $h_f<h_i$, avremo $W_p>0$, quindi lavoro motore.\\
            Questo era il caso di spostamento verticale, se invece lo spostamento non è verticale il lavoro è comunque lo stesso.\\
            Osserviamo ora che $\vec{F}$ e $d\vec{s}$ non dipendono dal sistema di riferimento, se però consideriamo la forma $\vec{F_p}=-mg\hat{h}$, questa dipende dal sistema di riferimento dato che c'è $\hat{h}$.\\
            Osserviamo inoltre che il lavoro della forza peso dipende solo dal punto iniziale e dal punto finale e non dal percorso.
            \begin{mdframed}
                Facciamo ora un'esercizio molto semplice:
                \begin{align*}
                    &m = 100kg\\
                    &\Delta h = 1000m\\
                    &W_p=-9.8\frac{m}{s^2}\cdot 100kg \cdot 1000m =-9.8\cdot 10^5kg\frac{m^2}{s^2}=-9.8\cdot 10^5J=-9.8\cdot 10^2KJ
                \end{align*}
            \end{mdframed}

        \subsubsection{Lavoro forza elastica}
            Data l'equazione della forza elastica:
            \begin{align*}
                \vec{F_{el}}=-K\vec{x}
            \end{align*}
            abbiamo lavoro:
            \begin{align*}
                W_{el}=\int_i^f F_{el}dx=\int_i^f -Kxdx=-\frac{K}{2}({x_f}^2 - {x_i}^2)
            \end{align*}
            Osserviamo ora che se ${x_f}^2>{x_i}^2$, avremo $W_{el}<0$, infatti allungo la molla che si oppone, mentre se ${x_f}^2<{x_i}^2$, avremo $W_{el}>0$, infatti accorcio la molla che aiuta.\\
            Osserviamo inoltre che il lavoro della forza elastica dipende solo dal punto iniziale e dal punto finale e non dal percorso.

        \subsubsection{Lavoro forza attrito}
            Data l'equazione della forza d'attrito:
            \begin{align*}
                \vec{F_{att}}=-\mu_dN=-\mu_dmg
            \end{align*}
            abbiamo lavoro:
            \begin{align*}
                W_{att}=\int_i^f F_{att}dx=F_{att}\int_i^f dx=-\mu_dmg\int_i^f dx=-\mu_dmgL
            \end{align*}
            dove $L$ è la lunghezza dello spostamento ($x_f - x_i$).\\
            Osserviamo che l'attrito si oppone sempre al movimento, quindi non vale quello che valeva per il lavoro della forza peso e della forza elastica, ovvero che a seconda del punto finale e del punto iniziale il lavoro poteva essere positivo o negativo.

    \subsection{Energia cinetica}
        Consideriamo ora il lavoro infinitesimo,
        \begin{align*}
            dW=\vec{F}\cdot d\vec{s}= m\frac{d\vec{v}}{dt}d\vec{s}= md\vec{v} \frac{d\vec{s}}{dt}=md\vec{v}\cdot\vec{v}
        \end{align*}
        e quindi,
        \begin{align*}
            dW=d[\frac{1}{2}mv^2]
        \end{align*}
        L'\textbf{energia cinetica} è la seguente quantità:
        \begin{align*}
            E_k=\frac{1}{2}mv^2
        \end{align*}
        e quindi ho che:
        \begin{align*}
            dW=d[E_k]
        \end{align*}
        \subsubsection{Teorema delle forze vive}
            Il \textbf{Teorema delle forze vive} afferma che se un corpo possiede un'energia cinetica iniziale e una forza agisce su di esso effettuando un lavoro, l'energia cinetica finale del corpo è uguale alla somma dell'energia cinetica iniziale e del lavoro compiuto dalla forza lungo la traiettoria del moto. In formula:
            \begin{align*}
                E_{k,f}=W_{i\rightarrow f}+E_{k,i}=>W_{i\rightarrow f}=E_{k,f}-E_{k,i}
            \end{align*}
            Osserviamo ora che se $v_i=0$ e $v_f=0$, allora per come è formulata $E_k$, vorrà dire che $E_{k,i}=0$ e $E_{k,f}=0$ e quindi $W_{i\rightarrow f}=0$.

    \subsection{Potenza}
        La \textbf{potenza} è definita come:
        \begin{align*}
            P=\frac{dW}{dt}=\frac{\vec{F}\cdot d\vec{s}}{dt}=\vec{F}\cdot\vec{v}
        \end{align*}
        L'unità di misura della potenza è il Watt, $W=1\frac{J}{s}$.

    \subsection{Forze conservative}
        Una \textbf{forza conservativa} è una forza per cui il lavoro dipende solamente dal punto iniziale, $p_i$, e dal punto finale, $p_f$.\\
        Se vado quindi da $p_i$ a $p_f$ seguendo due percorsi diversi il lavoro sarà lo stesso.\\
        Se invece vado da $p_f$ a $p_i$, il lavoro sarà sempre uguale ma avrà segno opposto, indipendentemente dal percorso. Questo è descritto come segue,
        \begin{align*}
            \oint \vec{f}\cdot d\vec{s}=0
        \end{align*}
        ovvero l'integrale chiuso.

        \subsubsection{Energia potenziale}
            L'\textbf{energia potenziale} di un corpo è l'energia che esso possiede a causa della sua posizione o del suo orientamento rispetto ad un campo di forze ed è denotata con $U$.

        \subsubsection{Energia meccanica}
            L'\textbf{energia meccanica} è la somma di energia cinetica ed energia potenziale ovvero,
            \begin{align*}
                E=E_k+U
            \end{align*}

        \subsubsection{Principio di conservazione dell'energia meccanica}
            Consideriamo il caso 1-DIM,
            \begin{align*}
                \int_{p_i}^{p_f} F dx = U(p_f) - U(p_i) = W_{p_i\rightarrow p_f}
            \end{align*}
            Ora per il teorema delle forze vive ho che,
            \begin{align*}
                W_{p_i\rightarrow p_f}=E_{k,f} - E_{k,i}
            \end{align*}
            e quindi, usando una notazione semplificata, ovvero per esempio invece che $U(p_f)$ uso $U_f$, ottengo che,
            \begin{align*}
                U_f - U_i=E_{k,f} - E_{k,i} => U_f - E_{k,f}=U_i - E_{k,i} => E_f = E_i
            \end{align*}
            ovvero l'energia meccanica iniziale è uguale a quella finale, e quindi
            \begin{align*}
                \Delta E = 0
            \end{align*}
            ovvero la variazione di energia meccanica è $0$, quindi nel caso in cui si hanno solo forze conservative, l'energia meccanica non varia, si conserva.\\
            Se ora consideriamo,
            \begin{align*}
                -(U(p_f) - U(p_i)) = W_{p_i\rightarrow p_f}
            \end{align*}
            ed deriviamo, otteniamo che,
            \begin{align*}
                F=- \frac{dU}{ds}
            \end{align*}
            Nel caso a più dimensioni invece, per esempio consideriamo 3-DIM, abbiamo che,
            \begin{align*}
                F=- \nabla U =-(\frac{\partial U}{\partial x}\frac{\partial U}{\partial y}\frac{\partial U}{\partial z})
            \end{align*}

        \subsubsection{Esercizio forze conservative}
            Consideriamo ora l'esercizio fatto in passato [pag.\pageref{esercizioDinamicaMolla}].\\
            Tutte le forze che agiscono sul sistema sono conservative, quindi possiamo usare il principio di conservazione dell'energia meccanica e risolverlo in modo molto più semplice.\\
            Ora abbiamo quindi che le forze che agiscono sul sistema sono,
            \begin{align*}
                \vec{F}_p=-mg\hat{z}
                \vec{F}_el=-k(z-z_i)\hat{z}
            \end{align*}
            e considerando la conservazione dell'energia meccanica abbiamo che,
            \begin{align*}
                E_k+U=cost => E_k+U_p+U_{el}=cost
            \end{align*}
            Calcoliamo ora quindi le energie potenziali,
            \begin{align*}
                &U_p=-W_p=mg(z-z_0)\\
                &U_{el}=\frac{k}{2}(z^2-z_0^2)
            \end{align*}
            e quindi ora sostituisco nella formula precedente e ottengo,
            \begin{align*}
                \frac{1}{2}mv^2+mgz+\frac{k}{2}z^2=cost
            \end{align*}
            Ora considerando il sistema di riferimento scelto, nel punto iniziale abbiamo $v_i=0$ e $z=0$ e quindi, la formula precedente è $=0$.\\
            Considerando il punto $z_{max}$, ho sempre che $v=0$, e ottengo che,
            \begin{align*}
                mg z_{max}+\frac{k}{2} {z_{max}}^2=0 => z_{max}(\frac{k}{2}z_{max} + mg)=0
            \end{align*}
            Le due soluzioni sono quindi $z_{max}=0$, che non viene considerata, e
            \begin{align*}
                z_{max}=-\frac{2mg}{k}=\Delta z_{max}
            \end{align*}
            Ora osserviamo che $z_{v_{max}}$ è $z_{eq}$, ovvero la $z$ nel punto di equilibrio, e quindi,
            \begin{align*}
                z_{v_{max}}=z_{eq}=\frac{z_{max}}{2}=-\frac{mg}{k}
            \end{align*}
            Per calcolare la $v_{max}$ ora, consideriamo sempre il fatto che essa è in $z_{eq}$ e quindi,
            \begin{align*}
                &\frac{1}{2}mv_{max}^2+mg z_{eq}+\frac{k}{2}z_{eq}^2=0 &&=> \frac{1}{2}mv_{max}^2-\frac{(mg)^2}{k}+\frac{(mg)^2}{2K}=0\\
                & &&=>\frac{1}{2}mv_{max}^2-\frac{(mg)^2}{2K}=0\\
                & &&=>\frac{1}{2}mv_{max}^2=\frac{(mg)^2}{2K}\\
                & &&=>v_{max}^2=\frac{mg^2}{K}\\
                & &&=>v_{max}=\sqrt{\frac{mg^2}{K}}
            \end{align*}

    \subsection{Forze non conservative e lavoro}
        Consideriamo ora il fatto in cui sul sistema agiscono anche forze non conservative e ne calcoliamo il lavoro.\\
        Abbiamo quindi che,
        \begin{align*}
            &\vec{R}=\vec{R}_c + \vec{R}_{nc}\\
            &W_{TOT}=\int_i^f \vec{R}d\vec{s}=\int_i^f \vec{R}_cd\vec{s} + \int_i^f \vec{R}_{nc}d\vec{s}=W_c+W_{nc}
        \end{align*}
        Ora però sappiamo che,
        \begin{align*}
            &W_{TOT}=\Delta E_k=>W_c+W_{nc}=\Delta E_k=>W_{nc}=\Delta E_k - W_c
        \end{align*}
        ma ora ci ricordiamo che $W_c=-\Delta U$ e quindi,
        \begin{align*}
            W_{nc}=\Delta E_k+\Delta U=>W_{nc}=\Delta E
        \end{align*}
        Quindi se sul sistema agiscono delle forze non conservative, il lavoro di queste sarà uguale alla variazione di energia meccanica.\\
        Osserviamo ora quindi che se $W_{nc}<0$, vuol dire che $E_f-E_i<0$ e quindi $E_f<E_i$. Allo stesso modo, se $W_{nc}>0$, vuol dire che $E_f-E_i>0$ e quindi $E_f>E_i$, e inoltre $E_f=E_i+W_{nc}$.

        \subsubsection{Esempio lavoro forze non conservative}
            Supponiamo che un corpo di massa $m$ scali una montagna di altezza $h_{max}$ partendo da $h_0=0$. Ora non considerando gli attriti, abbiamo solo la forza peso che agisce sulla massa. Calcoliamo quindi,
            \begin{align*}
                &\Delta U_p=mgh_{max}+mg0=mgh_{max}\\
                &\Delta E_k=\frac{1}{2}m{v_f}^2-\frac{1}{2}m{v_i}^2=\frac{1}{2}m0^2-\frac{1}{2}m{0}^2=0\\
                &\Delta E=mgh_{max}=W_{nc}
            \end{align*}
            L'energia cinetica è $=0$ dato che sia la velocità iniziale che quella finale sono $=0$ e quindi l'energia meccanica sarà uguale all'energia potenziale, quest'ultima è uguale all'energia potenziale finale dato che, essendo l'altezza iniziale $=0$, anche l'energia potenziale iniziale $=0$.
