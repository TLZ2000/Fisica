\newpage
\section{Funzioni}
	Sostanzialmente, \definizione{una funzione f è composta da 3 elementi:
	\begin{itemize}
		\item un insieme di partenza A (dominio);
		\item un insieme di arrivo B (co-dominio);
		\item un insieme di regole che ad ogni elemento di $a \in A$ associa un (importante, uno solo) elemento $b \in B$ (b=f(a));
	\end{itemize}}
	In matematica, una funzione si identifica con la forma "$f: A -> B$", ovvero una funzione f che "parte" da A e "arriva" in B. Vediamo uno schema molto semplice:
	\framedImg{6}{L02-img001}
	
	\subsection{Grafico di una funzione}
		In genere, associamo una funzione ad un \textbf{grafico} rappresentato sul \textbf{piano cartesiano} con numeri su entrambi gli assi. Non sempre questo è possibili: come insieme di partenza/arrivo possiamo usare qualsiasi cosa, non solo numeri. Comunque, \definizione{un grafico di una funzione non è altro che un insieme ordinato $\{(a, b) \in AxB: b=(f()a)\} \subseteq AxB$}.
		
	\subsection{Iniettività e suriettività (surgettività)}
		Queste sono 2 proprietà importanti delle funzioni. Supponendo di avere $f:A->B$:
		\begin{itemize}
			\item \definizione{f è iniettiva se ogni elemento di A viene "mandato" su un elemento di B diverso}, in matematichese possiamo dire $a1\in A, a2\in A, a1\neq a2=>f(a1)\neq f(a2)$, ovvero il fatto che a1 e a2 siano diversi implica che anche i valori delle rispettive funzioni siano diversi. In modo equivalente, possiamo dire $a1\in A, a2\in A, a1=a2=>f(a1)=f(a2)$ (anche se è una cosa abbastanza ovvia :P);
			\item \definizione{f è suriettiva (o surgettiva) se ogni elemento di B (l'insieme di arrivo) è immagine di almeno un elemento di A}. Nota che la suriettività non implica anche iniettività, quindi 2 elemnti di A possono essere associati allo stesso elemento di B. In matematichese possiamo rappresentare questa proprietà come "$\forall b\in B \exists a\in A : b=f(a)$", ovvero \textit{per ogni elemento dell'insieme B esiste almeno un elemento dell'insieme A tale che b è immagine di a};
			\item \definizione{se f è iniettiva e suriettiva, possiamo dire che f è biettiva (o bigettiva)}.
		\end{itemize}
	
		\subsubsection{Alcuni esempi}
			Vediamo alcuni esempi interessanti:
			\framedImg{6}{L02-img002}
			In questo caso abbiamo una funzione sia \textbf{iniettiva} (ogni elemento di A punta ad elementi diversi di B) che \textbf{suriettiva} (tutti gli elementi di B sono puntati da almeno un elemento di A), in altre parole la funzione è \textbf{biettiva}.
			\framedImg{6}{L02-img003}
			Qui invece abbiamo una funzione iniettiva ma \textbf{non suriettiva} (un elemento di B non è puntato da alcun elemento di A);
			\framedImg{6}{L02-img004}
			Qui la funzione è suriettiva ma \textbf{non iniettiva} (2 elementi diversi si A puntano allo stesso elemento di B);
			\framedImg{6}{L02-img005}
			E finiamo con qualcosa che \textbf{non è una funzione}, questo per 2 motivi:
			\begin{itemize}
				\item Un elemento di A \textbf{punta a 2 elementi diversi di B} (questo non vuol dire che in matematica questo comportamente non sia permesso, semplicemente noi non lo studiamo);
				\item Un elemento di A \textbf{non punta ad alcun elemento di B};
			\end{itemize}
		
	\subsection{Biettività (bigettività)}
		Sostanzialmente, una funzione è \textbf{biettiva} se è \textbf{sia iniettiva che suriettiva}. Dando una definizione più formale, possiamo dire: \definizione{una funzione f:A->B è biettiva se e solo se (questo per indicare uan condizione necessaria e sufficiente) è invertibile, ovvero se e solo esiste una funzione g:B->A tale che "$g(f(a))=a, \forall a\in A$" e "$f(g(b))=b, \forall b\in B$"}. Queste ultime 2 condizioni sono delle "\textbf{composizioni di funzioni}", a far bene si dovrebbe scrivere "$((g \circ f)(a))$" per dire $g(f(a))$. Vediamo un veloce schema di una funzione inversa:		
		\framedImg{95}{L02-img006}
		Ora, è importante notare che \textbf{non tutte le funzioni hanno una funzione inversa}. Come detto prima, una funzione è biettiva iff (se e solo se) è sia iniettiva che suriettiva, basti pensare che \textbf{non tutte le funzioni sono sia iniettive che suriettive}.
		
	\subsection{Immagine e controimmagine}
		Sia $f:A->B$ una funzione e $C\subseteq A$:
		\framedImg{6}{L02-img007}
		Possiamo dire che \definizione{l'immagine di C è l'insieme dei punti di B raggiunti da frecce che partono dagli elementi di C}, più formalmente possiamo dire $f(C)=\{f(a):a\in C\}$ (nota che questo insieme è sottoinsieme di B). Volendo possiamo anche calcolare l'immagine dell'intero inseme A, non dobbiamo necessariamente fare un sottoinsieme.
		\medskip\\
		Il concetto di controimmagine è molto simile a quello di immagine. Sia $D\subseteq B$, possiamo dire che \definizione{la controimmagine di D è l'insieme di punti di A da cui partono frecce che arrivano in D}, più formalmente possiamo scrivere $f\textsuperscript{-1}(D)=\{a\in A:f(a)\in D\}$.
		\medskip\\
		Ora dobbiamo fare 2 osservazioni importanti:
		\begin{enumerate}
			\item la funzione f è suriettiva se e solo se $f(A)=B$, ovvero se l'\textbf{immagine di A corrisponde all'intero insieme B};
			\item per definire f\textsuperscript{-1}(D) \textbf{non è necessario che f sia invertibile}. Nota che f\textsuperscript{-1} in matematica viene usato per rappresentare la funzione inversa, ma noi la usiamo per definire la \textbf{controimmagine}.
		\end{enumerate}