\section{Dinamica}
    \subsection{Leggi della dinamica}
        La \textbf{dinamica} si occupa dello studio del moto dei corpi a partire dalle sue cause(\textbf{forze}), ovvero delle circostanze che lo determinano e lo modificano nel \textbf{tempo} e nello \textbf{spazio} del suo sistema di riferimento. \href{https://it.wikipedia.org/wiki/Dinamica}{Wikipedia}
        Le leggi della dinamica sono 3 e sono le seguenti:
        \begin{enumerate}
            \item \underline{\textbf{Legge di Inerzia (I legge)}}: un corpo rimane nel suo stato di quiete finchè non intervengono agenti esterni a modificarne questo stato. Questa legge vale sono in sistemi di riferimento inerziali;\\
            \item \underline{\textbf{Legge di Newton (II legge)\label{II_legge_dinamica}}}:
            Viene definita la \textbf{quantità di moto} come $\vec{p}=m\vec{v}$, ovvero massa per velocità. Successivamente viene definita la forza ($\vec{F}$) come segue:
            \begin{align*}
                \vec{F}=\frac{d\vec{p}}{dt}=\frac{d(m\vec{v})}{dt}=\frac{dm}{dt}\vec{v} + m\frac{d\vec{v}}{dt}
            \end{align*}
            dove $m$ è la \textbf{massa inerziale}, ovvero la capacità di un corpo di opporsi alle variazioni del suo stato di moto, questa mette in relazione la velocità alla forza.\\
            Nel caso in cui la massa non varia, allora la forza può essere definita come $\vec{F}=m\cdot\vec{a}$.\\
            L'unità di misura della \textbf{forza} è il Newton ($N$) definito come $\frac{kg\cdot m}{s^2}$.\\
            \item \underline{\textbf{Principio di azione e reazione (III legge)}}: Quando il corpo 1 esercita una forza $\vec{F}$ sul corpo 2, quest'ultimo esercita sul corpo 1 una forza $-\vec{F}$, uguale e opposta.
            \begin{align*}
                \vec{F}_{1\rightarrow 2}=-\vec{F}_{2\rightarrow 1}
            \end{align*}
        \end{enumerate}
        Osserviamo che la \textbf{prima legge} potrebbe sembrare un caso particolare della \textbf{seconda legge}, con $\vec{F}=\vec{0}$, ma in realtà non è così, infatti la seconda e la terza legge sono valide solo all'interno di sistemi di riferimento inerziali, che sono definiti dalla prima legge.

    \subsection{Forze impulsive}
        L'\textbf{impulso} $\vec{P}$ è definito come la variazione di quantità di moto $\Delta\vec{p}$ in un $\Delta t$ piccolo, ovvero:
        \begin{align*}
            \vec{P}=\Delta\vec{p}=\int_0^t{\vec{F}dt}
        \end{align*}
        E la \textbf{forza impulsiva} come:
        \begin{align*}
            \vec{F}_{imp}=\frac{\Delta\vec{p}}{\Delta t}
        \end{align*}

        \subsubsection{Esempio forze impulsive}
            Supponiamo di avere un pavimento ed una palla che viene lasciata in aria. Questa palla cadrà verso il pavimento fino a raggiungerlo, rimbalzare su esso e tornare in sù (assumiamo che la velocità con cui torna in sù sia la stessa con cui cade, quindi non agiscono fattori esterni come attriti, ecc.).

            \framedImg{35}{L04-img004}

            Se ho un vettore velocità $\vec{v}$, allora ho che:
            \begin{align*}
                &\vec{p_i}=m\vec{v_i}=-m\vec{v}\\
                &\vec{p_f}=m\vec{v_f}=m\vec{v}\\
                &\Delta\vec{p}=\vec{p_f}-\vec{p_i}=2m\vec{v}
            \end{align*}

        \subsubsection{Esercizio su forze impulsive}
            Supponiamo di avere i seguenti dati e di dover calcolare $\vec{F}_{imp}$ (forza impulsiva):
            \begin{align*}
                &m=98g&&v=10.2\frac{m}{s}&&\Delta t=100ms\\
            \end{align*}
            Procediamo ora quindi con calcolare $\Delta\vec{p}$ usando la formula appena calcolata sopra e una volta ottenuto il valore calcoliamo la $\vec{F}_{imp}$:
            \begin{align*}
                &\Delta\vec{p}=2m\vec{v}=2\cdot 10.2\frac{m}{s}\cdot 0.098kg=0.99\frac{kg\cdot m}{s}\\
                &\vec{F}_{imp}=\frac{\Delta\vec{p}}{\Delta t}=\frac{0.99\frac{kg\cdot m}{s}}{0.1s}=9.99\frac{kg\cdot m}{s^2}=9.99N
            \end{align*}

    \subsection{Esercizi sulla dinamica}
        Supponiamo di avere un'oggetto appeso a due fili, che sono appesi al tetto, alla stessa distanza dall'oggetto e vogliamo trovare $\vec{T_1}$ e $\vec{T_2}$ tensioni dei fili, avendo i seguenti dati:
        \begin{align*}
            &m=100g&&\theta=60^{\circ}\\
        \end{align*}

        \framedImg{60}{L04-img005}

        Notiamo che l'oggetto resta fermo, quindi oltre a $\vec{p}$ (\textbf{forza peso}), su esso agiscono altre forze la cui somma è uguale e opposta a $\vec{p}$. Abbiamo quindi che la \textbf{risultante delle forze} $\vec{R}=\vec{0}$.\\
        Ora possiamo notare che $|\vec{T_1}|=|\vec{T_2}|=T$ e abbiamo le seguenti forze:
        \begin{align*}
            &\vec{p}=-mg\hat{y}\\
            &\vec{T_1}=T_x\hat{x}+T_y\hat{x}=Tsin\theta \hat{x}+Tcos\theta \hat{y}\\
            &\vec{T_2}=-T_x\hat{x}+T_y\hat{y}=-Tsin\theta \hat{x}+Tcos\theta \hat{y}
        \end{align*}
        Ora ci ricordiamo che $\vec{R}=\vec{0}$ quindi:
        \begin{align*}
            &\vec{R}=\vec{0}=\vec{P}+\vec{T_1}+\vec{T_2}=>
            \begin{cases}
                &R_x=0\\
                &R_y=0
            \end{cases}=>\\
            &=>
            \begin{cases}
                &R_x=0=Tsin\theta-Tsin\theta\\
                &R_y=0=-mg+Tcos\theta+Tcos\theta=-mg+2Tcos\theta
            \end{cases}
            \end{align*}
                La prima equazione del sistema vale zero, ora dalla seconda ricaviamo $T$:
            \begin{align*}
            mg=2Tcos\theta=>T=\frac{mg}{2Tcos\theta}=\frac{0.1kg\cdot 9.8\frac{m}{s^2}}{2\cdot\frac{1}{2}}=0.98N
        \end{align*}

    \subsection{Forze fondamentali}
        Le forze fondamentali sono le seguenti (scritte in ordine di intensità, dalla più debole alla più intensa):
        \begin{itemize}
            \item \underline{\textbf{Forza gravitazionale}}: descrive l'interazione tra le masse gravitazionali. È una forza onnipresente, non esiste quindi nessuna componente di materia che ha massa, che non la risente;
            \item \underline{\textbf{Forza debole}}: è la forza responsabile dell'interazione per cui i nuclei cambiano di natura (e.g. decadimento dei nuclei);
            \item \underline{\textbf{Forza elettromagnetica}}: deriva dall'unificazione di forza elettrica e forza magnetica. È la forza responsabile della trazione della ripulsione di cariche ed è alla base delle forze che aggregano la materia;
            \item \underline{\textbf{Forza forte/nucleare}}: è la forza responsabile della stabilità dei nuclei. Inizialmente si è definita come \textbf{forza nucleare}, ovvero che descrive un'interazione nucleare, tra protoni e neutroni, tra neutroni e neutroni e tra protoni e protoni. Poi è stato scoperto che i protoni e i neutroni non sono particelle fondamentali ma sono fatti di quark, e l'interazione è sentita dai quark, quindi è stata definita \textbf{forza forte}.
            \\
        \end{itemize}

        La forza elettromagnetica e la forza debole unificate formano la \textbf{forza elettrodebole}.
        Osserviamo che il quadro precedente, ovvero l'ordine di intensità, è \textbf{attuale}, era diverso nel passato e lo sarà anche nel futuro.\\\\
        Quando due corpi/sistemi fisici interagiscono tra loro per una delle forze fondamentali, lo fanno perchè hanno una sensibilità a quel tipo di forza, che è detta \textbf{carica} ed è rappresentata dalla lettera $q$. Per la forza gravitazionale per esempio, un corpo/sistema fisico subisce una trazione gravitazionale se ha una \textbf{massa} (\textbf{carica gravitazionale}, $q_G$), ovvero la misura dell'inclinazione del corpo ad interagire con questa gravitazione. Per la forza elettromagnetica è uguale, se due oggetti sono neutri, non c'è trazione ne repulsione. E la stessa cosa vale per forza debole e forza forte.\\\\
        Quella che chiamiamo carica in senso comune, in realtà è la \textbf{carica elettrica} ($q_E$).


    \subsection{Forze}

        \subsubsection{Forza peso}
            La \textbf{forza peso} è descritta come segue, dato $\vec{g} = -g\hat{z}$:
            \begin{align*}
                \vec{F_p} = - cost\cdot \vec{g}
            \end{align*}
            Ora applico la seconda legge della dinamica [pag.\pageref{II_legge_dinamica}] e ottengo:
            \begin{align*}
                \vec{F_p} = - cost\cdot \vec{g} = - m_G\cdot \vec{g}
            \end{align*}
            dove $m_G$ è la \textbf{massa gravitazionale}.\\\\
            Ora calcolo le forze esterne ($\vec{F}_{EXT}$):
            \begin{align*}
                \vec{F}_{EXT} = m_i\cdot\vec{a}
            \end{align*}
            dove $m_i$ è la \textbf{massa inerziale}, e posso assumere che $\vec{a} = \vec{g}$, e quindi di conseguenza, $m_i = m_G$.\\
            Osserviamo che $m_i$ e $m_G$ sono uguali dal punto di vista quantitativo, ma non dal punto di vista concettuale. Infatti il primo è la capacità del corpo di opporsi al movimento, mentre il secondo è la carica dell'interazione gravitazionale che il corpo ha.

            \paragraph{Esempio sensazione del peso}
                Supponiamo di avere un'ascensore che ha 100 piani. La sensazione che si ha è di essere più "pesanti" quando l'ascensore parte per salire e più "leggeri" quando si ferma in alto.\\
                Ora studiamo in particolare alcuni casi interessanti per vedere le differenze che ci sono:
                \begin{itemize}
                    \item \underline{\textbf{CASO INTERMEDIO}} (e.g. da piano 25 a piano 75):\\
                    Supponiamo che l'ascensore sia ben isolata dall'ambiente, quindi che non ci siano vibrazioni, rumori, indicatore del piano, ecc., allora non si ha la percezione se ci si sta muovendo o se si è fermi.
                    In questo caso abbiamo:
                    \begin{align*}
                        &\vec{v}=cost\\
                        &\vec{a}=\vec{0}\\
                        &\vec{R}=\vec{0}\\
                        &|\vec{F_p}|=|\vec{N}|=N_0
                    \end{align*}

                    \item \underline{\textbf{CASO PARTENZA IN SALITA}}:\\
                    Se siamo fermi in un piano inferiore al piano 100 e premiamo un piano più alto di quello in cui siamo, quando l'ascensore parte si ha la sensazione di pesare di più.
                    In questo caso abbiamo:
                    \begin{align*}
                        &\vec{a}=a\hat{z}\\
                        &\vec{g}=-g\hat{z}\\
                        &\vec{R}\neq\vec{0}\\
                        &\vec{F_p}+\vec{N}=m\vec{a} => -mg\hat{z}+N\hat{z}=ma\hat{z} => -mg+N=ma => N=ma+mg = N_0 + ma
                    \end{align*}
                    con $ma > 0$.

                    \item \underline{\textbf{CASO PARTENZA IN DISCESA}}:\\
                    Se siamo fermi in un piano superiore al piano 0 e premiamo un piano più basso di quello in cui siamo, quando l'ascensore parte si ha la sensazione di pesare di meno. Questo caso è uguale a quello della \textbf{partenza in salita} solo che c'è una decelerazione.
                    In questo caso abbiamo:
                    \begin{align*}
                        &\vec{a}<0\\
                        &\vec{F_p}+\vec{N}=m\vec{a} => -mg\hat{z}+N\hat{z}=ma\hat{z} => -mg+N=ma => N=ma+mg = N_0 + ma
                    \end{align*}
                    con $ma < 0$.

                    \item \underline{\textbf{CASO PARTICOLARE}}:\\
                    Nel caso in cui si tagli in cavo dell'ascensore, sarà in caduta libera. Non si percepisce nessun effetto di reazione vincolare da parte del suolo/pavimento dell'ascensore. In caduta libera non si percepisce il peso.
                    In questo caso abbiamo:
                    \begin{align*}
                        &\vec{a}<0\\
                        &|\vec{a}|=|\vec{g}|\\
                        &N = N_0 + ma = mg-mg=0
                    \end{align*}
                \end{itemize}



        \subsubsection{Forza gravitazionale}
            la \textbf{forza gravitazionale} ci dice che c'è un'attrazione con una certa costante di proporzionalità $G$, e l'intensità dell'attrazione è inversamente proporzionale al quadrato della distanza tra le due masse e direttamente proporzionale al prodotto delle due masse, ovvero come descritto nella seguente formula:
            \begin{align*}
                \vec{F_G} = \frac{m_g M}{r^2}\hat{r}
            \end{align*}
            Assumendo ora che $M=M_E$ e $r=r_E$, ovvero consideriamo massa e raggio della Terra, allora $\vec{F_G}$ è una costante uguale a $\vec{g}$.

        \subsubsection{Forza elastica}
            Supponiamo di avere una molla vincolata ad un supporto non movibile e fermo e all'altra estremità della molla attaccata una massa $m$. Assumiamo che non ci siano attriti.\\
            Quando ci allontaniamo dalla \textbf{lunghezza di riposo}, $l_0$, della molla, quest'ultima si allunga o accorcia tramite una \textbf{sollecitazione esterna}.\\
            Supponendo ora di avere una forza $\vec{F_{ext}}$ che agisce sulla massa, per far si che la massa rimanga ferma, vorrà dire che sulla massa agisce un'altra forza uguale e opposta esercitata dalla molla chiamata \textbf{forza elastica} $\vec{F_{el}}$.\\
            Osserviamo che se la molla si allunga, allora $\vec{F_{ext}}$ e $\vec{F_{el}}$ avranno un verso, se invece la molla si accorcia, avranno verso opposto.\\
            Osserviamo che la forza elastica è sempre opposta a $\Delta\vec{l}=\vec{l}-\vec{l_0}$, quindi se il verso di $\Delta\vec{l}$ è positivo, allora il verso di $\vec{F_{el}}$ è negativo, e viceversa.\\
            La $\vec{F_{el}}$ e $\Delta\vec{l}$ sono proporzionali, infatti la formula della $\vec{F_{el}}$ è la seguente:
            \begin{align*}
                \vec{F_{el}}=-K(\vec{l}-\vec{l_0})=-K\Delta\vec{l}
            \end{align*}
            Dove $K$ è detta \textbf{costante elastica} e la sua unità di misura è quindi $\frac{N}{m}$ (Newton/metro).\\
            Osservando la formula precedente si può facilmente notare che la costante elastica è indipendente dalla massa del corpo, infatti indica solo la durezza della molla.\\
            Le \textbf{forze di richiamo}, non solo la forza elastica, sono proporzionali allo spostamento, ovvero quando il corpo si allontana dal suo equilibrio viene richiamato verso di esso tramite una forza di richiamo che è sempre direttamente proporzionale allo spostamento, cioè a quanto il corpo si allontana dall'equilibrio.\\
            Ora consideriamo la massa $m$ e abbiamo:
            \begin{align*}
                F=-K\cdot x
            \end{align*}
            ovvero la forza è inversamente proporzionale allo spostamento.\\
            Ora considerando la seconda legge della dinamica [pag.\pageref{II_legge_dinamica}], sostituiamo la definizione di forza e otteniamo:
            \begin{align*}
                -Kx=m\frac{d^2x}{{dt}^2}=>\frac{d^2x}{{dt}^2}+\frac{K}{m}x=0
            \end{align*}
            Ora definiamo $\omega^2=\frac{K}{m}$ e otteniamo:
            \begin{align*}
                \frac{d^2x}{{dt}^2}+\omega^2x=0
            \end{align*}
            da cui ci ricaviamo $x(t)$, $v(t)$ e $a(t)$ come segue:
            \begin{align*}
                &x(t)=Asin(\omega t + \varphi)\\
                &v(t)=A\omega cos(\omega t + \varphi)\\
                &a(t)=-A\omega^2sin(\omega t + \varphi)
            \end{align*}
            Per dimostrare come abbiamo ricavato $x(t)$, osserviamo la sua derivata seconda, ovvero $a(t)$, e notiamo che togliendo $-\omega^2$ abbiamo esattamente $x(t)$ quindi sostituendo quella con $x$, otteniamo:
            \begin{align*}
                \frac{d^2x}{{dt}^2}=-\omega^2x=>\frac{d^2x}{{dt}^2}+\omega^2x=0
            \end{align*}
            Ovvero l'ipotesi iniziale.\\
            Osserviamo ora quindi che se applico una forza esterna sulla molla e poi la lascio andare, essa si muove in moto armonico intorno al punto di riposo. Questo moto dato che compare $sin$ nella formula sarà periodico, e il periodo è il seguente:
            \begin{align*}
                &\omega t+\varphi=\omega (t+T)+\varphi + 2k\pi=>\omega t=\omega t+\omega T+2k\pi\\
                &=>\omega T=2k\pi=>T=\frac{2k\pi}{\omega}
            \end{align*}
            Osserviamo ora che se fisso i valori di $x(t)$ e $v(t)$, riesco a ricavare $\varphi$ e $A$. Quindi se per esempio fisso:
            \begin{align*}
                &v(0)=v_M\\
                &x(0)=0
            \end{align*}
            dove $v_M$ è la velocità massima, mi ricavo:
            \begin{align*}
                &A\omega cos(\varphi)=v_M\\
                &Asin(\varphi)=0
            \end{align*}
            e da qui mi ricavo:
            \begin{align*}
                &A\omega=v_M=>A=\frac{v_M}{\omega}\\
                &\varphi=0
            \end{align*}


        \subsubsection{Forza di attrito (radente)}
