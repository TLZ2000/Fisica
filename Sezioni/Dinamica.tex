\section{Dinamica}
    \subsection{Leggi della dinamica}
        La \textbf{dinamica} si occupa dello studio del moto dei corpi a partire dalle sue cause(\textbf{forze}), ovvero delle circostanze che lo determinano e lo modificano nel \textbf{tempo} e nello \textbf{spazio} del suo sistema di riferimento. \href{https://it.wikipedia.org/wiki/Dinamica}{Wikipedia}
        Le leggi della dinamica sono 3 e sono le seguenti:
        \begin{enumerate}
            \item \underline{\textbf{Legge di Inerzia (I legge)}}: un corpo rimane nel suo stato di quiete finchè non intervengono agenti esterni a modificarne questo stato. Questa legge vale sono in sistemi di riferimento inerziali;\\
            \item \underline{\textbf{Legge di Newton (II legge)}}: \label{II_legge_dinamica}
            Viene definita la \textbf{quantità di moto} come $\vec{p}=m\vec{v}$, ovvero massa per velocità. Successivamente viene definita la forza ($\vec{F}$) come segue:
            \begin{align*}
                \vec{F}=\frac{d\vec{p}}{dt}=\frac{d(m\vec{v})}{dt}=\frac{dm}{dt}\vec{v} + m\frac{d\vec{v}}{dt}
            \end{align*}
            dove $m$ è la \textbf{massa inerziale}, ovvero la capacità di un corpo di opporsi alle variazioni del suo stato di moto, questa mette in relazione la velocità alla forza.\\
            Nel caso in cui la massa non varia, allora la forza può essere definita come $\vec{F}=m\cdot\vec{a}$.\\
            L'unità di misura della \textbf{forza} è il Newton ($N$) definito come $\frac{kg\cdot m}{s^2}$.\\
            \item \underline{\textbf{Principio di azione e reazione (III legge)}}: Quando il corpo 1 esercita una forza $\vec{F}$ sul corpo 2, quest'ultimo esercita sul corpo 1 una forza $-\vec{F}$, uguale e opposta.
            \begin{align*}
                \vec{F}_{1\rightarrow 2}=-\vec{F}_{2\rightarrow 1}
            \end{align*}
        \end{enumerate}
        Osserviamo che la \textbf{prima legge} potrebbe sembrare un caso particolare della \textbf{seconda legge}, con $\vec{F}=\vec{0}$, ma in realtà non è così, infatti la seconda e la terza legge sono valide solo all'interno di sistemi di riferimento inerziali, che sono definiti dalla prima legge.

    \subsection{Forze impulsive}
        L'\textbf{impulso} $\vec{P}$ è definito come la variazione di quantità di moto $\Delta\vec{p}$ in un $\Delta t$ piccolo, ovvero:
        \begin{align*}
            \vec{P}=\Delta\vec{p}=\int_0^t{\vec{F}dt}
        \end{align*}
        E la \textbf{forza impulsiva} come:
        \begin{align*}
            \vec{F}_{imp}=\frac{\Delta\vec{p}}{\Delta t}
        \end{align*}

        \subsubsection{Esempio forze impulsive}
            Supponiamo di avere un pavimento ed una palla che viene lasciata in aria. Questa palla cadrà verso il pavimento fino a raggiungerlo, rimbalzare su esso e tornare in su (assumiamo che la velocità con cui torna in su sia la stessa con cui cade, quindi non agiscono fattori esterni come attriti, ecc.).

            \framedImg{35}{L04-img004}

            Se ho un vettore velocità $\vec{v}$, allora ho che:
            \begin{align*}
                &\vec{p_i}=m\vec{v_i}=-m\vec{v}\\
                &\vec{p_f}=m\vec{v_f}=m\vec{v}\\
                &\Delta\vec{p}=\vec{p_f}-\vec{p_i}=2m\vec{v}
            \end{align*}

        \subsubsection{Esercizio su forze impulsive}
            Supponiamo di avere i seguenti dati e di dover calcolare $\vec{F}_{imp}$ (forza impulsiva):
            \begin{align*}
                &m=98g&&v=10.2\frac{m}{s}&&\Delta t=100ms\\
            \end{align*}
            Procediamo ora quindi con calcolare $\Delta\vec{p}$ usando la formula appena calcolata sopra e una volta ottenuto il valore calcoliamo la $\vec{F}_{imp}$:
            \begin{align*}
                &\Delta\vec{p}=2m\vec{v}=2\cdot 10.2\frac{m}{s}\cdot 0.098kg=0.99\frac{kg\cdot m}{s}\\
                &\vec{F}_{imp}=\frac{\Delta\vec{p}}{\Delta t}=\frac{0.99\frac{kg\cdot m}{s}}{0.1s}=9.99\frac{kg\cdot m}{s^2}=9.99N
            \end{align*}

    \subsection{Esercizi sulla dinamica}
        Supponiamo di avere un oggetto appeso a due fili, che sono appesi al tetto, alla stessa distanza dall'oggetto e vogliamo trovare $\vec{T_1}$ e $\vec{T_2}$ tensioni dei fili, avendo i seguenti dati:
        \begin{align*}
            &m=100g&&\theta=60^{\circ}\\
        \end{align*}

        \framedImg{60}{L04-img005}

        Notiamo che l'oggetto resta fermo, quindi oltre a $\vec{p}$ (\textbf{forza peso}), su esso agiscono altre forze la cui somma è uguale e opposta a $\vec{p}$. Abbiamo quindi che la \textbf{risultante delle forze} $\vec{R}=\vec{0}$.\\
        Ora possiamo notare che $|\vec{T_1}|=|\vec{T_2}|=T$ e abbiamo le seguenti forze:
        \begin{align*}
            &\vec{p}=-mg\hat{y}\\
            &\vec{T_1}=T_x\hat{x}+T_y\hat{x}=T sin\theta \hat{x}+T cos\theta \hat{y}\\
            &\vec{T_2}=-T_x\hat{x}+T_y\hat{y}=-T sin\theta \hat{x}+T cos\theta \hat{y}
        \end{align*}
        Ora ci ricordiamo che $\vec{R}=\vec{0}$ quindi:
        \begin{align*}
            &\vec{R}=\vec{0}=\vec{P}+\vec{T_1}+\vec{T_2}=>
            \begin{cases}
                &R_x=0\\
                &R_y=0
            \end{cases}=>\\
            &=>
            \begin{cases}
                &R_x=0=T sin\theta-T sin\theta\\
                &R_y=0=-mg+T cos\theta+T cos\theta=-mg+2Tcos\theta
            \end{cases}
            \end{align*}
                La prima equazione del sistema vale zero, ora dalla seconda ricaviamo $T$:
            \begin{align*}
            mg=2Tcos\theta=>T=\frac{mg}{2Tcos\theta}=\frac{0.1kg\cdot 9.8\frac{m}{s^2}}{2\cdot\frac{1}{2}}=0.98N
        \end{align*}

    \subsection{Forze fondamentali}






    \subsection{Forze}








        \subsubsection{Forza di attrito (radente)}
            Iniziamo col dire che esistono molti tipi diversi di attrito, noi però ci concentriamo sul'\textbf{attrito radente}, ovvero quello che si ottiene con \textbf{2 superfici a contatto}. In genere, l'attrito si comporta in 2 modi diversi, facciamo un esempio con una sedia sul pavimento:
            \begin{itemize}
                \item \textbf{attrito statico}: se applichiamo una certa forza $\vec{F}_{ext}$ alla sedia e quella \textbf{resta ferma} abbiamo una forza di \textbf{attrito statico ($\vec{A}_s$) che bilancia} la forza che applichiamo noi. In particolare:
                \begin{align*}
                    \begin{rcases*}
                        \vec{F}_{ext}=F_{ext}*\hat{x}\\
                        \vec{A}_{s}= -|\vec{A}_s|*\hat{x}=-|F_{ext}|*\hat{x}
                    \end{rcases*}
                    \vec{A}_s=-|\vec{F}_{ext}|*\hat{x} \#\#\#NON SICURO SU HAT!!!
                \end{align*}
                Da quest'ultimo pezzo possiamo capire che l'attrito, finché resta statico, è una \textbf{uguale ed opposta} alla forza che applichiamo noi sulla sedia, quindi quest'ultima resta ferma!
                \framedImg{35}{L06-img001}
                L'attrito statico esiste fino ad un certo punto, identificato con la \textbf{soglia} "$\vec{A}_{s, max}$", che possiamo calcolare con la formula:
                \begin{align*}
                    &\textcolor{Red}{\vec{A}_{s, max} = -\mu_s*|\vec{N}|*\hat{F}_{ext}}
                \end{align*}
                In particolare $\mu_s$ rappresenta il \textbf{coefficiente di attrito statico} (dipende dalle 2 superfici a contatto), $\vec{N}$ è la \textbf{forza vincolate} mentre il segno - è dato dal fatto che \textbf{l'attrito è sempre opposto al moto dell'oggetto} (in questo caso il "moto" è rappresentato dalla forza esterna che applichiamo noi alla sedia). Una volta che la forza esterna supera questa soglia, subentra l'\textbf{attrito dinamico};

                \item \textbf{attrito dinamico}: ad un certo punto, la forza che applichiamo noi sarà tale da \textbf{muovere} la sedia, a questo punto entriamo un una fase di \textbf{attrito dinamico}. Possiamo calcolarlo con la formula:
                \begin{align*}
                    &\textcolor{Red}{\vec{A}_D = -\mu_D*|\vec{N}|*\hat{v}}
                \end{align*}
                Anche qui $\mu_D$ rappresenta il \textbf{coefficiente di attrito dinamico} (dipende dalle 2 superfici a contatto), $\vec{N}$ è la \textbf{forza vincolate} mentre il segno - è dato dal fatto che \textbf{l'attrito è sempre opposto al moto dell'oggetto}.
            \end{itemize}

            Ora, all'intersezione tra attrito statico e dinamico succede una cosa particolare: la forza dell'attrito \textbf{diminuisce}. Vediamo un grafico:
            \framedImg{80}{L06-img002}
            Il distacco tra $A_{s, max}$ e $A_D$ dipende dal distacco tra $\mu_s$ e $\mu_D$

            \paragraph{Esempio di calcolo del coefficiente di attrito statico}
                Vediamo un esempio per il calcolo del coefficiente di attrito statico, supponiamo di avere una sedia con \textbf{massa} $m_s$ e un'\textbf{attrito statico massimo} $A_{s,max}$ calcolato usando una molla con \textbf{allungamento} $\Delta x$ e \textbf{coefficiente elastico} $K$, quanto vale il \textbf{coefficiente di attrito statico}?
                \begin{align*}
                    &m_s=6Kg && |A_{s, max}|=
                    \begin{cases}
                        \Delta x = 6 cm\\
                        K = 1,2 N/m
                    \end{cases}
                    &&\mu_s = ?
                \end{align*}
                \begin{align*}
                    &\vec{A}_{s,max} = -\mu_s * |\vec{N}| * \hat{F}_{ext} ===> \mu_s = \frac{|\vec{A}_{s, max}|}{|\vec{N}|} = \frac{K*\Delta x}{m_s * g} = \frac{7,2*10^{-2}N}{5,9*10^1N}=1,2*10^{-3}
                \end{align*}
                Nota che $g$ corrisponde all'accelerazione di gravità terrestre ($9,8 m/s^2$) o di qualsiasi altro pianeta su cui facciamo le misurazioni.

    \subsection{Piano inclinato}
        Introduciamo ora il concetto di \textbf{piano inclinato rispetto alla verticale} (con verticale intendiamo l'asse su cui giace la gravità).

        \subsubsection{Sistema di riferimento}
            Vediamo subito il sistema di riferimento:
            \framedImg{80}{L06-img003}
            Come asse x usiamo la \textbf{retta su cui giace il piano inclinato} mentre l'asse y è rappresentato dalla \textbf{normale all'asse x}. Nota inoltre che $\frac{H}{B} = tan(\Theta)$.
            Ora posizioniamo sul piano inclinato un qualche oggetto: questo avrà una \textbf{forza peso} con verso che giace sulla retta parallela alla verticale. Possiamo scomporre questa forza come proiezione sugli assi che abbiamo stabilito prima, in particolare $\vec{P}_\parallel$ per l'asse x e $\vec{P}_\bot$ per l'asse y. A questo punto, ora che abbiamo comoda la forza $\vec{P}_\bot$ possiamo anche stabilire la \textbf{forza vincolante} $\vec{N}$ che il piano oppone a questo oggetto (ricorda che la forza vincolante \textbf{può solo essere ortogonale al piano}). Vediamo un grafico:
            \framedImg{80}{L06-img004}
            L'obiettivo ora è calcolare le \textbf{risultanti delle forze} per gli assi x e y (con risultanti intendiamo la somma di tutte le forze "parallele" ad un certo asse), in particolare:
            \label{Risultanti_piano_inclinato}
            \begin{align*}
                &\textcolor{Red}{\vec{F}_{ext} =
                \begin{cases}
                    \vec{R}_x = \vec{P}_\parallel = m*g*sin(\Theta)\\
                    \vec{R}_y = \vec{P}_\bot + \vec{N} = m*g*cos(\Theta) - N = 0\\
                \end{cases}}
            \end{align*}

        \subsubsection{Esempio}
            Vediamo ora un esempio: abbiamo un bambino di \textbf{massa} $m$ e uno scivolo di \textbf{lunghezza} $L$ e con \textbf{inclinazione} $\Theta$. Quanto \textbf{tempo} impiega il bambino a scivolare sullo scivolo? Faremo 2 versioni di questo problema, la prima utilizzando un normale \textbf{moto uniformemente accelerato} e la seconda \textbf{introducendo anche l'attrito}. Vediamo una rappresentazione grafica del problema:
            \framedImg{5}{L06-img005}

            \paragraph{Versione senza attrito}
                Se proviamo a risolvere il problema senza considerare l'attrito, ci troviamo in presenza di un "semplice" \textbf{moto uniformemente accelerato}. Come prima cosa, \textbf{analizziamo le forze in campo} e calcoliamo le forze risultanti parallele agli assi. Iniziamo col dire che quella sull'asse y (che ricordo essere inclinato!) non ci interessa, infatti abbiamo la forza vincolante che bilancia, concentriamoci quindi solo sulla forza sull'asse x. Ricordiamo, per la \textit{seconda legge della dinamica} [pag. \pageref{II_legge_dinamica}]che:
                \begin{align*}
                    &\vec{F}_x = m*a_x
                \end{align*}
                A questo punto a noi interessa trovare l'accelerazione $a_x$, quindi rigiriamo un po' questa formula ed "\textbf{espandiamo}" la nostra $\vec{F}_x$ utilizzando le formule viste prima [pag. \pageref{Risultanti_piano_inclinato}]:
                \begin{align*}
                    &\vec{F}_x = m*a_x => a_x = \frac{\vec{F}_x}{m} = \frac{m * g * sin(\Theta)}{m} = g * sin(\Theta)
                \end{align*}
                Ora che abbiamo l'accelerazione, possiamo recuperare le formule viste per il moto uniformemente accelerato [pag. \pageref{Formule_moto_unif_acc}]. In questo caso ci interessa la formula dello \textbf{spazio percorso} (dato che sappiamo che il nostro scivolo è lungo $L$), ovvero:
                \begin{align*}
                    &L = s_0+v_0(t-t_0)+\frac{1}{2}*a*(t-t_0)^2
                \end{align*}
                A questa formula possiamo fare delle semplificazioni, in particolare togliere il I ($s_0 = 0$) e il II ($v_0 = 0$) termine, ottenendo:
                \begin{align*}
                    &L = \frac{1}{2}*a*(t-t_0)^2
                \end{align*}
                Che sostituendo l'accelerazione e, per comodità, il tempo diventa:
                \begin{align*}
                    &L = \frac{1}{2}*a*(\Delta t)^2
                \end{align*}
                A noi interessa calcolare $\Delta t$, quindi ci rigiriamo la formula in questo modo:
                \begin{align*}
                    &\Delta t = \sqrt{\frac{2L}{g*sin(\Theta)}}
                \end{align*}
                Introduciamo ora un po' di numeri, supponendo:
                \begin{align*}
                    &m = 20Kg && \Theta = 30^\circ && L = 4 m && \Delta t = ?
                \end{align*}
                Otteniamo il tempo
                \begin{align*}
                    &\Delta t = \sqrt{\frac{2L}{g*sin(\Theta)}} = \sqrt{\frac{2*4m}{9,8m/s^2*sin(30^\circ)}}=\sqrt{\frac{8 s^2}{9,8*1/2}} \approx 1,27 s
                \end{align*}

            \paragraph{Versione con attrito}
                Se aggiungiamo l'attrito, dobbiamo modificare un po' le nostre formule in modo da considerare l'attrito statico e dinamico:
                \begin{itemize}
                    \item \textbf{attrito statico}:
                    \begin{align*}
                        \begin{cases}
                            R_x = m*g*sin(\Theta_{max})-A_{s, max} &=> m*g*sin(\Theta_{max})-\mu_s * N= 0 \\
                            R_y = m*g*cos(\Theta_{max})-N &=> N = m*g*cos(\Theta_{max}) = 0
                        \end{cases}
                    \end{align*}
                    Da qui otteniamo che
                    \begin{align*}
                        &m*g*sin(\Theta_{max})-\mu_s * N= 0 &&=> m*g*sin(\Theta_{max})-\mu_s * m*g*cos(\Theta_{max})= 0\\
                        & &&=> m*g*[sin(\Theta_{max}) - \mu_s *cos(\Theta_{max})] = 0
                    \end{align*}
                    Ora, dato che sappiamo che $m*g$ non è nullo, possiamo dedurre che sia la seconda parte ad annullarsi, quindi:
                    \begin{align*}
                        & sin(\Theta_{max}) - \mu_s *cos(\Theta_{max}) = 0 && => \mu_s *cos(\Theta_{max}) = sin(\Theta_{max})\\
                        & && => \mu_s = \frac{sin(\Theta_{max})}{cos(\Theta_{max})}\\
                        & && => \textcolor{Red}{\mu_s = tan(\Theta_{max})}
                    \end{align*}
                    \item \textbf{attrito dinamico}:
                \end{itemize}
