\section{Dinamica}
    \subsection{Leggi della dinamica}
        La \textbf{dinamica} si occupa dello studio del moto dei corpi a partire dalle sue cause(\textbf{forze}), ovvero delle circostanze che lo determinano e lo modificano nel \textbf{tempo} e nello \textbf{spazio} del suo sistema di riferimento. \href{https://it.wikipedia.org/wiki/Dinamica}{Wikipedia}
        Le leggi della dinamica sono 3 e sono le seguenti:
        \begin{enumerate}
            \item \underline{\textbf{Legge di Inerzia (I legge)}}: un corpo rimane nel suo stato di quiete finchè non intervengono agenti esterni a modificarne questo stato. Questa legge vale sono in sistemi di riferimento inerziali;\\
            \item \underline{\textbf{Legge di Newton (II legge)}}: \label{II_legge_dinamica}
            Viene definita la \textbf{quantità di moto} come $\vec{p}=m\vec{v}$, ovvero massa per velocità. Successivamente viene definita la forza ($\vec{F}$) come segue:
            \begin{align*}
                \vec{F}=\frac{d\vec{p}}{dt}=\frac{d(m\vec{v})}{dt}=\frac{dm}{dt}\vec{v} + m\frac{d\vec{v}}{dt}
            \end{align*}
            dove $m$ è la \textbf{massa inerziale}, ovvero la capacità di un corpo di opporsi alle variazioni del suo stato di moto, questa mette in relazione la velocità alla forza.\\
            Nel caso in cui la massa non varia, allora la forza può essere definita come $\vec{F}=m\cdot\vec{a}$.\\
            L'unità di misura della \textbf{forza} è il Newton ($N$) definito come $\frac{kg\cdot m}{s^2}$.\\
            \item \underline{\textbf{Principio di azione e reazione (III legge)}}: \label{IIILeggeDinamica} Quando il corpo 1 esercita una forza $\vec{F}$ sul corpo 2, quest'ultimo esercita sul corpo 1 una forza $-\vec{F}$, uguale e opposta.
            \begin{align*}
                \vec{F}_{1\rightarrow 2}=-\vec{F}_{2\rightarrow 1}
            \end{align*}
        \end{enumerate}
        Osserviamo che la \textbf{prima legge} potrebbe sembrare un caso particolare della \textbf{seconda legge}, con $\vec{F}=\vec{0}$, ma in realtà non è così, infatti la seconda e la terza legge sono valide solo all'interno di sistemi di riferimento inerziali, che sono definiti dalla prima legge.

    \subsection{Forze impulsive}
        L'\textbf{impulso} $\vec{P}$ è definito come la variazione di quantità di moto $\Delta\vec{p}$ in un $\Delta t$ piccolo, ovvero:
        \begin{align*}
            \vec{P}=\Delta\vec{p}=\int_0^t{\vec{F}dt}
        \end{align*}
        E la \textbf{forza impulsiva} come:
        \begin{align*}
            \vec{F}_{imp}=\frac{\Delta\vec{p}}{\Delta t}
        \end{align*}

        \subsubsection{Esempio forze impulsive}
            Supponiamo di avere un pavimento ed una palla che viene lasciata in aria. Questa palla cadrà verso il pavimento fino a raggiungerlo, rimbalzare su esso e tornare in su (assumiamo che la velocità con cui torna in su sia la stessa con cui cade, quindi non agiscono fattori esterni come attriti, ecc.).

            \framedImg{35}{L04-img004}

            Se ho un vettore velocità $\vec{v}$, allora ho che:
            \begin{align*}
                &\vec{p_i}=m\vec{v_i}=-m\vec{v}\\
                &\vec{p_f}=m\vec{v_f}=m\vec{v}\\
                &\Delta\vec{p}=\vec{p_f}-\vec{p_i}=2m\vec{v}
            \end{align*}

        \subsubsection{Esercizio su forze impulsive}
            Supponiamo di avere i seguenti dati e di dover calcolare $\vec{F}_{imp}$ (forza impulsiva):
            \begin{align*}
                &m=98g&&v=10.2\frac{m}{s}&&\Delta t=100ms\\
            \end{align*}
            Procediamo ora quindi con calcolare $\Delta\vec{p}$ usando la formula appena calcolata sopra e una volta ottenuto il valore calcoliamo la $\vec{F}_{imp}$:
            \begin{align*}
                &\Delta\vec{p}=2m\vec{v}=2\cdot 10.2\frac{m}{s}\cdot 0.098kg=0.99\frac{kg\cdot m}{s}\\
                &\vec{F}_{imp}=\frac{\Delta\vec{p}}{\Delta t}=\frac{0.99\frac{kg\cdot m}{s}}{0.1s}=9.99\frac{kg\cdot m}{s^2}=9.99N
            \end{align*}

    \subsection{Esercizi sulla dinamica}
        Supponiamo di avere un oggetto appeso a due fili, che sono appesi al tetto, alla stessa distanza dall'oggetto e vogliamo trovare $\vec{T_1}$ e $\vec{T_2}$ tensioni dei fili, avendo i seguenti dati:
        \begin{align*}
            &m=100g&&\theta=60^{\circ}\\
        \end{align*}

        \framedImg{60}{L04-img005}

        Notiamo che l'oggetto resta fermo, quindi oltre a $\vec{p}$ (\textbf{forza peso}), su esso agiscono altre forze la cui somma è uguale e opposta a $\vec{p}$. Abbiamo quindi che la \textbf{risultante delle forze} $\vec{R}=\vec{0}$.\\
        Ora possiamo notare che $|\vec{T_1}|=|\vec{T_2}|=T$ e abbiamo le seguenti forze:
        \begin{align*}
            &\vec{p}=-mg\hat{y}\\
            &\vec{T_1}=T_x\hat{x}+T_y\hat{x}=T sin\theta \hat{x}+T cos\theta \hat{y}\\
            &\vec{T_2}=-T_x\hat{x}+T_y\hat{y}=-T sin\theta \hat{x}+T cos\theta \hat{y}
        \end{align*}
        Ora ci ricordiamo che $\vec{R}=\vec{0}$ quindi:
        \begin{align*}
            &\vec{R}=\vec{0}=\vec{P}+\vec{T_1}+\vec{T_2}=>
            \begin{cases}
                &R_x=0\\
                &R_y=0
            \end{cases}=>\\
            &=>
            \begin{cases}
                &R_x=0=T sin\theta-T sin\theta\\
                &R_y=0=-mg+T cos\theta+T cos\theta=-mg+2Tcos\theta
            \end{cases}
            \end{align*}
                La prima equazione del sistema vale zero, ora dalla seconda ricaviamo $T$:
            \begin{align*}
            mg=2Tcos\theta=>T=\frac{mg}{2Tcos\theta}=\frac{0.1kg\cdot 9.8\frac{m}{s^2}}{2\cdot\frac{1}{2}}=0.98N
        \end{align*}

    \subsection{Forze fondamentali}
        Le forze fondamentali sono le seguenti (scritte in ordine di intensità, dalla più debole alla più intensa):
        \begin{itemize}
            \item \underline{\textbf{Forza gravitazionale}}: descrive l'interazione tra le masse gravitazionali. È una forza onnipresente, non esiste quindi nessuna componente di materia che ha massa, che non la risente;
            \item \underline{\textbf{Forza debole}}: è la forza responsabile dell'interazione per cui i nuclei cambiano di natura (e.g. decadimento dei nuclei);
            \item \underline{\textbf{Forza elettromagnetica}}: deriva dall'unificazione di forza elettrica e forza magnetica. È la forza responsabile della trazione della ripulsione di cariche ed è alla base delle forze che aggregano la materia;
            \item \underline{\textbf{Forza forte/nucleare}}: è la forza responsabile della stabilità dei nuclei. Inizialmente si è definita come \textbf{forza nucleare}, ovvero che descrive un'interazione nucleare, tra protoni e neutroni, tra neutroni e neutroni e tra protoni e protoni. Poi è stato scoperto che i protoni e i neutroni non sono particelle fondamentali ma sono fatti di quark, e l'interazione è sentita dai quark, quindi è stata definita \textbf{forza forte}.
            \\
        \end{itemize}

        La forza elettromagnetica e la forza debole unificate formano la \textbf{forza elettrodebole}.
        Osserviamo che il quadro precedente, ovvero l'ordine di intensità, è \textbf{attuale}, era diverso nel passato e lo sarà anche nel futuro.\\\\
        Quando due corpi/sistemi fisici interagiscono tra loro per una delle forze fondamentali, lo fanno perchè hanno una sensibilità a quel tipo di forza, che è detta \textbf{carica} ed è rappresentata dalla lettera $q$. Per la forza gravitazionale per esempio, un corpo/sistema fisico subisce una trazione gravitazionale se ha una \textbf{massa} (\textbf{carica gravitazionale}, $q_G$), ovvero la misura dell'inclinazione del corpo ad interagire con questa gravitazione. Per la forza elettromagnetica è uguale, se due oggetti sono neutri, non c'è trazione ne repulsione. E la stessa cosa vale per forza debole e forza forte.\\\\
        Quella che chiamiamo carica in senso comune, in realtà è la \textbf{carica elettrica} ($q_E$).


    \subsection{Forze}

        \subsubsection{Forza peso}
            La \textbf{forza peso} è descritta come segue, dato $\vec{g} = -g\hat{z}$:
            \begin{align*}
                \vec{F_p} = - cost\cdot \vec{g}
            \end{align*}
            Ora applico la seconda legge della dinamica [pag.\pageref{II_legge_dinamica}] e ottengo:
            \begin{align*}
                \vec{F_p} = - cost\cdot \vec{g} = - m_G\cdot \vec{g}
            \end{align*}
            dove $m_G$ è la \textbf{massa gravitazionale}.\\\\
            Ora calcolo le forze esterne ($\vec{F}_{EXT}$):
            \begin{align*}
                \vec{F}_{EXT} = m_i\cdot\vec{a}
            \end{align*}
            dove $m_i$ è la \textbf{massa inerziale}, e posso assumere che $\vec{a} = \vec{g}$, e quindi di conseguenza, $m_i = m_G$.\\
            Osserviamo che $m_i$ e $m_G$ sono uguali dal punto di vista quantitativo, ma non dal punto di vista concettuale. Infatti il primo è la capacità del corpo di opporsi al movimento, mentre il secondo è la carica dell'interazione gravitazionale che il corpo ha.

            \paragraph{Esempio sensazione del peso}
                Supponiamo di avere un'ascensore che ha 100 piani. La sensazione che si ha è di essere più "pesanti" quando l'ascensore parte per salire e più "leggeri" quando si ferma in alto.\\
                \framedImg{50}{L05-img001}
                Ora studiamo in particolare alcuni casi interessanti per vedere le differenze che ci sono:
                \begin{itemize}
                    \item \underline{\textbf{CASO INTERMEDIO}} (e.g. da piano 25 a piano 75):\\
                    Supponiamo che l'ascensore sia ben isolata dall'ambiente, quindi che non ci siano vibrazioni, rumori, indicatore del piano, ecc., allora non si ha la percezione se ci si sta muovendo o se si è fermi.
                    In questo caso abbiamo:
                    \begin{align*}
                        &\vec{v}=cost\\
                        &\vec{a}=\vec{0}\\
                        &\vec{R}=\vec{0}\\
                        &|\vec{F_p}|=|\vec{N}|=N_0
                    \end{align*}

                    \item \underline{\textbf{CASO PARTENZA IN SALITA}}:\\
                    Se siamo fermi in un piano inferiore al piano 100 e premiamo un piano più alto di quello in cui siamo, quando l'ascensore parte si ha la sensazione di pesare di più.
                    In questo caso abbiamo:
                    \begin{align*}
                        &\vec{a}=a\hat{z}\\
                        &\vec{g}=-g\hat{z}\\
                        &\vec{R}\neq\vec{0}\\
                        &\vec{F_p}+\vec{N}=m\vec{a} => -mg\hat{z}+N\hat{z}=ma\hat{z} => -mg+N=ma => N=ma+mg = N_0 + ma
                    \end{align*}
                    con $ma > 0$.

                    \item \underline{\textbf{CASO PARTENZA IN DISCESA}}:\\
                    Se siamo fermi in un piano superiore al piano 0 e premiamo un piano più basso di quello in cui siamo, quando l'ascensore parte si ha la sensazione di pesare di meno. Questo caso è uguale a quello della \textbf{partenza in salita} solo che c'è una decelerazione.
                    In questo caso abbiamo:
                    \begin{align*}
                        &\vec{a}<0\\
                        &\vec{F_p}+\vec{N}=m\vec{a} => -mg\hat{z}+N\hat{z}=ma\hat{z} => -mg+N=ma => N=ma+mg = N_0 + ma
                    \end{align*}
                    con $ma < 0$.

                    \item \underline{\textbf{CASO PARTICOLARE}}:\\
                    Nel caso in cui si tagli in cavo dell'ascensore, sarà in caduta libera. Non si percepisce nessun effetto di reazione vincolare da parte del suolo/pavimento dell'ascensore. In caduta libera non si percepisce il peso.
                    In questo caso abbiamo:
                    \begin{align*}
                        &\vec{a}<0\\
                        &|\vec{a}|=|\vec{g}|\\
                        &N = N_0 + ma = mg-mg=0
                    \end{align*}
                \end{itemize}



        \subsubsection{Forza gravitazionale}
            la \textbf{forza gravitazionale} ci dice che c'è un'attrazione con una certa costante di proporzionalità $G$, e l'intensità dell'attrazione è inversamente proporzionale al quadrato della distanza tra le due masse e direttamente proporzionale al prodotto delle due masse, ovvero come descritto nella seguente formula:
            \begin{align*}
                \vec{F_G} = \frac{m_g M}{r^2}\hat{r}
            \end{align*}
            Assumendo ora che $M=M_E$ e $r=r_E$, ovvero consideriamo massa e raggio della Terra, allora $\vec{F_G}$ è una costante uguale a $\vec{g}$.

        \subsubsection{Forza elastica}
            Supponiamo di avere una molla vincolata ad un supporto non movibile e fermo e all'altra estremità della molla attaccata una massa $m$. Assumiamo che non ci siano attriti.\\
            Quando ci allontaniamo dalla \textbf{lunghezza di riposo}, $l_0$, della molla, quest'ultima si allunga o accorcia tramite una \textbf{sollecitazione esterna}.\\
            \framedImg{50}{L05-img002}
            Supponendo ora di avere una forza $\vec{F_{ext}}$ che agisce sulla massa, per far si che la massa rimanga ferma, vorrà dire che sulla massa agisce un'altra forza uguale e opposta esercitata dalla molla chiamata \textbf{forza elastica} $\vec{F_{el}}$.\\
            Osserviamo che se la molla si allunga, allora $\vec{F_{ext}}$ e $\vec{F_{el}}$ avranno un verso, se invece la molla si accorcia, avranno verso opposto.\\
            Osserviamo che la forza elastica è sempre opposta a $\Delta\vec{l}=\vec{l}-\vec{l_0}$, quindi se il verso di $\Delta\vec{l}$ è positivo, allora il verso di $\vec{F_{el}}$ è negativo, e viceversa.\\
            La $\vec{F_{el}}$ e $\Delta\vec{l}$ sono proporzionali, infatti la formula della $\vec{F_{el}}$ è la seguente:
            \begin{align*}
                \vec{F_{el}}=-K(\vec{l}-\vec{l_0})=-K\Delta\vec{l}
            \end{align*}
            Dove $K$ è detta \textbf{costante elastica} e la sua unità di misura è quindi $\frac{N}{m}$ (Newton/metro).\\
            Osservando la formula precedente si può facilmente notare che la costante elastica è indipendente dalla massa del corpo, infatti indica solo la durezza della molla.\\
            Le \textbf{forze di richiamo}, non solo la forza elastica, sono proporzionali allo spostamento, ovvero quando il corpo si allontana dal suo equilibrio viene richiamato verso di esso tramite una forza di richiamo che è sempre direttamente proporzionale allo spostamento, cioè a quanto il corpo si allontana dall'equilibrio.\\
            Ora consideriamo la massa $m$ e abbiamo:
            \begin{align*}
                F=-K\cdot x
            \end{align*}
            ovvero la forza è inversamente proporzionale allo spostamento.\\
            Ora considerando la seconda legge della dinamica [pag.\pageref{II_legge_dinamica}], sostituiamo la definizione di forza e otteniamo:
            \begin{align*}
                -Kx=m\frac{d^2x}{{dt}^2}=>\frac{d^2x}{{dt}^2}+\frac{K}{m}x=0
            \end{align*}
            Ora definiamo $\omega^2=\frac{K}{m}$ e otteniamo:
            \begin{align*}
                \frac{d^2x}{{dt}^2}+\omega^2x=0
            \end{align*}
            da cui ci ricaviamo $x(t)$, $v(t)$ e $a(t)$ come segue:
            \begin{align*}
                &x(t)=Asin(\omega t + \varphi)\\
                &v(t)=A\omega cos(\omega t + \varphi)\\
                &a(t)=-A\omega^2sin(\omega t + \varphi)
            \end{align*}
            Per dimostrare come abbiamo ricavato $x(t)$, osserviamo la sua derivata seconda, ovvero $a(t)$, e notiamo che togliendo $-\omega^2$ abbiamo esattamente $x(t)$ quindi sostituendo quella con $x$, otteniamo:
            \begin{align*}
                \frac{d^2x}{{dt}^2}=-\omega^2x=>\frac{d^2x}{{dt}^2}+\omega^2x=0
            \end{align*}
            Ovvero l'ipotesi iniziale.\\
            Osserviamo ora quindi che se applico una forza esterna sulla molla e poi la lascio andare, essa si muove in moto armonico intorno al punto di riposo. Questo moto dato che compare $sin$ nella formula sarà periodico, e il periodo è il seguente:
            \begin{align*}
                &\omega t+\varphi=\omega (t+T)+\varphi + 2k\pi=>\omega t=\omega t+\omega T+2k\pi\\
                &=>\omega T=2k\pi=>T=\frac{2k\pi}{\omega}
            \end{align*}
            Osserviamo ora che se fisso i valori di $x(t)$ e $v(t)$, riesco a ricavare $\varphi$ e $A$. Quindi se per esempio fisso:
            \begin{align*}
                &v(0)=v_M\\
                &x(0)=0
            \end{align*}
            dove $v_M$ è la velocità massima, mi ricavo:
            \begin{align*}
                &A\omega cos(\varphi)=v_M\\
                &Asin(\varphi)=0
            \end{align*}
            e da qui mi ricavo:
            \begin{align*}
                &A\omega=v_M=>A=\frac{v_M}{\omega}\\
                &\varphi=0
            \end{align*}
            e quindi ottengo:
            \begin{align*}
                &x(t)=\frac{v_M}{\omega}sin(\omega t)\\
                &v(t)=v_Mcos(\omega t)
            \end{align*}


        \subsubsection{Forza di attrito (radente)}
            Iniziamo col dire che esistono molti tipi diversi di attrito, noi però ci concentriamo sul'\textbf{attrito radente}, ovvero quello che si ottiene con \textbf{2 superfici a contatto}. In genere, l'attrito si comporta in 2 modi diversi, facciamo un esempio con una sedia sul pavimento:
            \begin{itemize}
                \item \textbf{attrito statico}: se applichiamo una certa forza $\vec{F}_{ext}$ alla sedia e quella \textbf{resta ferma} abbiamo una forza di \textbf{attrito statico ($\vec{A}_s$) che bilancia} la forza che applichiamo noi. In particolare:
                \begin{align*}
                    \begin{rcases*}
                        \vec{F}_{ext}=F_{ext}*\hat{x}\\
                        \vec{A}_{s}= -|\vec{A}_s|*\hat{x}=-|F_{ext}|*\hat{x}
                    \end{rcases*}
                    \vec{A}_s=-|\vec{F}_{ext}|*\hat{x} \#\#\#NON SICURO SU HAT!!!
                \end{align*}
                Da quest'ultimo pezzo possiamo capire che l'attrito, finché resta statico, è una \textbf{uguale ed opposta} alla forza che applichiamo noi sulla sedia, quindi quest'ultima resta ferma!
                \framedImg{35}{L06-img001}
                L'attrito statico esiste fino ad un certo punto, identificato con la \textbf{soglia} "$\vec{A}_{s, max}$", che possiamo calcolare con la formula:
                \begin{align*}
                    &\textcolor{Red}{\vec{A}_{s, max} = -\mu_s*|\vec{N}|*\hat{F}_{ext}}
                \end{align*}
                In particolare $\mu_s$ rappresenta il \textbf{coefficiente di attrito statico} (dipende dalle 2 superfici a contatto), $\vec{N}$ è la \textbf{forza vincolate} mentre il segno - è dato dal fatto che \textbf{l'attrito è sempre opposto al moto dell'oggetto} (in questo caso il "moto" è rappresentato dalla forza esterna che applichiamo noi alla sedia). Una volta che la forza esterna supera questa soglia, subentra l'\textbf{attrito dinamico};

                \item \textbf{attrito dinamico}: ad un certo punto, la forza che applichiamo noi sarà tale da \textbf{muovere} la sedia, a questo punto entriamo un una fase di \textbf{attrito dinamico}. Possiamo calcolarlo con la formula:
                \begin{align*}
                    &\textcolor{Red}{\vec{A}_D = -\mu_D*|\vec{N}|*\hat{v}}
                \end{align*}
                Anche qui $\mu_D$ rappresenta il \textbf{coefficiente di attrito dinamico} (dipende dalle 2 superfici a contatto), $\vec{N}$ è la \textbf{forza vincolate} mentre il segno - è dato dal fatto che \textbf{l'attrito è sempre opposto al moto dell'oggetto}.
            \end{itemize}

            Ora, all'intersezione tra attrito statico e dinamico succede una cosa particolare: la forza dell'attrito \textbf{diminuisce}. Vediamo un grafico:
            \framedImg{80}{L06-img002}
            Il distacco tra $A_{s, max}$ e $A_D$ dipende dal distacco tra $\mu_s$ e $\mu_D$

            \paragraph{Esempio di calcolo del coefficiente di attrito statico}
                Vediamo un esempio per il calcolo del coefficiente di attrito statico, supponiamo di avere una sedia con \textbf{massa} $m_s$ e un'\textbf{attrito statico massimo} $A_{s,max}$ calcolato usando una molla con \textbf{allungamento} $\Delta x$ e \textbf{coefficiente elastico} $K$, quanto vale il \textbf{coefficiente di attrito statico}?
                \begin{align*}
                    &m_s=6Kg && |A_{s, max}|=
                    \begin{cases}
                        \Delta x = 6 cm\\
                        K = 1,2 N/m
                    \end{cases}
                    &&\mu_s = ?
                \end{align*}
                \begin{align*}
                    &\vec{A}_{s,max} = -\mu_s * |\vec{N}| * \hat{F}_{ext} ===> \mu_s = \frac{|\vec{A}_{s, max}|}{|\vec{N}|} = \frac{K*\Delta x}{m_s * g} = \frac{7,2*10^{-2}N}{5,9*10^1N}=1,2*10^{-3}
                \end{align*}
                Nota che $g$ corrisponde all'accelerazione di gravità terrestre ($9,8 m/s^2$) o di qualsiasi altro pianeta su cui facciamo le misurazioni.

    \subsection{Piano inclinato}
        Introduciamo ora il concetto di \textbf{piano inclinato rispetto alla verticale} (con verticale intendiamo l'asse su cui giace la gravità).

        \subsubsection{Sistema di riferimento}
            Vediamo subito il sistema di riferimento:
            \framedImg{80}{L06-img003}
            Come asse x usiamo la \textbf{retta su cui giace il piano inclinato} mentre l'asse y è rappresentato dalla \textbf{normale all'asse x}. Nota inoltre che $\frac{H}{B} = tan(\theta)$.
            Ora posizioniamo sul piano inclinato un qualche oggetto: questo avrà una \textbf{forza peso} con verso che giace sulla retta parallela alla verticale. Possiamo scomporre questa forza come proiezione sugli assi che abbiamo stabilito prima, in particolare $\vec{P}_\parallel$ per l'asse x e $\vec{P}_\bot$ per l'asse y. A questo punto, ora che abbiamo comoda la forza $\vec{P}_\bot$ possiamo anche stabilire la \textbf{forza vincolante} $\vec{N}$ che il piano oppone a questo oggetto (ricorda che la forza vincolante \textbf{può solo essere ortogonale al piano}). Vediamo un grafico:
            \framedImg{80}{L06-img004}
            L'obiettivo ora è calcolare le \textbf{risultanti delle forze} per gli assi x e y (con risultanti intendiamo la somma di tutte le forze "parallele" ad un certo asse), in particolare:
            \label{Risultanti_piano_inclinato}
            \begin{align*}
                &\textcolor{Red}{\vec{F}_{ext} =
                \begin{cases}
                    \vec{R}_x = \vec{P}_\parallel = m*g*sin(\theta)\\
                    \vec{R}_y = \vec{P}_\bot + \vec{N} = m*g*cos(\theta) - N = 0\\
                \end{cases}}
            \end{align*}

        \subsubsection{Esempio}
            Vediamo ora un esempio: abbiamo un bambino di \textbf{massa} $m$ e uno scivolo di \textbf{lunghezza} $L$ e con \textbf{inclinazione} $\theta$. Quanto \textbf{tempo} impiega il bambino a scivolare sullo scivolo? Faremo 2 versioni di questo problema, la prima utilizzando un normale \textbf{moto uniformemente accelerato} e la seconda \textbf{introducendo anche l'attrito}. Vediamo una rappresentazione grafica del problema:
            \framedImg{5}{L06-img005}

            \paragraph{Versione senza attrito}
                Se proviamo a risolvere il problema senza considerare l'attrito, ci troviamo in presenza di un "semplice" \textbf{moto uniformemente accelerato}. Come prima cosa, \textbf{analizziamo le forze in campo} e calcoliamo le forze risultanti parallele agli assi. Iniziamo col dire che quella sull'asse y (che ricordo essere inclinato!) non ci interessa, infatti abbiamo la forza vincolante che bilancia, concentriamoci quindi solo sulla forza sull'asse x. Ricordiamo, per la \textit{seconda legge della dinamica} [pag. \pageref{II_legge_dinamica}]che:
                \begin{align*}
                    &\vec{F}_x = m*a_x
                \end{align*}
                A questo punto a noi interessa trovare l'accelerazione $a_x$, quindi rigiriamo un po' questa formula ed "\textbf{espandiamo}" la nostra $\vec{F}_x$ utilizzando le formule viste prima [pag. \pageref{Risultanti_piano_inclinato}]:
                \begin{align*}
                    &\vec{F}_x = m*a_x => a_x = \frac{\vec{F}_x}{m} = \frac{m * g * sin(\theta)}{m} = g * sin(\theta)
                \end{align*}
                Ora che abbiamo l'accelerazione, possiamo recuperare le formule viste per il moto uniformemente accelerato [pag. \pageref{Formule_moto_unif_acc}]. In questo caso ci interessa la formula dello \textbf{spazio percorso} (dato che sappiamo che il nostro scivolo è lungo $L$), ovvero:
                \begin{align*}
                    &L = s_0+v_0(t-t_0)+\frac{1}{2}*a*(t-t_0)^2
                \end{align*}
                A questa formula possiamo fare delle semplificazioni, in particolare togliere il I ($s_0 = 0$) e il II ($v_0 = 0$) termine, ottenendo:
                \begin{align*}
                    &L = \frac{1}{2}*a*(t-t_0)^2
                \end{align*}
                Che sostituendo l'accelerazione e, per comodità, il tempo diventa:
                \begin{align*}
                    &L = \frac{1}{2}*a*(\Delta t)^2
                \end{align*}
                A noi interessa calcolare $\Delta t$, quindi ci rigiriamo la formula in questo modo:
                \begin{align*}
                    &\Delta t = \sqrt{\frac{2L}{g*sin(\theta)}}
                \end{align*}
                Introduciamo ora un po' di numeri, supponendo:
                \label{EsercizioConAttrito}
                \begin{align*}
                    &m = 20Kg && \theta = 30^\circ && L = 4 m && \Delta t = ?
                \end{align*}
                Otteniamo il tempo
                \begin{align*}
                    &\Delta t = \sqrt{\frac{2L}{g*sin(\theta)}} = \sqrt{\frac{2*4m}{9,8m/s^2*sin(30^\circ)}}=\sqrt{\frac{8 s^2}{9,8*1/2}} \approx 1,27 s
                \end{align*}

            \paragraph{Versione con attrito}
                Se aggiungiamo l'attrito, dobbiamo modificare un po' le nostre formule in modo da considerare l'attrito statico e dinamico:
                \begin{itemize}
                    \item \textbf{attrito statico}:
                    \begin{align*}
                        \begin{cases}
                            R_x = m*g*sin(\theta_{max})-A_{s, max} &=> m*g*sin(\theta_{max})-\mu_s * N= 0 \\
                            R_y = m*g*cos(\theta_{max})-N &=> N = m*g*cos(\theta_{max}) = 0
                        \end{cases}
                    \end{align*}
                    Da qui otteniamo che
                    \begin{align*}
                        &m*g*sin(\theta_{max})-\mu_s * N= 0 &&=> m*g*sin(\theta_{max})-\mu_s * m*g*cos(\theta_{max})= 0\\
                        & &&=> m*g*[sin(\theta_{max}) - \mu_s *cos(\theta_{max})] = 0
                    \end{align*}
                    Ora, dato che sappiamo che $m*g$ non è nullo, possiamo dedurre che sia la seconda parte ad annullarsi, quindi:
                    \begin{align*}
                        & sin(\theta_{max}) - \mu_s *cos(\theta_{max}) = 0 && => \mu_s *cos(\theta_{max}) = sin(\theta_{max})\\
                        & && => \mu_s = \frac{sin(\theta_{max})}{cos(\theta_{max})}\\
                        & && => \textcolor{Red}{\mu_s = tan(\theta_{max})}
                    \end{align*}
                    Nota che con "$\theta_{max}$" indica l'angolo massimo oltre al quale il nostro oggetto sul piano inclinato inizia a muoversi, superando la soglia di attrito statico, senza l'applicazione di forze esterne;
                    \item \textbf{attrito dinamico}: in questo caso, la componente che ci interessa trovare è l'accelerazione sull'asse x ($a_x$). Partiamo quindi da quello che conosciamo e cerchiamo di rigirarlo un po' per ottenere l'accelerazione:
                    \begin{align*}
                        \begin{cases}
                            R_x = m*g*sin(\theta)-A_{d} &=> m*g*sin(\theta)-\mu_d * N= m*a_x \\
                            R_y = m*g*cos(\theta)-N = 0 &=> N = m*g*cos(\theta)
                        \end{cases}
                    \end{align*}
                    Da qui otteniamo che:
                    \begin{align*}
                        &m*g*sin(\theta)-\mu_d*m*g*cos(\theta) = m*a_x&&=>g*sin(\theta)-\mu_d*g*cos(\theta) = a_x\\
                        & && => \textcolor{Red}{a_x = g*(sin(\theta)-\mu_d*cos(\theta))}
                    \end{align*}
                \end{itemize}

        \subsubsection{Esempio di calcolo del coefficiente di attrito dinamico}
            Supponiamo di avere lo stesso identico esercizio di prima, però ora \textbf{entra in gioco anche l'attrito}. Supponendo che l'oggetto impieghi $\Delta t_{att}$ secondi a percorrere tutto il piano, a quanto corrisponde il coefficiente di attrito dinamico?\\
            Iniziamo con \textbf{l'equazione dello spazio per il moto uniformemente accelerato} e la rigiriamo un po':
            \begin{align*}
                &s = s_0 + v_0*\Delta t + \frac{1}{2}*a*(\Delta t)^2 &&=> L = 0 + 0*\Delta t + \frac{1}{2}*a*(\Delta t)^2\\
                & && => L = \frac{1}{2}*a*(\Delta t)^2\\
                & && => L = \frac{1}{2}*a*(\Delta t)^2
            \end{align*}
            Ora, sappiamo che, in un contesto di moto con attrito dinamico, $\vec{R}_x = \vec{P}_\parallel -\vec{A}_D = m * a_x$, utilizzando i calcoli visti prima otteniamo che l'accelerazione vale:
            \begin{align*}
                a = g*(sin(\theta) - \mu_d*cos(\theta))
            \end{align*}
            Mettiamo tutti insieme e otteniamo:
            \begin{align*}
                L = \frac{1}{2}*g*(sin(\theta) - \mu_d*cos(\theta))*(\Delta t)^2
            \end{align*}
            La modifichiamo per ottenere il valore di $\mu_d$:
            \begin{align*}
                &L = \frac{1}{2}*g*(sin(\theta) - \mu_d*cos(\theta))*(\Delta t)^2 && \mu_d =\frac{sin(\theta)}{cos(\theta)}-\frac{2L}{g*\Delta t^2*cos(\theta)}
            \end{align*}
            Ora ci basta semplicemente inserire i valori (li puoi ritrovare all'esercizio prima, [pag. \pageref{EsercizioConAttrito}]) ed otteniamo il nostro risultato $\mu_d \approx 0,28$.

    \subsection{Pendolo semplice}
        Introduciamo il pendolo, ovvero una \textbf{massa} agganciata ad un punto di ancoraggio tramite una \textbf{fune inestensibile}. Supponiamo che \textbf{non ci sia alcun attrito}, quindi il nostro pendolo continuerà ad oscillare tramite un \textbf{moto perpetuo}. Cominciamo introducendo i 2 sistemi di riferimento: il primo ci serivirà per specificare delle cose nel secondo.

        \subsubsection{Sistema di riferimento sul peso}
            \framedImg{50}{L07-img001}
            Nota che questo sistema di riferiemento \textbf{si muove assieme al peso}, in particolare abbiamo che l'asse y che è \textbf{normale alla traiettoria} e l'asse x che è \textbf{tangente alla traiettoria}. Iniziamo analizzando le forze in gioco. In questo caso abbiamo solo la \textbf{forza peso} che può essere scomposta in questo modo:
            \begin{align*}
                &\vec{P}=
                \begin{cases}
                    \vec{P}_\parallel &= -m*g*sin(\theta)\\
                    \vec{P}_\bot &= m*g*cos(\theta)
                \end{cases}
            \end{align*}
            Inoltre, dato che consideriamo \textbf{il cavo inestensibile}, abbiamo una \textbf{forza di tensione} ($\vec{T}$) uguale e opposta a $\vec{P}_bot$ che la bilancia. A questo punto possiamo analizzare \textbf{le risultanti}:
            \begin{align*}
                &\vec{R}_x =\vec{P}_\parallel= -m*g*sin(\theta)\\
                &\vec{R}_y =\vec{P}_\bot + \vec{T}= m*g*cos(\theta) - T = 0
            \end{align*}

        \subsubsection{Sistema di riferimento nella posizione di equilibrio}
            Ora, cambiamo sistema di riferimento e utilizziamo quello "cartesiano" con centro nella posizione di equilibri del nostro pendolo:
            \framedImg{50}{L07-img002}
            Con questo nuovo riferimento, possiamo trovare facilmente la poszione (x, y) del nostro peso:
            \begin{align*}
                &x(t) = R * sin[\theta(t)]\\
                &y(t) = R - R*cos[\theta(t)]
            \end{align*}
            Supponendo che $R$ sia la lunghezza del nostro cavo. Nota anche che le coordinate dipendono dal tempo, infatti \textbf{sarà l'angolo $\theta$ a variare col tempo}. Il problema è \textbf{in che modo}? Ci torneremo in seguito. Questo sistema di riferimento ci permette inoltre di scomporre le nostre 2 forze (peso e tensione) in questo modo:
            \begin{align*}
                &\vec{P} =
                \begin{cases}
                    \vec{P}_x = 0\\
                    \vec{P}_y = -m*g
                \end{cases}
                &&\vec{T} =
                \begin{cases}
                    \vec{T}_x = -\textcolor{Red}{T}sin(\theta) = -\textcolor{Red}{m*g*cos(\theta)}sin(\theta)\\
                    \vec{T}_y = \textcolor{Red}{T}cos(\theta) =  \textcolor{Red}{m*g*cos(\theta)}*cos(\theta)
                \end{cases}
            \end{align*}
            Nella scomposizione della tensione possiamo usare la formula che abbiamo trovato prima, ovvero "$m*g*cos(\theta) - T = 0$" (prima dobbiamo rigirarla un po'). Con queste "nuove" scomposizioni, possiamo calcolare le forze risultanti sugli assi:
            \begin{align*}
                \vec{R}=
                \begin{cases}
                    \vec{R}_x = \vec{T}_x = -m*g*sin(\theta)*cos(\theta)\\
                    \vec{R}_y = \vec{P} + \vec{T}_y = -m*g + m*g*cos^2(\theta) = -m*g*(1-cos^2(\theta)) = -m*g*sin^2(\theta)
                \end{cases}
            \end{align*}
            A questo punto, per la \textbf{II legge della dinamica}, sappiamo per definizione che $R = m*a$, quindi:
            \begin{align*}
                \vec{R}=
                \begin{cases}
                    \vec{R}_x = m*a_x = -m*g*sin(\theta)*cos(\theta) & => a_x = -g*sin(\theta)*cos(\theta)\\
                    \vec{R}_y = m*a_y = -m*g*sin^2(\theta) & => a_y = -g*sin^2(\theta)
                \end{cases}
            \end{align*}

            \paragraph{Calcolare spazio, velocità e accelerazione}
                Con le operazioni precedenti abbiamo scoperto che;
                \begin{align*}
                    &\begin{cases}
                        a_x = -g*sin(\theta)*cos(\theta) = \frac{d^2x}{dt^2}\\
                        a_y = -g*sin^2(\theta) = \frac{d^2y}{dt^2}
                    \end{cases}&&
                    \begin{cases}
                        x(t) = R*sin[\textcolor{Red}{\theta(t)}]\\
                        y(t) = R-R*cos[\textcolor{Red}{\theta(t)}]
                    \end{cases}
                \end{align*}
                Il nostro problema è "$\textcolor{Red}{\theta(t)}$", \textbf{come varia $\theta$ in funzione del tempo?} Noi però abbiamo il valore dell'accelerazione e sappiamo che \textbf{la derivata seconda ($\frac{d^2x}{dt^2}$/$\frac{d^2y}{dt^2}$) dello spostamento corrisponde all'accelerazione!} Quindi deriviamo la funzione dello spostamento, ricordando però che noi \textbf{non conosciamo la derivata di $\theta(t)$}:
                \begin{align*}
                    &
                    \begin{cases}
                        x(t)=R*sin[\theta(t)]\\
                        y(t)=R-R*cos[\theta(t)]
                    \end{cases}
                    && =>
                    \begin{cases}
                        x^{(1)}(t)=R*cos(\theta)*\theta^{(1)}\\
                        y^{(1)}(t)=R*sin(\theta)*\theta^{(1)}
                    \end{cases}\\
                    & && =>
                    \begin{cases}
                        x^{(2)}(t)=R*[-sin(\theta)*\theta^{(1)}*\theta^{(1)}+cos(\theta)*\theta^{(2)}] \\
                        y^{(2)}(t)=R*[cos(\theta)*\theta^{(1)}*\theta^{(1)}+sin(\theta)*\theta^{(2)}]
                    \end{cases}\\
                    & && =>
                    \begin{cases}
                        x\rightarrow -g*sin(\theta)*cos(\theta) = R*[cos(\theta)*\theta^{(2)}-sin(\theta)*\theta^{(1)2}] \\
                        y\rightarrow -g*sin^2(\theta) = R*[sin(\theta)*\theta^{(2)}+cos(\theta)*\theta^{(1)2}]
                    \end{cases}\\
                    & && =>
                    \begin{cases}
                        x\rightarrow -g*sin(\theta)*cos(\theta) = l*[cos(\theta)*\theta^{(2)}-sin(\theta)*\theta^{(1)2}] \\
                        y\rightarrow -g*sin^2(\theta) = l*[sin(\theta)*\theta^{(2)}+cos(\theta)*\theta^{(1)2}]
                    \end{cases}\\
                    & && =>
                    \begin{cases}
                        x\rightarrow -\frac{g}{l}*sin(\theta)*cos(\theta) = [cos(\theta)*\theta^{(2)}-sin(\theta)*\theta^{(1)2}] \\
                        y\rightarrow -\frac{g}{l}*sin^2(\theta) = [sin(\theta)*\theta^{(2)}+cos(\theta)*\theta^{(1)2}]
                    \end{cases}
                \end{align*}

                Adesso, prendiamo il "$\frac{g}{l}$" e lo chiamiamo "$\omega^2$", perche? Perchè SI, lo vuole Iuppa. Di conseguenza le nostre formule diventano:
                \begin{align*}
                    \begin{cases}
                        x\rightarrow -\omega^2*sin(\theta)*cos(\theta) = [cos(\theta)*\theta^{(2)}-sin(\theta)*\theta^{(1)2}] \\
                        y\rightarrow -\omega^2*sin^2(\theta) = [sin(\theta)*\theta^{(2)}+cos(\theta)*\theta^{(1)2}]
                    \end{cases}
                \end{align*}

            \paragraph{Le piccole oscillazioni}
                Il problema con questa formula è che \textbf{è un casino calcolarsi seni e coseni}, quindi noi possiamo \textbf{ragionare in termini di piccole oscillazioni}, in modo da approssimare seni e coseni con \textbf{Taylor}!

                \begin{mdframed}
                    \begin{align*}
                        & sin(\theta) \approx \theta && cos(\theta)\approx 1-\textcolor{Red}{\frac{\theta^2}{2}}\\
                        & tg(\theta)\approx\theta && (1+\theta)^\alpha \approx +\alpha\theta\\
                        & e^{\alpha\theta}\approx1+\alpha\theta&&\sqrt{1+\theta}\approx 1+\frac{\theta}{2}
                    \end{align*}
                    \begin{center}
                        Nota che le \textcolor{Red}{parti in rosso}, dato che consideriamo solo $\theta$ piccoli, le possiamo togliere!
                    \end{center}
                \end{mdframed}

                Con "piccole oscillazioni" intendiamo $\theta << 1$, ovvero "$\theta$ molto minore di 1" (inteso come radianti, in gradi possiamo immaginare che $\theta$ non superi i 4 gradi circa). Ora, considerando solo queste piccole oscillazioni, possiamo approssimare le formule in questo modo:

                \begin{align*}
                    &\begin{cases}
                        x\rightarrow -\omega^2*sin(\theta)*cos(\theta) = [cos(\theta)*\theta^{(2)}-sin(\theta)*\theta^{(1)2}] \\
                        y\rightarrow -\omega^2*sin^2(\theta) = [sin(\theta)*\theta^{(2)}+cos(\theta)*\theta^{(1)2}]
                    \end{cases}
                    &&=>
                    \begin{cases}
                        x\rightarrow -\omega^2*\theta*1 = [1*\theta^{(2)}-\textcolor{Red}{\theta*\theta^{(1)2}}] \\
                        y\rightarrow -\omega^2*\theta^2 = [\textcolor{Red}{\theta*\theta^{(2)}}+1*\theta^{(1)2}]
                    \end{cases}\\
                    & &&=>
                    \begin{cases}
                        x\rightarrow -\omega^2*\theta = \theta^{(2)} \\
                        y\rightarrow -\omega^2*\theta^2 = \theta^{(1)2}
                    \end{cases}
                \end{align*}
                \begin{center}
                    Nota che le \textcolor{Red}{parti in rosso}, in quanto \textbf{moltiplicate per $\theta<<1$} diventano trascurabili, quindi le togliamo
                \end{center}

                Prendiamo in considerazione solo la formula per la x, abbiamo quindi che:
                \begin{align*}
                    -\omega^2*\theta = \theta^{(2)} = \textcolor{Red}{\frac{d^2\theta}{dt^2}+\omega^2*\theta = 0}
                \end{align*}
                Ovvero \textbf{l'equazione del moto armonico}! Da qui possiamo dire quindi che:
                \begin{align*}
                    & \omega = \frac{2\pi}{T} && => \frac{2\pi}{T} = \sqrt{\frac{g}{l}}\\
                4& && => \textcolor{Red}{T = 2\pi*\sqrt{\frac{l}{g}}}\\
                \end{align*}
                Ricorda che $T$ in questo caso corrisponde al \textbf{periodo} e non alla forza di tensione.

    \subsection{Esercizio sulla dinamica}\label{esercizioDinamicaMolla}
        Vediamo ora un esercizio riassuntivo sulla dinamica, che cerca di raggruppare tutto quello che abbiamo visto. Supponiamo di essere in questa situazione:
        \framedImg{40}{L07-img003}
        Abbiamo un \textbf{peso di massa $\bf{m}$} attaccato ad un \textbf{filo inestensibile di massa trascurabile} che poggia su una \textbf{molla in posizione di riposo con coefficiente elastico $\bf{k}$}. Ad un certo punto, il cavo \textbf{viene tagliato}, quindi il peso verrà \textbf{sostenuto interamente dalla molla}, che \textbf{inizierà a comprimersi ed allungarsi} (come si comporta una molla normale)
        \framedImg{40}{L07-img004}
        Supponiamo di avere i seguenti dati e di dover trovare i seguenti valori:
        \begin{align*}
            & k = 70 N/m&& m = 0,5 Kg && T = ?\\
            &\Delta l_{max} = ? && z_{vMax}=? && v_{max}=?
        \end{align*}
        Con:
        \begin{itemize}
            \item $T$ periodo;
            \item $\Delta l_{max}$ accorciamento massimo;
            \item $z_{vMax}$ z dove la velocità è massima;
            \item $v_{max}$ velocità massima;
        \end{itemize}
        Allora, come prima cosa \textbf{visualizziamo quali forze agiscono nel sistema}, in particolare abbiamo:
        \begin{itemize}
            \item la \textbf{forza peso} $\vec{P}=-m*g$;
            \item la \textbf{forza elastica} $\vec{F}_{el}=-k*(\vec{z}-\vec{z}_0)$. La parte tra parentesi viene chiamata "\textbf{forza di richiamo}" e \textbf{dipende dal sistema di riferimento che adottiamo} (in particolare la $z_0$, che indica la \textbf{posizione di riposo} della nostra molla, nel nostro caso [in base al sistema di riferimento che abbiamo adottato] vale 0).
        \end{itemize}
        \framedImg{40}{L07-img005}


        \subsubsection{Studio della dinamica del problema}
            Iniziamo considerando le varie risultanti delle forze che agiscono sul nostro asse z (semplicemente agiscono solo forze verticali, quindi ignoriamo l'asse orizzontale). Iniziamo considerando le 2 forze che abbiamo già visto prima:
            \begin{align*}
                \begin{cases}
                    \vec{P} = -m*g*\hat{z}\\
                    \vec{F}_{el}=-k*(\vec{z}-\vec{z}_0) = \vec{F}_{el}=-k*(\vec{z}-0) = \vec{F}_{el}=-k*z*\hat{z}
                \end{cases}
            \end{align*}
            Nota che i \textbf{versori} $\hat{z}$ servono solo per descrivere la direzione della forza, poi li possiamo togliere per semplicità. Ora, avremmo che la risultante di queste forze corrisponde alla loro somma:
            \begin{align*}
                \vec{R} = \vec{P}+\vec{R}_{el} = -m*g*\hat{z}-k*z*\hat{z} = m*\vec{a}\ (per\ definizione)
            \end{align*}
            A questo punto noi sappiamo però che l'accelerazione corrisponde alla \textbf{derivata seconda dello spazio}, quindi possiamo scrivere:
            \begin{align*}
                &-m*g*\hat{z}-k*z*\hat{z} = m*\frac{d^2z}{dt^2}*\hat{z} &&=> -g - \frac{k}{m}*z = \frac{d^2z}{dt^2}\\
                & && => \textcolor{Red}{\frac{d^2z}{dt^2}+\frac{k}{m}*z=-g}
            \end{align*}
            La formula \textcolor{Red}{in rosso} sappiamo che è \textbf{l'equazione armonica}! In questo caso ponimao $\omega^2=\frac{k}{m}$, ottenendo:
            \begin{align*}
                \frac{d^2z}{dt^2}+\omega^2*z=-g
            \end{align*}
            Ci sono un po' di "problemi" però, rispetto all'equazione "classica" non abbiamo $=0$ ma $=-g$, quello viene definito \textbf{membro forzante} e semplicemente \textbf{sfasa la nostra posizione di equilibrio}, in questo caso la sposta verso il basso. Per definizione del moto armonico [pag.\pageref{Moto armonico}] che possiamo indicare lo spostamento nel tempo in questo modo:
            \begin{align*}
                z(t) = \textcolor{Orange}{A*sin(\omega t+\varphi)+\textcolor{Red}{w(t)}}
            \end{align*}
            In questo caso dobbiamo \textcolor{Red}{aggiungere questa parte in rosso} perché dobbiamo \textbf{compensare in qualche modo quell' "=-g"}. Ora, per calcolare velocità ed accelerazione (quest'ultima è quella che ci interessa) ci basta semplicemente derivare la formula trovata prima:
            \begin{align*}
                &z(t) = A*sin(\omega t+\varphi)+\textcolor{Red}{w(t)}&&=>\frac{dz}{dt}=A\omega*cos(\omega t+\varphi)+\frac{dw}{dt}\\
                & &&=>\textcolor{OliveGreen}{\frac{d^2z}{dt^2}=-A\omega^2*sin(\omega t+\varphi)+\frac{d^2w}{dt^2}}\\
            \end{align*}
            \textcolor{OliveGreen}{Questa} è la formula che ci interessa! Infatti ora possiamo sostituirla all'interno dell'equazione iniziale, ovvero:
            \begin{align*}
                &\frac{d^2z}{dt^2}+\frac{k}{m}*z=-g&&=> \textcolor{OliveGreen}{-A\omega^2*sin(\omega t+\varphi)+\frac{d^2w}{dt^2}} +\omega^2(\textcolor{Orange}{A*sin(\omega t+\varphi)+w(t)}) = -g
            \end{align*}
            Bene, abbiamo la formula matematica che descrive la dinamica del nostro sistema. Ora dobbiamo fissare un po' di valori che non conosciamo al momento, in particolare $\varphi, \omega$ e $ w(t)$, come facciamo? Studiamo la nostra formula, che ricordiamo essere in \textbf{funzione del tempo}, in alcuni punti (possibilmente comodi). Cominciamo considerando $t=0$:
            \begin{align*}
                &-A\omega^2*sin(\omega*0+\varphi)+\frac{d^2w}{dt^2} +\omega^2(A*sin(\omega *0+\varphi)+w(0)) = -g &&=> \frac{d^2w}{dt^2} +\omega^2*w(0) = -g
            \end{align*}
            Ci siamo semplificati i seni, ora dobbiamo soltanto trovare il valore della nostra $w(0)$: per toglierci la derivata (che è abbastanza scomoda) possiamo supporre che $w$ sia \textbf{costante in funzione del tempo}, quindi otterremo:
            \begin{align*}
                &\frac{d^2w}{dt^2} +\omega^2*w(0) = -g &&=> 0 + \omega^2*w=-g\\
                & &&=>\textcolor{Red}{w=-\frac{g}{\omega^2}}
            \end{align*}
            Ecco fatto, ora possiamo tornare un po' di equazioni prima e \textbf{sostituire il valore che abbiamo trovato di w}:
            \begin{align*}
                &z(t) = A*sin(\omega t+\varphi)+w(t)&&=> z(t) = A*sin(\omega t+\varphi)-\frac{g}{\omega^2}
            \end{align*}
            Ora ci manca solo da trovare il valore di $A$ e di $\varphi$, per trovarli possiamo usare il fatto che \textbf{sia lo spazio che la velocità in t=0 valgono 0}! Vediamo:
            \begin{align*}
                &
                \begin{cases}
                    z(t) = A*sin(\omega*0+\varphi)+-\frac{g}{\omega^2} = 0\\
                    \frac{dz}{dt} = v(t)=A\omega*cos(\omega*0+\varphi) = 0
                \end{cases}
                &&=>
                \begin{cases}
                    z(t) = A*sin(\varphi)+-\frac{g}{\omega^2} = 0\\
                    \frac{dz}{dt} = v(t)=A\omega*cos(\varphi) = 0
                \end{cases}\\
                & &&=>
                \begin{cases}
                    A = \frac{g}{\omega^2}\\
                    \varphi=\frac{\pi}{2}
                \end{cases}
            \end{align*}
            Quindi arriviamo alla fine con questa formula che descrive il nostro moto:
            \begin{align*}
                z(t)=\frac{g}{\omega^2}*sin(\omega t + \frac{\pi}{2})-\frac{g}{\omega^2} = \textcolor{Red}{\frac{g}{\omega^2}*cos(\omega t)-\frac{g}{\omega^2}}
            \end{align*}
            Come la interpretiamo? Sostanzialmente come un normalissimo \textbf{moto armonico}, con:
            \begin{itemize}
                \item $a = \frac{g}{\omega^2}\rightarrow$ampiezza, il valore massimo (o minimo) della curva;
                \item $\omega =\frac{2\pi}{T}=>T=\frac{2\pi}{\omega}\rightarrow$periodo, misurato in secondi;
                \item $\varphi = \frac{\pi}{2}\rightarrow$fase, indica da che punto parte il grafico, nel caso di questo esercizio l'abbiamo tolta trasformando il $sin$ in $cos$;
            \end{itemize}
            \framedImg{5}{L07-img006}
            Ora che abbiamo questa formula, abbiamo tutto! Possiamo facilmente calcolare tutti i punti richiesti dal problema, vediamo come.

        \subsubsection{Allungamento massimo}
            Come possiamo vedere dall'immagine precedente, l'ampiezza è la distanza massima dal punto di equilibrio, quindi ci basta prendere \textbf{2 volte l'ampiezza} (nota che il "-" è dovuto al sistema di riferimento adottato):
            \begin{align*}
                \Delta l_{max}=-2\frac{g}{\omega^2}=-2*\frac{m*g}{K}\approx-14cm
            \end{align*}

        \subsubsection{Posizione di velocità massima}
            All'istante 0, anche la velocità vale 0. Successivamente la velocità inizierà ad aumentare finché non raggiungerà il massimo. Per capire dove si trova il massimo possiamo \textbf{studiare la derivata associata e vedere dove questa vale 0}, quindi:
            \begin{align*}
                &\frac{dz}{dt}=v(t)=-\frac{g}{\omega^2}*\omega*sin(\omega t)+ 0 &&=>\frac{dv}{dt}=-\frac{g}{\omega^2}*\omega^2*cos(\omega t)\\
                & &&=>\textcolor{Red}{\frac{dv}{dt}=-g*cos(\omega t)}
            \end{align*}
            Quando \textcolor{Red}{questa} equazione vale 0? Dobbiamo \textbf{annullare il coseno}, quindi porre $\omega t = \frac{\pi}{2}=>t=\frac{\pi}{2\omega}$. In che posizione ci troviamo con $t=\frac{\pi}{2\omega}$? Vediamo:
            \begin{align*}
                z(\frac{\pi}{2\omega})=\frac{g}{\omega^2}*cos(\omega * \frac{\pi}{2\omega})-\frac{g}{\omega^2} = \frac{g}{\omega^2}*0-\frac{g}{\omega^2} = \textcolor{Red}{-\frac{g}{\omega^2}}
            \end{align*}
            Come si può facilmente notare, abbiamo la velocità nella \textbf{posizione di equilibro della molla}! Potevamo anche arrivarci osservando il disegno, infatti nel punto di equilibrio il peso \textbf{smette di prendere velocità ed inizia a rallentare in seguito alla forza elastica}! Calcoliamo il numero:
            \begin{align*}
                z_{vMax}=-\frac{g}{\omega^2} = -\frac{g*m}{k}\approx -7cm
            \end{align*}

        \subsubsection{Velocità massima}
            Prima abbiamo scoperto che la velocità massima si ottiene nella $z_{eq}$, che si raggiunge in $t=\frac{\pi}{2*\omega^2}$
            \begin{align*}
                v(\frac{\pi}{2*\omega^2})=-\frac{g}{\omega}*sin(\omega \frac{\pi}{2*\omega^2})=-\frac{g}{\omega}*1 = -\frac{g}{\omega} = -g*\sqrt{\frac{m}{k}}\approx-0,84m/s
            \end{align*}
            Di nuovo, il "-" della velocità è sempre dovuto al sitema di riferimento che abbiamo adottato.

        \subsubsection{Calcoliamo la T}
            C'è un problema: nella lezione Iuppa considera T come \textbf{il periodo} mentre nel video di Youtube la considera come \textbf{forza di tensione del cavo iniziale}. Calcoliamo entrambe.

            \paragraph{Il periodo}
                Nulla di complesso, noi sappiamo che:
                \begin{align*}
                    &\omega=\frac{2\pi}{T}&&=>T=\frac{2\pi}{\omega}\\
                    & &&=> T=2\pi*\sqrt{\frac{m}{k}}\approx 0,52s
                \end{align*}

            \paragraph{La tensione}
                Ancora, niente di particolare. Sappiamo semplicemente che la \textbf{tensione è uguale ed opposta alla forza peso}, quindi:
                \begin{align*}
                    &\vec{T}=-\vec{P} &&=> \vec{T}=-m*g\approx 4,9N
                \end{align*}
