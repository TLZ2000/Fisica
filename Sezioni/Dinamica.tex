\section{Dinamica}
  \subsection{Leggi della dinamica}
    La \textbf{dinamica} si occupa dello studio del moto dei corpi a partire dalle sue cause(\textbf{forze}), ovvero delle circostanze che lo determinano e lo modificano nel \textbf{tempo} e nello \textbf{spazio} del suo sistema di riferimento. \href{https://it.wikipedia.org/wiki/Dinamica}{Wikipedia}
    Le leggi della dinamica sono 3 e sono le seguenti:
    \begin{enumerate}
      \item \underline{\textbf{Legge di Inerzia (I legge)}}: un corpo rimane nel suo stato di quiete finchè non intervengono agenti esterni a modificarne questo stato. Questa legge vale sono in sistemi di riferimento inerziali;\\
      \item \underline{\textbf{Legge di Newton (II legge)}}:
      Viene definita la \textbf{quantità di moto} come $\vec{p}=m\vec{v}$, ovvero massa per velocità. Successivamente viene definita la forza ($\vec{F}$) come segue:
      \begin{align*}
        \vec{F}=\frac{d\vec{p}}{dt}=\frac{d(m\vec{v})}{dt}=\frac{dm}{dt}\vec{v} + m\frac{d\vec{v}}{dt}
      \end{align*}
      dove $m$ è la \textbf{massa inerziale}, ovvero la capacità di un corpo di opporsi alle variazioni del suo stato di moto, questa mette in relazione la velocità alla forza.\\
      Nel caso in cui la massa non varia, allora la forza può essere definita come $\vec{F}=m\cdot\vec{a}$.\\
      L'unità di misura della \textbf{forza} è il Newton ($N$) definito come $\frac{kg\cdot m}{s^2}$.\\
      \item \underline{\textbf{Principio di azione e reazione (III legge)}}: Quando il corpo 1 esercita una forza $\vec{F}$ sul corpo 2, quest'ultimo esercita sul corpo 1 una forza $-\vec{F}$, uguale e opposta.
      \begin{align*}
        \vec{F}_{1\rightarrow 2}=-\vec{F}_{2\rightarrow 1}
      \end{align*}
    \end{enumerate}
    Osserviamo che la \textbf{prima legge} potrebbe sembrare un caso particolare della \textbf{seconda legge}, con $\vec{F}=\vec{0}$, ma in realtà non è così, infatti la seconda e la terza legge sono valide solo all'interno di sistemi di riferimento inerziali, che sono definiti dalla prima legge.

  \subsection{Forze impulsive}
    L'\textbf{impulso} $\vec{P}$ è definito come la variazione di quantità di moto $\Delta\vec{p}$ in un $\Delta t$ piccolo, ovvero:
    \begin{align*}
      \vec{P}=\Delta\vec{p}=\int_0^t{\vec{F}dt}
    \end{align*}
    E la \textbf{forza impulsiva} come:
    \begin{align*}
      \vec{F}_{imp}=\frac{\Delta\vec{p}}{\Delta t}
    \end{align*}

    \subsubsection{Esempio forze impulsive}
      Supponiamo di avere un pavimento ed una palla che viene lasciata in aria. Questa palla cadrà verso il pavimento fino a raggiungerlo, rimbalzare su esso e tornare in sù (assumiamo che la velocità con cui torna in sù sia la stessa con cui cade, quindi non agiscono fattori esterni come attriti, ecc.).

      \framedImg{35}{L04-img004}

      Se ho un vettore velocità $\vec{v}$, allora ho che:
      \begin{align*}
        &\vec{p_i}=m\vec{v_i}=-m\vec{v}\\
        &\vec{p_f}=m\vec{v_f}=m\vec{v}\\
        &\Delta\vec{p}=\vec{p_f}-\vec{p_i}=2m\vec{v}
      \end{align*}

    \subsubsection{Esercizio su forze impulsive}
      Supponiamo di avere i seguenti dati e di dover calcolare $\vec{F}_{imp}$ (forza impulsiva):
      \begin{align*}
        &m=98g&&v=10.2\frac{m}{s}&&\Delta t=100ms\\
      \end{align*}
      Procediamo ora quindi con calcolare $\Delta\vec{p}$ usando la formula appena calcolata sopra e una volta ottenuto il valore calcoliamo la $\vec{F}_{imp}$:
      \begin{align*}
        &\Delta\vec{p}=2m\vec{v}=2\cdot 10.2\frac{m}{s}\cdot 0.098kg=0.99\frac{kg\cdot m}{s}\\
        &\vec{F}_{imp}=\frac{\Delta\vec{p}}{\Delta t}=\frac{0.99\frac{kg\cdot m}{s}}{0.1s}=9.99\frac{kg\cdot m}{s^2}=9.99N
      \end{align*}

  \subsection{Esercizi sulla dinamica}
    Supponiamo di avere un'oggetto appeso a due fili, che sono appesi al tetto, alla stessa distanza dall'oggetto e vogliamo trovare $\vec{T_1}$ e $\vec{T_2}$ tensioni dei fili, avendo i seguenti dati:
    \begin{align*}
      &m=100g&&\theta=60^{\circ}\\
    \end{align*}

    \framedImg{60}{L04-img005}

    Notiamo che l'oggetto resta fermo, quindi oltre a $\vec{p}$ (\textbf{forza peso}), su esso agiscono altre forze la cui somma è uguale e opposta a $\vec{p}$. Abbiamo quindi che la \textbf{risultante delle forze} $\vec{R}=\vec{0}$.\\
    Ora possiamo notare che $|\vec{T_1}|=|\vec{T_2}|=T$ e abbiamo le seguenti forze:
    \begin{align*}
      &\vec{p}=-mg\hat{y}\\
      &\vec{T_1}=T_x\hat{x}+T_y\hat{x}=Tsin\theta \hat{x}+Tcos\theta \hat{y}\\
      &\vec{T_2}=-T_x\hat{x}+T_y\hat{y}=-Tsin\theta \hat{x}+Tcos\theta \hat{y}
    \end{align*}
    Ora ci ricordiamo che $\vec{R}=\vec{0}$ quindi:
    \begin{align*}
      &\vec{R}=\vec{0}=\vec{P}+\vec{T_1}+\vec{T_2}=>
      \begin{cases}
        &R_x=0\\
        &R_y=0
      \end{cases}=>\\
      &=>\begin{cases}
        &R_x=0=Tsin\theta-Tsin\theta\\
        &R_y=0=-mg+Tcos\theta+Tcos\theta=-mg+2Tcos\theta
      \end{cases}
    \end{align*}
    La prima equazione del sistema vale zero, ora dalla seconda ricaviamo $T$:
    \begin{align*}
    mg=2Tcos\theta=>T=\frac{mg}{2Tcos\theta}=\frac{0.1kg\cdot 9.8\frac{m}{s^2}}{2\cdot\frac{1}{2}}=0.98N
    \end{align*}
